\anchor{anchor-instructor-preface}
\section*{Preface for the instructor}% not numbered because in frontmatter; has to be section so page number appears
\addcontentsline{toc}{section}{\protect\link{instructor-preface}{Preface for the instructor}}
\label{instructor-preface}

I've attempted an innovation in the order of topics for freshman E\&M,
the goal being to follow the logical sequence while also providing
plenty of opportunities for relating abstract ideas to hands-on
experience. The typical sequence starts by slogging through Coulomb's
law, the electric field, and Gauss's law, none of which are well
suited to practical exploration in the laboratory. In this book, each of the 
first 5 chapters is short and includes a laboratory exercise that can
be completed in about an hour and a half. The approach I've taken is
to introduce the electric and magnetic field on an equal footing
(which is in fact the way the subject was developed historically). As
empirically motivated postulates, we take some primitive ideas about
relativity along with the expressions for the energy and momentum
density of the fields. 

Another goal is to introduce the laws of physics in their natural,
local form, i.e., Maxwell's equations in differential rather than
integral form, without getting bogged down in an extensive development
of the toolbox of vector calculus that would be more appropriate in an
honors text like Purcell. Much of the necessary apparatus of div,
grad, and curl is developed first in visual or qualitative form. At
the end of the book we circle back and do some problem solving using
the integral forms of Maxwell's equations. 
