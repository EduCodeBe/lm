A capacitor is a device that stores energy in an electric field.
The simplest example consists of two parallel conducting plates.
The energy is proportional to the square of the field
strength, which is proportional to the charges on the  
plates. If we assume the plates carry charges that are the 
same in magnitude, $+q$ and $-q$, then the energy stored in
the capacitor must be proportional to $q^2$. For historical 
reasons, we write the constant of proportionality as $1/2C$,
\begin{equation*}
                E    =   \frac{1}{2C}q^2\eqquad.
\end{equation*}
The constant $C$ is a geometrical property of the
capacitor, called its capacitance.
Based on this definition, the units of capacitance must be
coulombs squared per joule, and this combination is more
conveniently abbreviated as the farad\index{farad!defined},
$1\ \zu{F}=1\ \zu{C}^2/\zu{J}$.

Voltage is electrical potential energy per unit charge, so
the voltage difference across a 
capacitor is related to the amount by which its energy would
increase if we increased the absolute values of the charges on the plates
 from $q$ to $q+\Delta q$:
\begin{align*}
                V         &=    (E_{q+\Delta q}-E_q)/\Delta q   \\
                        &= \frac{\Delta}{\Delta q}\left(\frac{1}{2C}q^2\right) \\
                        &= \frac{q}{C} 
\end{align*}
Many books
use this as the definition of capacitance.
It follows from this relation that
capacitances in parallel add, $C_\text{equivalent}=C_1+C_2$, whereas when
they are wired in series, it is their inverses that add,
$C_\text{equivalent}^{-1}=C_1^{-1}+C_2^{-1}$.
