When an unbalanced distribution of charges is subject to an external
electric field $\vc{E}$, it experiences a torque $\btau$. We define the electric dipole
moment $\vc{d}$ to be the vector such that
\begin{equation}
  \btau=\vc{d}\times\vc{E}.
\end{equation}
When the total charge is zero, this relation uniquely defines $\vc{d}$, regardless of
the point chosen as the axis.
In the simplest case, of charges $+q$ and $-q$ at opposite ends of a stick of
length $\ell$, the dipole moment has magnitude $q\ell$ and points from the negative
charge to the positive one.
The potential energy of a dipole in an external field is
\begin{equation}
  U = -\vc{d}\cdot\vc{E}.
\end{equation}
