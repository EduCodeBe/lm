When we look at the ``sea of arrows'' representation of a field,
\ref{fig:field-line-concept-stacked}/1, there is a natural visual tendency to
imagine connecting the arrows as in \ref{fig:field-line-concept-stacked}/2.
The curves formed in this way are called field lines, and they have a
direction, shown by the arrowheads. 

% fig {"name":"field-line-concept-stacked","caption":"Two different representations of an electric field."}

Electric field lines originate from positive
charges and terminate on negative ones. We can choose a constant of proportionality that fixes
how coarse or fine the ``grain of the wood'' is, but once this choice is made 
the strength of each charge is shown by the number of lines that begin or end on it. For example, figure
\ref{fig:field-line-concept-stacked}/2
shows eight lines at each charge, so we know that $q_1/q_2=(-8)/8=-1$. Because lines never begin or end except on a charge,
we can always find the total charge inside any given region by subtracting the number of lines that go in from the number
that come out and multiplying by the appropriate constant of proportionality. Ignoring the constant, we can apply this
technique to figure \ref{fig:gauss-field-lines} to find $q_A=-8$, $q_B=2-2=0$, and $q_C=5-5=0$.

% fig {"name":"gauss-field-lines","caption":"The number of field lines coming in and out of each region depends on the total charge it encloses."}

Let us now make this description more mathematically precise.
Given a smooth, closed surface such as the ones in figure \ref{fig:gauss-field-lines},
we have an inside and an outside, so that at any point on the surface we can
define a unit normal $\hat{\vc{n}}$ (i.e., a vector with magnitude 1, perpendicular to the
surface) that points outward. Given an infinitesimally small piece of the surface, with
area $\der A$, we define an area vector $\der\vc{A}=\hat{\vc{n}}\der A$.
The infinitesimal flux $\der\Phi$ through this infinitesimal patch of the surface is defined
as $\der\Phi=\vc{E}\cdot\der\vc{A}$, and integrating over the entire surface gives the total flux
$\Phi=\int\der\Phi=\int \vc{E}\cdot\der\vc{A}$. Intuitively, the flux
measures how many field lines pierce the surface. Gauss's law states that
\begin{equation}
  q_\text{in}=\frac{\Phi}{4\pi k},
\end{equation}
where $q_\text{in}$ is the total charge inside a closed surface, and $\Phi$ is the
flux through the surface. (In terms of the constant $\epsilon_0=1/(4\pi k)$, we have 
$q_\text{in}=\epsilon_0\Phi$.)

Unlike Coulomb's law, Gauss's law holds in all circumstances, even when there are charges
moving in complicated ways and electromagnetic waves flying around. Gauss's law can be
thought of as a definition of electric charge. 
