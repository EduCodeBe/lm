Figure \ref{fig:swing-rock-simple} shows an overhead view of a person
swinging a rock on a rope. A force from the string is required to make
the rock's velocity vector keep changing direction.  If the string
breaks, the rock will follow Newton's first law and go straight
instead of continuing around the circle. Circular motion requires
a force with a component toward the center of the circle.

% fig {"name":"swing-rock-simple","caption":"Overhead view of a person swinging a rock on a rope."}

Uniform circular motion is the special case in which the speed is constant.
In uniform circular motion, the acceleration vector is toward the
center, and therefore total force acting on the object
must point \emph{directly} toward the center. We can define the
\intro{angular velocity}\index{angular velocity} $\omega$, which is the number of radians per second
by which the object's angle changes, $\omega=\der\theta/\der t$.
For uniform circular motion, $\omega$ is constant, and the magnitude of the acceleration is
\begin{equation}
  a = \omega^2 r = \frac{v^2}{r},
\end{equation}
where $r$ is the radius of the circle (see problem \ref{hw:circularcalculus},
p.~\pageref{hw:circularcalculus}). These expressions can also be related
to the \intro{period}\index{period} of the rotation $T$, which is the time for
one revolution. We have $\omega=2\pi/T$.
