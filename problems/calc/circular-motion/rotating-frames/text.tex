When you're in the back seat of a car going around a curve,
not looking out the window,
there is a strong tendency to adopt a frame of reference in
which the car is at rest. This is a noninertial frame of reference,
because the car is accelerating. In a noninertial frame, Newton's
laws are violated. For example, the air freshener hanging from the mirror in figure
\ref{fig:accelerometer} on p.~\pageref{fig:accelerometer} will swing as the
car enters the curve, but this motion is not caused by a force made by any identifiable
object. In the
inertial frame of someone standing by the side of the road, the air freshener
simply continued straight while the \emph{car} accelerated.
