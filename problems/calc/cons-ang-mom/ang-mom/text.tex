We have discarded Newton's laws of motion and begun the process of
rebuilding the laws of mechanics from scratch using conservation laws.
So far we have encountered conservation of energy and momentum.

It is not hard to come up with examples to show that this list of
conservation laws is incomplete. The earth has been
rotating about its own axis
at very nearly the same speed (once every 24 hours) for all of human history.
Why hasn't the planet's rotation slowed down and come to a halt? Conservation of energy doesn't
protect us against this unpleasant scenario. 
The kinetic energy tied up in the earth's spin could be transformed into some
other type of energy, such as heat. Nor is such a deceleration prevented by
conservation of momentum, since the total momentum of the earth due to its
rotation cancels out.

There is a third important conservation law in mechanics, which is conservation of
\intro{angular momentum}.\index{angular momentum} 
A spinning body such as the earth has angular momentum. Conservation of angular
momentum arises from symmetry of the laws of physics with respect to rotation.
That is, there is no special direction built into the laws of physics, such as the
direction toward the constellation Sagittarius. Suppose that a non-spinning asteroid
were to gradually start spinning. Even if there were some source of energy to initiate
this spin (perhaps the heat energy stored in the rock), it wouldn't make sense for
the spin to start spontaneously, because then the axis of spin would have to point
in some direction, but there is no way to determine why one particular direction
would be preferred.

As a concrete example, suppose that a bike wheel of radius $r$ and mass $m$ is
spinning at a rate such that a point on the rim (where all the mass is concentrated)
moves at speed $v$. We then define the wheel's angular momentum $L$ to have magnitude
$mvr$.

More generally, we define the angular momentum of a system of particles to
be the sum of the quantity $\vc{r}\times\vc{p}$, where $\vc{r}$ is the position of
a particle relative to an arbitrarily chosen point called the axis, $\vc{p}$ is the
particle's momentum vector, and $\times$ represents the vector cross product. This
definition is chosen both because experiments show that this is the quantity that is
conserved. It
follows from the definition that angular momentum is a vector, and that its direction
is defined by the same right-hand rule used to define the cross product.
