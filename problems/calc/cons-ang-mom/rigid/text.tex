\timetraveltohere

In the special case where a rigid body rotates about an axis of symmetry and its
center of mass is at rest, the dimension parallel to the axis becomes irrelevant
both kinematically and dynamically. We can imagine squashing the system flat
so that the object rotates in a two-dimensional plane about a fixed point.
The kinetic energy in this situation is
\begin{align*}
  K &= \frac{1}{2}I\omega^2 \\
\intertext{and the angular momentum about the center of mass is}
  L &= I\omega,
\end{align*}
where $I$ is a constant of proportionality, called the \intro{moment of inertia}.
These relations are analogous to $K=(1/2)mv^2$ and $\vc{p}=m\vc{v}$ for motion of
a particle.
The angular velocity can also be made into a vector $\bomega$, which points along
the axis in the right-handed direction. We then have $\vc{L}=I\bomega$.

For a system of particles, the moment of inertia is given by
\begin{equation*}
  I = \sum m_i r_i^2,
\end{equation*}
where $r_i$ is the $i$th particle's distance from the axis. For a continuous distribution
of mass,
\begin{equation*}
  I = \int r^2 \:\der m.
\end{equation*}
Figure \ref{fig:moments-of-inertia} gives the moments of inertia of some commonly encountered
shapes.

% fig {"name":"moments-of-inertia","caption":"Moments of inertia of some geometric shapes."}
