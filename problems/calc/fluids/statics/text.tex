\subsection{Fluids}

In physics, the term \intro{fluid}\index{fluid}
is used to mean either a gas or a liquid. The important
feature of a fluid can be demonstrated by comparing with a
cube of jello on a plate. The jello is a solid. If you shake
the plate from side to side, the jello will respond by
shearing, i.e., by slanting its sides, but it will tend to
spring back into its original shape. A solid can sustain
shear forces, but a fluid cannot. A fluid does not resist a
change in shape unless it involves a change in volume.

\subsection{Pressure}

We begin by restricting ourselves to the case of fluid statics,
in which the fluid is at rest and in equilibrium. A small
chunk or ``parcel'' of the fluid has forces acting on it
from the adjacent portions of the fluid. We have assumed that the parcel is
in equilibrium, and if no external forces are present then these forces must cancel.
By the definition
of a fluid these forces are perpendicular to any part of the imaginary
boundary surrounding the parcel. Since force is an additive
quantity, the force the fluid exerts on any surface must be proportional to the
surface's area. We therefore define the \intro{pressure}\index{pressure}
to be the (perpendicular) force per unit area,
\begin{equation}
  P = \frac{F_\perp}{A}.
\end{equation}
If the forces are to cancel, then this force must be the
same on all sides on an object such as a cube, so it follows that pressure
has no direction: it is a scalar.
The SI units of pressure are newtons per square meter, which can be
abbreviated as pascals, $1\ \zu{Pa}=1\ \nunit/\munit^2$. The pressure
of the earth's atmosphere at sea level is about 100 kPa.

Only pressure \emph{differences} are ordinarily of any importance.
For example, your ears hurt when you fly in an airplane because there
is a pressure difference between your inner ear and the cabin; once the
pressures are equalized, the pain stops.

\subsection{Variation of pressure with depth}

If a fluid is subjected to a gravitational field and is in equilibrium, then
the pressure can only depend on depth (figure \ref{fig:funkycontainer}).

% fig {"name":"funkycontainer","caption":"The pressure is the same at all the points marked with dots."}

To find the variation with depth, we consider the vertical
forces acting on a tiny, cubical parcel of the fluid having infinitesimal
height $\der y$, where positive $y$ is up.
By requiring equilibrium, we find that the difference in pressure
between the top and bottom is
$\der P  =  -\rho g \der y$.
A more elegant way of writing this is in terms
of a dot product, 
\begin{equation}
  \der P = \rho\vc{g}\cdot\der\vc{y}
\end{equation}
which automatically takes care of the plus or minus sign, and avoids
any requirements about the coordinate system. By integrating this equation,
we can find the change in pressure $\Delta P$ corresponding to any change
in depth $\Delta y$.

\subsection{Archimedes' principle}

A helium balloon or a submarine experiences unequal pressure above and below, due to the
variation of pressure with depth. The total force of the surrounding fluid does not vanish,
and is called the buoyant force. In a fluid in equilibrium that does not contain any foreign object,
any parcel of fluid evidently has its weight canceled out by the buoyant force on it. This buoyant
force is unchanged if another object is substituted for the parcel of fluid, so
the buoyant force on a submerged object is upward and equal to the weight of
the displaced fluid. This is called \intro{Archimedes' principle}.\index{Archimedes' principle}
