We have seen that a force is always an interaction between two objects.
\intro{Newton's third law}\index{Newton's laws!third}
states that these forces come in pairs. If object A exerts a
force on object B, then B also exerts a force on A.
The two forces have equal magnitudes but are in opposite directions.
In symbols,
\begin{equation}
  \vc{F}_\text{A on B} = -  \vc{F}_\text{B on A}.
\end{equation}

Newton's third law holds regardless of whether everything is
in a state of equilibrium. It might seem as though the two forces
would cancel out, but they can't cancel out because it doesn't
even make sense to add them in the first place. They act on different
objects, and it only makes sense to add forces if they act on the
same object.

% fig {"name":"skaters","caption":"Newton's third law does not mean that
%      forces always cancel out so that nothing can ever move. If these two ice
%      skaters, initially at rest, push against
%      each other, they will both move."}

The pair of forces related by Newton's third law are always of the
same type. For example, 
the hand in the left side of figure \ref{fig:force-operational}
makes a frictional force to the right on the rope. Newton's third law
tells us that the rope exerts a force on the hand that is to the
left and of the same strength. Since one of these forces is frictional,
the other is as well.
