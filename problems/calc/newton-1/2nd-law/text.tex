\intro{Newton's second law}\index{Newton's laws!second} tells us what happens when the forces acting
on an object do \emph{not} cancel out.
The object's acceleration is then given by
\begin{equation}
  \vc{a} = \frac{\vc{F}_\text{total}}{m},
\end{equation}
where $F_\text{total}$ is the vector sum of all the forces, and $m$ is the
object's mass. Mass is a permanent property of an object that measures
its inertia, i.e., how much it resists a change in its motion.
Since the SI unit of mass is the kilogram, it follows from Newton's
second law that the newton is related to the base units of the SI
as $1\ \nunit=1\ \kgunit\unitdot\munit/\sunit^2$.

The force that the earth's gravity exerts on an object is called its
\intro{weight}\index{weight}, which is not the same thing as its mass.
An object of mass $m$ has weight $mg$ (problem \ref{hw:prove-mg}, p.~\pageref{hw:prove-mg}).
