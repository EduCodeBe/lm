\subsection{Velocity and acceleration}
If an object undergoes an infinitesimal displacement $\der\vc{r}$ in an
infinitesimal time interval $\der t$, then its velocity vector is the derivative
$\vc{v}=\der\vc{r}/\der t$. This type of derivative of a vector can be computed
by differentiating each component separately. The acceleration is the
second derivative $\der^2\vc{r}/\der t^2$.

% fig {"name":"greyhound","caption":"The racing greyhound's velocity vector is
% in the direction of its motion, i.e., tangent to its curved path."}


The velocity vector has a magnitude that indicates the speed of motion, and
a direction that gives the direction of the motion. We saw in section
\ref{sec:kinematics-1d-velocity} that velocities add in relative motion.
To generalize this to more than one dimension, we use vector addition.

The acceleration vector does \emph{not} necessarily point in the direction
of motion. It points in the direction that an accelerometer would point,
as in figure \ref{fig:accelerometer}.

% fig {"name":"accelerometer","caption":"The car has just swerved to the right.
% The air freshener hanging from the rear-view mirror acts as an accelerometer, showing
% that the acceleration vector is to the right."}

\subsection{Projectiles and the inclined plane}

Forces cause accelerations, not velocities. In particular, the downward force of gravity
causes a downward acceleration vector. After a projectile\index{projectile motion}
is launched, the only force
on it is gravity, so its acceleration vector points straight down. Therefore the horizontal
part of its motion has constant velocity. The vertical and horizontal motions of a projectile
are independent. Neither affects the other.
