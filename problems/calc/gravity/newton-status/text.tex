Newton's theory of gravity, according to which masses act on one another
\emph{instantaneously} at a distance, is not consistent with Einstein's theory of
relativity, which requires that all physical influences travel no faster than the
speed of light. Einstein generalized Newton's description of gravity in his
general theory of relativity. Newton's theory is a good approximation to the
general theory when the masses that interact move at speeds small compared to the
speed of light, when the gravitational fields are weak, and when the distances
involved are small in cosmological terms. General relativity is needed in order
to discuss phenomena such as neutron stars and black holes, the big bang and the
expansion of the universe. The effects of general relativity also become important,
for example, in the Global Positioning System (GPS), where extremely high precision
is required, so that even extremely small deivations from the Newtonian picture are
important.

General relativity shares with Newtonian gravity the prediction that free fall
is universal. High-precision tests of this universality are therefore stringent
tests of both theories.
