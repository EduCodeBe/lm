Kepler's laws were a beautifully simple explanation of what the
planets did, but they didn't address why they moved as they did. Once
Newton had formulated his laws of motion and taught them to some of
his friends, they began trying to connect them to Kepler's laws. It
was clear now that an inward force would be needed to bend the
planets' paths.  Since the outer planets were moving slowly along more
gently curving paths than the inner planets, their accelerations were
apparently less. This could be explained if the sun's force was
determined by distance, becoming weaker for the farther planets.  In
the approximation of a circular orbit, it is not difficult to show
that Kepler's law of periods requires that this weakening with distance
vary according to $F\propto 1/r^2$. We know that objects near the earth's
surface feel a gravitational force that is also in proportion to their masses.
Newton therefore hypothesized a universal law of gravity,
\begin{equation*}
  F  =  \frac{Gm_1m_2}{r^2},
\end{equation*}
which states that any two massive particles, anywhere in the universe, attract each other
with a certain amount of force. The universal constant $G$
is equal to $6.67\times10^{-11}\ \nunit\unitdot\munit^2/\kgunit^2$.
