If you look carefully at a piece of rope or yarn while tightening it,
you will see a physical change in the fibers. This is a manifestation
of the fact that there is tension in the rope. Tension is a scalar with
units of newtons. For a rope of negligible mass, the tension is constant
throughout the rope, and it equals the magnitudes of the forces at its
ends. This is still true if the rope goes around a frictionless post,
or a massless pulley with a frictionless axle. You can't push with a
rope, you can only pull. A rigid object such as a pencil can, however,
sustain compression, which is equivalent to negative tension.

A pulley\index{pulley} is an example of a \intro{simple machine},\index{simple machine}
which is a device that can
amplify a force by some factor, while reducing the amount of motion
by the inverse of that factor. Another example of a simple machine is
the gear system on a bicycle.

We can put more than one simple machine together in order
to give greater
amplification of forces or to redirect forces in different directions.
For an idealized system,\footnote{In such a system: (1) The ropes and pulleys have negligible mass. (2)
Friction in the pulleys' bearings is negligible. (3) The ropes don't stretch.} the fundamental principles are:

\begin{enumerate}\label{pulley-rules}
\item The total force acting on any pulley is zero.\footnote{$F=ma$, and $m=0$ since the pulley's mass is assumed to be negligible.}
\item The tension in any given piece of rope is
      constant throughout its length.
\item The length of every piece of rope remains the same.
\end{enumerate}

% fig {"name":"eg-pulleys-three-quarters","width":"medium","caption":"A complicated pulley system. The bar is massless."}

As an example, let us find the mechanical advantage $T_5/F$ of the pulley system shown in
figure \ref{fig:eg-pulleys-three-quarters}. 
By rule 2, $T_1=T_2$, and by rule 1, $F=T_1+T_2$, so $T_1=T_2=F/2$. Similarly, $T_3=T_4=F/4$.
Since the bar is massless, the same reasoning that led to rule 1 applies to the bar as well, and $T_5=T_1+T_3$.
The mechanical advantage is $T_5/F=3/4$, i.e., this pulley system \emph{reduces} the input force.
