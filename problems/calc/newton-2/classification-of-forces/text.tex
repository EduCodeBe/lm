A fundamental and still unsolved problem in physics is the classification
of the forces of nature. Ordinary experience suggests to us that forces
come in different types, which behave differently. Frictional forces seem
clearly different from magnetic forces.

But some forces that appear distinct
are actually the same. For instance, the friction that holds a nail into the wall
seems different from the kind of friction that we observe when fluids are involved ---
you can't drive a nail into a waterfall and make it stick. But at the atomic level,
both of these types of friction arise from atoms bumping into each other.
The force that holds a magnet on your fridge also seems different from the
force that makes your socks cling together when they come out of the dryer, but
it was gradually realized, starting around 1800 and culminating with Einstein's
theory of relativity in 1905, that electricity and magnetism are actually closely
related things, and observers in different states of motion do not even agree on
what is an electric force and what is a magnetic one. The tendency has been for
more and more superficially disparate forces to become unified in this way.

Today we have whittled the list down to only three types of interactions at the subatomic
level (called the gravitational, electroweak, and strong forces). It is possible
that some future theory of physics will reduce the list to only one --- which Star Wars fans
would then probably want to call ``The Force.''

Nevertheless, there is a practical classification of forces that works pretty well
for objects on the human scale, and that is usually more convenient. Figure \ref{fig:force-tree}
on p.~\pageref{fig:force-tree} shows this scheme in the form of a tree.

% fig {"name":"force-tree","timetravel":0,"caption":"A practical classification scheme for forces."}
