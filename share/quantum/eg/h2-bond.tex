\begin{eg}{Chemical bonds}\label{h2-bond}\index{chemical bonds!quantum explanation for hydrogen}\index{bond|see {chemical bonds}}\index{hydrogen molecule|see {chemical bonds}}
I began this section with a classical argument that chemical
bonds, as in an $\zu{H}_2$ molecule, should not exist. Quantum
physics explains why this type of bonding does in fact
occur. There are actually two effects going on, one due to kinetic
energy and one due to electrical energy. We'll concentrate on the kinetic energy
effect in this example. m4_ifelse(m4_eval(__mod || __sn),1,[: Example \ref{eg:h2-details} on page 
\pageref{eg:h2-details} revisits the $\zu{H}_2$ bond in more detail.:],[::])
m4_ifelse(__mod,1,[:(There is also a qualitatively different type of bonding called ionic
bonding that occurs when an atom of one element steals the electron from an atom of another element.):],[:(A qualitatively different
type of bonding is discussed on page \pageref{ionicbonds}.):])

The kinetic energy effect is pretty simple.
When the atoms are next to each other, the electrons
are shared between them. The ``box'' is about twice as wide,
and a larger box allows a smaller kinetic energy. Energy is required
in order to separate the atoms. 
\end{eg}
