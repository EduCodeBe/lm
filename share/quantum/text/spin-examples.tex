<% begin_sec("Odds and evens, and how they add up",nil,'odds-and-evens') %>
From grade-school arithmetic, we have the rules
\begin{align*}
  \text{even}+\text{even} &= \text{even} \\
  \text{odd}+\text{even} &= \text{odd} \\
  \text{odd}+\text{odd} &= \text{even}.
\end{align*}
Thus we know that $123456789+987654321$ is even, without having to actually compute the result.
Dividing by two gives similar relationships for integer and half-integer angular momenta.
For example, a half-integer plus an integer gives a half-integer, and therefore when we add
the intrinsic spin 1/2 of an electron to any additional, integer spin that the electron
has from its motion through space,
we get a half-integer angular momentum. That is, the \emph{total} angular momentum of an electron
will always be a half-integer. Similarly, when we add the intrinsic spin 1 of a photon to its
angular momentum due to its integral motion through space, we will always get an integer. Thus
the integer or half-integer character of any particle's \emph{total} angular momentum 
($\text{spin}+\text{motion}$) is determined entirely by the particle's spin.

These relationships tell us things about the spins we can make by putting together different
particles to make bigger particles, and they also tell us things about decay processes.

\begin{eg}{Spin of the helium atom}
A helium-4 atom consists of two protons, two neutrons, and two electrons.
A proton, a neutron, and an electron each have spin 1/2. Since the atom is
a composite of six particles, each of which has half-integer spin, the atom as
a whole has an integer angular momentum.
\end{eg}

m4_ifelse(__mod,1,[:\begin{eg}{Silver atoms in the Stern-Gerlach experiment}\label{eg:stern-gerlach-total-spin}
The silver atoms used in the Stern-Gerlach experiment had an odd number of protons (47),
an even number of neutrons (two isotopes), and an odd number of electrons (47).
The result is that the atom as a whole has an integer spin. However,
only the electrons contribute significantly to the magnetic dipole moment of an atom (\notewithoutbackref{stern-gerlach-details}),
so the experiment only probed their angular momentum, which was a half-integer value because 47 is odd.
In principle this could be as high as $47/2$, but in the ground state it turns out to be only $1/2$, which
can be interpreted as the intrinsic spin of one of the electrons. The other 46 electrons' orbital and
intrinsic angular momentum end up canceling out.
\end{eg}
:])

\begin{eg}{Emission of a photon from an atom}
An atom can emit light,
\begin{equation*}
  \text{atom} \rightarrow \text{atom}+\text{photon}.
\end{equation*}
This works in terms of angular momentum because the photon's spin 1 is an integer.
Thus, regardless of whether the atom's angular momentum is an integer or a half-integer,
the process is allowed by conservation of angular momentum. If the atom's angular momentum
is an integer, then we have $\text{integer}=\text{integer}+1$, and if it's
a half-integer, $\text{half-integer}=\text{half-integer}+1$; either of these is possible.
If not for this logic, it would be impossible for matter to emit light. In general,
if we want a particle such as a photon to pop into existence like this, it must
have an integer spin.
\end{eg}

\begin{eg}{Beta decay}
When a free neutron undergoes beta decay, we have
\begin{equation*}
  \text{n} \rightarrow \text{p}+\text{e}^-+\bar{\nu}.
\end{equation*}
All four of these particles have spin 1/2, so the angular momenta go like
\begin{equation*}
  \text{half-integer} \rightarrow \text{half-integer}+\text{half-integer}+\text{half-integer},
\end{equation*}
which is possible, e.g., $1/2=3/2-5/2+3/2$. Because the neutrino has almost no interaction with
normal matter, it normally flies off undetected, and the reaction was originally thought to be
\begin{equation*}
  \text{n} \rightarrow \text{p}+\text{e}.
\end{equation*}
With hindsight, this is impossible, because we can never have
\begin{equation*}
  \text{half-integer} \rightarrow \text{half-integer}+\text{half-integer}.
\end{equation*}
The reasoning holds not just for the beta decay of a free neutron, but for any beta decay: a neutrino or antineutrino
must be emitted in order to conserve angular momentum.
But historically, this was not understood at first, and when Enrico Fermi proposed the
existence of the neutrino in 1934, the journal to which he first submitted his paper
rejected it as ``too remote from reality.''
\end{eg}

<% end_sec('odds-and-evens') %>
