Americium-241 is an artificial isotope used in smoke detectors. It undergoes alpha decay, with a half-life of 432 years.
As discussed in example \ref{eg:alpha-tunneling} on page \pageref{eg:alpha-tunneling}, alpha decay can be understood
as a tunneling process, and although the barrier is not rectangular in shape, the equation for the tunneling
probability on page \pageref{tunneling-probability} can still be used as a rough guide to our thinking.
For americium-241, the tunneling probability is about $1\times10^{-29}$. % http://hyperphysics.phy-astr.gsu.edu/hbase/Nuclear/alpdec.html
Suppose that this nucleus were to decay by emitting a tritium (helium-3) nucleus instead of an alpha particle (helium-4).
Estimate the relevant tunneling probability, assuming that the total energy $E$ remains the same.
This higher probability is contrary to the empirical observation that this
nucleus is not observed to decay by tritium emission with any significant probability, and in general
tritium emission is almost unknown in nature; this is mainly because the tritium nucleus is far less stable
than the helium-4 nucleus, and the difference in binding energy reduces the energy available for the decay.
