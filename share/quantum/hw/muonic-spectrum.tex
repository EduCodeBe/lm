m4_ifelse(__lm_series,1,[:This question requires that you read optional section \ref{sec:hydrogen-energies}.:])
A muon is a subatomic particle that acts exactly like
an electron except that its mass is 207 times greater. Muons
can be created by cosmic rays, and it can happen that one of
an atom's electrons is displaced by a muon, forming a muonic
atom. If this happens to a hydrogen atom, the resulting
system consists simply of a proton plus a muon.\hwendpart
 (a) 
m4_ifelse(__sn,1,[:Based on the results of section \ref{subsec:hydrogen-energies}, how:],[:How:])
would the size of a muonic hydrogen atom in its ground state
compare with the size of the normal atom?\hwendpart
 (b) If you were
searching for muonic atoms in the sun or in the earth's
atmosphere by spectroscopy, in what part of the electromagnetic
spectrum would you expect to find the absorption lines?
