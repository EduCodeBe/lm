New isotopes are continually being produced and studied. A common method is
that experimenters produce a beam of nuclei in an accelerator, and the beam
strikes a target such as a thin metal foil. If a beam nucleus happens to hit
a target nucleus, nuclear fusion can occur. Once the fused nucleus is formed, it is common for
several neutrons to boil off, and the number of neutrons lost can be random, so that
more than one isotope can be produced in the same experiment.

Liza carries out such an experiment and observes beta particles being emitted afterward,
meaning that she has produced an isotope that is radioactive. She counts the number of
betas observed in her detector for the first three hours after the isotope has been
produced, with the following results:

\begin{tabular}{ll}
first hour  &  770,336 \\
second hour &  662,901 \\
third hour  &  582,813
\end{tabular}

Is this a decay curve that could be statistically consistent with a single half-life,
i.e., with the production of a single isotope?
