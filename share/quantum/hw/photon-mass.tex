As far as we know, the mass of the photon is zero. However, it's not possible to prove by experiments that
anything is zero; all we can do is put an upper limit on the number. As of 2008, the best experimental upper limit on
the mass of the photon is about $1\times 10^{-52}$ kg. Suppose that the photon's mass really isn't zero, and that
the value is at the top of the range that is consistent with the present experimental evidence.
In this case, the $c$ occurring in relativity would no longer be interpreted as the speed of light.
As with material particles, the speed $v$ of a photon would depend on its
energy, and could never be as great as $c$. Estimate the relative size $(c-v)/c$ of the discrepancy in speed,
in the case of a photon of visible light.
<% hw_answer %>

