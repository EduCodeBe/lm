(a) Consider the function defined by $f(x,y)=(x-y)^2$. Visualize the graph
of this function as a surface. (This is a simple enough example that you
should not have to resort to computer software.)  Use this visualization to
determine the behavior of the sign of the Laplacian, as in 
example \ref{eg:laplacian-2d} on p.~\pageref{eg:laplacian-2d}.\hwendpart
(b) Consider the following incorrect calculation of this Laplacian. We take
the first derivatives and find
\begin{equation*}
  \frac{\partial f}{\partial x}+\frac{\partial f}{\partial y} = 0.
\end{equation*}
Next we take the second derivatives, but those are zero as well, so the
Laplacian is zero.
Critique this calculation in two ways: (1) by comparing with part a;
(2) by comparing with a correct calculation.\hwendpart
\hwremark{In general, if we have a function $f$ of two variables, the quantity
$Q=\partial f/\partial x+\partial f/\partial y$ can never be of physical
interest, because it doesn't behave in a sensible way when we rotate our coordinate axes.
You may want to prove this by showing that by rotating your
coordinate system, you can get a completely different answer than the
one calculated in part b.}
<% hw_solution %>
