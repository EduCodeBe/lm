On pp.~\pageref{start-approx-hydrogen-energies}-\pageref{end-approx-hydrogen-energies}
of __subsection_or_section(hydrogen-energies), we used simple algebra to derive
an approximate expression for the energies of states in hydrogen, without having to
explicitly solve the Schr\"odinger equation.
As input to the calculation, we used the the proportionality $__pe \propto r^{-1}$,
which is a characteristic of the electrical interaction. The result for the
energy of the $n$th standing wave pattern was $E_n \propto n^{-2}$.

There are other systems of physical interest in which we have
$__pe \propto r^k$ for values of $k$ besides $-1$. Problem
\ref{hw:quantumho} discusses the ground state of the harmonic oscillator,
with $k=2$ (and a positive constant of proportionality). In particle physics,
systems called charmonium and bottomonium are made out of pairs of subatomic
particles called quarks, which interact according to $k=1$, i.e., a force that
is independent of distance. (Here we have a positive
constant of proportionality, and $r>0$ by definition.
The motion turns out not to be too relativistic, so 
the Schr\"odinger equation is a reasonable approximation.)\index{quark}\index{charmonium}\index{bottomonium}
The figure shows actual energy levels for these three systems, drawn with
different energy scales so that they can all be shown side by side.
The sequence of energies in hydrogen approaches a limit, which is the energy
required to ionize the atom. In charmonium, only the first three levels are known.\footnote{See
Barnes et al., ``The XYZs of Charmonium at BES,'' \url{arxiv.org/abs/hep-ph/0608103}. To avoid complication,
the levels shown are only those in the group known for historical reasons as the $\Psi$ and $J/\Psi$.}
% http://www.astro.phys.ethz.ch/astro1/Users/fluri/private/lectures/MolecUniv/3_Molecular_Spectroscopy_script_part1.pdf
% http://arxiv.org/abs/hep-ph/0608103

Generalize the method used for $k=-1$ to any value of $k$, and find the
exponent $j$ in the resulting proportionality $E_n \propto n^j$.
Compare the theoretical calculation with the behavior of the actual energies shown in the figure.
Comment on the limit $k\rightarrow\infty$.
\answercheck
