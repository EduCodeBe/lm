Hydrogen-2 (${}^2\zu{H}$) is referred to as deuterium. It contains one proton
and one neutron. Its ground state is its only bound state, and in this state
the neutron and proton have the following quantum numbers:

\begin{tabular}{ll}
neutron:   &  $\ell=0$, $\ell_z=0$, $s=1/2$, $s_z=1/2$ \\
proton:    &  $\ell=0$, $\ell_z=0$, $s=1/2$, $s_z=1/2$.
\end{tabular}

\noindent Here the $z$ axis has been chosen parallel to the total angular momentum,
and the total angular momentum is 1. Another state in which we could put the system
is this one:

\begin{tabular}{ll}
neutron:   &  $\ell=0$, $\ell_z=0$, $s=1/2$, $s_z=1/2$ \\
proton:    &  $\ell=0$, $\ell_z=0$, $s=1/2$, $s_z=-1/2$.
\end{tabular}

\noindent This state's total angular momentum is 0. (This state is observed to
be unbound, but we're not concerned in this problem with whether states are
bound or unbound.)

Suppose that our system instead consisted of two neutrons and no protons at all.
Could you put them in the spin-1 state? In the spin-0 state?
(Don't worry about whether these states are bound.)
