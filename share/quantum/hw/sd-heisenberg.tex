If $x$ has an average value of zero, then
the standard deviation of the probability distribution $D(x)$ is defined by
\begin{equation*}
  \sigma^2 = \sqrt{\int D(x)x^2 \der x}\eqquad,
\end{equation*}
where the integral ranges over all possible values of $x$.

Interpretation: if $x$ only has a high probability of having values close to the average
(i.e., small positive and negative values), the thing being integrated will always
be small, because $x^2$ is always a small number; the standard deviation will therefore
be small. Squaring $x$ makes sure that either a number below the average ($x<0$) or a number above
the average ($x>0$) will contribute a positive amount to the standard deviation. We take
the square root of the whole thing so that it will have the same units as $x$, rather
than having units of $x^2$.

Redo problem \ref{hw:fwhm-heisenberg} using the standard deviation rather than
the FWHM.

Hints: (1) You need to determine the amplitude of the wave based on normalization.
(2) You'll need the following definite integral:
$\int_{-\pi/2}^{\pi/2} u^2 \cos^2 u \der u = (\pi^3-6\pi)/24$.
\answercheck
