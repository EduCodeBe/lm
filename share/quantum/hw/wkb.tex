m4_ifelse(__sn,1,[:On page \pageref{quantitativetunneling}:],[:In section \ref{sec:schrodinger}:])
we derived an expression for the
probability that a particle would tunnel through a
rectangular barrier, i.e., a region in which the interaction
energy $U(x)$ has a graph that looks like a rectangle. Generalize this to a barrier
of any shape. [Hints: First try generalizing to two
rectangular barriers in a row, and then use a series of
rectangular barriers to approximate the actual curve of an
arbitrary function $U(x)$. Note that the width and height of the
barrier in the original equation occur in such a way that
all that matters is the area under the $U$-versus-$x$
curve. Show that this is still true for a series of
rectangular barriers, and generalize using an integral.]  If
you had done this calculation in the 1930's you could have
become a famous physicist.
