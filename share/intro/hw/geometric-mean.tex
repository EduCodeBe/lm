% meta {"stars":0}
 <% hw_solution %> The usual definition of the mean (average) of two
numbers $a$ and $b$ is $(a+b)/2$. This is called the
arithmetic mean. The geometric mean, however, is defined as
$(ab)^{1/2}$ (i.e., the square root of $ab$). For the sake of definiteness, let's say both
numbers have units of mass. (a) Compute the arithmetic mean
of two numbers that have units of grams. Then convert the
numbers to units of kilograms and recompute their mean. Is
the answer consistent? (b) Do the same for the geometric
mean. (c) If $a$ and $b$ both have units of grams, what
should we call the units of \emph{ab}? Does your answer make
sense when you take the square root? (d) Suppose someone
proposes to you a third kind of mean, called the superduper
mean, defined as $(ab)^{1/3}$. Is this reasonable?
