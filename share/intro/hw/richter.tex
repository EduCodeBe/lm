<% hw_solution %>  One step on the Richter scale corresponds to a
factor of 100 in terms of the energy absorbed by something
on the surface of the Earth, e.g., a house. For instance, a
9.3-magnitude quake would release 100 times more energy than
an 8.3. The energy spreads out from the epicenter as a wave,
and for the sake of this problem we'll assume we're dealing
with seismic waves that spread out in three dimensions, so
that we can visualize them as hemispheres spreading out
under the surface of the earth. If a certain 7.6-magnitude
earthquake and a certain 5.6-magnitude earthquake produce
the same amount of vibration where I live, compare the
distances from my house to the two epicenters.
