% meta {"stars":1}
 In Europe, a piece of paper of the standard size,
called A4, is a little narrower and taller than its American
counterpart. The ratio of the height to the width is the
square root of 2, and this has some useful properties. For
instance, if you cut an A4 sheet from left to right, you get
two smaller sheets that have the same proportions. You can
even buy sheets of this smaller size, and they're called A5.
There is a whole series of sizes related in this way, all
with the same proportions. (a) Compare an A5 sheet to an A4
in terms of area and linear size. (b) The series of paper
sizes starts from an A0 sheet, which has an area of one
square meter. Suppose we had a series of boxes defined in a
similar way: the B0 box has a volume of one cubic meter, two
B1 boxes fit exactly inside an B0 box, and so on. What would
be the dimensions of a B0 box? \answercheck
