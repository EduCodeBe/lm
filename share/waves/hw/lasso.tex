The simplest trick with a lasso is to spin a flat loop in a horizontal
plane. The whirling loop of a lasso is kept under tension mainly due
to its own rotation. Although the spoke's force on the loop has an
inward component, we'll ignore it. The purpose of this problem,
which is based on one by A.P. French, is to prove a cute fact
about wave disturbances moving around the loop. As far as I
know, this fact has no practical implications for trick roping!
Let the loop have radius $r$ and mass per unit length $\mu$, 
and let its angular velocity be $\omega$.\hwendpart
(a) Find the
tension, $T$, in the loop in terms of $r$, $\mu$, and $\omega$. Assume
the loop is a perfect circle, with no wave disturbances on it yet.
<% hw_hint("lasso") %>\quad<% hw_answer %>\hwendpart
(b) Find the velocity of a wave pulse traveling around the
loop. Discuss what happens when the pulse moves is in the
same direction as the rotation, and when it travels contrary to the
rotation.
