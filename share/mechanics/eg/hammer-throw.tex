\begin{eg}{The hammer throw}
\egquestion In the men's Olympic hammer throw, a steel ball of radius 6.1 cm is swung on the
end of a wire of length 1.22 m. What fraction of the ball's angular momentum
comes from its rotation, as opposed to its motion through space?

\eganswer It's always important to solve problems symbolically first, and plug in numbers
only at the end, so let the radius of the ball be $b$, and the length of the wire $\ell$.
If the time the ball takes to go once around the circle is $T$, then
this is also the time it takes to revolve once around its own axis. Its speed
is $v=2\pi\ell/T$, so its angular momentum due to its motion through space
is $mv\ell=2\pi m\ell^2/T$. Its angular momentum due to its rotation around its
own center is $(4\pi/5)mb^2/T$. The ratio of these two angular momenta is
$(2/5)(b/\ell)^2=1.0\times10^{-3}$. The angular momentum due to the ball's
spin is extremely small.
\end{eg}
