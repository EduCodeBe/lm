\begin{summary}

\begin{vocab}

\vocabitem{angular momentum}{a measure of rotational motion; a conserved
quantity for a closed system}

\vocabitem{axis}{An arbitrarily chosen point used in the definition of
angular momentum. Any object whose direction changes
relative to the axis is considered to have angular momentum.
No matter what axis is chosen, the angular momentum of a
closed system is conserved.}

\vocabitem{torque}{the rate of change of angular momentum; a numerical
measure of a force's ability to twist on an object}

\vocabitem{equilibrium}{a state in which an object's momentum and
angular momentum are constant}

\vocabitem{stable equilibrium}{one in which a force always acts to bring
the object back to a certain point}

\vocabitem{unstable equilibrium}{one in which any deviation of the
object from its equilibrium position results in a force
pushing it even farther away}

\end{vocab}

\begin{notation}

\notationitem{$L$}{angular momentum}

\notationitem{$t$}{torque}

\notationitem{$T$}{m4_ifelse(__me,1,the period,) the time required for a rigidly rotating body to
complete one rotation}

m4_ifelse(__me,1,[:
\notationitem{$\omega$}{the angular velocity, $\der\theta/\der t$}

\notationitem{moment of inertia, $I$}{the proportionality constant in the
equation $L  =  I\omega$}
:])

\end{notation}

m4_ifelse(__me,1,[::],[:
\begin{othernotation}

\notationitem{period}{a name for the variable $T$ defined above}

\notationitem{moment of inertia, $I$}{the proportionality constant in the
equation $L  =  2\pi I / T$}
\end{othernotation}
:])

\begin{summarytext}

Angular momentum is a measure of rotational motion which is
conserved for a closed system. This book only discusses
angular momentum for rotation of material objects in two
dimensions. Not all rotation is rigid like that of a wheel
or a spinning top. An example of nonrigid rotation is a
cyclone, in which the inner parts take less time to complete
a revolution than the outer parts. In order to define a
measure of rotational motion general enough to include
nonrigid rotation, we define the angular momentum of a
system by dividing it up into small parts, and adding up all
the angular momenta of the small parts, which we think of as
tiny particles. We arbitrarily choose some point in space,
the \emph{axis}, and we say that anything that changes its
direction relative to that point possesses angular momentum.
The angular momentum of a single particle is
\begin{equation*}
                L  =  mv_{\perp}r\eqquad,
\end{equation*}
where $v_{\perp}$ is the component of its velocity perpendicular to
the line joining it to the axis, and $r$ is its distance
from the axis. Positive and negative signs of angular
momentum are used to indicate clockwise and counterclockwise rotation.

The \emph{choice of axis theorem} states that any axis may
be used for defining angular momentum. If a system's angular
momentum is constant for one choice of axis, then it is also
constant for any other choice of axis.

The \emph{spin theorem} states that an object's angular
momentum with respect to some outside axis A can be found by
adding up two parts:

(1) The first part is the object's angular momentum found by
using its own center of mass as the axis, i.e., the angular
momentum the object has because it is spinning.

(2) The other part equals the angular momentum that the
object would have with respect to the axis A if it had all
its mass concentrated at and moving with its center of mass.

Torque is the rate of change of angular momentum. The torque
a force can produce is a measure of its ability to twist on
an object. The relationship between force and torque is
\begin{equation*}
                |\btau |    =    r |F_{\perp} |\eqquad,
\end{equation*}
where $r$ is the distance from the axis to the point where
the force is applied, and $F_{\perp}$ is the component of the force
perpendicular to the line connecting the axis to the point
of application. Statics problems can be solved by setting
the total force and total torque on an object equal to zero
and solving for the unknowns.

m4_ifelse(__me,1,[:
	In the special case of a \sumem{rigid
	body} rotating in a single plane, we define
	\begin{align*}
		\omega	&=  	\frac{\der\theta}{\der t} 	\qquad \text{[angular velocity]}\\
	\intertext{and}
		\alpha	&=  	\frac{\der\omega}{\der t}\eqquad,	\qquad \text{[angular acceleration]}\\
	\intertext{in terms of which we have}
		L &= I\omega
	\intertext{and}
		\tau &= I\alpha\eqquad,
	\intertext{where the \sumem{moment of inertia}, $I$, is defined as}
		I	&=  	\sum{m_i r_i^2}\eqquad,
	\end{align*}
	summing over all the atoms in the object (or using calculus to perform a continuous
	sum, i.e. an integral). The relationship between the angular quantities and
	the linear ones is
	\begin{multline*}
		v_t	= \omega r \qquad 	\hfill \shoveright{\text{[tangential velocity of a point]}}\\
		v_r	= 0 \qquad 	\hfill \shoveright{\text{[radial velocity of a point]}}\\
		a_t	=  \alpha r	\eqquad. \hfill \shoveright{\text{[radial acceleration of a point]}}\\
			\hfill\text{at a distance $r$ from the axis]}\\
		a_r	=  \omega^2 r	 \qquad 	\hfill \shoveright{\text{[radial acceleration of a point]}}\\
			\hfill\text{at a distance $r$ from the axis]}\\
	\end{multline*}

	In three dimensions, torque and angular momentum are vectors, and are expressed
	in terms of the vector \sumem{cross product}, which is the only
	rotationally invariant way of defining a multiplication of two vectors
	that produces a third vector:
	\begin{align*}
		\vc{L}	&= \vc{r}\times\vc{p}\\
		\btau	&= \vc{r}\times\vc{F}
	\end{align*}
	In general, the cross product of vectors $\vc{b}$ and $\vc{c}$ has
	magnitude
	\begin{equation*}
		|\vc{b}\times\vc{c}|	= |\vc{b}|\,|\vc{c}|\,\sin\theta_{bc}\eqquad,\\
	\end{equation*}
	which can be interpreted geometrically as the area of the parallelogram
	formed by the two vectors when they are placed tail-to-tail.
	The direction of the cross product lies along the line which is perpendicular
	to both vectors; of the two such directions, we choose the one that is right-handed,
	in the sense that if we point the fingers of the flattened right hand along
	$\vc{b}$, then bend the knuckles to point the fingers along $\vc{c}$, the thumb
	gives the direction of $\vc{b}\times\vc{c}$. In terms of components, the
	cross product is
	\begin{align*}
	(\vc{b}\times\vc{c})_x	&=  b_yc_z - c_yb_z\\
	(\vc{b}\times\vc{c})_y	&=  b_zc_x - c_zb_x\\
	(\vc{b}\times\vc{c})_z	&=  b_xc_y - c_xb_y
	\end{align*}
	The  cross product 
	has the disconcerting properties
	\begin{align*}
		\vc{a}\times\vc{b} &= -\vc{b}\times\vc{a} \qquad \text{[noncommutative]}\\
		\intertext{and}
		\vc{a}\times(\vc{b}\times\vc{c}) &\ne (\vc{a}\times\vc{b})\times\vc{c} \qquad \text{[nonassociative]}\eqquad,
	\end{align*}
	and there is no ``cross-division.''
	
	For rigid-body rotation in three dimensions, we define an angular velocity
	vector $\bomega$, which lies along the axis of rotation and bears a right-hand
	relationship to it. Except in special cases, there is no scalar moment of
	inertia for which $\vc{L}=I\bomega$; the moment of inertia must be expressed
	as a matrix.
:])

\end{summarytext}

\end{summary}
