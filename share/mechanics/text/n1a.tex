<% begin_sec("Types of motion",0,'types-of-motion') %>\index{motion!types of}

If you had to think consciously in order to move your body,
you would be severely disabled.  Even walking, which we
consider to be no great feat, requires an intricate series
of motions that your cerebrum would be utterly incapable of
coordinating.  The task of putting one foot in front of the
other is controlled  by the more primitive parts of your
brain, the ones that have not changed much since the mammals
and reptiles went their separate evolutionary ways.  The
thinking part of your brain limits itself to general
directives such as ``walk faster,'' or ``don't step on her
toes,'' rather than micromanaging every contraction and
relaxation of the hundred or so muscles of your hips, legs, and feet.

<% marg(110) %>
<%
  fig(
    'bike',
    %q{Rotation.}
  )
%>
\spacebetweenfigs
<%
  fig(
    'rolling-soccer-ball',
    %q{Simultaneous rotation and motion through space.}
  )
%>
\spacebetweenfigs
<%
  fig(
    'chair',
    %q{%
      One person might say that the
      tipping chair was only rotating
      in a circle about its point of
      contact with the floor, but
      another could describe it as
      having both rotation and
      motion through space.
    }
  )
%>
<% end_marg %>%
Physics is all about the conscious understanding of motion,
but we're obviously not immediately prepared to understand
the most complicated types of motion.  Instead, we'll use
the divide-and-conquer technique.  We'll first classify the
various types of motion, and then begin our campaign with an
attack on the simplest cases.  To make it clear what we are
and are not ready to consider, we need to examine and define
carefully what types of motion can exist.

  <% begin_sec("Rigid-body motion distinguished from motion that changes an object's shape") %>\index{motion!rigid-body}

Nobody, with the possible exception of Fred Astaire, can
simply glide forward without bending their joints.  Walking
is thus an example in which there is both a general motion
of the whole object and a change in the shape of the object.
 Another example is the motion of a jiggling water balloon
as it flies through the air.  We are not presently
attempting a mathematical description of the way in which
the shape of an object changes.  Motion without a change in
shape is called rigid-body motion.  (The word ``body'' is
often used in physics as a synonym for ``object.'')

  <% end_sec() %>
  <% begin_sec("Center-of-mass motion as opposed to rotation",nil,'cm-qualitative') %>\index{rotation}

A ballerina leaps into the air and spins around once before
landing.  We feel intuitively that her rigid-body motion
while her feet are off the ground consists of two kinds of
motion going on simultaneously: a rotation and a motion of
her body as a whole through space, along an arc.  It is not
immediately obvious, however, what is the most useful way to
define the distinction between rotation and motion through
space.  Imagine that you attempt to balance a chair and it
falls over.  One person might say that the only motion was a
rotation about the chair's point of contact with the floor,
but another might say that there was both rotation and
motion down and to the side.

<%
  fig(
    'jete-side',
    %q{%
      The leaping dancer's motion is
      complicated, but the motion of her center of mass is simple.
    },
    {
      'width'=>'wide'
    }
  )
%>

<% marg(140) %>
<%
  fig(
    'pears',
    %q{%
      No matter what point you hang the
      pear from, the string lines up with the
      pear's center of mass. The center of
      mass can therefore be defined as the
      intersection of all the lines made by
      hanging the pear in this way.  Note that
      the X in the figure should not be
      interpreted as implying that the center
      of mass is on the surface --- it is
      actually inside the pear.
    }
  )
%>
\spacebetweenfigs
<%
  fig(
    'trapeze',
    %q{%
      The circus performers hang with the ropes passing
      through their centers of mass.
    }
  )
%>

<% end_marg %>%
It turns out that there is one particularly natural and
useful way to make a clear definition, but it requires a
brief digression.  Every object has a balance point,
referred to in physics as the \emph{center of mass}.  For a
two-dimensional object such as a cardboard cutout, the
center of mass is the point at which you could hang the
object from a string and make it balance.  In the case of
the ballerina (who is likely to be three-dimensional unless
her diet is particularly severe), it might be a point either
inside or outside her body, depending on how she holds her
arms.  Even if it is not practical to attach a string to the
balance point itself, the \index{center of mass}center of
mass can be defined as shown in figure \figref{pears}.

Why is the center of mass concept relevant to the question
of classifying rotational motion as opposed to motion
through space?  As illustrated in figures \figref{jete-side} and \figref{jete-overhead},
 it turns
out that the motion of an object's center of mass is nearly
always far simpler than the motion of any other part of the
object.  The ballerina's body is a large object with a
complex shape. We might expect that her motion would be much
more complicated than the motion of a small, simply-shaped
object, say a marble, thrown up at the same angle as the
angle at which she leapt.  But it turns out that the motion
of the ballerina's center of mass is exactly the same as the
motion of the marble.  That is, the motion of the center of
mass is the same as the motion the ballerina would have if
all her mass was concentrated at a point.  By restricting
our attention to the motion of the center of mass, we can
therefore simplify things greatly.

<%
  fig(
    'jete-overhead',
    %q{%
      %
      The same leaping dancer, viewed from
      above.  Her center of mass traces a
      straight line, but a point away from her
      center of mass, such as her elbow,
      traces the much more  complicated
      path shown by the dots.
      
    },
    {
      'width'=>'wide'
    }
  )
%>

We can now replace the ambiguous idea of ``motion as a whole
through space'' with the more useful and better defined
concept of ``\index{center of mass!motion of}\index{center-of-mass
motion}center-of-mass motion.''  The motion of any rigid
body can be cleanly split into rotation and center-of-mass
motion.  By this definition, the tipping chair does have
both rotational and center-of-mass motion. Concentrating on
the center of mass motion allows us to make a simplified
\index{models}model of the motion, as if a complicated
object like a human body was just a marble or a point-like
particle. Science really never deals with reality; it deals
with models of reality.

<% marg(105) %>
<%
  fig(
    'unbalanced-wheel',
    %q{%
      An improperly balanced wheel has a
      center of mass that is not at its
      geometric center. When you get a new
      tire, the mechanic clamps little weights
      to the rim to balance the wheel.
    }
  )
%>
\spacebetweenfigs
<%
  fig(
    'jumping-toy',
    %q{%
      This toy was intentionally designed so that the mushroom-shaped
      piece of metal on top would throw off the center of mass. When you wind it
      up, the mushroom spins, but the center of mass doesn't want to move, so the
      rest of the toy tends to counter the mushroom's motion, causing the whole
      thing to jump around.
    }
  )
%>
<% end_marg %>
Note that the word ``center'' in ``center of mass'' is not
meant to imply that the center of mass must lie at the
geometrical center of an object.  A car wheel that has not
been balanced properly has a center of mass that does not
coincide with its geometrical center.  An object such as the
human body does not even have an obvious geometrical center.

It can be helpful to think of the center of mass as the
average location of all the mass in the object. With this
interpretation, we can see for example that raising your
arms above your head raises your center of mass, since the
higher position of the arms' mass raises the average. We won't
be concerned right now with calculating centers of mass
mathematically; the relevant equations are in ch.~\ref{ch:momentum}.

<%
  fig(
    'jete-illusion',
    %q{%
      A fixed point on the dancer's body
      follows a trajectory that is flatter than
      what we expect, creating an illusion
      of flight.
    },
    {
      'width'=>'wide'
    }
  )
%>

Ballerinas and professional basketball players can create an
illusion of flying horizontally through the air because our
brains intuitively expect them to have rigid-body motion,
but the body does not stay rigid while executing a
\index{grand jete}grand jete or a \index{slam dunk}slam
dunk.  The legs are low at the beginning and end of the
jump, but come up higher at the middle.  Regardless of what
the limbs do, the center of mass will follow the same arc,
but the low position of the legs at the beginning and end
means that the torso is higher compared to the center of
mass, while in the middle of the jump it is lower compared
to the center of mass.  Our eye follows the motion of the
torso and tries to interpret it as the center-of-mass motion
of a rigid body.  But since the torso follows a path that is
flatter than we expect, this attempted interpretation fails,
and we experience an illusion that the person is flying
horizontally.
<%
  fig(
    'cm-examples',
    %q{Example \ref{eg:cm-examples}.},
    {
      'width'=>'wide',
      'sidecaption'=>true
    }
  )
%>

\begin{eg}{The center of mass as an average}\label{eg:cm-examples}
\egquestion Explain how we know that the center of mass of each
object is at the location shown in figure \figref{cm-examples}.

\eganswer The center of mass is a sort of average, so the height
of the centers of mass in 1 and 2 has to be midway between the
two squares, because that height is the average of the heights
of the two squares. Example 3 is a combination of examples 1
and 2, so we can find its center of mass by averaging the horizontal
positions of their centers of mass. In example 4, each 
square has been skewed a little, but just as much mass has been
moved up as down, so the average vertical position of the mass
hasn't changed. Example 5 is clearly not all that different from
example 4, the main difference being a slight clockwise rotation,
so just as in example 4, the center of mass must be hanging in
empty space, where there isn't actually any mass. Horizontally,
the center of mass must be between the heels and toes, or else
it wouldn't be possible to stand without tipping over.
\end{eg}

Another interesting example from the sports
world is the \index{high jump}high jump, in which the
jumper's curved body passes over the bar, but the center of
mass passes under the bar!  Here the jumper lowers his legs
and upper body at the peak of the jump in order to bring his
waist higher compared to the center of mass.

<% marg(0) %>
<%
  fig(
    'high-jump',
    %q{%
      The high-jumper's body passes over
      the bar, but his center of mass passes under it.
    }
  )
%>
\vspace{30mm}
<%
  fig(
    'gymnastics-wheel',
    %q{Self-check \ref{sc:gymnastics-wheel}.}
  )
%>

<% end_marg %>

Later in this course, we'll find that there are more
fundamental reasons (based on Newton's laws of motion) why
the center of mass behaves in such a simple way compared to
the other parts of an object.  We're also postponing any
discussion of numerical methods for finding an object's
center of mass. Until later in the course, we will only deal
with the motion of objects' centers of mass.

  <% end_sec() %>
  <% begin_sec("Center-of-mass motion in one dimension") %>


In addition to restricting our study of motion to center-of-mass
motion, we will begin by considering only cases in which the
center of mass moves along a straight line.  This will
include cases such as objects falling straight down, or a
car that speeds up and slows down but does not turn.

Note that even though we are not explicitly studying the
more complex aspects of motion, we can still analyze the
center-of-mass motion while ignoring other types of motion
that might be occurring simultaneously .  For instance, if a
cat is falling out of a tree and is initially upside-down,
it goes through a series of contortions that bring its feet
under it.  This is definitely not an example of rigid-body
motion, but we can still analyze the motion of the cat's
center of mass just as we would for a dropping rock.

<% self_check('rigid-body',<<-'SELF_CHECK'
Consider a person running, a person pedaling on a bicycle, a
person coasting on a bicycle, and a person coasting on ice
skates.  In which cases is the center-of-mass motion
one-dimensional?  Which cases are examples of rigid-body motion?
  SELF_CHECK
  ) %>

<% self_check('gymnastics-wheel',<<-'SELF_CHECK'
The figure shows a gymnast holding onto the inside of a big wheel.
From inside the wheel, how could he make it roll one way or the
other?
  SELF_CHECK
  ) %>

  <% end_sec() %>
<% end_sec('types-of-motion') %>
<% begin_sec("Describing distance and time",0,'describing-distance-and-time') %>

Center-of-mass motion in one dimension is particularly easy
to deal with because all the information about it can be
encapsulated in two variables: $x$, the position of the
center of mass relative to the origin, and $t$, which
measures a point in time. For instance, if someone supplied
you with a sufficiently detailed table of $x$ and $t$
values, you would know pretty much all there was to know
about the motion of the object's center of mass.

  <% begin_sec("A point in time as opposed to duration",nil,'t-vs-delta-t') %>\index{time!point in}\index{time!duration}

In ordinary speech, we use the word ``time'' in two
different senses, which are to be distinguished in physics.
It can be used, as in ``a short time'' or ``our time here on
earth,'' to mean a length or duration of time, or it can be
used to indicate a clock reading, as in ``I didn't know what
time it was,'' or ``now's the time.'' In symbols, $t$ is
ordinarily used to mean a point in time, while $\Delta t$
signifies an interval or duration in time. The capital Greek
letter delta, $\Delta $, means ``the change in...,'' i.e. a
duration in time is the change or difference between one
clock reading and another. The notation $\Delta t$ does not
signify the product of two numbers, $\Delta $ and $t$, but
rather one single number, $\Delta t$. If a matinee begins at
a point in time $t=1$ o'clock and ends at $t=3$ o'clock, the
duration of the movie was the change in $t$,
\begin{equation*}
     \Delta t=3\ \zu{hours} - 1\ \zu{hour} =  2\ \zu{hours}\eqquad.
\end{equation*}
To avoid the use of negative numbers for $\Delta t$, we
write the clock reading ``after'' to the left of the minus
sign, and the clock reading ``before'' to the right of the
minus sign. A more specific definition of the \index{delta
notation}delta notation is therefore that delta stands for
``after minus before.''

Even though our definition of the delta notation guarantees
that $\Delta t$ is positive, there is no reason why $t$
can't be negative. If $t$ could not be negative, what would
have happened one second before $t=0?$ That doesn't mean
that time ``goes backward'' in the sense that adults can
shrink into infants and retreat into the womb. It just means
that we have to pick a reference point and call it $t=0$,
and then times before that are represented by negative values of $t$.
An example is that a year like 2007 A.D. can be thought of as a positive
$t$ value, while one like 370 B.C. is negative. Similarly, when you
hear a countdown for a rocket launch, the phrase ``t minus ten seconds''
is a way of saying $t=-10\ \sunit$, where $t=0$ is the time of blastoff,
and $t>0$ refers to times after launch.

Although a point in time can be thought of as a clock
reading, it is usually a good idea to avoid doing computations
with expressions such as ``2:35'' that are combinations of
hours and minutes. Times can instead be expressed entirely
in terms of a single unit, such as hours. Fractions of an
hour can be represented by decimals rather than minutes, and
similarly if a problem is being worked in terms of minutes,
decimals can be used instead of seconds.

<% self_check('t-versus-delta-t',<<-'SELF_CHECK'
Of the following phrases, which refer to points in time,
which refer to time intervals, and which refer to time in
the abstract rather than as a measurable number?

(1) ``The time has come.''

(2) ``Time waits for no man.''

(3) ``The whole time, he had spit on his chin.''
  SELF_CHECK
  ) %>

m4_ifelse(__me,1,[:
    <% begin_sec("The Leibniz notation and infinitesimals",nil,'leibniz-notation') %>\label{infinitesimals}
$\Delta$ is the Greek version of ``D,'' suggesting that there is a
relationship between $\Delta t$ and the notation $\der t$ from calculus.
The ``d'' notation was invented by Leibniz\index{Leibniz, Gottfried} around 1675 to suggest the word ``difference.''
                % Cajori, A History of Mathematical Notations: Vol. II, p. 203. Available online through google bookx.
The idea was that a $\der t$ would be like a $\Delta t$ that was extremely small --- smaller than any real number, and
yet greater than zero. These infinitestimal numbers\index{infinitesimal number}
were the way the world's greatest mathematicians thought about calculus for the next two hundred
years. For example, $\der y/\der x$ meant the number you got when you divided $\der y$ by $\der x$.
The use of infinitesimal numbers was seen as a natural part of the process of generalization that had
already seen the invention of fractions and irrational numbers by the ancient Greeks,
zero and negative numbers in India, and complex numbers in Renaissance Italy.
By the end of the 19th century, mathematicians had begun making formal mathematical descriptions of number systems,
and they had succeeded in making nice tidy schemes out of all of these categories except for infinitesimals.
Having run into a brick wall, they decided to rebuild calculus using the notion of a limit. Depending on when and
where you got your education in calculus, you may have been warned severely that $\der y$ and $\der x$ were not
numbers, and that $\der y/\der x$ didn't mean dividing one by another.

But in the 1960's, the logician Abraham Robinson\index{Robinson, Abraham}
at Yale proved that infinitesimals could be tamed and domesticated; they were no more self-contradictory
than negative numbers or fractions. There is a handy rule for making sure that you don't come to incorrect conclusions
by using infinitesimals. The rule is that you can apply any axiom of the real number system to infinitesimals, and
the result will be correct, provided that the axiom can be put in a form like ``for any number \ldots,''
but not ``for any \emph{set} of numbers \ldots''
We carry over the axiom, reinterpreting ``number'' to mean any member of the enriched number system that includes both the
real numbers and the infinitesimals.

\begin{eg}{Logic and infinitesimals}
There is an axiom of the real number system that for any number $t$, $t+0=t$. This applies to infinitesimals as well,
so that $\der t +0=\der t$.
\end{eg}
    <% end_sec('leibniz-notation') %>
:])
  <% end_sec('t-vs-delta-t') %>
  <% begin_sec("Position as opposed to change in position",nil,'x-vs-delta-x') %>

As with time, a distinction should be made between a point
in space, symbolized as a coordinate $x$, and a change in
position, symbolized as $\Delta x$.

As with $t,x$ can be negative. If a train is moving down the
tracks, not only do you have the freedom to choose any point
along the tracks and call it $x=0$, but it's also up to you
to decide which side of the $x=0$ point is positive $x$ and
which side is negative $x$.

Since we've defined the delta notation to mean ``after minus
before,'' it is possible that $\Delta x$ will be negative,
unlike $\Delta t$ which is guaranteed to be positive.
Suppose we are describing the motion of a train on tracks
linking Tucson and Chicago. As shown in the figure, it is
entirely up to you to decide which way is positive.

<%
  fig(
    'tucson',
    %q{%
      Two equally valid ways of describing the motion of a train from Tucson to
      Chicago. In example 1, the train has a positive 
      $\Delta x$ as it goes from Enid to
      Joplin. In 2, the same train going forward in the same
      direction has a negative $\Delta x$.
    },
    {
      'width'=>'wide',
      'sidecaption'=>true
    }
  )
%>

Note that in addition to $x$ and $\Delta x$, there is a
third quantity we could define, which would be like an
odometer reading, or actual distance traveled. If you drive
10 miles, make a U-turn, and drive back 10 miles, then your
$\Delta x$ is zero, but your car's odometer reading has
increased by 20 miles. However important the odometer
reading is to car owners and used car dealers, it is not
very important in physics, and there is not even a standard
name or notation for it. The change in position, $\Delta x$,
is more useful because it is so much easier to calculate: to
compute $\Delta x$, we only need to know the beginning and
ending positions of the object, not all the information
about how it got from one position to the other.

<% self_check('delta-x-bouncing',<<-'SELF_CHECK'
A ball falls vertically, hits the floor, bounces to a height of one meter,
falls, and hits the floor again. Is the $\\Delta x$ between
the two impacts equal to zero, one, or two meters?
  SELF_CHECK
  ) %>

  <% end_sec('x-vs-delta-x') %>
  <% begin_sec("Frames of reference",nil,'frames-of-reference') %>

The example above shows that there are two arbitrary choices
you have to make in order to define a position variable,
$x$. You have to decide where to put $x=0$, and also which
direction will be positive. This is referred to as choosing
a \index{coordinate system!defined}coordinate system or
choosing a \index{frame of reference!defined}frame of
reference. (The two terms are nearly synonymous, but the
first focuses more on the actual $x$ variable, while the
second is more of a general way of referring to one's point
of view.) As long as you are consistent, any frame is
equally valid. You just don't want to change coordinate
systems in the middle of a calculation.

m4_ifelse(__me,1,[:\enlargethispage{\baselineskip}:])

Have you ever been sitting in a train in a station when
suddenly you notice that the station is moving backward?
Most people would describe the situation by saying that you
just failed to notice that the train was moving --- it only
seemed like the station was moving. But this shows that
there is yet a third arbitrary choice that goes into
choosing a coordinate system: valid frames of reference can
differ from each other by moving relative to one another. It
might seem strange that anyone would bother with a
coordinate system that was moving relative to the earth, but
for instance the frame of reference moving along with a
train might be far more convenient for describing things
happening inside the train.

  <% end_sec('frames-of-reference') %>
<% end_sec('describing-distance-and-time') %>
<% begin_sec("Graphs of motion; velocity",0,'graphs-of-motion') %>\index{graphs!of position versus time}

<% marg(100) %>
<%
  fig(
    'xt-graph-1',
    %q{Motion with constant velocity.}
  )
%>
\spacebetweenfigs
<%
  fig(
    'xt-graph-2',
    %q{%
      Motion that decreases $x$ is
      represented with negative values of $\Delta x$
      and $v$.
    }
  )
%>
m4_ifelse(__lm_series,1,[:
\spacebetweenfigs
<%
  fig(
    'xt-graph-3',
    %q{Motion with changing velocity. How can we find the velocity at the time indicated by the dot?}
  )
%>
:])
<% end_marg %>%

<% begin_sec("Motion with constant velocity",nil,'constant-velocity-motion') %>

In example \figref{xt-graph-1}, an object is moving at constant speed in one
direction.  We can tell this because every two seconds, its
position changes by five meters.

In algebra notation, we'd say that the graph of $x$ vs. $t$
shows the same change in position, $\Delta x=5.0$ m, over
each interval of $\Delta t=2.0$ s.  The object's velocity
or speed is obtained by calculating 
$v=\Delta x/\Delta t=(5.0\ \munit)/(2.0\ \sunit)=2.5\ \munit/\sunit$.
In graphical terms, the
velocity can be interpreted as the slope of the line.  Since
the graph is a straight line, it wouldn't have mattered if
we'd taken a longer time interval and calculated 
$v=\Delta x/\Delta t=(10.0\ \munit)/(4.0\ \sunit)$.  The answer would still have
been the same, 2.5 m/s.

Note that when we divide a number that has units of meters
by another number that has units of seconds, we get units of
meters per second, which can be written m/s.  This is
another case where we treat units as if they were algebra
symbols, even though they're not.

m4_ifelse(__me,1,[:\enlargethispage{\baselineskip}:])

In example \figref{xt-graph-2}, the object is moving in the opposite
direction: as time progresses, its $x$ coordinate decreases.
Recalling the definition of the $\Delta$ notation as
``after minus before,'' we find that $\Delta t$ is still
positive, but $\Delta x$ must be negative.  The slope of the
line is therefore negative, and we say that the object has a
negative velocity, 
$v=\Delta x/\Delta t=(-5.0\ \munit)/(2.0\ \sunit)=-2.5\ \munit/\sunit$.
We've already seen that the plus and minus
signs of $\Delta x$ values have the interpretation of
telling us which direction the object moved.  Since $\Delta t$
is always positive, dividing by $\Delta t$ doesn't change
the plus or minus sign, and the plus and minus signs of
velocities are to be interpreted in the same way.  In
graphical terms, a positive slope characterizes a line that
goes up as we go to the right, and a negative slope tells us
that the line went down as we went to the right.

m4_ifelse(__lm_series,1,[:\worked{light-year-to-meters}{light-years}:])

m4_ifelse(__me,1,[:\pagebreak[4]:])

<% end_sec('constant-velocity-motion') %>
<% begin_sec("Motion with changing velocity",nil,'changing-velocity') %>
m4_ifelse(__lm_series,1,,[:
<% marg(m4_ifelse(__me,1,50,80)) %>
<%
  fig(
    'xt-graph-3',
    %q{Motion with changing velocity. How can we find the velocity at the time indicated by the dot?}
  )
%>
<% end_marg %>
:])

Now what about a graph like figure \figref{xt-graph-3}?  This might be a
graph of a car's motion as the driver cruises down the
freeway, then slows down to look at a car crash by the side
of the road, and then speeds up again, disappointed that
there is nothing dramatic going on such as flames or babies
trapped in their car seats.  (Note that we are still talking
about one-dimensional motion.  Just because the graph is
curvy doesn't mean that the car's path is curvy.  The graph
is not like a map, and the horizontal direction of the graph
represents the passing of time, not distance.)
m4_ifelse(__me,1,[:
%------------------------ begin ME version ------------------------
If we apply the equation $v=\Delta x/\Delta t$ to this example, we will get
the wrong answer, because the  $\Delta x/\Delta t$ gives a single number, but the velocity
is clearly changing. This is an example of a good general rule that tells you when you need to use your differential
calculus. Any time a rate of change is measured by an expression of the form $\Delta\ldots/\Delta\ldots$,
the result will only be right when the rate of change is constant. When the rate of change is varying, we
need to generalize the expression by making it into a derivative.
Just as an infinitesimally small\footnote{see p.~\pageref{infinitesimals}} $\Delta t$ is notated $\der t$, an infinitesimally
small $\Delta x$ is a $\der x$. The velocity is then the derivative $\der x/\der t$.

\begin{eg}{Units of velocity}
\egquestion
Verify that the units of $v=\der x/\der t$ make sense.

\eganswer
We expect the velocity to have units of meters per second, and it does come out to have those units, since
$\der x$ has units of meters and $\der t$ seconds. This ability to check the units of derivatives is one of the
main reasons that Leibniz designed his notation for derivatives the way he did.
\end{eg}

\begin{eg}{An insect pest}\label{eg:pest}
\egquestion An insect pest from the United States is inadvertently released in
a village in rural China. The pests spread outward at a rate of $s$ kilometers
per year, forming a widening circle of contagion. Find the number of square
kilometers per year that become newly infested. Check that the units of the result
make sense. Interpret the result.

\eganswer Let $t$ be the time, in years, since the pest was introduced.
The radius of the circle is $r=st$, and its area is $a=\pi r^2=\pi(st)^2$.
The derivative is
\begin{equation*}
  \frac{\der a}{\der t} = (2\pi s^2) t
\end{equation*}

The units of $s$ are km/year, so squaring it gives $\zu{km}^2/\zu{year}^2$.
The 2 and the $\pi$ are unitless, and multiplying by $t$ gives units
of $\zu{km}^2/\zu{year}$, which is what we expect for $\der a/\der t$, since
it represents the number of square kilometers per year that become infested.

Interpreting the result, we notice a couple of things. First, the rate
of infestation isn't constant; it's proportional to $t$, so people might not
pay so much attention at first, but later on the effort required to combat the
problem will grow more and more quickly. Second, we notice that the
result is proportional to $s^2$. This suggests that anything that could be
done to reduce $s$ would be very helpful. For instance, a measure that cut
$s$ in half would reduce $\der a/\der t$ by a factor of four.
\end{eg}
<% end_sec('changing-velocity') %>
%------------------------ end ME version ------------------------
:],[:
%------------------------ begin LM version ------------------------

<% marg(100) %>
<%
  fig(
    'xt-graph-4',
    %q{%
      The velocity at any given moment
      is defined as the slope of the tangent
      line through the relevant point on the
      graph.
    }
  )
%>
<% end_marg %>%
Example \figref{xt-graph-3} is similar to example \figref{xt-graph-1}
in that the object
moves a total of 25.0 m in a period of 10.0 s, but it is
no longer true that it makes the same amount of progress
every second.  There is no way to characterize the entire
graph by a certain velocity or slope, because the velocity
is different at every moment.  It would be incorrect to say
that because the car covered 25.0 m in 10.0 s, its
velocity was 2.5 m/s.  It moved faster than that at the
beginning and end, but slower in the middle.  There may have
been certain instants at which the car was indeed going 2.5
m/s, but the speedometer swept past that value without
``sticking,'' just as it swung through various other values
of speed.  (I definitely want my next car to have a
speedometer calibrated in m/s and showing both negative
and positive values.)

We assume that our speedometer tells us what is happening to
the speed of our car at every instant, but how can we define
speed mathematically in a case like this?  We can't just
define it as the slope of the curvy graph, because a curve
doesn't have a single well-defined slope as does a line.  A
mathematical definition that corresponded to the speedometer
reading would have to be one that assigned a
velocity value to a single point on the curve, i.e., a single
instant in time, rather than to the entire graph.  If we
wish to define the speed at one instant such as the one
marked with a dot, the best way to proceed is illustrated in
\figref{xt-graph-4}, where we have drawn the line through that point called
the tangent line, the line that ``hugs the curve.'' We can
then adopt the following definition of \index{velocity!definition}velocity:

\begin{important}[definition of velocity]
The velocity of an object at any given moment is the slope of the tangent
line through the relevant point on its $x-t$ graph.
\end{important}

One interpretation of this definition is that the velocity
tells us how many meters the object would have traveled in
one second, if it had continued moving at the same speed for
at least one second.

<%
  fig(
    'microscope',
    %q{The original graph, on the left, is the one from figure \figref{xt-graph-2}.
       Each successive magnification to the right is by a factor of four.
    },
    {
      'width'=>'fullpage'
    }
  )
%>

A good way of thinking about the tangent-line definition is shown in figure \figref{microscope}.
We zoom in on our point of interest more and more, as if through a microscope capable of unlimited
magnification. As we zoom in, the curviness of the graph becomes less and less apparent.
(Similarly, we don't notice in everyday life that the earth is a sphere.)
In the figure, we zoom in by 400\%, and then again by 400\%, and so on.
After a series of these zooms, the graph appears indistinguishable from a line, and we can
measure its slope just as we would for a line.

If all we saw was the ultra-magnified view,
we would assume that the object was moving at a constant speed, which is 2.5 m/s in our example,
and that it would continue to move at that speed. Therefore the speed of 2.5 m/s can be interpreted
as meaning that if the object had continued at constant speed for a further time interval of 1 s, it would have
traveled 2.5 m.

<% marg(-300) %>
<%
  fig(
    'xt-graph-5',
    %q{%
      Example \ref{eg:slope-of-tangent-line}: finding the velocity at the
      point indicated with the dot.
    }
  )
%>
<% end_marg %>%

\begin{eg}{The slope of the tangent line}\label{eg:slope-of-tangent-line}
\egquestion What is the velocity at the point shown with a dot on the graph?

\eganswer First we draw the tangent line through that point.
To find the slope of the tangent line, we need to pick two
points on it. Theoretically, the slope should come out the
same regardless of which two points we pick, but in
practical terms we'll be able to measure more accurately if
we pick two points fairly far apart, such as the two white
diamonds. To save work, we pick points that are directly
above labeled points on the $t$ axis, so that 
$\Delta t=4.0\ \sunit$ is easy to read off. One diamond lines up with
$x\approx17.5$ m, the other with $x\approx26.5$ m,
so $\Delta x=9.0\ \munit$. The velocity is $\Delta x/\Delta t=2.2\ \munit/\sunit$.
\end{eg}

Looking at the tangent line in figure \figref{xt-graph-5}, we can see that it
touches the curve at the point marked with a dot, but without cutting through it at that point.
No other line through that point has this ``no-cut'' property; if we rotated the line either clockwise
or counterclockwise about the point, it would cut through. Except in certain unusual
cases, there is always exactly one such no-cut line at any given point on a smooth curve,
and that no-cut line is the tangent line. It's as though the region below the curve were
a solid block of wood, and the tangent line were the edge of a ruler. The ruler can't penetrate
the block.

<% end_sec('changing-velocity') %>
<% begin_sec("Conventions about graphing",nil,'conventions-about-graphing') %>\index{graphing}

The placement of $t$ on the horizontal axis and $x$ on the
upright axis may seem like an arbitrary convention, or may
even have disturbed you, since your algebra teacher always
told you that $x$ goes on the horizontal axis and $y$ goes
on the upright axis.  There is a reason for doing it this
way, however.  In example \figref{xt-graph-5}, we have an object that
reverses its direction of motion twice.  It can only be in
one place at any given time, but there can be more than one
time when it is at a given place.  For instance, this object
passed through $x=17$ m on three separate occasions, but
there is no way it could have been in more than one place at
$t=5.0\ \sunit$.  Resurrecting some terminology you learned in
your trigonometry course, we say that $x$ is a function of
$t$, but $t$ is not a function of $x$.  In situations such
as this, there is a useful convention that the graph should
be oriented so that any vertical line passes through the
curve at only one point.  Putting the $x$ axis across the
page and $t$ upright would have violated this convention.
To people who are used to interpreting graphs, a graph that
violates this convention is as annoying as fingernails
scratching on a chalkboard.  We say that this is a graph of
``$x$ versus $t$.''  If the axes were the other way around,
it would be a graph of ``$t$ versus $x$.''   I remember the
``versus'' terminology by visualizing the labels on the $x$
and $t$ axes and remembering that when you read, you go from
left to right and from top to bottom.

m4_ifelse(__me,1,,\vfill)

\startdqs

\begin{dq}
Park is running slowly in gym class, but then he notices Jenna
watching him, so he speeds up to try to impress her. Which
of the graphs could represent his motion?
\end{dq}

<%
  fig(
    'dq-gym',
    '',
    {
      'width'=>'wide',
      'anonymous'=>true,
      'float'=>false
    }
  )
%>

\pagebreak

\begin{dq}
The figure shows a sequence of positions for two racing
tractors. Compare the tractors' velocities as the race
progresses. When do they have the same velocity? [Based on
a question by Lillian McDermott.]
\end{dq}

<%
  fig(
    'dq-tractor-race',
    '',
    {
      'width'=>'wide',
      'anonymous'=>true,
      'float'=>false
    }
  )
%>

\begin{dq}
If an object had an $x-t$ graph that was a straight line with
$\Delta x$=0 and $\Delta t\ne0$, what would you say about its
velocity?  Sketch an example of such a graph. What about
$\Delta t$=0 and $\Delta x\ne0$?
\end{dq}

\begin{dq}
If an object has a wavy motion graph like the one in
figure \figref{xt-graph-6} on p.~\pageref{fig:xt-graph-6}, what are the times at
which the object reverses its direction?  Describe
the object's velocity at these points.
\end{dq}

\begin{dq}
Discuss anything unusual about the following three graphs.
\end{dq}

<%
  fig(
    'dq-unphysical-xt',
    '',
    {
      'width'=>'wide',
      'anonymous'=>true,
      'float'=>false
    }
  )
%>

\begin{dq}
I have been using the term ``velocity'' and avoiding the
more common English word ``speed,'' because introductory
physics texts typically define them to mean different things.  They
use the word ``speed,'' and the symbol ``$s$'' to mean the
absolute value of the velocity, $s=|v|$.  Although I've chosen not to
emphasize this distinction in technical vocabulary, there
are clearly two different concepts here.  Can you think of
an example of a graph of $x$-versus-$t$ in which the object has
constant speed, but not constant velocity?
\end{dq}

<% marg(300) %>
<%
  fig(
    'xt-graph-6',
    %q{Reversing the direction of motion.}
  )
%>
\spacebetweenfigs
<%
  fig(
    'dq-interpret-xt',
    %q{Discussion question \ref{dq:interpret-xt}.},
    {
      'anonymous'=>true
    }
  )
%>
<% end_marg %>
\begin{dq}\label{dq:interpret-xt}
For the graph shown in the figure, describe how the object's velocity changes.
\end{dq}

\begin{dq}
Two physicists duck out of a boring scientific conference.  On the street, they witness an
accident in which a pedestrian is injured by a hit-and-run
driver.  A criminal trial results, and they must testify.
In her testimony, Dr. Transverz Waive says, ``The car was
moving along pretty fast, I'd say the velocity was +40
mi/hr.  They saw the old lady too late, and even though they
slammed on the brakes they still hit her before they
stopped.  Then they made a $U$ turn and headed off at a
velocity of about -20 mi/hr, I'd say.''  Dr. Longitud N.L.
Vibrasheun says, ``He was really going too fast, maybe his
velocity was -35 or -40 mi/hr.  After he hit Mrs. Hapless,
he turned around and left at a velocity of, oh, I'd guess
maybe +20 or +25 mi/hr.''  Is their testimony contradictory?  Explain.
\end{dq}

<% end_sec('conventions-about-graphing') %>
%------------------------ end LM version ------------------------
:])

<% end_sec('graphs-of-motion') %>
<% begin_sec("The principle of inertia",m4_ifelse(__me,1,0,4),'principle-of-inertia') %>

<% begin_sec("Physical effects relate only to a change in velocity",nil,'effects-from-changing-v') %>

Consider two statements of a kind that was at one time made with
the utmost seriousness:

\epigraphnobyline{People like Galileo and \index{Copernicus}Copernicus who say
the earth is rotating must be crazy.  We know the earth
can't be moving.  Why, if the earth was really turning once
every day, then our whole city would have to be moving
hundreds of leagues in an hour.  That's impossible!
Buildings would shake on their foundations.  Gale-force
winds would knock us over.  Trees would fall down.  The
Mediterranean would come sweeping across the east coasts of
Spain and Italy.  And furthermore, what force would be
making the world turn?}

\epigraphnobyline{All this talk of passenger trains moving at forty miles an
hour is sheer hogwash!  At that speed, the air in a
passenger compartment would all be forced against the back
wall.  People in the front of the car would suffocate, and
people at the back would die because in such concentrated
air, they wouldn't be able to expel a breath.}

Some of the effects predicted in the first quote are clearly
just based on a lack of experience with rapid motion that is
smooth and free of vibration.  But there is a deeper
principle involved.  In each case, the speaker is assuming
that the mere fact of motion must have dramatic physical
effects.  More subtly, they also believe that a force is
needed to keep an object in motion: the first person thinks
a force would be needed to maintain the earth's rotation,
and the second apparently thinks of the rear wall as pushing
on the air to keep it moving.

<% marg(0) %>
<%
  fig(
    'aristotle',
    %q{%
      Why does Aristotle look so sad? Has he realized that
      his entire system of physics is wrong?
    }
  )
%>
 %\spacebetweenfigs
<%
  fig(
    'shanghai-and-anaheim',
    %q{%
      The earth spins. People in Shanghai say they're at rest and
      people in Los Angeles are moving. Angelenos say the same about the Shanghainese.
    }
  )
%>
 %\spacebetweenfigs
<%
  fig(
    'jets-in-formation-over-ny',
    %q{The jets are at rest. The Empire State Building is moving.}
  )
%>
<% end_marg %>

Common modern knowledge and experience tell us that these
people's predictions must have somehow been based on
incorrect reasoning, but it is not immediately obvious where
the fundamental flaw lies.  It's one of those things a
four-year-old could infuriate you by demanding a clear
explanation of.  One way of getting at the fundamental
principle involved is to consider how the modern concept of
the universe differs from the popular conception at the time
of the Italian Renaissance.  To us, the word ``earth''
implies a planet, one of the nine planets of our solar
system, a small ball of rock and dirt that is of no
significance to anyone in the universe except for members of
our species, who happen to live on it.  To Galileo's
contemporaries, however, the earth was the biggest, most
solid, most important thing in all of creation, not to be
compared with the wandering lights in the sky known as
planets.  To us, the earth is just another object, and when
we talk loosely about ``how fast'' an object such as a car
``is going,'' we really mean the car-object's velocity
relative to the earth-object.

<%
  fig(
    'rocket-sled',
    %q{%
      This Air Force doctor volunteered to ride a rocket sled as 
      a medical experiment. The obvious effects on
      his head and face are not because of the sled's speed but because of its rapid \emph{changes} in speed: increasing
      in 2 and 3, and decreasing in 5 and 6.
      In 4 his speed is greatest, but because his speed is not
      increasing or decreasing very much at this moment, there is little effect on him.
    },
    {
      'width'=>'wide',
      'sidecaption'=>m4_ifelse(__me,1,false,true)
    }
  )
%>

<% end_sec('effects-from-changing-v') %>
<% begin_sec("Motion is relative",nil,'motion-is-relative') %>

m4_ifelse(__me,1,[:\enlargethispage{\baselineskip}:])

According to our modern world-view, it isn't 
reasonable to expect that a special force should be required
to make the air in the train have a certain velocity
relative to our planet.  After all, the ``moving'' air in
the ``moving'' train might just happen to have zero velocity
relative to some other planet we don't even know about.
Aristotle claimed that things ``naturally'' wanted to be at
rest, lying on the surface of the earth.  But experiment
after experiment has shown that there is really nothing so
special about being at rest relative to the earth.  For
instance, if a mattress falls out of the back of a truck on
the freeway, the reason it rapidly comes to rest with
respect to the planet is simply because of friction forces
exerted by the asphalt, which happens to be attached to the planet.

Galileo's insights are summarized as follows:

\begin{important}[The principle of inertia]\index{inertia, principle of}\label{principle-of-inertia}
No force is required to maintain motion with constant velocity in
a straight line, and absolute motion does not cause any
observable physical effects.
\end{important}

There are many examples of situations that seem to disprove
the principle of inertia, but these all result from
forgetting that friction is a force. For instance, it seems
that a force is needed to keep a sailboat in motion. If the
wind stops, the sailboat stops too. But the wind's force is
not the only force on the boat; there is also a frictional
force from the water. If the sailboat is cruising and the
wind suddenly disappears, the backward frictional force
still exists, and since it is no longer being counteracted
by the wind's forward force, the boat stops. To disprove the
principle of inertia, we would have to find an example where
a moving object slowed down even though no forces whatsoever were acting on it.
Over the years since Galileo's lifetime, physicists have done more and more precise
experiments to search for such a counterexample, but the results have always been negative.
Three such tests are described on pp.~\pageref{sec:galileo-ramps-inertia},
\pageref{eg:clock-comparison-inertia}, and
m4_ifelse(__lm_series,1,[:\pageref{sec:battat}:],[:\pageref{battat}:]).\label{first-law-evidence}

m4_ifelse(__me,1,[:\enlargethispage{\baselineskip}:])

<% self_check('inertia-counterexamples',<<-'SELF_CHECK'
What is incorrect about the following supposed counterexamples
to the principle of inertia?

(1) When astronauts blast off in a rocket, their huge
velocity does cause a physical effect on their bodies ---
they get pressed back into their seats, the flesh on their
faces gets distorted, and they have a hard time lifting their arms.

(2) When you're driving in a convertible with the top down,
the wind in your face is an observable physical effect of
your absolute motion.
  SELF_CHECK
  ) %>

\worked{cycloid}{a bug on a wheel}

\startdqs

<% marg(m4_ifelse(__me,1,[:-300:],[:140:])) %>
<%
  fig(
    'dq-cruise-ship',
    %q{Discussion question \ref{dq:cruise-ship}.},
    {
      'anonymous'=>true
    }
  )
%>
\spacebetweenfigs
<%
  fig(
    'dq-flag-in-balloon',
    %q{Discussion question \ref{dq:flag-in-balloon}.},
    {
      'anonymous'=>true
    }
  )
%>
<% end_marg %>
\begin{dq}\label{dq:cruise-ship}
A passenger on a cruise ship finds, while the ship is
docked, that he can leap off of the upper deck and just
barely make it into the pool on the lower deck.  If the ship
leaves dock and is cruising rapidly, will this adrenaline
junkie still be able to make it?
\end{dq}

\begin{dq}\label{dq:flag-in-balloon}
You are a passenger in the open basket hanging under a
helium balloon. The balloon is being carried along by the
wind at a constant velocity. If you are holding a flag in
your hand, will the flag wave? If so, which way? [Based on a
question from PSSC Physics.]
\end{dq}

\begin{dq}
Aristotle stated that all objects naturally wanted to
come to rest, with the unspoken implication that ``rest''
would be interpreted relative to the surface of the earth.
Suppose we go back in time and transport Aristotle to the
moon. Aristotle knew, as we do, that the moon circles the
earth; he said it didn't fall down because, like everything
else in the heavens, it was made out of some special
substance whose ``natural'' behavior was to go in circles
around the earth. We land, put him in a space suit, and kick
him out the door. What would he expect his fate to be in
this situation? If intelligent creatures inhabited the moon,
and one of them independently came up with the equivalent of
Aristotelian physics, what would they think about objects coming to rest?
\end{dq}

\begin{dq}\label{dq:beer}
The glass is sitting on a level table in a train's
dining car, but the surface of the water is tilted. What can
you infer about the motion of the train?
\end{dq}
<% marg(80) %>
<%
  fig(
    'beer',
    %q{Discussion question \ref{dq:beer}.},
    {
      'anonymous'=>true
    }
  )
%>
<% end_marg %>

<% end_sec('motion-is-relative') %>
<% end_sec('principle-of-inertia') %>
<% begin_sec("Addition of velocities",m4_ifelse(__me,1,0,4),'addition-of-velocities') %>\index{velocity!addition of}

<% begin_sec("Addition of velocities to describe relative motion",nil,'newtonian-v-addition') %>\label{vel-addition-newtonian}

Since absolute motion cannot be unambiguously measured, the
only way to describe motion unambiguously is to describe the
motion of one object relative to another. Symbolically, we
can write $v_{PQ}$ for the velocity of object $P$ relative to object $Q$.

Velocities measured with respect to different reference
points can be compared by addition. In the figure below, the
ball's velocity relative to the couch equals the ball's
velocity relative to the truck plus the truck's velocity
relative to the couch:
\begin{align*}
      v_{BC}   &=  v_{BT}+v_{TC}  \\
         &=  5\ \zu{cm}/\sunit + 10\ \zu{cm}/\sunit \\
         &=  15\ \zu{cm}/\sunit
\end{align*}

The same equation can be used for any combination of three
objects, just by substituting the relevant subscripts for
B, T, and C. Just remember to write the equation so
that the velocities being added have the same subscript
twice in a row. In this example, if you read off the
subscripts going from left to right, you get $\zu{BC}\ldots=\ldots\zu{BTTC}$.
 The fact that the two ``inside'' subscripts on the right
are the same means that the equation has been set up
correctly. Notice how subscripts on the left look just like
the subscripts on the right, but with the two T's eliminated.

<%
  fig(
    'dinos',
    %q{%
      These two highly competent physicists disagree on absolute
       velocities, but they would agree on relative
      velocities. Purple Dino considers the couch to be at rest, while Green Dino
       thinks of the truck as being at rest.
      They agree, however, that the truck's velocity relative to the couch is 
      $v_{TC}=10$ cm/s, the ball's velocity relative
      to the truck is $v_{BT}=5$ cm/s, and the ball's velocity relative to the couch is 
      $v_{BC}=v_{BT}+v_{TC}=15$ cm/s.
    },
    {
      'width'=>'wide'
    }
  )
%>

<% end_sec('newtonian-v-addition') %>
<% begin_sec("Negative velocities in relative motion",nil,'negative-v') %>\index{velocity!negative}

My discussion of how to interpret positive and negative
signs of velocity may have left you wondering why we should
bother.  Why not just make velocity positive by definition?
The original reason why negative numbers were invented was
that bookkeepers decided it would be convenient to use the
negative number concept for payments to distinguish them
from receipts.  It was just plain easier than writing
receipts in black and payments in red ink.  After adding up
your month's positive receipts and negative payments, you
either got a positive number, indicating profit, or a
negative number, showing a loss.  You could then show 
that total with a high-tech ``$+$'' or ``$-$'' sign, instead
of looking around for the appropriate bottle of ink.

Nowadays we use positive and negative numbers for all kinds
of things, but in every case the point is that it makes
sense to add and subtract those things according to the
rules you learned in grade school, such as ``minus a minus
makes a plus, why this is true we need not discuss.'' Adding
velocities has the significance of comparing relative
motion, and with this interpretation negative and positive
velocities can be used within a consistent framework. For
example, the truck's velocity relative to the couch equals
the truck's velocity relative to the ball plus the ball's
velocity relative to the couch:
\begin{align*}
     v_{TC}    &=  v_{TB}+v_{BC}      \\
         &=  -5\ \zu{cm}/\sunit + 15\ \zu{cm}/\sunit  \\
         &=  10\ \zu{cm}/\sunit
\end{align*}
If we didn't have the technology of negative numbers, we
would have had to remember a complicated set of rules for
adding velocities: (1) if the two objects are both moving
forward, you add, (2) if one is moving forward and one is
moving backward, you subtract, but (3) if they're both
moving backward, you add.  What a pain that would have been.

\worked{cross-deck}{two dimensions}

\begin{eg}{Airspeed}\label{eg:airspeed}
On June 1, 2009, Air France flight 447 disappeared without warning over the Atlantic Ocean.
All 232 people aboard were killed.
Investigators believe the disaster was triggered because the pilots lost the ability to
accurately determine their speed relative to the air. This is done using sensors called Pitot
tubes, mounted outside the plane on the wing. Automated radio signals showed that these sensors
gave conflicting readings before the crash, possibly because they iced up. For fuel efficiency,
modern passenger jets fly at a very high altitude, but in the thin air they
can only fly within a very narrow range of speeds. If the speed is too low, the plane stalls, and
if it's too high, it breaks up. If the pilots can't tell what their airspeed is, they can't
keep it in the safe range.

Many people's reaction to this story is to wonder why planes don't just use GPS to measure their
speed. One reason is that GPS tells you your speed relative to the ground, not relative to the
air. Letting P be the plane, A the air, and G the ground, we have
\begin{equation*}
  v_{PG} = v_{PA}+v_{AG}\eqquad,
\end{equation*}
where $v_{PG}$ (the ``true ground speed'') is what GPS would measure, $v_{PA}$ (``airspeed'')
is what's critical for stable flight, and $v_{AG}$ is the velocity of the wind relative to
the ground 9000 meters below. Knowing $v_{PG}$ isn't enough to determine $v_{PA}$ unless
$v_{AG}$ is also known.
\end{eg}

<%
  fig(   
    'air-france',
    'Example '+ref_workaround('eg:airspeed')+'. 1.~The aircraft before the disaster. 2.~A Pitot tube. 3.~The flight path of flight 447. 4.~Wreckage being recovered.',
    {
      'width'=>'fullpage'
    }
  )
%>

\startdqs

\begin{dq}
Interpret the general rule $v_{AB}=-v_{BA}$ in words.
\end{dq}

\begin{dq}
Wa-Chuen slips away from her father at the mall and walks
up the down escalator, so that she stays in one place. Write
this in terms of symbols.
\end{dq}

<% end_sec('negative-v') %>
<% end_sec('addition-of-velocities') %>
m4_ifelse(__me,1,,[:
%----- LM only -------
<% begin_sec("Graphs of velocity versus time",0,'v-t-graphs') %>\index{graphs!velocity versus time}

Since changes in velocity play such a prominent role in
physics, we need a better way to look at changes in velocity
than by laboriously drawing tangent lines on $x$-versus-$t$
graphs.  A good method is to draw a graph of velocity versus
time.  The examples on the left show the $x-t$ and $v-t$
graphs that might be produced by a car starting from a
traffic light, speeding up, cruising for a while at constant
speed, and finally slowing down for a stop sign.  If you
have an air freshener hanging from your rear-view mirror,
then you will see an effect on the air freshener during the
beginning and ending periods when the velocity is changing,
but it will not be tilted during the period of constant
velocity represented by the flat plateau in the middle of the $v-t$ graph.

Students often mix up the things being represented on these
two types of graphs.  For instance, many students looking at
the top graph say that the car is speeding up the whole
time, since ``the graph is becoming greater.''  What is
getting greater throughout the graph is $x$, not $v$.

<% marg(40) %>
<%
  fig(
    'xt-vt-traffic-light',
    %q{%
      Graphs of $x$ and $v$ versus $t$ for
      a car accelerating away from a traffic light, and then stopping for
      another red light.
    }
  )
%>
<% end_marg %>

Similarly, many students would look at the bottom graph and
think it showed the car backing up, because ``it's going
backwards at the end.''  But what is decreasing at the end
is $v$, not $x$. Having both the $x-t$ and $v-t$ graphs in
front of you like this is often convenient, because one
graph may be easier to interpret than the other for a
particular purpose.  Stacking them like this means that
corresponding points on the two graphs' time axes are lined
up with each other vertically.  However, one thing that is a
little counterintuitive about the arrangement is that in a
situation like this one involving a car, one is tempted to
visualize the landscape stretching along the horizontal axis
of one of the graphs.  The horizontal axes, however,
represent time, not position.  The correct way to visualize
the landscape is by mentally rotating the horizon 90 degrees
counterclockwise and imagining it stretching along the
upright axis of the $x$-$t$ graph, which is the only axis that
represents different positions in space.

<% end_sec('v-t-graphs') %> % Graphs of Velocity Versus Time
:])%----------- end if LM
m4_ifelse(__lm_series,1,[:<% begin_sec("Applications of calculus",3,'calculus-for-velocity',{'calc'=>true}) %>\index{calculus!invention by Newton}

%------------ begin LM version, Applications of Calculus
The integral symbol, $\int$, in the heading for this
section indicates that it is meant to be read by students in
calculus-based physics. Students in an algebra-based physics
course should skip these sections. The calculus-related
sections in this book are meant to be usable by students who
are taking calculus concurrently, so at this early point in
the physics course I do not assume you know any calculus
yet. This section is therefore not much more than a quick
preview of calculus, to help you relate what you're
learning in the two courses.

Newton was the first person to figure out the tangent-line
definition of velocity for cases where the $x-t$ graph is
nonlinear. Before Newton, nobody had conceptualized the
description of motion in terms of $x-t$ and $v-t$ graphs. In
addition to the graphical techniques discussed in this
chapter, Newton also invented a set of symbolic techniques
called calculus. If you have an equation for $x$ in terms of
$t$, calculus allows you, for instance, to find an equation
for $v$ in terms of $t$. In calculus terms, we say that the
function $v(t)$ is the derivative of the function $x(t)$. In
other words, the derivative of a function is a new function
that tells how rapidly the original function was changing.
We now use neither Newton's name for his technique (he
called it ``the method of fluxions'') nor his notation. The
more commonly used notation is due to Newton's German
contemporary Leibnitz, whom the English accused of
plagiarizing the calculus from Newton. In the Leibnitz notation, we write
\begin{equation*}
  v = \frac{\der x}{\der t}
\end{equation*}
to indicate that the function $v(t)$ equals the slope of the
tangent line of the graph of $x(t)$ at every time $t$. The
Leibnitz notation is meant to evoke the delta notation, but
with a very small time interval. Because the $\der x$ and
$\der t$ are thought of as very small $\Delta x$'s and
$\Delta t$'s, i.e., very small differences, the part of
calculus that has to do with derivatives is called
differential calculus.

Differential \index{calculus!differential}calculus
consists of three things:

\begin{itemize}

\item  The concept and definition of the \index{derivative}derivative,
which is covered in this book, but which will be discussed
more formally in your math course.

\item  The \index{calculus!Leibnitz notation}\index{Leibnitz}Leibnitz
notation described above, which you'll need to get more
comfortable with in your math course.

\item  A set of rules that allows you to find an equation for
the derivative of a given function. For instance, if you
happened to have a situation where the position of an object
was given by the equation $x=2t^7$, you would be able to use
those rules to find $\der x/\der t=14t^6$. This bag of
tricks is covered in your math course.

\end{itemize}

<% end_sec() %>
%------------ end LM version, Applications of Calculus
:])
