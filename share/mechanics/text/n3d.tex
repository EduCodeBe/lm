<% begin_sec("Conceptual framework",0) %>\index{circular motion}

I now live fifteen minutes from Disneyland, so my friends
and family in my native Northern California think it's a
little strange that I've never visited the Magic Kingdom
again since a childhood trip to the south. The truth is that
for me as a preschooler, Disneyland was not the Happiest
Place on Earth. My mother took me on a ride in which little
cars shaped like rocket ships circled rapidly  around a
central pillar. I knew I was going to die. There was a force
trying to throw me outward, and the safety features of the
ride would surely have been inadequate if I hadn't screamed
the whole time to make sure Mom would hold on to me.
Afterward, she seemed surprisingly indifferent to the
extreme danger we had experienced.

<% begin_sec("Circular motion does not produce an outward force") %>

My younger self's understanding of circular motion was
partly right and partly wrong. I was wrong in believing that
there was a force pulling me outward, away from the center
of the circle. The easiest way to understand this is to
bring back the parable of the bowling ball in the pickup
truck from chapter 4. As the truck makes a left turn, the
driver looks in the rearview mirror and thinks that some
mysterious force is pulling the ball outward, but the truck
is accelerating, so the driver's frame of reference is not
an inertial frame. Newton's laws are violated in a
noninertial frame, so the ball appears to accelerate without
any actual force acting on it. Because we are used to
inertial frames, in which accelerations are caused by
forces, the ball's acceleration creates a vivid illusion
that there must be an outward force.\index{frame of reference!rotating}

<%
  fig(
    'pickup-truck-circular',
    %q{%
      %
      1. In the turning truck's frame of reference, the
      ball appears to violate Newton's laws, displaying
      a sideways acceleration that is not the result
      of a force-interaction with any other object.
      %
      2. In an inertial frame of reference, such as the
      frame fixed to the earth's surface, the ball obeys
      Newton's first law. No forces are acting on it, and
      it continues moving in a straight line. It is the
      truck that is participating in an interaction with
      the asphalt, the truck that accelerates as it should
      according to Newton's second law.
    },
    {
      'width'=>'wide',
      'sidecaption'=>true
    }
  )
%>

In an inertial frame everything makes more sense. The ball
has no force on it, and goes straight as required by
Newton's first law. The truck has a force on it from the
asphalt, and responds to it by accelerating (changing the
direction of its velocity vector) as Newton's second law says it should.

\vspace{15mm}

\begin{eg}{The halteres}
Another interesting example is an insect organ called the halteres,
a pair of small knobbed limbs behind the wings, which vibrate up and down
and help the insect to maintain its orientation in flight. The halteres
evolved from a second pair of wings possessed by earlier insects. Suppose, for
example, that the halteres are on their upward stroke, and at that moment
an air current causes the fly to pitch its nose down. The halteres
follow Newton's first law, continuing to rise vertically, but in the fly's
rotating frame of reference, it seems as though they have been subjected to
a backward force. The fly has special sensory organs that perceive this twist,
and help it to correct itself by raising its nose.
\end{eg}
<% marg(50) %>
<%
  fig(
    'halteres',
    %q{%
      This crane fly's halteres help
      it to maintain its orientation in flight.
    }
  )
%>
<% end_marg %>

\vspace{15mm}

<% end_sec() %>
<% begin_sec("Circular motion does not persist without a force") %>

<%
  fig(
    'swing-rock',
    %q{%
      1. An overhead view of a person
      swinging a rock on a rope. A force from
      the string is required to make the rock's
      velocity vector keep changing direction.
      2. If the string breaks, the rock will
      follow Newton's first law and go
      straight instead of continuing around
      the circle.
    },
    {
      'width'=>'wide',
      'sidecaption'=>true
    }
  )
%>%
I was correct, however, on a different point about the Disneyland ride. To make me curve
around with the car, I really did need some force such as a
force from my mother, friction from the seat, or a normal
force from the side of the car. (In fact, all three forces
were probably adding together.) One of the reasons why
Galileo failed to refine the principle of inertia into a
quantitative statement like Newton's first law is that he
was not sure whether motion without a force would naturally
be circular or linear. In fact, the most impressive examples
he knew of the persistence of motion were mostly circular:
the spinning of a top or the rotation of the earth, for
example. Newton realized that in examples such as these,
there really were forces at work. Atoms on the surface of
the top are prevented from flying off straight by the
ordinary force that keeps atoms stuck together in solid
matter. The earth is nearly all liquid, but gravitational
forces pull all its parts inward.
<% marg(42) %>
<%
  fig(
    'sparks-fly-on-tangent',
    %q{%
      Sparks fly away along tangents to a grinding wheel.
    }
  )
%>
<% end_marg %>

\index{circular motion!uniform}
\index{circular motion!nonuniform}
<% end_sec() %>
<% begin_sec("Uniform and nonuniform circular motion") %>

Circular motion always involves a change in the direction of
the velocity vector, but it is also possible for the
magnitude of the velocity to change at the same time.
Circular motion is referred to as \emph{uniform} if $|\vc{v}|$ is
constant, and \emph{nonuniform} if it is changing.

Your speedometer tells you the magnitude of your car's
velocity vector, so when you go around a curve while keeping
your speedometer needle steady, you are executing uniform
circular motion. If your speedometer reading is changing as
you turn, your circular motion is nonuniform. Uniform
circular motion is simpler to analyze mathematically, so we
will attack it first and then pass to the nonuniform case.

<% self_check('uniform-or-nonuniform',<<-'SELF_CHECK'
Which of these are examples of uniform circular motion and
which are nonuniform?

    (1) the clothes in a clothes dryer (assuming they remain
against the inside of the drum, even at the top)

    (2) a rock on the end of a string being whirled in a vertical circle
  SELF_CHECK
  ) %>

<% end_sec() %>
<% begin_sec("Only an inward force is required for uniform circular motion.",4) %>

<% marg(0) %>
<%
  fig(
    'brick-on-a-rope',
    %q{%
      To make the brick go in a circle, I had
      to exert an inward force on the rope.
    }
  )
%>
<% end_marg %>
Figure \figref{swing-rock} showed the string pulling
in straight along a radius of the circle, but many people
believe that when they are doing this they must be
``leading'' the rock a little to keep it moving along. That
is, they believe that the force required to produce uniform
circular motion is not directly inward but at a slight angle
to the radius of the circle. This intuition is incorrect,
which you can easily verify for yourself now if you have
some string handy. It is only while you are getting the
object going that your force needs to be at an angle to the
radius. During this initial period of speeding up, the
motion is not uniform. Once you settle down into uniform
circular motion, you only apply an inward force.

If you have not done the experiment for yourself, here is a
theoretical argument to convince you of this fact. We have
discussed in chapter 6 the principle that forces have no
perpendicular effects. To keep the rock from speeding up or
slowing down, we only need to make sure that our force is
perpendicular to its direction of motion. We are then
guaranteed that its forward motion will remain unaffected:
our force can have no perpendicular effect, and there is no
other force acting on the rock which could slow it down. The
rock requires no forward force to maintain its forward
motion, any more than a projectile needs a horizontal force
to ``help it over the top'' of its arc.

<%
  fig(
    'hammer-polygons',
    %q{%
      A series of three hammer taps makes
      the rolling ball trace a triangle, seven
      hammers a heptagon. If the number
      of hammers was large enough, the ball
      would essentially be experiencing a
      steady inward force, and it would go
      in a circle. In no case is any forward
      force necessary.
    },
    {
      'width'=>'wide',
      'sidecaption'=>false
    }
  )
%>

<% marg(70) %>
<%
  fig(
    'car-in-circle',
    %q{%
      When a car is going straight at 
      constant speed, the forward and backward
      forces on it are canceling out, producing
      a total force of zero. When it moves
      in a circle at constant speed, there are
      three forces on it, but the forward and
      backward forces cancel out, so the
      vector sum is an inward force.
    }
  )
%>
<% end_marg %>

\pagebreak[4]

Why, then, does a car driving in circles in a parking lot
stop executing uniform circular motion if you take your foot
off the gas? The source of confusion here is that Newton's
laws predict an object's motion based on the \emph{total}
force acting on it. A car driving in circles has three forces on it

(1) an inward force from the asphalt, controlled with the steering wheel;

(2) a forward force from the asphalt, controlled with the gas pedal; and

(3) backward forces from air resistance and rolling resistance.

You need to make sure there is a forward force on the car so
that the backward forces will be exactly canceled out,
creating a vector sum that points directly inward.

\begin{eg}{A motorcycle making a turn}\label{eg:motorcycle}
The motorcyclist in figure \figref{motorcyclist} is moving along an
arc of a circle. It looks like he's chosen to ride the slanted surface
of the dirt at a place where it makes just the angle he wants, allowing
him to get the force he needs on the tires as a normal force, without
needing any frictional force. The dirt's normal force on the tires
points up and to our left. The vertical component of that force is
canceled by gravity, while its horizontal component causes him to
curve.
\end{eg}
<% marg(100) %>
<%
  fig(
    'motorcyclist',
    %q{Example \ref{eg:motorcycle}.}
  )
%>

<% end_marg %>

<% end_sec() %>
<% begin_sec("In uniform circular motion, the acceleration vector is inward.") %>

Since experiments show that the force vector points directly
inward, Newton's second law implies that the acceleration
vector points inward as well. This fact can also be proven
on purely kinematical grounds, and we will do so in the next section.

\begin{eg}{Clock-comparison tests of Newton's first law}\label{eg:clock-comparison-inertia}
Immediately after his original statement of the first 
law in the \emph{Principia Mathematica},\index{Newton's laws of motion!first law!test of}
Newton offers the supporting example of a spinning top, which only slows down because of
friction. He describes the different parts of the top as being held together by ``cohesion,''
i.e., internal forces. Because these forces act toward the center, they don't speed up
or slow down the motion. The applicability of the first law, which only describes linear motion,
may be more clear if we simply take figure \figref{hammer-polygons} as a model of rotation.
Between hammer taps, the ball experiences no force, so by the first law it doesn't speed
up or slow down.

Suppose that we want to subject the first law to 
a stringent experimental test.\footnote{Page \pageref{first-law-evidence} lists places in this
book where we describe experimental tests of Newton's first law.} The law predicts that
if we use a clock to measure the rate of rotation of an object spinning frictionlessly, it won't
``naturally'' slow down as Aristotle would have expected. But what is a clock but something with hands
that rotate at a fixed rate? In other words, we are comparing one clock with another. This is called
a clock-comparison experiment. Suppose that the laws of
physics weren't purely Newtonian, and there really was a very slight Aristotelian tendency for
motion to slow down in the absence of friction. If we compare two clocks, they should both slow
down, but if they aren't the same type of clock,
then it seems unlikely that they would slow down at exactly the same rate, and over time they should
drift further and further apart.

High-precision clock-comparison experiments have been done using a variety of clocks. In atomic clocks,
the thing spinning is an atom. Astronomers can observe the rotation of collapsed stars called pulars,
which, unlike the earth, can rotate with almost no disturbance due to geological activity or friction
induced by the tides. In these experiments, the pulsars are observed to match the rates of the atomic
clocks with a drift of less than about $10^{-6}$ seconds over a 
period of 10 years.\footnote{Matsakis \emph{et al.}, Astronomy and Astrophysics 326 (1997) 924.
Freely available online at \url{adsabs.harvard.edu}.}
Atomic clocks using atoms of different elements drift relative to one another by no more than 
about $10^{-16}$ per year.\footnote{Gu\'{e}na \emph{et al.}, \url{arxiv.org/abs/1205.4235}}

It is not presently possible to do experiments with a similar level of precision using human-scale rotating
objects. However, a set of
gyroscopes aboard the Gravity Probe B satellite were allowed to spin weightlessly in a vacuum, without any
physical contact that would have caused kinetic friction. Their rotation was extremely accurately
monitored for the purposes of another experiment (a test of Einstein's theory of general relativity,
which was the purpose of the mission), and they were found to be spinning down so gradually
that they would have taken about 10,000 years to slow down by a factor of two. This rate was
consistent with estimates of the amount of friction to be expected from the small amount of residual
gas present in the vacuum chambers.

A subtle point in the interpretation of these experiments is that if there was a slight tendency
for motion to slow down, we would have to decide what it was supposed to slow down relative to.
A straight-line motion that is slowing down in some frame of reference can always
be described as \emph{speeding up} in some other appropriately chosen frame (problem \ref{hw:decel-accel-frames},
p.~\pageref{hw:decel-accel-frames}). If the laws of physics did have this slight Aristotelianism
mixed in, we could wait for the anomalous acceleration or deceleration to stop. The object we were
observing would then define a special or ``preferred'' frame of reference. Standard theories of
physics do not have such a preferred frame, and clock-comparison experiments can be viewed as
tests of the existence of such a frame.\index{frame of reference!preferred}
Another test for the existence of a 
preferred frame m4_ifelse(__lm_series,1,[:is described on p.~\pageref{sec:battat}.:],[:was described on p.~\pageref{battat}.:])
\end{eg}

\pagebreak[4]

<% marg(13) %>
<%
  fig(
    'crack-the-whip',
    %q{Discussion questions \ref{dq:crack-the-whip-1}-\ref{dq:crack-the-whip-4}},
    {
      'anonymous'=>true
    }
  )
%>
\spacebetweenfigs
<%
  fig(
    'dq-tilt-a-whirl',
    %q{Discussion question \ref{dq:tilt-a-whirl}.},
    {
      'anonymous'=>true
    }
  )
%>
<% end_marg %>

\startdqs

\begin{dq}\label{dq:crack-the-whip-1}
In the game of crack the whip, a line of people stand
holding hands, and then they start sweeping out a circle.
One person is at the center, and rotates without changing
location.  At the opposite end is the person who is running
the fastest, in a wide circle.  In this game, someone always
ends up losing their grip and flying off.  Suppose the
person on the end loses her grip.  What path does she follow
as she goes flying off? Draw an overhead view.  (Assume she is going so fast that
she is really just trying to put one foot in front of the
other fast enough to keep from falling; she is not able to
get any significant horizontal force between her feet and the ground.)
\end{dq}

\begin{dq}
Suppose the person on the outside is still holding on,
but feels that she may loose her grip at any moment.  What
force or forces are acting on her, and in what directions
are they?  (We are not interested in the vertical forces,
which are the earth's gravitational force pulling down, and
the ground's normal force pushing up.)
Make a table in the format shown in __subsection_or_section(analysis-of-forces).
\end{dq}

\begin{dq}
Suppose the person on the outside is still holding on,
but feels that she may loose her grip at any moment.  What
is wrong with the following analysis of the situation?
``The person whose hand she's holding exerts an inward force
on her, and because of Newton's third law, there's an equal
and opposite force acting outward.  That outward force is
the one she feels throwing her outward, and the outward
force is what might make her go flying off, if it's strong enough.''
\end{dq}

\begin{dq}\label{dq:crack-the-whip-4}
If the only force felt by the person on the outside is an
inward force, why doesn't she go straight in?
\end{dq}

\begin{dq}\label{dq:tilt-a-whirl}
In the amusement park ride shown in the figure, the
cylinder spins faster and faster until the customer can pick
her feet up off the floor without falling. In the old Coney
Island version of the ride, the floor actually dropped out
like a trap door, showing the ocean below. (There is also a
version in which the whole thing tilts up diagonally, but
we're discussing the version that stays flat.) If there is
no outward force acting on her, why does she stick to the
wall? Analyze all the forces on her.
\end{dq}

\begin{dq}
What is an example of circular motion where the inward
force is a normal force?  What is an example of circular
motion where the inward force is friction?  What is an
example of circular motion where the inward force is the sum
of more than one force?
\end{dq}

\begin{dq}
Does the acceleration vector always change continuously
in circular motion? The velocity vector?
\end{dq}

<% end_sec() %>
<% end_sec() %>
<% begin_sec("Uniform circular motion",4,'uniform-circular-motion') %>
m4_ifelse(__me,1,[:%
\label{eg:circularaccel}% Not really an example, but rigid-body text, shared with SN, refers to it that way.
In this section I derive some convenient results, which you will use frequently, for
the acceleration of an object performing uniform circular motion.

        An object moving in a circle of radius $r$ in the $x$-$y$ plane has
        \begin{align*}
                 x        &=  r\ \zu{cos}\ \omega t \qquad \text{and}\\
                 y        &=  r\ \zu{sin}\ \omega t\eqquad,
        \end{align*} 
        where $\omega$ is the number of radians traveled per second, and the positive or negative
        sign indicates whether the motion is clockwise or counterclockwise.

        Differentiating, we find that the components of the velocity are
        \begin{align*}
                 v_{x}        &= -\omega r\ \zu{sin}\ \omega t \qquad \text{and}\\
                 v_{y}        &= \ \ \ \omega r\ \zu{cos}\ \omega t\eqquad,
        \end{align*} 
        and for the acceleration we have
        \begin{align*}
                 a_{x}        &= -\omega^2 r\ \zu{cos}\ \omega t \qquad \text{and}\\
                 a_{y}        &= -\omega^2 r\ \zu{sin}\ \omega t\eqquad.
        \end{align*} 
        The acceleration vector has cosines and sines in the same places as the
        \vc{r} vector, but with minus signs in front, so it points in the opposite direction,
        i.e., toward the center of the circle. By Newton's second law, \vc{a}=\vc{F}/$m$,
        this shows that the force must be inward as well; without this force, the object
        would fly off straight.
<% marg(m4_ifelse(__me,1,70,30)) %>
<%
  fig(
    'hammer',
    %q{%
      This figure shows an intuitive justification for the fact proved mathematically in this section,
      that the direction of the force and acceleration in circular motion is inward.
      The heptagon, 2, is a better approximation to a circle than the
              triangle, 1. To make an infinitely good approximation to circular motion,
              we would need to use an infinitely large number of infinitesimal taps, which
              would amount to a steady inward force.
    }
  )
%>
\spacebetweenfigs
<%
  fig(
    'carcircleforces',
    %q{%
      The
                       total force in the forward-backward direction is zero in both
              cases.
    }
  )
%>
<% end_marg %>

        The magnitude of the acceleration is
        \begin{align*}
                |\vc{a}|        &= \sqrt{ a_x^2+ a_{y}^2}\\
                                &= \omega^2 r\eqquad.
        \end{align*}
        It makes sense that $\omega$ is squared, since reversing the sign of $\omega$
        corresponds to reversing the direction of motion, but the acceleration is toward the
        center of the circle, regardless of whether the motion is clockwise or counterclockwise.
        This result can also be rewritten in the form
        \begin{equation*}
                |\vc{a}|        = \frac{|\vc{v}|^2}{r}\eqquad.
        \end{equation*}

        \index{circular motion!inward force}
        \index{circular motion!no forward force}
        These results are
        counterintuitive. Until Newton, physicists and laypeople alike had assumed
        that the planets would need a force to push them \emph{forward} in their orbits.
        Figure \figref{hammer} may help to make it more plausible that only an inward force
        is required. A forward force might be needed in order to cancel out a backward
        force such as friction, \figref{carcircleforces}, but the total force in the forward-backward
        direction needs to be exactly zero for constant-speed motion.
        \index{circular motion!no outward force}
        When you are in a car undergoing circular motion, there is also a strong illusion of an
        \emph{outward} force. But what object could be making such a force? The car's seat
        makes an inward force on you, not an outward one. There is no object that could be
        exerting an outward force on your body. In reality, this force is an illusion that comes
        from our brain's intuitive efforts to interpret the situation within a noninertial frame of
        reference. As shown in figure \figref{truckcircular}, we can describe everything perfectly
        well in an inertial frame of reference, such as the frame attached to the sidewalk.
        In such a frame, the bowling ball goes straight because there is \emph{no} force
        on it. The wall of the truck's bed hits the ball, not the other way around.
<% marg(m4_ifelse(__me,1,70,100)) %>
<%
  fig(
    'truckcircular',
    %q{%
      There is no outward force on the bowling ball, but in the noninertial
                      frame it seems like one exists.
    }
  )
%>
<% end_marg %>
:],[:

<% marg(5) %>
<%
  fig(
    'law-of-sines',
    %q{The law of sines.}
  )
%>
\spacebetweenfigs
<%
  fig(
    'v2-r-derivation',
    %q{%
      Deriving $|\vc{a}|=|\vc{v}|^2/r$ for uniform
      circular motion.
    }
  )
%>

<% end_marg %>
In this section I derive a simple and very useful equation
for the magnitude of the acceleration of an object
undergoing constant acceleration. The law of sines is
involved, so I've recapped it in figure \figref{law-of-sines}.

The derivation is brief, but the method requires some
explanation and justification. The idea is to calculate a
$\Delta\vc{v}$ vector describing the change in the velocity
vector as the object passes through an angle $\theta $. We
then calculate the acceleration, $\vc{a}=\Delta\vc{v}/\Delta t$. The
astute reader will recall, however, that this equation is
only valid for motion with constant acceleration. Although
the magnitude of the acceleration is constant for uniform
circular motion, the acceleration vector changes its
direction, so it is not a constant vector, and the equation
$\vc{a}=\Delta\vc{v}/\Delta t$ does not apply. The justification for
using it is that we will then examine its behavior when we
make the time interval very short, which means making the
angle $\theta $ very small. For smaller and smaller time
intervals, the $\Delta\vc{v}/\Delta t$ expression becomes a
better and better approximation, so that the final result of
the derivation is exact.

In figure \figref{v2-r-derivation}/1, the object sweeps out an angle $\theta $. Its
direction of motion also twists around by an angle $\theta$,
from the vertical dashed line to the tilted one. Figure
\figref{v2-r-derivation}/2
shows the initial and final velocity vectors, which have
equal magnitude, but directions differing by $\theta $. In
\figref{v2-r-derivation}/3,
I've reassembled the vectors in the proper
positions for vector subtraction. They form an isosceles
triangle with interior angles $\theta$, $\eta$, and $\eta$.
(Eta, $\eta$, is my favorite Greek letter.) The law of sines gives
\begin{equation*}
  \frac{|\Delta\vc{v}|}{\sin\theta} = \frac{|\vc{v}|}{\sin\eta}\eqquad.
\end{equation*}
This tells us the magnitude of $\Delta \vc{v}$, which is one of
the two ingredients we need for calculating the magnitude of
$\vc{a}=\Delta\vc{v}/\Delta t$. The other ingredient is $\Delta t$.
The time required for the object to move through the angle $\theta $ is
\begin{equation*}
  \Delta t = \frac{\text{length of arc}}{|\vc{v}|}\eqquad.
\end{equation*}
Now if we measure our angles in radians we can use the
definition of radian measure, which is 
$(\text{angle})=(\text{length of arc})/(\text{radius})$,
 giving $\Delta t=\theta r/|\vc{v}|$. Combining this
with the first expression involving $|\Delta v|$ gives
\begin{align*}
        |\vc{a}|     &=  |\Delta \vc{v}|/\Delta t  \\
             &=   \frac{|\vc{v}|^2}{r} \: \cdot \: \frac{\sin\theta}{\theta} \: \cdot \: \frac{1}{\sin\eta}\eqquad.
\end{align*}
When $\theta $ becomes very small, the small-angle
approximation $\sin \theta\approx \theta$ applies, and also $\eta $ becomes
close to 90\degunit, so $\sin  \eta \approx 1$, and we have an equation for $|\vc{a}|$:
\begin{equation*}
                |\vc{a}|  =  \frac{|\vc{v}|^2}{r}\eqquad. \qquad \shoveright{\text{[uniform circular motion]}}
\end{equation*}
:])

\begin{eg}{Force required to turn on a bike}
\egquestion A bicyclist is making a turn along an arc of a
circle with radius 20 m, at a speed of 5 m/s. If the
combined mass of the cyclist plus the bike is 60 kg, how
great a static friction force must the road be able
to exert on the tires?

\eganswer Taking the magnitudes of both sides of Newton's second law gives
\begin{align*}
        |\vc{F}|     &=  |m\vc{a}|  \\
             &=  m|\vc{a}|\eqquad.
\end{align*}
Substituting $|\vc{a}|=|\vc{v}|^2/r$ gives
\begin{align*}
        |\vc{F}|     &= m|\vc{v}|^2/r \\
                     &\approx 80\ \nunit
\end{align*}
(rounded off to one sig fig).
\end{eg}


\begin{eg}{Don't hug the center line on a curve!}
\egquestion You're driving on a mountain road with a steep
drop on your right. When making a left turn, is it safer to
hug the center line or to stay closer to the outside of the road?

\eganswer You want whichever choice involves the least
acceleration, because that will require the least force and
entail the least risk of exceeding the maximum force of
static friction. Assuming the curve is an arc of a circle
and your speed is constant, your car is performing uniform
circular motion, with $|\vc{a}|=|\vc{v}|^2/r$. The dependence on the
square of the speed shows that driving slowly is the main
safety measure you can take, but for any given speed you
also want to have the largest possible value of $r$. Even
though your instinct is to keep away from that scary
precipice, you are actually less likely to skid if you keep
toward the outside, because then you are describing a larger circle.
\end{eg}

\begin{eg}{Acceleration related to radius and period of rotation}\label{eg:accel-rt}
\egquestion How can the equation for the acceleration in
uniform circular motion be rewritten in terms of the radius
of the circle and the \index{period!of uniform circular
motion}period, $T$, of the motion, i.e., the time required to go around once?

\eganswer The period can be related to the speed as follows:
\begin{align*}
        |\vc{v}|     &=  \frac{\text{circumference}}{T}  \\
             &=  2\pi r/T\eqquad.
\end{align*}
Substituting into the equation $|\vc{a}|=|\vc{v}|^2/r$ gives
\begin{equation*}
        |\vc{a}|     =    \frac{4\pi^2r}{T^2}\eqquad.
\end{equation*}
\end{eg}

<% marg(0) %>
<%
  fig(
    'clothes-dryer',
    %q{Example \ref{eg:clothes-dryer}.}
  )
%>
<% end_marg %>

\begin{eg}{A clothes dryer}\label{eg:clothes-dryer}
\egquestion My clothes dryer has a drum with an inside radius
of 35 cm, and it spins at 48 revolutions per minute. What is
the acceleration of the clothes inside?

\eganswer We can solve this by finding the period and
plugging in to the result of the previous example. If it
makes 48 revolutions in one minute, then the period is 1/48
of a minute, or 1.25 s. To get an acceleration in mks
units, we must convert the radius to 0.35 m. Plugging in,
the result is 8.8 $\munit/\sunit^2$.
\end{eg}

\begin{eg}{More about clothes dryers!}
\egquestion In a discussion question in the previous section,
we made the assumption that the clothes remain against the
inside of the drum as they go over the top. In light of the
previous example, is this a correct assumption?

\eganswer No. We know that there must be some minimum speed
at which the motor can run that will result in the clothes
just barely staying against the inside of the drum as they
go over the top. If the clothes dryer ran at just this
minimum speed, then there would be no normal force on the
clothes at the top: they would be on the verge of losing
contact. The only force acting on them at the top would be
the force of gravity, which would give them an acceleration
of $g=9.8\ \munit/\sunit^2$. The actual dryer must be running slower
than this minimum speed, because it produces an acceleration
of only $8.8\ \munit/\sunit^2$. My theory is that this is done
intentionally, to make the clothes mix and tumble.
\end{eg}

\worked{tilt-a-whirl}{The tilt-a-whirl}

\worked{off-ramp}{An off-ramp}

\startdqs

\begin{dq}
A certain amount of force is needed to provide the
acceleration of circular motion. What if we are exerting a
force perpendicular to the direction of motion in an attempt
to make an object trace a circle of radius $r$, but the
force isn't as big as $m|\vc{v}|^2/r$?
\end{dq}

\begin{dq}\label{dq:space-colony}
Suppose a rotating space station, as in figure
\figref{space-colony} on page \pageref{fig:space-colony}, is built. It
gives its
occupants the illusion of ordinary gravity. What happens
when a person in the station lets go of a ball? What happens
when she throws a ball straight ``up'' in the air (i.e.,
towards the center)?
\end{dq}

<%
  fig(
    'space-colony',
    %q{%
      Discussion question
      \ref{dq:space-colony}. An artist's conception of a rotating space
      colony in the form of a giant wheel. A
      person living in this noninertial frame of
      reference has an illusion of a force pulling
       her outward, toward the deck, for the
      same reason that a person in the pickup
      truck has the illusion of a force pulling
      the bowling ball. By adjusting the speed
      of rotation,  the designers can make an
      acceleration $|\vc{v}|^2/r$ equal to the usual 
      acceleration of gravity on earth. On earth,
      your acceleration standing on the ground
      is zero, and a falling rock heads for your
      feet with an acceleration of 9.8 $\munit/\sunit^2$. A
      person standing on the deck of the space
      colony has an \emph{upward} acceleration of 9.8
      $\munit/\sunit^2$, and when she lets go of a rock,
      her feet head \emph{up} at the nonaccelerating
      rock. To her, it seems the same as true
      gravity.
    },
    {
      'width'=>'wide',
      'sidecaption'=>true,
      'narrowfigwidecaption'=>true,
      'float'=>false,
      'anonymous'=>false
    }
  )
%>

<% end_sec() %>
<% begin_sec("Nonuniform circular motion",3) %>

<% marg(5) %>
<%
  fig(
    'nonuniform',
    %q{%
      1. Moving in a circle while speeding up.
      2. Uniform circular motion. 3. Slowing down.
    }
  )
%>
<% end_marg %>

What about nonuniform circular motion? Although so far we
have been discussing components of vectors along fixed $x$
and $y$ axes, it now becomes convenient to discuss
components of the acceleration vector along the radial line
(in-out) and the tangential line (along the direction of
motion). For nonuniform circular motion, the \index{radial
component!defined}radial component of the acceleration obeys
the same equation as for uniform circular motion,
\begin{equation*}
        a_r     =  v^2/r\eqquad,
\end{equation*}
where $v=|\vc{v}|$, but the acceleration vector also has a tangential component,
\begin{equation*}
m4_ifelse(__me,1,[:%
            a_t     =  \frac{\der v}{\der t}\eqquad.
:],[:%
            a_t     =  \text{slope of the graph of $v$ versus $t$}\eqquad.
:])%
\end{equation*}
The latter quantity has a simple interpretation. If you are
going around a curve in your car, and the speedometer needle
is moving, the tangential component of the acceleration
vector is simply what you would have thought the acceleration
was if you saw the speedometer and didn't know you were
going around a curve.

\begin{eg}{Slow down before a turn, not during it.}
\egquestion When you're making a turn in your car and you're
afraid you may skid, isn't  it a good idea to slow down?

\eganswer If the turn is an arc of a circle, and you've
already completed part of the turn at constant speed without
skidding, then the road and tires are apparently capable of
enough static friction to supply an acceleration of
$|\vc{v}|^2/r$. There is no reason why you would skid out now if
you haven't already. If you get nervous and brake, however,
then you need to have a tangential acceleration component in
addition to the radial one you were already able to
produce successfully. This would require an acceleration
vector with a greater magnitude, which in turn would require
a larger force. Static friction might not be able to supply
that much force, and you might skid out. The safer thing to do is to
approach the turn at a comfortably low speed.
\end{eg}

\enlargethispage{\baselineskip}


\worked{cyclists}{A bike race}

<% end_sec() %>\begin{summary}

\begin{vocab}

\vocabitem{uniform circular motion}{circular motion in which the
magnitude of the velocity vector remains constant}

\vocabitem{nonuniform circular motion}{circular motion in which the
magnitude of the velocity vector changes}

\vocabitem{radial}{parallel to the radius of a circle; the in-out direction}

\vocabitem{tangential}{tangent to the circle, perpendicular to
the radial direction}

\end{vocab}

\begin{notation}

\notationitem{$a_r$}{radial acceleration; the component of the acceleration
vector along the in-out direction}

\notationitem{$a_t$}{tangential acceleration; the component of the
acceleration vector tangent to the circle}
\end{notation}

\begin{summarytext}

If an object is to have circular motion, a force must be
exerted on it toward the center of the circle. There is no
outward force on the object; the illusion of an outward
force comes from our experiences in which our point of view
was rotating, so that we were viewing things in a noninertial frame.

An object undergoing uniform circular motion has an inward
acceleration vector of magnitude
\begin{equation*}
        |\vc{a}|  =  v^2/r\eqquad,
\end{equation*}
where $v=|\vc{v}|$.
In nonuniform circular motion, the radial and tangential
components of the acceleration vector are
\begin{align*}
        a_r     &=  v^2/r   \\
        a_t     &=  %
m4_ifelse(__me,1,[:%
                   \frac{\der v}{\der t}\eqquad.
:],[:%
                   \text{slope of the graph of $v$ versus $t$}\eqquad.
:])%
\end{align*}

\end{summarytext}

\end{summary}
