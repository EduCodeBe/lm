\begin{summary}

\begin{vocab}

\vocabitem{work}{the amount of energy transferred into or out of a
system, excluding energy transferred by heat conduction}

\end{vocab}

\begin{notation}

\notationitem{$W$}{work}

\end{notation}

\begin{summarytext}

Work is a measure of the transfer of mechanical energy, i.e.,
the transfer of energy by a force rather than by heat
conduction. When the force is constant, work can usually be calculated as
\begin{equation*}
                W  =  F_{\parallel} |\vc{d}|\eqquad,           \qquad  \text{[only if the
force is constant]}
\end{equation*}
where $\vc{d}$ is simply a less cumbersome notation for $\Delta \vc{r}$,
the vector from the initial position to the final position. Thus,
\begin{itemize}
\item A force in the same direction as the motion does positive
work, i.e., transfers energy into the object on which it acts.

\item A force in the opposite direction compared to the motion
does negative work, i.e., transfers energy out of the
object on which it acts.

\item When there is no motion, no mechanical work is done. The
human body burns calories when it exerts a force without
moving, but this is an internal energy transfer of energy
within the body, and thus does not fall within the
scientific definition of work.

\item A force perpendicular to the motion does no work.
\end{itemize}
When the force is not constant, the above equation should be generalized as
m4_ifelse(__me,1,[:%
an integral, $\int F_{\parallel}\der x$.
:],[:%
the area under the graph of $F_{\parallel}$ versus $d$.
:])

m4_ifelse(__me,1,[:
	There is only one meaningful (rotationally invariant)
	way of defining a multiplication of vectors
	whose result is a scalar, and it is known as the vector \sumem{dot product:}
	\begin{align*}	
		\vc{b}\cdot\vc{c}	
			&=  b_{x} c_{x}+ b_y c_{y}+ b_{z} c_z \\
			&= |\vc{b}|\,|\vc{c}|\,\cos\theta_{bc}\eqquad.
	\end{align*}
	The dot product has most of the usual properties associated with multiplication,
	except that there is no ``dot division.''
        The dot product can be used to compute mechanical work as $W=\vc{F}\cdot\vc{d}$.
:])

Machines such as pulleys, levers, and gears may increase or
decrease a force, but they can never increase or decrease
the amount of work done. That would violate conservation of
energy unless the machine had some source of stored energy
or some way to accept and store up energy.

There are some situations in which the equation $W=F_{\parallel}$
$|\vc{d}|$ is ambiguous or not true, and these issues are
discussed rigorously in section \ref{sec:when-is-work-fd}. However, problems can
usually be avoided by analyzing the types of energy being
transferred before plunging into the math. In any case there
is no substitute for a physical understanding of the processes involved.

The techniques developed for calculating work can also be
applied to the calculation of potential energy. We fix some
position as a reference position, and calculate the
potential energy for some other position, $x$, as
\begin{equation*}
                PE_x  =  -W_{\text{ref}\rightarrow x}\eqquad.
\end{equation*}

The following two equations for potential energy have
broader significance than might be suspected based on the
limited situations in which they were derived:
\begin{equation*}
  PE = \frac{1}{2}k\left(x-x_\zu{o}\right)^2\eqquad.
\end{equation*}
\begin{longnoteafterequation}
[potential energy of a spring having spring constant $k$,
when stretched or compressed from the equilibrium position
$x_o;$ analogous equations apply for the twisting, bending,
compression, or stretching of any object.]
\end{longnoteafterequation}
\begin{equation*}
                PE  = -\frac{GMm}{r}
\end{equation*}
\begin{longnoteafterequation}
[gravitational potential energy of objects of masses $M$
and $m$, separated by a distance $r$; an analogous equation
applies to the electrical potential energy of an electron in an atom.]
\end{longnoteafterequation}
\end{summarytext}

\end{summary}
