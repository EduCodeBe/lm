In many subfields of physics these days, it is possible to
read an entire issue of a journal without ever encountering
an equation involving force or a reference to Newton's laws
of motion. In the last hundred and fifty years, an entirely
different framework has been developed for physics, based
on conservation laws.

The new approach is not just preferred because it is in
fashion. It applies inside an atom or near a black hole,
where Newton's laws do not. Even in everyday situations the
new approach can be superior. We have already seen how
perpetual motion machines could be designed that were too
complex to be easily debunked by Newton's laws. The beauty
of conservation laws is that they tell us something must
remain the same, regardless of the complexity of the process.

So far we have discussed only two conservation laws, the
laws of conservation of mass and energy. Is there any reason
to believe that further conservation laws are needed in
order to replace Newton's laws as a complete description of
nature? Yes. Conservation of mass and energy do not relate
in any way to the three dimensions of space, because both
are scalars. Conservation of energy, for instance, does not
prevent the planet earth from abruptly making a 90-degree
turn and heading straight into the sun, because kinetic
energy does not depend on direction. In this chapter, we
develop a new conserved quantity, called momentum, which is a vector.

<% begin_sec("Momentum",0,'momentum') %>

  <% begin_sec("A conserved quantity of motion") %>

Your first encounter with conservation of momentum may have
come as a small child unjustly confined to a shopping cart.
You spot something interesting to play with, like the
display case of imported wine down at the end of the aisle,
and decide to push the cart over there. But being imprisoned
by Dad in the cart was not the only injustice that day.
There was a far greater conspiracy to thwart your young id,
one that originated in the laws of nature. Pushing forward
did nudge the cart forward, but it pushed you backward. If
the wheels of the cart were well lubricated, it wouldn't
matter how you jerked, yanked, or kicked off from the back
of the cart. You could not cause any overall forward motion
of the entire system consisting of the cart with you inside.

\enlargethispage{-4\baselineskip}

In the Newtonian framework, we describe this as arising from
Newton's third law. The cart made a force on you that was
equal and opposite to your force on it. In the framework of
conservation laws, we cannot attribute your frustration to
conservation of energy. It would have been perfectly
possible for you to transform some of the internal chemical
energy stored in your body to kinetic energy of the cart and your body.

The following characteristics of the situation suggest that
there may be a new conservation law involved:

\oneofaseriesofpoints{A closed system is involved.}{All conservation laws deal with
closed systems. You and the cart are a closed system, since
the well-oiled wheels prevent the floor from making any
forward force on you.}

\oneofaseriesofpoints{Something remains unchanged.}{The overall velocity of the
system started out being zero, and you cannot change it.
This vague reference to ``overall velocity'' can be made
more precise: it is the velocity of the system's center of
mass that cannot be changed.}

\oneofaseriesofpoints{Something can be transferred back and forth without changing
the total amount.}{If we define forward as positive and
backward as negative, then one part of the system can gain
positive motion if another part acquires negative motion. If
we don't want to worry about positive and negative signs, we
can imagine that the whole cart was initially gliding
forward on its well-oiled wheels. By kicking off from the
back of the cart, you could increase your own velocity, but
this inevitably causes the cart to slow down.}

\noindent It thus appears that there is some numerical measure of an
object's quantity of motion that is conserved when you add
up all the objects within a system.

<% end_sec() %>
<% begin_sec("Momentum") %>

Although velocity has been referred to, it is not the total
velocity of a closed system that remains constant. If it
was, then firing a gun would cause the gun to recoil at the
same velocity as the bullet! The gun does recoil, but at a
much lower velocity than the bullet. Newton's third law tells us
\begin{equation*}
                F_{gun\ on\ bullet}    =    - F_{bullet\ on\ gun}\eqquad,
\end{equation*}
and assuming a constant force for simplicity, Newton's
second law allows us to change this to
\begin{equation*}
        m_{bullet}\frac{\Delta v_{bullet}}{\Delta t}
            =    -m_{gun}\frac{\Delta v_{gun}}{\Delta t}\eqquad.
\end{equation*}
Thus if the gun has 100 times more mass than the bullet, it
will recoil at a velocity that is 100 times smaller and in
the opposite direction, represented by the opposite sign.
The quantity $mv$ is therefore apparently a useful
measure of motion, and we give it a name, $\index{momentum!defined}momentum$,
and a symbol, $p$. (As far as I know, the letter ``p'' was
just chosen at random, since ``m'' was already being used
for mass.) The situations discussed so far have been
one-dimensional, but in three-dimensional situations it is
treated as a vector.

\begin{important}[definition of momentum for material objects]
The momentum of a material object, i.e., a piece of
matter, is defined as
\begin{equation*}
                \vc{p}    =    m\vc{v}\eqquad,
\end{equation*}
the product of the object's mass and its velocity vector.
\end{important}
\noindent The units of momentum are $\kgunit\unitdot\munit/\sunit$, and there is unfortunately
no abbreviation for this clumsy combination of units.

The reasoning leading up to the definition of momentum was
all based on the search for a conservation law, and the only
reason why we bother to define such a quantity is that
experiments show it is conserved:

\begin{important}[the law of conservation of momentum]
In any closed system, the vector sum of all the momenta remains constant,
\begin{equation*}
                \vc{p}_{1i}+ \vc{p}_{2i}+ \ldots    =    \vc{p}_{1f} + \vc{p}_{2f} + \ldots\eqquad,
\end{equation*}
where $i$ labels the initial and $f$ the final momenta. (A
closed system is one on which no external forces act.)
\end{important}
\noindent This chapter first addresses the one-dimensional case, in
which the direction of the momentum can be taken into
account by using plus and minus signs. We then pass to three
dimensions, necessitating the use of vector addition.

\enlargethispage{-4\baselineskip}

A subtle point about conservation laws is that they all
refer to ``closed systems,'' but ``closed'' means different
things in different cases. When discussing conservation of
mass, ``closed'' means a system that doesn't have matter
moving in or out of it. With energy, we mean that there is
no work or heat transfer occurring across the boundary of
the system. For momentum conservation, ``closed'' means
there are no external \emph{forces} reaching into the system.

\begin{eg}{A cannon}
\egquestion A cannon of mass 1000 kg fires a 10-kg shell at a
velocity of 200 m/s. At what speed does the cannon recoil?

\eganswer The law of conservation of momentum tells us that
\begin{equation*}
                p_{cannon,i} + p_{shell,i}  =  p_{cannon,f} + p_{shell,f}\eqquad.
\end{equation*}
Choosing a coordinate system in which the cannon points in
the positive direction, the given information is
\begin{align*}
 p_{cannon,i} &= 0 \\
 p_{shell,i} &= 0 \\
 p_{shell,f} &= 2000\ \kgunit\unitdot\munit/\sunit\eqquad.
\end{align*}
We must have $p_{cannon,f}=-2000\ \kgunit\unitdot\munit/\sunit$, so the recoil
velocity of the cannon is $-2$ m/s.
\end{eg}

<%
  fig(
    'ion-drive',
    %q{%
      The ion drive engine of the NASA Deep Space 1 probe, shown under construction (left) and being tested in a vacuum
      chamber (right) prior to its October 1998 launch. Intended mainly as a test vehicle for new technologies, the craft nevertheless
      carried out a successful scientific program that included a flyby of a comet.
    },
    {
      'width'=>'wide'
    }
  )
%>

\begin{eg}{Ion drive for propelling spacecraft}\label{eg:ion-drive}
\egquestion The experimental solar-powered ion drive of the
Deep Space 1 space probe expels its xenon gas exhaust at a
speed of 30,000 m/s, ten times faster than the exhaust
velocity for a typical chem\-ical-fuel rocket engine. Roughly
how many times greater is the maximum speed this spacecraft
can reach, compared with a chem\-ical-fueled probe with the
same mass of fuel (``reaction mass'') available for pushing
out the back as exhaust?

\eganswer Momentum equals mass multiplied by velocity. Both
spacecraft are assumed to have the same amount of reaction
mass, and the ion drive's exhaust has a velocity ten times
greater, so the momentum of its exhaust is ten times
greater. Before the engine starts firing, neither the probe
nor the exhaust has any momentum, so the total momentum of
the system is zero. By conservation of momentum, the total
momentum must also be zero after all the exhaust has been
expelled. If we define the positive direction as the
direction the spacecraft is going, then the negative
momentum of the exhaust is canceled by the positive momentum
of the spacecraft. The ion drive allows a final speed that
is ten times greater. (This simplified analysis ignores the
fact that the reaction mass expelled later in the burn is
not moving backward as fast, because of the forward speed of
the already-moving spacecraft.)
\end{eg}

<% end_sec() %>
<% begin_sec("Generalization of the momentum concept") %>

As with all the conservation laws, the law of conservation
of momentum has evolved over time. In the 1800's it was
found that a beam of light striking an object would give it
some momentum, even though light has no mass, and would
therefore have no momentum according to the above definition.
Rather than discarding the principle of conservation of
momentum, the physicists of the time decided to see if the
definition of momentum could be extended to include momentum
carried by light. The process is analogous to the process
outlined on page
\pageref{subsec:new-forms-of-energy} for identifying new forms of energy.
The first step was the discovery that light could impart
momentum to matter, and the second step was to show that the
momentum possessed by light could be related in a definite
way to observable properties of the light. They found that
conservation of momentum could be successfully generalized
by attributing to a beam of light a momentum vector in the
direction of the light's motion and having a magnitude
proportional to the amount of energy the light possessed.
The momentum of \index{momentum!of light}light is negligible
under ordinary circumstances, e.g., a flashlight left on for
an hour would only absorb about $10^{-5}\ \kgunit\unitdot\munit/\sunit$ of
momentum as it recoiled.
<% marg(5) %>

<%
  fig(
    'halley-nucleus',
    %q{%
      Steam and other gases boiling off of the nucleus of Halley's comet.
      This close-up photo was taken by the European Giotto space probe, which
      passed within 596 km of the nucleus on March 13, 1986. 
    }
  )
%>
\spacebetweenfigs
<%
  fig(
    'halley',
    %q{Halley's comet, in a much less magnified view from a ground-based telescope.}
  )
%>

<% end_marg %>

\begin{eg}{The tail of a comet}\label{eg:halley}
Momentum is not always equal to $mv$.
Like many comets, Halley's
comet has a very elongated elliptical orbit.
 About once per century, its orbit
brings it close to the sun. The comet's head,
or nucleus, is composed of dirty ice, so the
energy deposited by the intense sunlight boils
off steam and dust, \figref{halley-nucleus}.
The sunlight does not
just carry energy, however --- it also carries
momentum. The momentum of the sunlight impacting on the smaller dust particles pushes
them away from the sun, forming a tail, \figref{halley}. By analogy with matter, for
which momentum equals $mv$, you would expect that massless light would have zero
momentum, but the equation $p=mv$ is not the
correct one for light, and light does have
momentum. (The gases typically form a second, distinct tail whose motion
is controlled by the sun's magnetic field.)
\end{eg}

The reason for bringing this up is not so that you can plug
numbers into a formulas in these exotic situations. The
point is that the conservation laws have proven so sturdy
exactly because they can easily be amended to fit new
circumstances. Newton's laws are no longer at the center of
the stage of physics because they did not have the same
adaptability. More generally, the moral of this story is the
provisional nature of scientific truth.

It should also be noted that conservation of momentum is not
a consequence of Newton's laws, as is often asserted in
textbooks. Newton's laws do not apply to light, and
therefore could not possibly be used to prove anything about
a concept as general as the conservation of momentum in its modern form.

<% end_sec() %>
<% begin_sec("Momentum compared to kinetic energy") %>
\index{kinetic energy!compared to momentum}\index{momentum!compared to kinetic energy}

Momentum and kinetic energy are both measures of the
quantity of motion, and a sideshow in the Newton-Leibnitz
controversy over who invented calculus was an argument over
whether \emph{mv} (i.e., momentum) or $mv^2$ (i.e., kinetic
energy without the 1/2 in front) was the ``true'' measure of
motion. The modern student can certainly be excused for
wondering why we need both quantities, when their complementary
nature was not evident to the greatest minds of the 1700's.
The following table highlights their differences.

\vfill

\noindent\begin{tabular}{|p{60mm}|p{60mm}|}\hline
\textbf{kinetic energy \ldots} & \textbf{momentum \ldots} \\ \hline
is a scalar. &
                               is a vector. \\ \hline
 %
is not changed by a force perpendicular to the motion, which changes only the direction of the velocity vector. &
is changed by any force, since a change in either the magnitude or the direction of the velocity vector will
result in a change in the momentum vector. \\ \hline
 %
is always positive, and cannot cancel out. &
cancels with momentum in the opposite direction. \\ \hline
 %
can be traded for other forms of energy that do not involve motion. KE is not a conserved quantity by itself. &
is always conserved in a closed system. \\ \hline
is quadrupled if the velocity is doubled. & is doubled if the velocity is doubled. \\ \hline
\end{tabular}

\vfill

\begin{eg}{A spinning top}\label{eg:spinning-top}
A spinning top has zero total momentum, because for every
moving point, there is another point on the opposite side
that cancels its momentum. It does, however, have kinetic energy.
\end{eg}

<% marg(5) %>

<%
  fig(
    'eg-top-and-tuning-fork',
    %q{%
      Examples \ref{eg:spinning-top} and \ref{eg:tuning-fork}. The momenta cancel, but the energies don't.
    }
  )
%>
<% end_marg %>

\begin{eg}{Why a tuning fork has two prongs}\label{eg:tuning-fork}
A tuning fork is made with two prongs so that they can vibrate in opposite directions,
canceling their momenta. In a hypothetical version with only one prong, the momentum
would have to oscillate, and this momentum would have to come from somewhere, such as
the hand holding the fork. The result would be that vibrations would be transmitted
to the hand and rapidly die out. In a two-prong fork, the two momenta cancel, but the
energies don't.
\end{eg}

\begin{eg}{Momentum and kinetic energy in firing a rifle}
The rifle and bullet have zero momentum and zero kinetic
energy to start with. When the trigger is pulled, the bullet
gains some momentum in the forward direction, but this is
canceled by the rifle's backward momentum, so the total
momentum is still zero. The kinetic energies of the gun and
bullet are both positive scalars, however, and do not
cancel. The total kinetic energy is allowed to increase,
because kinetic energy is being traded for other forms of
energy. Initially there is chemical energy in the gunpowder.
This chemical energy is converted into heat, sound, and
kinetic energy. The gun's ``backward'' kinetic energy does
not refrigerate the shooter's shoulder!
\end{eg}

\pagebreak

\begin{eg}{The wobbly earth}
As the moon completes half a circle around the earth, its
motion reverses direction. This does not involve any change
in kinetic energy, and the earth's gravitational force does
not do any work on the moon. The reversed velocity vector
does, however, imply a reversed momentum vector, so
conservation of momentum in the closed earth-moon system
tells us that the earth must also change its momentum. In
fact, the earth wobbles in a little ``orbit'' about a point
below its surface on the line connecting it and the moon.
The two bodies' momentum vectors always point in opposite
directions and cancel each other out.
\end{eg}

\begin{eg}{The earth and moon get a divorce}
Why can't the moon suddenly decide to fly off one way and
the earth the other way? It is not forbidden by conservation
of momentum, because the moon's newly acquired momentum in
one direction could be canceled out by the change in the
momentum of the earth, supposing the earth headed the
opposite direction at the appropriate, slower speed. The
catastrophe is forbidden by conservation of energy, because
both their energies would have to increase greatly.
\end{eg}

\begin{eg}{Momentum and kinetic energy of a glacier}
A cubic-kilometer glacier would have a mass of about
$10^{12}$ kg. If it moves at a speed of $10^{-5}$ m/s,
then its momentum is $10^7\ \kgunit\unitdot\munit/\sunit$. This is the kind of
heroic-scale result we expect, perhaps the equivalent of the
space shuttle taking off, or all the cars in LA driving in
the same direction at freeway speed. Its kinetic energy,
however, is only 50 J, the equivalent of the calories
contained in a poppy seed or the energy in a drop of
gasoline too small to be seen without a microscope. The
surprisingly small kinetic energy is because kinetic energy
is proportional to the square of the velocity, and the
square of a small number is an even smaller number.
\end{eg}

\startdqs

\begin{dq}
If all the air molecules in the room settled down in a thin film on the floor,
would that violate conservation of momentum? Conservation of energy?
\end{dq}

\begin{dq}
A refrigerator has coils in the back that get hot, and heat
is molecular motion. These moving molecules have both energy
and momentum. Why doesn't the refrigerator need to be tied
to the wall to keep it from recoiling from the momentum it loses out the
back?
\end{dq}

<% end_sec() %>
<% end_sec('momentum') %>
<% begin_sec("Collisions in one dimension",4,'collisions-1d') %>

<% marg(4) %>
<%
  fig(
    'colliding-galaxies',
    %q{%
      This Hubble Space Telescope photo
      shows a small galaxy (yellow blob in
      the lower right) that has collided with
      a larger galaxy (spiral near the center), producing a wave of star formation (blue track) due to the shock
      waves passing through the galaxies'
      clouds of gas. This is considered a
      collision in the physics sense, even
      though it is statistically certain that no
      star in either galaxy ever struck a star
      in the other. (This is because the stars
      are very small compared to the distances between them.)
    }
  )
%>
<% end_marg %>%
Physicists employ the term ``\index{collision!defined}collision''
in a broader sense than ordinary usage, applying it to any
situation where objects interact for a certain period of
time. A bat hitting a baseball, a radioactively emitted
particle damaging DNA, and a gun and a bullet going their
separate ways are all examples of collisions in this sense.
Physical contact is not even required. A comet swinging past
the sun on a hyperbolic orbit is considered to undergo a
collision, even though it never touches the sun. All that
matters is that the comet and the sun exerted gravitational
forces on each other.

The reason for broadening the term ``collision'' in this way
is that all of these situations can be attacked mathematically
using the same conservation laws in similar ways. In the
first example, conservation of momentum is all that is required.

\begin{eg}{Getting rear-ended}\label{eg:rear-ended}
\egquestion Ms. Chang is rear-ended at a stop light by Mr.
Nelson, and sues to make him pay her medical bills. He
testifies that he was only going 35 miles per hour when he
hit Ms. Chang. She thinks he was going much faster than
that. The cars skidded together after the impact, and
measurements of the length of the skid marks and the
coefficient of friction show that their joint velocity
immediately after the impact was 19 miles per hour. Mr.
Nelson's Nissan weighs 3100 pounds, and Ms. Chang 's
Cadillac weighs 5200 pounds. Is Mr. Nelson telling the truth?

\eganswer Since the cars skidded together, we can write down
the equation for conservation of momentum using only two
velocities, $v$ for Mr. Nelson's velocity before the crash,
and $v'$ for their joint velocity afterward:
\begin{equation*}
                m_N v  =  m_N v' + m_C v'\eqquad.
\end{equation*}
Solving for the unknown, $v$, we find
\begin{equation*}
                v  =   \left(1+\frac{m_C}{m_N}\right)v'\eqquad.
\end{equation*}
Although we are given the weights in pounds, a unit of
force, the ratio of the masses is the same as the ratio of
the weights, and we find $v=51$ miles per hour. He is lying.
\end{eg}

The above example was simple because both cars had the same
velocity afterward. In many one-dimensional collisions,
however, the two objects do not stick. If we wish to predict
the result of such a collision, conservation of momentum
does not suffice, because both velocities after the
collision are unknown, so we have one equation in two unknowns.

Conservation of energy can provide a second equation, but
its application is not as straightforward, because kinetic
energy is only the particular form of energy that has to do
with motion. In many collisions, part of the kinetic energy
that was present before the collision is used to create heat
or sound, or to break the objects or permanently bend them.
Cars, in fact, are carefully designed to crumple in a
collision. Crumpling the car uses up energy, and that's good
because the goal is to get rid of all that kinetic energy in
a relatively safe and controlled way. At the opposite
extreme, a superball is ``super'' because it emerges from a
collision with almost all its original kinetic energy,
having only stored it briefly as potential energy while it
was being squashed by the impact.

Collisions of the superball type, in which almost no kinetic
energy is converted to other forms of energy, can thus be
analyzed more thoroughly, because they have $KE_f=KE_i$, as
opposed to the less useful inequality $KE_f<KE_i$ for a
case like a tennis ball bouncing on grass.\label{elastic-and-inelastic}
m4_ifelse(__me,1,[:These two types of collisions are referred to, respectively, as elastic
and inelastic.\index{collision!elastic}\index{collision!inelastic}
The extreme inelastic case is discussed further on p.~\pageref{subsubsec:totally-inelastic}.:])

\begin{eg}{Pool balls colliding head-on}\label{eg:pool-balls}
\egquestion Two pool balls collide head-on, so that the
collision is restricted to one dimension. Pool balls are
constructed so as to lose as little kinetic energy as
possible in a collision, so under the assumption that no
kinetic energy is converted to any other form of energy,
what can we predict about the results of such a collision?

\eganswer Pool balls have identical masses, so we use the
same symbol $m$ for both. Conservation of momentum and no loss
of kinetic energy give us the two equations
\begin{align*}
  mv_{1i}+mv_{2i} &=  mv_{1f}+mv_{2f} \\
  \frac{1}{2}mv_{1i}^2+\frac{1}{2}mv_{2i}^2 &=  \frac{1}{2}mv_{1f}^2+\frac{1}{2}mv_{2f}^2
\end{align*}
The masses and the factors of 1/2 can be divided out, and we
eliminate the cumbersome subscripts by replacing the symbols
$v_{1i}$,... with the symbols $A,B,C$, and $D$:
\begin{align*}
 A+B &= C+D \\
 A^2+B^2 &= C^2+D^2\eqquad.
\end{align*}
A little experimentation with numbers shows that given
values of $A$ and $B$, it is impossible to find $C$ and $D$
that satisfy these equations unless $C$ and $D$ equal $A$
and $B$, or $C$ and $D$ are the same as $A$ and $B$ but
swapped around. A formal proof of this fact is given in the
sidebar.
In the special case where ball 2 is initially at
rest, this tells us that ball 1 is stopped dead by the
collision, and ball 2 heads off at the velocity originally
possessed by ball 1. This behavior will be familiar to players of pool.
\end{eg}
<% marg(100) %>
\formatlikecaption{%
\begin{flushleft}\textit{Gory details of the proof in example \ref{eg:pool-balls}}\end{flushleft}
The equation $A+B = C+D$ says that
the change in one ball's velocity is
equal and opposite to the change
in the other's. We invent a symbol
$x=C-A$ for the change in ball 1's velocity. The second equation can
then be rewritten as
$A^2+B^2 = (A+x)^2+(B-x)^2$.
Squaring out the quantities in parentheses and then simplifying, we get
$0 = Ax-Bx+x^2$.
The equation has the trivial solution $x=0$, i.e., neither ball's velocity
is changed, but this is physically
impossible because the balls can't travel through each other like
ghosts. Assuming $x\ne 0$, we can divide by $x$ and solve for $x=B-A$. This
means that ball 1 has gained an
amount of velocity exactly right to match ball 2's
initial velocity, and vice-versa. The
balls must have swap\-ped velocities.}%
<% end_marg %>

Often, as in the example above, the details of the algebra
are the least interesting part of the problem, and
considerable physical insight can be gained simply by
counting the number of unknowns and comparing to the number
of equations. Suppose a beginner at pool notices a case
where her cue ball hits an initially stationary ball and
stops dead. ``Wow, what a good trick,'' she thinks. ``I bet
I could never do that again in a million years.'' But she
tries again, and finds that she can't help doing it even if
she doesn't want to. Luckily she has just learned about
collisions in her physics course. Once she has written down
the equations for conservation of energy and no loss of
kinetic energy, she really doesn't have to complete the
algebra. She knows that she has two equations in two
unknowns, so there must be a well-defined solution. Once she
has seen the result of one such collision, she knows that
the same thing must happen every time. The same thing would
happen with colliding marbles or croquet balls. It doesn't
matter if the masses or velocities are different, because
that just multiplies both equations by some constant factor.

<% begin_sec("The discovery of the neutron") %>

This was the type of reasoning employed by James \index{Chadwick,
James!discovery of neutron}Chadwick in his 1932 discovery of
the \index{neutron!discovery of}neutron. At the time, the
atom was imagined to be made out of two types of fundamental
particles, \index{proton}protons and \index{electron}electrons.
The protons were far more massive, and clustered together in
the atom's core, or \index{nucleus}nucleus. Attractive
\index{electrical force!in atoms}electrical forces caused
the electrons to orbit the nucleus in circles, in much the
same way that gravitational forces kept the planets from
cruising out of the solar system. Experiments showed that
the helium nucleus, for instance, exerted exactly twice as
much electrical force on an electron as a nucleus of
hydrogen, the smallest atom, and this was explained by
saying that helium had two protons to hydrogen's one. The
trouble was that according to this model, helium would have
two electrons and two protons, giving it precisely twice the
mass of a hydrogen atom with one of each. In fact, helium
has about four times the mass of hydrogen.

Chadwick suspected that the helium nucleus possessed two
additional particles of a new type, which did not participate
in electrical forces at all, i.e., were electrically neutral.
If these particles had very nearly the same mass as protons,
then the four-to-one mass ratio of helium and hydrogen could
be explained. In 1930, a new type of radiation was
discovered that seemed to fit this description. It was
electrically neutral, and seemed to be coming from the
nuclei of light elements that had been exposed to other
types of radiation. At this time, however, reports of new
types of particles were a dime a dozen, and most of them
turned out to be either clusters made of previously known
particles or else previously known particles with higher
energies. Many physicists believed that the ``new'' particle
that had attracted Chadwick's interest was really a
previously known particle called a \index{gamma ray}gamma
ray, which was electrically neutral. Since gamma rays have
no mass, Chadwick decided to try to determine the new
particle's mass and see if it was nonzero and approximately
equal to the mass of a proton.

Unfortunately a subatomic particle is not something you can
just put on a scale and weigh. Chadwick came up with an
ingenious solution. The masses of the nuclei of the various
chemical elements were already known, and techniques had
already been developed for measuring the speed of a rapidly
moving nucleus. He therefore set out to bombard samples of
selected elements with the mysterious new particles. When a
direct, head-on collision occurred between a mystery
particle and the nucleus of one of the target atoms, the
nucleus would be knocked out of the atom, and he would
measure its velocity.
<%
  fig(
    'chadwick',
    %q{%
      %
      Chadwick's subatomic pool table. A disk of the naturally occurring metal polonium provides a source 
      of radiation capable of kicking neutrons out of the beryllium nuclei. The
      type of radiation emitted by the polonium is easily absorbed
      by a few mm of air, so the air has to be pumped out of the
      left-hand chamber. The neutrons, Chadwick's mystery particles, penetrate matter far more readily, and fly out through
      the wall and into the chamber on the right, which is filled with
      nitrogen or hydrogen gas. When a neutron collides with a
      nitrogen or hydrogen nucleus, it kicks it out of its atom at
      high speed, and this recoiling nucleus then rips apart thousands of other atoms of the gas. The result is an electrical
      pulse that can be detected in the wire on the right. Physicists
      had already calibrated this type of apparatus so that they
      could translate the strength of the electrical pulse into the
      velocity of the recoiling nucleus. The whole apparatus shown
      in the figure would fit in the palm of your hand, in dramatic
      contrast to today's giant particle accelerators.%
      
    },
    {
      'width'=>'wide'
    }
  )
%>

Suppose, for instance, that we bombard a sample of hydrogen
atoms with the mystery particles. Since the participants in
the collision are fundamental particles, there is no way for
kinetic energy to be converted into heat or any other form
of energy, and Chadwick thus had two equations in three unknowns:

equation \#1: conservation of momentum

equation \#2: no loss of kinetic energy

unknown \#1: mass of the mystery particle

unknown \#2: initial velocity of the mystery particle

unknown \#3: final velocity of the mystery particle

The number of unknowns is greater than the number of
equations, so there is no unique solution. But by creating
collisions with nuclei of another element, nitrogen, he
gained two more equations at the expense of only one more unknown:

equation \#3: conservation of momentum in the new collision

equation \#4: no loss of kinetic energy in the new collision

unknown \#4: final velocity of the mystery particle
in the new collision

He was thus able to solve for all the unknowns, including
the mass of the mystery particle, which was indeed within
1\% of the mass of a proton. He named the new particle the
neutron, since it is electrically neutral.

\startdq

\begin{dq}
Good pool players learn to make the cue ball spin, which can
cause it not to stop dead in a head-on collision with a
stationary ball. If this does not violate the laws of
physics, what hidden assumption was there in the example above?
\end{dq}

<% end_sec() %>
<% end_sec('collisions-1d') %>
<% begin_sec("Relationship of momentum to the center of mass",nil,'p-and-cm',{'optional'=>true}) %>
\index{momentum!related to center of mass}\index{center of mass!related to momentum}

<%
  fig(
    'wrench',
    %q{%
      In this multiple-flash photograph, we
      see the wrench from above as it flies
      through the air, rotating as it goes. Its
      center of mass, marked with the black
      cross, travels along a straight line,
      unlike the other points on the wrench,
      which execute loops.
    },
    {
      'width'=>'wide',
      'sidecaption'=>true
    }
  )
%>

We have already discussed the idea of the center of mass on p.~\pageref{subsec:cm-qualitative},
but using the concept of
momentum we can now find a mathematical method for defining
the center of mass, explain why the motion of an object's
center of mass usually exhibits simpler motion than any
other point, and gain a very simple and powerful way of
understanding collisions.

The first step is to realize that the center of mass concept
can be applied to systems containing more than one object.
Even something like a wrench, which we think of as one
object, is really made of many atoms. The center of mass is
particularly easy to visualize in the case shown on the
left, where two identical hockey pucks collide. It is clear
on grounds of symmetry that their center of mass must be at
the midpoint between them. After all, we previously defined
the center of mass as the balance point, and if the two
hockey pucks were joined with a very lightweight rod whose
own mass was negligible, they would obviously balance at the
midpoint. It doesn't matter that the hockey pucks are two
separate objects. It is still true that the motion of their
center of mass is exceptionally simple, just like that of
the wrench's center of mass.

<% marg(80) %>
<%
  fig(
    'hockey-pucks-lab-frame',
    %q{%
      Two hockey pucks collide. Their
      mutual center of mass traces the straight
      path shown by the dashed line.
    }
  )
%>
<% end_marg %>
The $x$ coordinate of the hockey pucks' center of mass is
thus given by $x_{cm}=(x_1+x_2)/2$, i.e., the arithmetic
average of their $x$ coordinates. Why is its motion so
simple? It has to do with conservation of momentum. Since
the hockey pucks are not being acted on by any net external
force, they constitute a closed system, and their total
momentum is conserved. Their total momentum is
\begin{align*}
 mv_1+mv_2 &= m(v_1+v_2) \\
 &= m \left(\frac{\Delta x_1}{\Delta t}+\frac{\Delta x_2}{\Delta t}\right)\\
 &= \frac{m}{\Delta t}\Delta\left(x_1+x_2\right)\\
 &= m\frac{2\Delta x_{cm}}{\Delta t}\\
 &= m_{total}v_{cm}
\end{align*}
In other words, the total momentum of the system is the same
as if all its mass was concentrated at the center of mass
point. Since the total momentum is conserved, the $x$
component of the center of mass's velocity vector cannot
change. The same is also true for the other components, so
the center of mass must move along a straight line at constant speed.

The above relationship between the total momentum and the
motion of the center of mass applies to any system, even
if it is not closed.

\begin{lessimportant}[total momentum related to center of mass motion]
The total momentum of any system is related to its total mass and the
velocity of its center of mass by the equation
\begin{equation*}
  \vc{p}_{total} = m_{total}\vc{v}_{cm}\eqquad.
\end{equation*}
\end{lessimportant}

What about a system containing objects with unequal masses,
or containing more than two objects? The reasoning above can
be generalized to a weighted average\label{cm-equation}
\begin{equation*}
                x_{cm}  =   \frac{m_1x_1+m_2x_2+\ldots}{m_1+m_2+\ldots}\eqquad,
\end{equation*}
with similar equations for the $y$ and $z$ coordinates.

<% begin_sec("Momentum in different frames of reference") %>

Absolute motion is supposed to be undetectable, i.e., the
laws of physics are supposed to be equally valid in all
inertial frames of reference. If we first calculate some
momenta in one frame of reference and find that momentum is
conserved, and then rework the whole problem in some other
frame of reference that is moving with respect to the first,
the numerical values of the momenta will all be different.
Even so, momentum will still be conserved. All that matters
is that we work a single problem in one consistent frame of reference.

One way of proving this is to apply the equation $\vc{p}_{total}=$\linebreak[4] % Splitting the equation in two fixes an unfortunate line break.
$m_{total}\vc{v}_{cm}$.
If the velocity of frame B relative to frame A is
$\vc{v}_{BA}$, then the only effect of changing frames of
reference is to change $\vc{v}_{cm}$ from its original value to
$\vc{v}_{cm}+\vc{v}_{BA}$. This adds a constant onto the momentum
vector, which has no effect on conservation of momentum.

<% end_sec() %>
<% begin_sec("The center of mass frame of reference") %>
\index{center of mass!frame of reference}

<% marg(5) %>
<%
  fig(
    'hockey-pucks-cm-frame',
    %q{%
      %
      Moving your head so that you are always
      looking down from right above the center of mass, you observe the collision
      of the two hockey pucks in the center
      of mass frame.
    }
  )
%>
<% end_marg %>
A particularly useful frame of reference in many cases is
the frame that moves along with the center of mass, called
the center of mass (c.m.) frame. In this frame, the total
momentum is zero. The following examples show how the center
of mass frame can be a powerful tool for simplifying our
understanding of collisions.

\begin{eg}{A collision of pool balls viewed in the c.m. frame}
If you move your head so that your eye is always above the
point halfway in between the two pool balls, you are viewing
things in the center of mass frame. In this frame, the balls
come toward the center of mass at equal speeds. By symmetry,
they must therefore recoil at equal speeds along the lines
on which they entered. Since the balls have essentially
swapped paths in the center of mass frame, the same must
also be true in any other frame. This is the same result
that required laborious algebra to prove previously without
the concept of the center of mass frame.
\end{eg}

\begin{eg}{The slingshot effect}\index{slingshot effect}
It is a counterintuitive fact that a spacecraft can pick up
speed by swinging around a planet, if it arrives in the
opposite direction compared to the planet's motion. Although
there is no physical contact, we treat the encounter as a
one-dimensional collision, and analyze it in the center of
mass frame. Figure \figref{slingshot-sun-frame} shows such a
``collision,'' with a space probe whipping around Jupiter.
In the sun's frame of reference, Jupiter is moving.

<% marg(25) %>
<%
  fig(
    'slingshot-sun-frame',
    %q{%
      The slingshot effect viewed in the
      sun's frame of reference. Jupiter is
      moving to the left, and the collision is
      head-on.
    }
  )
%>
\spacebetweenfigs
<%
  fig(
    'slingshot-cm-frame',
    %q{%
      The slingshot viewed in the frame
      of the center of mass of the Jupiter-spacecraft system.
    }
  )
%>

<% end_marg %>%

What about the center of mass frame?
Since Jupiter is so much more massive than the
spacecraft, the center of mass is essentially fixed at
Jupiter's center, and Jupiter has zero velocity in the
center of mass frame, as shown in figure \figref{slingshot-cm-frame}. The c.m. frame
is moving to the left compared to the sun-fixed frame used
in \figref{slingshot-sun-frame}, so the spacecraft's initial velocity is greater in this frame.

Things are simpler in the center of mass frame, because it
is more symmetric. In the complicated sun-fixed frame, the incoming leg
of the encounter is rapid, because the two bodies are
rushing toward each other, while their separation on the
outbound leg is more gradual, because Jupiter is trying to
catch up. In the c.m. frame, Jupiter is sitting still, and
there is perfect symmetry between the incoming and outgoing
legs, so by symmetry we have $v_{1f}=-v_{1i}$. Going
back to the sun-fixed frame, the spacecraft's final velocity
is increased by the frames' motion relative to each other.
In the sun-fixed frame, the spacecraft's velocity has increased greatly.

The result can also be understood in terms of work and
energy. In Jupiter's frame, Jupiter is not doing any work on
the spacecraft as it rounds the back of the planet, because
the motion is perpendicular to the force. But in the sun's
frame, the spacecraft's velocity vector at the same moment
has a large component to the left, so Jupiter is doing work on it.
\end{eg}

\startdqs

\begin{dq}
Make up a numerical example of two unequal masses moving
in one dimension at constant velocity, and verify the
equation $p_{total}=m_{total}v_{cm}$ over a time
interval of one second.
\end{dq}

\begin{dq}
A more massive tennis racquet or baseball bat makes the
ball fly off faster. Explain why this is true, using the
center of mass frame. For simplicity, assume that the
racquet or bat is simply sitting still before the collision,
and that the hitter's hands do not make any force large
enough to have a significant effect over the short
duration of the impact.
\end{dq}

<% end_sec() %>

m4_ifelse(__me,1,[:
<% begin_sec("Totally inelastic collisions",0,'totally-inelastic') %>
On p.~\pageref{elastic-and-inelastic} we discussed collisions that were totally elastic (no
conversion of KE into other types of energy). A useful application of the
center of mass frame of reference is to the description of the opposite extreme,
a totally \emph{inelastic} collision.

A totally inelastic collision cannot just be defined
as one in which all the KE is converted into other forms, both because the definition
would depend on our frame of reference and because
there is a constraint imposed by conservation of momentum. Let's say that
a golfer hits a ball. In the frame of reference of the grass,
it would violate conservation of momentum if
the ball were to stay put while the club simply stopped moving. If such a complete
cessation of motion is to happen, then it must occur in the center of mass frame
of reference. In the c.m.~frame, there is zero total momentum both before and after
the collision. Thus if we observe no motion at all after the collision, we must be in the
c.m.~frame. 

Therefore we define a totally inelastic collision as one in which there is no
motion in the c.m.~frame in the final state.\index{collision!totally inelastic}
An observer watching such a collision, in any frame, will see that
the amount of KE transformed into other forms of energy is
as great as possible subject to conservation of momentum.

When objects touch physically (possibly crumpling or changing shape during
the collision) in a totally elastic collision, the final state in the c.m.~frame
is one in which the two objects are at rest and touching. In other frames of reference,
we see the objects stick to each other and travel away together after the collision.
An example of this type was example \ref{eg:rear-ended}
on p.~\pageref{eg:rear-ended}, in which one car rear-ended another, and they stuck
together as a unit after the crash.
<% end_sec() %>
:])

<% end_sec('p-and-cm') %>
<% begin_sec("Momentum transfer",0,'momentum-transfer') %>
\index{momentum!rate of change of}\index{momentum!transfer of}

  <% begin_sec("The rate of change of momentum",nil,'dp-dt') %>

As with conservation of energy, we need a way to measure and
calculate the transfer of momentum into or out of a system
when the system is not closed. In the case of energy, the
answer was rather complicated, and entirely different
techniques had to be used for measuring the transfer of
mechanical energy (work) and the transfer of heat by
conduction. For momentum, the situation is far simpler.

In the simplest case, the system consists of a single object
acted on by a constant external force. Since it is only the
object's velocity that can change, not its mass, the
momentum transferred is
\begin{equation*}
                \Delta\vc{p}         =    m\Delta\vc{v}\eqquad,
\end{equation*}
which with the help of $\vc{a}=\vc{F}/m$ and the constant-acceleration
equation $\vc{a}=\Delta\vc{v}/\Delta t$ becomes
\begin{align*}
 \Delta\vc{p} &= m\vc{a}\Delta t \\
 &= \vc{F}\Delta t\eqquad.
\end{align*}
Thus the rate of transfer of momentum, i.e., the number of
$\kgunit\unitdot\munit/\sunit$ absorbed per second, is simply the external force,
\begin{equation*}
        \vc{F}         =    \frac{\Delta\vc{p}}{\Delta t}\eqquad.
\end{equation*}
\begin{longnoteafterequation}
[relationship between the force on an object and the rate
of change of its momentum; valid only if the force is constant]
\end{longnoteafterequation}
\noindent This is just a restatement of Newton's
second law, and in fact Newton originally stated it this
way. As shown in figure \figref{e-and-p-transfer}, the relationship between force
and momentum is directly analogous to that between power and energy.

<% marg(50) %>
<%
  fig(
    'e-and-p-transfer',
    %q{%
      Power and force are the rates at which
      energy and momentum are transferred.
    }
  )
%>
<% end_marg %>
The situation is not materially altered for a system
composed of many objects. There may be forces between the
objects, but the internal forces cannot change the system's
momentum. (If they did, then removing the external forces would result in
a closed system that could change its
own momentum,  like the mythical man
who could pull himself up by his own bootstraps. That would violate conservation of momentum.) The
equation above becomes
\begin{equation*}
                \vc{F}_{total}         =    \frac{\Delta\vc{p}_{total}}{\Delta t}\eqquad.
\end{equation*}
\begin{longnoteafterequation}
[relationship between the total external force on a system
and the rate of change of its total momentum; valid only if
the force is constant]
\end{longnoteafterequation}


\begin{eg}{Walking into a lamppost}
\egquestion Starting from rest, you begin walking, bringing
your momentum up to $100\ \kgunit\unitdot\munit/\sunit$. You walk straight into a
lamppost. Why is the momentum change of $-100\ \kgunit\unitdot\munit/\sunit$ caused
by the lamppost so much
more painful than the change of $+100\ \kgunit\unitdot\munit/\sunit$ when you started walking?

\eganswer The situation is one-dimensional, so we can
dispense with the vector notation. It probably takes you
about 1 s to speed up initially, so the ground's force on
you is $F=\Delta p/\Delta t\approx 100\ \nunit$. Your impact with the
lamppost, however, is over in the blink of an eye, say 1/10
s or less. Dividing by this much smaller $\Delta t$ gives
a much larger force, perhaps thousands of newtons. (The
negative sign simply indicates that the force is in the
opposite direction.)
\end{eg}

This is also the principle of airbags in cars. The time
required for the airbag to decelerate your head is fairly
long, the time required for your face to travel 20 or 30 cm.
Without an airbag, your face would hit the
dashboard, and the time interval would be the much
shorter time taken by your skull to move a couple of
centimeters while your face compressed. Note that either
way, the same amount of mechanical work has to be done on
your head: enough to eliminate all its kinetic energy.
<% marg(100) %>
<%
  fig(
    'airbag',
    %q{%
      The airbag increases $\Delta t$ so as to reduce $F=\Delta p/\Delta t$.
    }
  )
%>
<% end_marg %>

\begin{eg}{Ion drive for spacecraft}
\egquestion The ion drive of the Deep Space 1 spacecraft,
pictured on page \pageref{fig:ion-drive} and discussed in example
\ref{eg:ion-drive}, produces a thrust of 90 mN
(millinewtons). It carries about 80 kg of reaction mass,
which it ejects at a speed of 30,000 m/s. For how long can
the engine continue supplying this amount of thrust before
running out of reaction mass to shove out the back?

\eganswer Solving the equation $F=\Delta p/\Delta t$ for
the unknown $\Delta t$, and treating force and momentum as
scalars since the problem is one-dimensional, we find
\begin{align*}
 \Delta t &= \frac{\Delta p}{F} \\
 &= \frac{m_{exhaust}\Delta v_{exhaust}}{F} \\
 &= \frac{(80\ \kgunit)(30,000\ \munit/\sunit)}{0.090\ \nunit}\\
 &= 2.7\times10^7\ \sunit \\
 &= 300\ \zu{days}
\end{align*}
\end{eg}

<% marg(65) %>
<%
  fig(
    'toppling-box',
    %q{Example \ref{eg:toppling-box}.}
  )
%>
<% end_marg %>
\begin{eg}{A toppling box}\label{eg:toppling-box}
If you place a box on a frictionless surface, it will fall
over with a very complicated motion that is hard to predict
in detail. We know, however, that its center of mass moves
in the same direction as its momentum vector points. There
are two forces, a normal force and a gravitational force,
both of which are vertical. (The gravitational force is
actually many gravitational forces acting on all the atoms
in the box.) The total force must be vertical, so the
momentum vector must be purely vertical too, and the center
of mass travels vertically. This is true even if the box
bounces and tumbles. [Based on an example by Kleppner and Kolenkow.]
\end{eg}

  <% end_sec('dp-dt') %>
% -------- LM has area stuff here, then a later, optional calc-based section. Me has only the later calc-based section.
m4_ifelse(__me,1,[::],[:
  <% begin_sec("The area under the force-time graph",nil,'integral-of-f-dt') %>

Few real collisions involve a constant force. For example,
when a tennis ball hits a racquet, the strings stretch and
the ball flattens dramatically. They are both acting like
springs that obey Hooke's law, which says that the force is
proportional to the amount of stretching or flattening. The
force is therefore small at first, ramps up to a maximum
when the ball is about to reverse directions, and ramps back
down again as the ball is on its way back out. The equation
$F=\Delta p/\Delta t$, derived under the assumption of
constant acceleration, does not apply here, and the force
does not even have a single well-defined numerical value
that could be plugged in to the equation.

As with similar-looking equations such as
$v=\Delta p/\Delta t$, the equation $F=\Delta p/\Delta t$ is
correctly generalized by saying that the force is the
slope of the $p-t$ graph.
<% marg(0) %>
<%
  fig(
    'tennis-f-t',
    %q{%
      The $F-t$ graph for a tennis racquet hitting a ball
      might look like this. The amount of momentum transferred equals the area under
      the curve.
    }
  )
%>
<% end_marg %>

Conversely, if we wish to find $\Delta p$ from a graph such
as the one in figure \figref{tennis-f-t}, one approach would be to
divide the force by the mass of the ball, rescaling the $F$
axis to create a graph of acceleration versus time. The area
under the acceleration-versus-time graph gives the change in
velocity, which can then be multiplied by the mass to find
the change in momentum. An unnecessary complication was
introduced, however, because we began by dividing by the
mass and ended by multiplying by it. It would have made just
as much sense to find the area under the original $F-t$
graph, which would have given us the momentum change directly.

  <% end_sec('integral-of-f-dt') %>
:])

\startdq

\begin{dq}
Many collisions, like the collision of a bat with a
baseball, appear to be instantaneous. Most people also would
not imagine the bat and ball as bending or being compressed
during the collision. Consider the following possibilities:

\begin{enumerate}
\item The collision is instantaneous.

\item The collision takes a finite amount of time, during
which the ball and bat retain their shapes and remain in contact.

\item The collision takes a finite amount of time, during
which the ball and bat are bending or being compressed.
\end{enumerate}

\noindent How can two of these be ruled out based on energy or
momentum considerations?
\end{dq}

<% end_sec('momentum-transfer') %>
<% begin_sec("Momentum in three dimensions",m4_ifelse(__me,1,4,nil),'momentum-in-3d') %>\index{momentum!examples in three dimensions}

In this section we discuss how the concepts applied
previously to one-dimensional situations can be used as well
in three dimensions. Often vector addition is all that is
needed to solve a problem:

<% marg(0) %>
<%
  fig(
    'kaboom',
    %q{Example \ref{eg:kaboom}.}
  )
%>
<% end_marg %>%

\begin{eg}{An explosion}\label{eg:kaboom}
\egquestion Astronomers observe the planet Mars as the
Martians fight a nuclear war. The Martian bombs are so
powerful that they rip the planet into three separate pieces
of liquified rock, all having the same mass. If one fragment
flies off with velocity components
\begin{align*}
  v_{1x}&=0  \\
  v_{1y}&=1.0\times10^4\ \zu{km}/\zu{hr}\eqquad,
\intertext{and the second with}
  v_{2x}&=1.0\times10^4\ \zu{km}/\zu{hr}\\
  v_{2y}&=0\eqquad,
\end{align*}
(all in the center of mass frame)
what is the magnitude of the third one's velocity?

\eganswer In the center of mass frame,
the planet initially had zero momentum. After the
explosion, the vector sum of the momenta must still be zero.
Vector addition can be done by adding components, so
\begin{align*}
 mv_{1x}+ mv_{2x}+ mv_{3x} &= 0\eqquad, \quad \text{and} \\
 mv_{1y}+ mv_{2y}+ mv_{3y} &= 0\eqquad,
\end{align*}
where we have used the same symbol $m$ for all the terms,
because the fragments all have the same mass. The masses can
be eliminated by dividing each equation by $m$, and we find
\begin{align*}
 v_{3x} &= -1.0\times10^4\ \zu{km}/\zu{hr} \\
 v_{3y} &= -1.0\times10^4\ \zu{km}/\zu{hr}
\end{align*}
which gives a magnitude of
\begin{align*}
 |\vc{v}_3| &= \sqrt{v_{3x}^2+v_{3y}^2}\\
 &= 1.4\times10^4\ \zu{km}/\zu{hr}
\end{align*}
\end{eg}

<% begin_sec("The center of mass") %>

In three dimensions, we have the vector equations
\begin{align*}
                \vc{F}_{total}  &=  \frac{\Delta\vc{p}_{total}}{\Delta t} \\
\intertext{and}
                \vc{p}_{total}  &=  m_{total}\vc{v}_{cm}\eqquad.
\end{align*}
The following is an example of their use.

\begin{eg}{The bola}\label{eg:bola}
The bola, similar to the North American lasso, is used by
South American gauchos to catch small animals by tangling up
their legs in the three leather thongs. The motion of the
whirling bola through the air is extremely complicated, and
would be a challenge to analyze mathematically. The motion
of its center of mass, however, is much simpler. The only
forces on it are gravitational, so
\begin{equation*}
                \vc{F}_{total} =  m_{total}\vc{g}\eqquad.
\end{equation*}
Using the equation $\vc{F}_{total}  =  \Delta\vc{p}_{total}/\Delta t$, we find
\begin{equation*}
                \Delta \vc{p}_{total}/\Delta t   =  m_{total}\vc{g}\eqquad,
\end{equation*}
and since the mass is constant, the equation $\vc{p}_{total}=m_{total}\vc{v}_{cm}$
allows us to change this to
\begin{equation*}
                m_{total}\Delta \vc{v}_{cm}/\Delta t  =  m_{total}\vc{g}\eqquad.
\end{equation*}
The mass cancels, and $\Delta \vc{v}_{cm}/\Delta t$ is simply
the acceleration of the center of mass, so
\begin{equation*}
                \vc{a}_{cm}  =  \vc{g}\eqquad.
\end{equation*}
In other words, the motion of the system is the same as if
all its mass was concentrated at and moving with the center
of mass. The bola has a constant downward acceleration equal
to $g$, and flies along the same parabola as any other
projectile thrown with the same initial center of mass
velocity. Throwing a bola with the correct rotation is
presumably a difficult skill, but making it hit its target
is no harder than it is with a ball or a single rock.

\noindent [Based on an example by Kleppner and Kolenkow.]
\end{eg}

<% marg(100) %>
<%
  fig(
    'bola',
    %q{Example \ref{eg:bola}.}
  )
%>
<% end_marg %>

<% end_sec() %>
<% begin_sec("Counting equations and unknowns") %>
Counting equations and unknowns is just as useful as in one
dimension, but every object's momentum vector has three
components, so an unknown momentum vector counts as three
unknowns. Conservation of momentum is a single vector
equation, but it says that all three components of the total
momentum vector stay constant, so we count it as three
equations. Of course if the motion happens to be confined to
two dimensions, then we need only count vectors as
having two components.

\begin{eg}{A two-car crash with sticking}
Suppose two cars collide, stick together, and skid off
together. If we know the cars' initial momentum vectors, we
can count equations and unknowns as follows:

unknown \#1: $x$ component of cars' final, total momentum

unknown \#2: $y$ component of cars' final, total momentum

equation \#1: conservation of the total $p_x$

equation \#2: conservation of the total $p_y$

\noindent Since the number of equations equals the number of unknowns,
there must be one unique solution for their total momentum
vector after the crash. In other words, the speed and
direction at which their common center of mass moves off
together is unaffected by factors such as whether the cars
collide center-to-center or catch each other a little off-center.
\end{eg}

m4_ifelse(__me,1,[:\vfill:])

\begin{eg}{Shooting pool}
Two pool balls collide, and as before we assume there is no
decrease in the total kinetic energy, i.e., no energy
converted from KE into other forms. As in the previous
example, we assume we are given the initial velocities and
want to find the final velocities. The equations and unknowns are:

unknown \#1: $x$ component of ball \#1's final momentum

unknown \#2: $y$ component of ball \#1's final momentum

unknown \#3: $x$ component of ball \#2's final momentum

unknown \#4: $y$ component of ball \#2's final momentum

equation \#1: conservation of the total $p_x$

equation \#2: conservation of the total $p_y$

equation \#3: no decrease in total KE

Note that we do not count the balls' final kinetic energies
as unknowns, because knowing the momentum vector, one can
always find the velocity and thus the kinetic energy. The
number of equations is less than the number of unknowns, so
no unique result is guaranteed. This is what makes pool an
interesting game. By aiming the cue ball to one side of the
target ball you can have some control over the balls' speeds
and directions of motion after the collision.

It is not possible, however, to choose any combination of
final speeds and directions. For instance, a certain shot
may give the correct direction of motion for the target
ball, making it go into a pocket, but may also have the
undesired side-effect of making the cue ball go in a pocket.
\end{eg}

m4_ifelse(__me,1,[:\vfill:])

<% end_sec() %>
<% begin_sec("Calculations with the momentum vector") %>

The following example illustrates how a force is required in order to
change the direction of the momentum vector, just as one
would be required to change its magnitude.

m4_ifelse(__me,1,[:\pagebreak:])

<% marg(0) %>
<%
  fig(
    'turbine',
    %q{Example \ref{eg:turbine}.}
  )
%>
<% end_marg %>
\begin{eg}{A turbine}\label{eg:turbine}
\egquestion In a hydroelectric plant, water flowing over a dam
drives a turbine, which runs a generator to make electric
power. The figure shows a simplified physical model of the
water hitting the turbine, in which it is assumed that the
stream of water comes in at a 45\degunit angle with respect to
the turbine blade, and bounces off at a 90\degunit angle at
nearly the same speed. The water flows at a rate $R$, in
units of kg/s, and the speed of the water is $v$. What are
the magnitude and direction of the water's force on the turbine?

\eganswer In a time interval $\Delta $t, the mass of water
that strikes the blade is $R\Delta $t, and the magnitude of
its initial momentum is $mv=vR\Delta t$. The
water's final momentum vector is of the same magnitude, but
in the perpendicular direction. By Newton's third law, the
water's force on the blade is equal and opposite to the
blade's force on the water. Since the force is constant, we
can use the equation
\begin{equation*}
                F_{\text{blade on water}}         =    \frac{\Delta \vc{p}_{water}}{\Delta t}\eqquad.
\end{equation*}
Choosing the $x$ axis to be to the right and the $y$ axis to
be up, this can be broken down into components as
\begin{align*}
 F_{\text{blade on water},x} &= \frac{\Delta p_{\text{water},x}}{\Delta t} \\
 &= \frac{-vR\Delta t-0}{\Delta t} \\
 &= -vR
\end{align*}
and
\begin{align*}
 F_{\text{blade on water},y} &= \frac{\Delta p_{\text{water},y}}{\Delta t} \\
 &= \frac{0-(-vR\Delta t)}{\Delta t} \\
 &= vR\eqquad.
\end{align*}
The water's force on the blade thus has components
\begin{align*}
 F_{\text{water on blade},x} &= vR \\
 F_{\text{water on blade},y} &= -vR\eqquad.
\end{align*}
In situations like this, it is always a good idea to check
that the result makes sense physically. The $x$ component of
the water's force on the blade is positive, which is correct
since we know the blade will be pushed to the right. The $y$
component is negative, which also makes sense because the
water must push the blade down. The magnitude of the water's
force on the blade is
\begin{equation*}
                |F_{\text{water on blade}}|    =    \sqrt{2}vR
\end{equation*}
and its direction is at a 45-degree angle down and to the right.
\end{eg}

\startdqs
\begin{dq}
The figures show a jet of water striking two different
objects. How does the total downward force compare in the
two cases? How could this fact be used to create a better
waterwheel? (Such a waterwheel is known as a Pelton wheel.)\label{dq:pelton}
\end{dq}

<%
  fig(
    'dq-pelton',
    %q{Discussion question \ref{dq:pelton}.},
    {
      'width'=>'wide',
      'anonymous'=>true
    }
  )
%>

<% end_sec() %>
<% end_sec('momentum-in-3d') %>
<% begin_sec("Applications of calculus",3,'momentum-calc-applications',{'calc'=>true}) %>
% -------- LM has area stuff earlier, and now this optional calc-based section. Me has only the this calc-based section.
m4_ifelse(__me,1,[:
Few real collisions involve a constant force. For example,
when a tennis ball hits a racquet, the strings stretch and
the ball flattens dramatically. They are both acting like
springs that obey Hooke's law, which says that the force is
proportional to the amount of stretching or flattening. The
force is therefore small at first, ramps up to a maximum
when the ball is about to reverse directions, and ramps back
down again as the ball is on its way back out. The equation
$F=\Delta p/\Delta t$, derived under the assumption of
constant acceleration, does not apply here, and the force
does not even have a single well-defined numerical value
that could be plugged in to the equation.

This is like every other situation where an equation of the form
$foo=\Delta bar/\Delta baz$ has to be generalized to the case
where the rate of change isn't constant. We have
$F=\der p/\der t$ and, by the fundamental theorem of calculus,
$\Delta p = \int F\der t$, which can be interpreted as the area
under the $F-t$ graph,  figure \figref{tennis-f-t}.
<% marg(0) %>
<%
  fig(
    'tennis-f-t',
    %q{%
      The $F-t$ graph for a tennis racquet hitting a ball
      might look like this. The amount of momentum transferred equals the area under
      the curve.
    }
  )
%>
<% end_marg %>
:],[:
By now you will have learned to recognize the circumlocutions
I use in the sections without calculus in order to introduce
calculus-like concepts without using the notation,
terminology, or techniques of calculus. It will therefore
come as no surprise to you that the rate of change of
momentum can be represented with a derivative,
\begin{equation*}
 \vc{F}_{total} = \frac{\der \vc{p}_{total}}{\der t}\eqquad.
\end{equation*}
And of course the business about the area under the $F-t$
curve is really an integral, $\Delta p_{total}=\int F_{total}\der t$, which can be made
into an integral of a vector in the more general three-dimensional case:
\begin{equation*}
 \Delta \vc{p}_{total} = \int \vc{F}_{total}\der t\eqquad.
\end{equation*}
In the case of a material object that is neither losing nor
picking up mass, these are just trivially rearranged
versions of familiar equations, e.g., $F=m\der v/\der t$ rewritten as $F=\der(mv)/\der t$. The
following is a less trivial example, where $F=ma$
alone would not have been very easy to work with.
:])

\begin{eg}{Rain falling into a moving cart}
\egquestion If 1 kg/s of rain falls vertically into a 10-kg
cart that is rolling without friction at an initial speed of
1.0 m/s, what is the effect on the speed of the cart when
the rain first starts falling?

\eganswer The rain and the cart make horizontal forces on
each other, but there is no external horizontal force on the
rain-plus-cart system, so the horizontal motion obeys
\begin{equation*}
        F = \frac{\der(mv)}{\der t}         =  0
\end{equation*}
We use the product rule to find
\begin{equation*}
 0 = \frac{\der m}{\der t}v + m\frac{\der v}{\der t}\eqquad.
\end{equation*}
We are trying to find how $v$ changes, so we solve for $dv/dt$,
\begin{align*}
 \frac{\der v}{\der t}&= -\frac{v}{m}\frac{\der m}{\der t} \\
 &= -\left(\frac{1\ \munit/\sunit}{10\ \kgunit}\right)\left(1\ \kgunit/\sunit\right)\\
 &= -0.1\ \munit/\sunit^2\eqquad.
\end{align*}
(This is only at the moment when the rain starts to fall.)
\end{eg}

Finally we note that there are cases where $F=ma$ is
not just less convenient than $F=\der p/\der t$ but in fact
$F=ma$ is wrong and $F=\der p/\der t$ is right. A good
example is the formation of a comet's tail by sunlight. We
cannot use $F=ma$ to describe this process, since we
are dealing with a collision of light with matter, whereas
Newton's laws only apply to matter. The equation $F=\der p/\der t$,
on the other hand, allows us to find the force experienced
by an atom of gas in the comet's tail if we know the rate at
which the momentum vectors of light rays are being turned
around by reflection from the atom.

<% end_sec('momentum-calc-applications') %>
