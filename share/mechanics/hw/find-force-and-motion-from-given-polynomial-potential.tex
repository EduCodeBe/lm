The potential energy of a particle moving
in a certain one-dimensional region of space is
\begin{multline*}
  U(x) = (1.00\ \junit/\munit^3) x^3 - (7.00\ \junit/\munit^2) x^2 \\ + (10.0\ \junit/\munit) x.
\end{multline*}
%
(a) Determine the force $F(x)$ acting on the particle as a function of
position.\hwendpart
%
(b) Is the force you found in part a conservative, or non-conservative?
Explain.\hwendpart
%
(c) Let ``$\mathcal{R}$'' refer to the region $x = -1.00\ \munit$
to $x= + 6.00\ \munit$.
Draw $U(x)$ on $\mathcal{R}$ (label your axes). On your plot,
label all points of stable and unstable equilibrium on $\mathcal{R}$,
and find their locations.\hwendpart
%
(d) What is the maximum force (in magnitude) experienced by a
particle on $\mathcal{R}$?\hwendpart
%
(e) The particle has mass $1.00\ \kgunit$ and is released from rest at
$x=2.00\ \munit$. Describe the subsequent motion. What is the maximum
KE that the particle achieves?
