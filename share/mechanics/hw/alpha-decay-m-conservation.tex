An atom of the most common naturally occurring uranium isotope breaks up spontaneously
        into a thorium atom plus a helium atom. The masses are as follows:

        \begin{tabular}{lr}
                uranium        & $3.95292849\times 10^{-25}\ \kgunit$ \\
                thorium & $3.88638748\times 10^{-25}\ \kgunit$ \\
                helium & $6.646481\times 10^{-27}\ \kgunit$ \\
        \end{tabular}

        \noindent{}Each of these experimentally determined masses is uncertain in its last decimal
        place. Is mass conserved in this process to within the accuracy of the experimental
        data? How would you interpret this?
