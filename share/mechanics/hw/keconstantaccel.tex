A car starts from rest at $t=0$, and starts speeding
up with constant acceleration. (a) Find the car's kinetic
energy in terms of its mass, $m$, acceleration, $a$, and the
time, $t$. (b) Your answer in the previous part also equals
the amount of work, $W$, done from $t=0$ until time $t$.
Take the derivative of the previous expression to find the
power expended by the car at time $t$. (c) Suppose two cars
with the same mass both start from rest at the same time,
but one has twice as much acceleration as the other. At any
moment, how many times more power is being dissipated by the
more quickly accelerating car? (The answer is not 2.)
\answercheck
