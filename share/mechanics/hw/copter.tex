  A helicopter of mass $m$ is taking off vertically. The
only forces acting on it are the earth's gravitational force
and the force, $F_{air}$, of the air pushing up on the
propeller blades.\hwendpart
 %
 (a) If the helicopter lifts off at $t=0$,
what is its vertical speed at time $t$?\hwendpart
 (b) Check that the units of your answer to part a make sense.\hwendpart
(c) Discuss how your answer to part a depends on all three
variables, and show that it makes sense. That is, for each
variable, discuss what would happen to the result if you
changed it while keeping the other two variables constant.
Would a bigger value give a smaller result, or a bigger
result? Once you've figured out this \emph{mathematical}
relationship, show that it makes sense \emph{physically}.\hwendpart
 (d) Plug numbers
into your equation from part a, using $m=2300$ kg,
$F_{air}=27000\ \nunit$, and $t=4.0\ \sunit$.
\answercheck
