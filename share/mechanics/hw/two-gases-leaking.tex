At a given
temperature, the average kinetic energy per molecule is a fixed value,
so for instance in air, the more massive oxygen molecules are moving more slowly
on the average than the nitrogen molecules. The ratio of the masses of
oxygen and nitrogen molecules is 16.00 to 14.01. Now suppose
a vessel containing some air is surrounded by a vacuum, and the vessel has
a tiny hole in it, which allows the air to slowly leak out. The molecules
are bouncing around randomly, so a given molecule will have to ``try'' many
times before it gets lucky enough to head out through the hole. 
Find the rate at which oxygen leaks divided by the rate at which nitrogen leaks.
(Define this rate according to the fraction of the gas that leaks out in a given
time, not the mass or number of molecules leaked per unit time.)\answercheck
