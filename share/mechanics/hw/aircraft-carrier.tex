Aircraft carriers originated in World War I, and the first landing on a carrier was performed by
E.H.~Dunning in a Sopwith Pup biplane, landing on HMS Furious. (Dunning was killed the second time he
attempted the feat.) In such a landing, the pilot slows down to just above the plane's stall speed,
which is the minimum speed at which the plane can fly without stalling. The plane then lands and is caught by
cables and decelerated as it travels the length of the flight deck.
Comparing a modern US F-14 fighter
jet landing on an Enterprise-class carrier to Dunning's original exploit, the stall speed is greater
by a factor of 4.8, and to accomodate this, the length of the flight deck is greater by a factor of 1.9.
Which deceleration is greater, and by what factor?\answercheck
