Most modern bow hunters in the U.S. use a fancy mechanical bow called a
compound bow, which looks nothing like what most people imagine when they
think of a bow and arrow. It has a system of pulleys designed to produce
the force curve shown in the figure, where $F$ is the force required to
pull the string back, and $x$ is the distance between the string and the
center of the bow's body. It is not a linear Hooke's-law graph, as it
would be for an old-fashioned bow. The big advantage of the design is
that relatively little force is required to hold the bow stretched to
point B on the graph. This is the force required from the hunter in order
to hold the bow ready while waiting for a shot. Since it may be necessary
to wait a long time, this force can't be too big. An old-fashioned bow,
designed to require the same amount of force when fully drawn, would
shoot arrows at much lower speeds, since its graph would be a straight line
from A to B. For the graph shown in the figure (taken from realistic
data), find the speed at which a 26 g arrow is released, assuming that
70\% of the mechanical work done by the hand is actually transmitted
to the arrow. (The other 30\% is lost to frictional heating inside the
bow and kinetic energy of the recoiling and vibrating bow.)
\answercheck
