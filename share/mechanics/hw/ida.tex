<% hw_solution %> The figure shows an image from the Galileo space
probe taken during its August 1993 flyby of the asteroid
Ida. Astronomers were surprised when Galileo detected a
smaller object orbiting Ida. This smaller object, the only
known satellite of an asteroid in our solar system, was
christened Dactyl, after the mythical creatures who lived on
Mount Ida, and who protected the infant Zeus. For scale, Ida
is about the size and shape of Orange County, and Dactyl the
size of a college campus. Galileo was unfortunately unable
to measure the time, $T$, required for Dactyl to orbit Ida.
If it had, astronomers would have been able to make the
first accurate determination of the mass and density of an
asteroid. Find an equation for the density, $\rho $, of Ida
in terms of Ida's known volume, $V$, the known radius, $r$,
of Dactyl's orbit, and the lamentably unknown variable $T$.
(This is the same technique that was used successfully for
determining the masses and densities of the planets that have moons.)
