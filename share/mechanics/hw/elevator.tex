Today's tallest buildings are really not that much
taller than the tallest buildings of the 1940's. One big
problem with making an even taller skyscraper is that every
elevator needs its own shaft running the whole height of the
building. So many elevators are needed to serve the
building's thousands of occupants that the elevator shafts
start taking up too much of the space within the building.
An alternative is to have elevators that can move both
horizontally and vertically: with such a design, many
elevator cars can share a few shafts, and they don't get in
each other's way too much because they can detour around
each other. In this design, it becomes impossible to hang
the cars from cables, so they would instead have to ride on
rails which they grab onto with wheels. Friction would keep
them from slipping. The figure shows such a frictional
elevator in its vertical travel mode. (The wheels on the
bottom are for when it needs to switch to horizontal
motion.)\hwendpart
 %
 (a) If the coefficient of static friction between
rubber and steel is $\mu_s$, and the maximum mass of the car
plus its passengers is $M$, how much force must there be
pressing each wheel against the rail in order to keep the
car from slipping? (Assume the car is not accelerating.)\answercheck\hwendpart
 %
 (b)
Show that your result has physically reasonable behavior
with respect to $\mu_s$. In other words, if there was less
friction, would the wheels need to be pressed more firmly or
less firmly? Does your equation behave that way?
