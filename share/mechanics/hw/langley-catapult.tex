At the turn of the 20th century, Samuel Langley engaged in a bitter rivalry with
the Wright brothers to develop human flight. Langley's design used a catapult for launching. For safety,
the catapult was built on the roof of a houseboat, so that any crash would be into the water.
This design required reaching cruising speed within a fixed, short distance, so large
accelerations were required, and the forces frequently damaged the craft, causing dangerous
and embarrassing accidents. Langley achieved several uncrewed, unguided flights,
but never succeeded with a human pilot. If the force of the catapult is fixed by the
structural strength of the plane, and the distance for acceleration by the size of the
houseboat, by what factor is the launch velocity reduced when the plane's
340 kg is augmented by the 60 kg mass of a small man?\answercheck
