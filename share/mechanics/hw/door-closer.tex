% meta {"stars":1}
Sometimes doors are built with mechanisms that automatically close them after
they have been opened. The designer can set both the strength of the spring
and the amount of friction. If there is too much friction in relation to the strength of the spring, the door takes
too long to close, but if there is too little, the door will oscillate.
For an optimal design, we get motion of the form $x=ct e^{-bt}$, where $x$ is the position of some point on the door,
and $c$ and $b$ are positive constants. (Similar systems are used for other mechanical devices, such as stereo
speakers and the recoil mechanisms of guns.) In this example, the door moves in the positive direction up until
a certain time, then stops and settles back in the negative direction, eventually approaching $x=0$. This would be
the type of motion we would get if someone flung a door open and the door closer then brought it back closed again.
(a) Infer the units of the constants $b$ and $c$.\hwendpart
(b) Find the door's maximum speed (i.e., the greatest absolute value of its velocity) as it 
comes back to the closed position.\answercheck\hwendpart
(c) Show that your answer has units that make sense.
