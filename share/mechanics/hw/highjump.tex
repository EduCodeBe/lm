In the 1964 Olympics in Tokyo, the best men's high jump
was 2.18 m. Four years later in Mexico City, the gold
medal in the same event was for a jump of 2.24 m. Because
of Mexico City's altitude (2400 m), the acceleration of
gravity there is lower than that in Tokyo by about
$0.01\ \munit/\sunit^2$. Suppose a high-jumper has a mass of 72 kg.\hwendpart
 %
(a) Compare his mass and weight in the two locations.\hwendpart
 %
(b) Assume that he is able to jump with the same initial
vertical velocity in both locations, and that all other
conditions are the same except for gravity. How much higher
should he be able to jump in Mexico City?\answercheck\hwendpart
 %
(Actually, the reason for the big change between '64 and '68
was the introduction of the ``Fosbury flop.'')
