A particle is undergoing uniform circular motion in the $xy$-plane
such that its distance to the origin does not change. At one instant,
the particle is moving with velocity 
$\vc{v} = (10.0\ \munit/\sunit) \hat{\vc{x}}$
and with acceleration $\vc{a} = (2.0\ \munit/\sunit^2) \hat{\vc{y}}$.\\
%
(a) What is the radius of the circle?\answercheck\hwendpart
%
(b) How long does it take for the particle to travel once around the
circle (i.e., what is the period of motion)?\answercheck\hwendpart
%
(c) In what direction will the particle be moving after a quarter of a
period? Give your answer as an angle $\theta$ ($0 \leq \theta <
360\degunit$) that the velocity vector makes with respect to the
$+\hat{\vc{x}}$ direction, measured counterclockwise from the $+\hat{\vc{x}}$
direction.\answercheck
