In metals, some electrons, called conduction electrons, are free to move around, rather than being bound to one
atom. Classical physics gives an adequate description of many of their properties.
Consider a metal at temperature $T$, and let $m$ be the mass of the electron.
Find expressions for (a) the average kinetic energy of a conduction electron,
and (b) the average square of its velocity, $\overline{v^2}$. (It would not be of much interest
to find $\overline{v}$, which is just zero.) Numerically, $\sqrt{\overline{v^2}}$, called the root-mean-square
velocity, comes out to be
surprisingly large --- about two orders of magnitude greater than the normal
thermal velocities we find for atoms in a gas. Why?
\answercheck
\hwremark{From this analysis, one would think that the conduction electrons would contribute greatly
to the heat capacities of metals. In fact they do not contribute very much in most cases; if they did,
Dulong and Petit's observations would not have come out as described in the text. The resolution of
this contradiction was only eventually worked out by Sommerfeld in 1933, and involves the fact that
electrons obey the Pauli exclusion principle.\index{Dulong-Petit law}\index{Sommerfeld, Arnold}\index{specific heat!electrons' contribution}}
