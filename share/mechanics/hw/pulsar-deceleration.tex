Neutron stars are the collapsed remnants of dead stars.
They rotate quickly, and their rotation can be measured extremely
accurately by radio astronomers. Some of them rotate at such a predictable
rate that they can be used to count time about as accurately as the
best atomic clocks. They do decelerate slowly, but this
deceleration can be taken into account. One of the best-studied
stars of this type\footnote{Verbiest \emph{et al.}, Astrophysical Journal 679 (675) 2008}
was observed continuously over a 10-year period. As of
the benchmark date April 5, 2001, it was found to have
\begin{align*}
  \omega &= 1.091313551502333\times10^{3}\ \sunit^{-1} \\
\intertext{and}
  \alpha &= -1.085991\times10^{-14}\ \sunit^{-2},
\end{align*}
where the error bars in the final digit of each number are about $\pm 1$.
Astronomers often use the Julian year as their unit of time, where one
Julian year is defined to be exactly $3.15576\times10^7\ \sunit$. Find the
number of revolutions that this pulsar made over a period of 10 Julian
years, starting from the benchmark date.\answercheck

% http://iopscience.iop.org/article/10.1086/529576
% not paywalled
% P=5.757451924362137 ms
% Pdot=5.729370 10^-20
% reference epoch = 52005 (JD(?))
%   = when j --now="2001 apr 5"
% calc -e "P=5.757451924362139 ms; Pdot=5.729370 10^-20; omega=2pi/P->s-1; alpha=-(1/(2pi))Pdotomega^2->s-2"
%    omega = 1.09131355150233*10^3 s-1
%    alpha = -1.08599067863924*10^-14 s-2
% omega calculated using WA
%   1.0913135515023333
%   with error bars of 2 in last digit of P =>
%   1.0913135515023327
% error bars on alpha:
%   calc -e "P=5.757451924362139 ms; Pdot=5.729372 10^-20; omega=2pi/P->s-1; alpha=-(1/(2pi))Pdotomega^2->s-2"
%    alpha = -1.08599105773526*10^-14 s-2
