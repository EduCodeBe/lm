 A skier of mass $m$ is coasting down a slope inclined at
an angle $\theta $ compared to horizontal. Assume for
simplicity that the treatment of kinetic friction given in
chapter 5 is appropriate here, although a soft and wet
surface actually behaves a little differently. The
coefficient of kinetic friction acting between the skis and
the snow is $\mu_k$, and in addition the skier experiences
an air friction force of magnitude $bv^2$, where $b$ is a
constant.\hwendpart
 %
 (a) Find the maximum speed that the skier will
attain, in terms of the variables $m$, $g$, $\theta$, $\mu_k$, and
$b$.\answercheck\hwendpart
 %
 (b) For angles below a certain minimum angle $\theta_{min}$,
the equation gives a result that is not mathematically
meaningful. Find an equation for $\theta_{min}$, and give a
physical explanation of what is happening for $\theta <\theta_{min}$.\answercheck\hwendpart
