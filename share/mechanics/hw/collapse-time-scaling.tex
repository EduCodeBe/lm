The structures that we see in the universe, such as solar systems,
galaxies, and clusters of galaxies, are believed to have condensed
from clumps that formed, due to gravitational attraction, in preexisting
clouds of gas and dust. Observations of the cosmic microwave background
radiation m4_ifelse(__problems,1,[::],[:(p.~\pageref{cmb}):]) suggest that the mixture of hot
hydrogen and helium that existed soon after the Big Bang was extremely
uniform, but not perfectly so. We can imagine that any region that started
out a little more dense would form a natural center for the collapse of
a clump. Suppose that we have a spherical region with density $\rho$
and radius $r$, and for simplicity let's just assume that it's surrounded
by vacuum. (a) Find the acceleration of the material at  the edge of the
cloud. To what power of $r$ is it proportional? 
\answercheck\hwendpart
(b) The cloud will take
a time $t$ to collapse to some fraction of its original size.
Show that $t$ is independent of $r$.

\hwremark{This result suggests that structures would get a chance to
form at all scales in the universe. That is, solar systems would not form
before galaxies got to, or vice versa. It is therefore physically natural
that when we look at the universe at essentially all scales less than
a billion light-years, we see structure.}
