A car accelerates from rest. At low speeds, its acceleration is limited
by static friction, so that if we press too hard on the gas, we will
``burn rubber'' (or, for many newer cars, a computerized traction-control
system will override the gas pedal). At higher speeds, the limit on
acceleration comes from the power of the engine, which puts a limit on
how fast kinetic energy can be developed.\\
(a) Show that if a force $F$ is applied to an object moving at
speed $v$, the power required is given by $P=vF$.\hwendpart
(b) Find the speed $v$ at which we cross over from the first regime
described above to the second. At speeds higher than this, the engine
does not have enough power to burn rubber. Express your result in terms of the
car's power $P$, its mass $m$, the  coefficient of static friction $\mu_s$,
and $g$.\answercheck\hwendpart
(c) Show that your answer to part b has units that make sense.\hwendpart
(d) Show that the dependence of your answer on each of the four variables
makes sense physically.\hwendpart
(e) The 2010 Maserati Gran Turismo Convertible has a maximum power of
$3.23\times10^5\ \zu{W}$ (433 horsepower) and a mass (including a 50-kg driver)
of $2.03\times10^3\ \kgunit$. (This power is the maximum the engine can supply
at its optimum frequency of 7600 r.p.m. Presumably the automatic transmission is designed so a gear is available
in which the engine will be running at very nearly this frequency when
the car is moving at $v$.) Rubber on asphalt has $\mu_s\approx0.9$.
Find $v$ for this car. Answer: $18\ \munit/\sunit$, or about 40 miles per hour.\hwendpart
(f) Our analysis has neglected air friction, which can probably be approximated
as a force proportional to $v^2$. The existence of this force is the reason that the
car has a maximum speed, which is 176 miles per hour. To get a feeling for how good an approximation
it is to ignore air friction, find what fraction of the engine's
maximum power is being used to overcome air resistance when the car is moving at
the speed $v$ found in part e. Answer: 1\%
