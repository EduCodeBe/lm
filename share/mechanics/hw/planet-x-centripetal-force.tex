Planet X rotates, as the earth does, and is perfectly spherical. An
astronaut who weighs $980.0\ \nunit$ on the earth steps on a scale at
the north pole of Planet X and the scale reads $600.0\ \nunit$; at
the equator of Planet X, the scale only reads $500.0\ \nunit$. The
distance from the north pole to the equator is 20,000 km, measured
along the surface of Planet X.\\
%
(a) Explain why the astronaut appears to weigh more at the north pole
of planet X than at the equator. Which is the ``actual''
weight of the astronaut? Analyze the forces acting on an astronaut
standing on a scale, providing one analysis for the north pole, and one for the
equator.\hwendpart
%
(b) Find the mass of planet X. Is planet X more massive than the earth, or less
massive? The radius of the earth is 6370 km, and its mass
is $5.97 \times 10^{24}\ \kgunit$.\answercheck\hwendpart
%
(c) If a 30,000 kg satellite is orbiting the planet very close to the
surface, what is its orbital period? Assume planet X has no
atmosphere, so that there's no air resistance.\answercheck\hwendpart
%
(d) How long is a day on planet X? Is this longer than, or shorter
than, the period of the satellite in part c?\answercheck\hwendpart
