% meta {"stars":0}
You're in your Honda on the freeway traveling
behind a Ford pickup truck. The truck is moving at a steady speed of
$30.0\ \munit/\sunit$, you're speeding at $40.0\ \munit/\sunit$, and you're cruising 45
meters behind the Ford. At $t=0$, the Ford slams on his/her brakes,
and decelerates at a rate of $5.0\ \munit/\sunit^2$. You don't notice this
until $t=1.0\ \sunit$, where you begin decelerating at $10.0\ \munit/\sunit^2$.
Let positive $x$ be the direction of motion, and let your position be
$x=0$ at $t=0$. The goal is to find the motion of each vehicle and determine
whether there is a collision.\\
%
(a) Doing this entire calculation purely numerically would be very cumbersome,
and it would be difficult to tell whether you had made mistakes. Translate the
given information into algebra symbols, and
find an equation for $x_\text{F}(t)$, the position of the Ford as a function of time.\answercheck\hwendpart
%
(b) Write a similar symbolic equation for $x_\text{H}(t)$ (for $t > 1\ \sunit$), the
position of the Honda as a function of time. Why isn't this
formula valid for $t < 1\ \sunit$?\answercheck\hwendpart
%
(c) By subtracting one from the other, find an expression for the
distance between the two vehicles as a function of time, $d(t)$
(valid for $t > 1\ \sunit$ until the truck stops). Does the equation
$d(t) = 0$ have any solutions? What does this tell you?\answercheck\hwendpart
%
(d) Because this is a fairly complicated calculation, we will find
the result in two different ways and check them against each other.
Plug numbers back in to the results of parts a and b, replacing the symbols in the constant coefficients,
and graph the two functions using a graphing calculator or an online utility such as desmos.com.
%
(e) As you should have discovered in parts c and d, the two
vehicles do not collide. At what time does the minimum distance occur,
and what is that distance?\answercheck
