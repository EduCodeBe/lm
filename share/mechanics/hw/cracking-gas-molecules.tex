A sample of gas is enclosed in a sealed chamber. The gas consists of
molecules, which are then split in half through some process such as
exposure to ultraviolet light, or passing an electric spark through the
gas. The gas returns to the same temperature as the surrounding
room, but the molecules remain split apart, at least for some amount
of time. (To achieve these conditions, we would need an extremely dilute gas.
Otherwise the recombination of the molecules would be faster than the cooling
down to the same temperature as the room.) How does its pressure now compare with its pressure before the
molecules were split?
