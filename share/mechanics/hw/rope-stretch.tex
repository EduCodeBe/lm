The rock climber in the figure has mass $m$ and is on a slope $\theta$ above the
horizontal. At a distance $x$ down the slope below him is a ledge. He is tied in
to a climbing rope and being belayed from above, so that if he slips he won't
simply plunge to his death. Climbing ropes are intentionally made out of stretchy
material so that in a fall, the climber gets a gentle catch rather than a violent
force that would
m4_ifelse(__sn,1,[:hurt.:],[:hurt (see example \ref{eg:fall-factor}, p.~\pageref{eg:fall-factor}).:])
However, the rope
should not be more stretchy than necessary because of situations like this one:
if the rope were to stretch by more than $x$, the climber would hit the ledge.\\
(a) Find the spring constant that the rope should have in order to limit the
amount of rope stretch to $x$.\answercheck\hwendpart
(b) Show that your answer to part a has the right units.\hwendpart
(c) Analyze the mathematical dependence of the result on each of the variables, and verify that
it makes sense physically.\hwendpart

