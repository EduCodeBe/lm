The following are the three most common ways in which gamma rays interact with matter:

\emph{Photoelectric effect:} The gamma ray hits an electron, is annihilated, and gives all of its energy to the electron.

\emph{Compton scattering:} The gamma ray bounces off of an electron,
exiting in some direction with some amount of energy.\index{Compton scattering}

\emph{Pair production:} The gamma ray is annihilated, creating an
electron and a positron.\index{pair production}

\noindent Example \ref{eg:no-pair-prod-in-vacuum} on p.~\pageref{eg:no-pair-prod-in-vacuum} shows
that pair production can't occur in a vacuum due to conservation of the energy-momentum
four-vector. What about the other two processes? Can the photoelectric effect occur without the
presence of some third particle such as an atomic nucleus? Can Compton scattering happen
without a third particle?
