\index{momentum!of light}\index{light!momentum of}\index{electromagnetic wave!momentum of}
An object moving at a speed very close to the speed of light is referred to as
ultrarelativistic. Ordinarily (luckily) the only ultrarelativistic objects
in our universe are subatomic particles, such as cosmic rays or particles
that have been accelerated in a particle accelerator.\hwendpart
(a) What kind of number is $\mygamma$ for an ultrarelativistic particle?\hwendpart
(b) Repeat example \ref{eg:massenergy-low-speed} on page \pageref{eg:massenergy-low-speed},
but instead of very low, nonrelativistic speeds, consider ultrarelativistic speeds.\hwendpart
(c) Find an equation for the ratio $\massenergy/p$. The speed may be relativistic, but don't
assume that it's ultrarelativistic.\answercheck\hwendpart
(d) Simplify your answer to part c for the case where the speed is ultrarelativistic.\answercheck\hwendpart
(e) We can think of a beam of light as an ultrarelativistic object --- it certainly moves at a speed
that's sufficiently close to the speed of light! Suppose you turn on a one-watt flashlight, leave it
on for one second, and then turn it off. Compute the momentum of the recoiling flashlight, in units
of $\kgunit\unitdot\munit/\sunit$. (Cf.~p.~\pageref{fig:maxwellian-momentum-of-light}.)\answercheck\hwendpart
(f) Discuss how your answer in part e relates to the correspondence principle.
