As discussed in
m4_ifelse(__sn,1,[:chapter \ref{ch:waves}:],[:m4_ifelse(__lm,1,[:chapter \ref{ch:free-waves}:],[:section \ref{sec:waves-on-a-string}:]):]), the speed at which a disturbance travels along
a string under tension is given by $v=\sqrt{T/\mu}$, where $\mu$ is the mass per unit
length, and $T$ is the tension.\\
 (a) Suppose a string has a density $\rho$, and a cross-sectional
area $A$. Find an expression for the maximum tension that could possibly exist in the string
without producing $v>c$, which is impossible according to relativity. Express your answer in
terms of $\rho$, $A$, and $c$. The interpretation is that relativity puts a limit on how
strong any material can be.\answercheck\hwendpart
(b) Every substance has a tensile strength, defined as the force
per unit area required to break it by pulling it apart. The tensile strength is measured in
units of $\zu{N}/\munit^2$, which is the same as the pascal (Pa), the mks unit of pressure.
Make a numerical estimate of the maximum tensile strength allowed by relativity in the case where
the rope is made out of ordinary matter, with a density on the same order of magnitude as
that of water. (For comparison, kevlar has a tensile strength of about $4\times10^9$ Pa,
and there is speculation that fibers made from carbon nanotubes could have
values as high as  $6\times10^{10}$ Pa.)\answercheck\hwendpart
(c) A black hole is a star that has collapsed and become very dense, so that
its gravity is too strong for anything ever to escape from it. For instance, the escape
velocity from a black hole is greater than $c$, so a projectile can't be shot out of it.
Many people, when they hear this description of a black hole in terms of an escape velocity,
wonder why it still wouldn't be possible to extract an object from a black
hole by other means.
For example, suppose we lower an astronaut into a black hole on a rope, and then pull him
back out again. Why might this not work?\hwendpart
