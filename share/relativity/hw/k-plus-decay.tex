Radiocative particle a decays, annihilating itself and producing two particles b and c, of unequal mass.
Consider this process in the frame of reference in which particle a was at rest before the decay.\\
(a) In the special case where very little energy is released in the decay, and particles b and c have
nonrelativistic speeds, prove using classical physics that the particle with the lower mass must have
the higher kinetic energy.\hwendpart
(b) Find an expression for the mass-energy $E_c$ of particle c, in terms of the masses $m_a$, $m_b$, and $m_c$.
Hint: work in natural units, and make use of the result of problem \ref{hw:rel-epm}a.\answercheck\hwendpart
(c) Show that the units of your answer make sense.\hwendpart
(d) Show that your expression has the correct behavior in the case of $m_b=m_c$.\hwendpart
(e) A process of this type is the decay of a $\zu{K}^+$ particle into a $\pi^+$ and a $\pi_0$ (called pions).
The masses are 493.7, 139.6, and 135.0 MeV, respectively. (MeV are a unit of energy, but in natural units, they
can also be a unit of mass.) Find the mass-energies and kinetic energies of the two pions, and verify that
the nonrelativistic prediction of part (a) is still correct, even in the fully relativistic case.
