We can't go faster than the speed of light,
but because of relativistic time dilation, it is theoretically possible to travel to
arbitrarily distant parts of the galaxy within a human lifetime.\\
(a) Suppose that we want to visit a star at distance $d$ in light-years,
and we want our one-way trip, at constant velocity, to seem like a time interval $\interval$
to us. Starting from the equation $\interval^2=t^2-d^2$ (the version expressed in natural units),
find the time $t$ that this takes according to observers on earth.\answercheck\hwendpart
(b) The photo shows the constellation Canis Major, including the stars Sirius and Adhara.
Because we can't see depth when we look at the celestial sphere, we may be misled into
imagining that these two stars are close to one another. Actually Sirius is one of
our nearest neighbors, at 8.6 light-years, while Adhara is 570 light-years away.
For each of these stars, take $\interval=10$ years and evaluate $t$.\answercheck\hwendpart
(c) The data in this problem are conveniently expressed in natural units, but
for practice, let's see how your answer to part a would have looked in SI units.
Insert factors of $c$ as demonstrated in examples \ref{eg:natural-to-si}-\ref{eg:natural-to-si-2},
p.~\pageref{eg:natural-to-si}. Don't redo the algebra for part a, which would be more
complicated. The point here is to practice
how to put factors of $c$ in at the end of the calculation, as demonstrated in those
examples.\answercheck
