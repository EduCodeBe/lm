On p.~\pageref{subsec:proper-time}, we defined a quantity $t^2-x^2$, which is
often referred to as the spacetime interval.\index{spacetime interval}\index{interval!spacetime}
Let's notate it as $\interval$ (cursive letter ``I''). The only reason this quantity is interesting is that it stays
the same in all frames of reference, but to define it, we first had to pick a frame
of reference in order to define an $x-t$ plane, and then turn around and prove that it didn't matter
what frame had been chosen. It might thus be nicer simply to define it as the
square of the gliding-clock time, in the case where B can be reached from A. Since this definition
never refers to any coordinates or frame of reference, we know automatically that it is
frame-independent. In this case where $\interval>0$, we say that the relationship between A and 
B is timelike;\index{interval!timelike}
there is enough time for cause and effect to propagate between A and B. An interval
$\interval<0$ is called spacelike.\index{interval!spacelike}

In the spacelike case, we can define $\interval$ using rulers, as 
on p.~\pageref{interval-using-ruler}, but it's awkward to have to introduce an entirely
new measuring instrument in order to complete the 
definition. Geroch\footnote{Robert Geroch, \emph{General Relativity from A to B}, University of Chicago Press, 1978}
suggests a cute alternative in which this case as well can be treated using clocks.
Let observer O move inertially (i.e., without accelerating), and let her initial position and state of motion
be chosen such that she will be present at event A. Before A, she emits a ray of light, choosing to emit it
at the correct time and in the correct direction so that it will reach B. At B, we arrange to have the ray
reflected so that O can receive the reflection at some later time. Let $t_1$ be the time elapsed on O's clock
from emission of the first ray until event A, and let $t_2$ be the time from A until she receives the second
ray. The goal of this problem is to show that if we define $\interval$ as $-t_1 t_2$, we obtain the same
result as with the previous definition. Since $t_1$ are $t_2$ are simply clock readings, not coordinates
defined in an arbitrary frame of reference, this definition is automatically frame-independent.

(a) Show that $\interval$, as originally defined on p.~\pageref{subsec:proper-time},
    has the same units as the expression $-t_1 t_2$.\hwendpart
(b) Pick an event in the $x-t$ plane, and sketch the regions that are timelike and spacelike in relation to it.\hwendpart
(c) The special case of $\interval=0$ is called a lightlike\index{interval!lightlike} interval. Such events
    lie on a cone in the diagram drawn in part b, and this cone is called the light cone.\index{light cone}
    Verify that the two definitions of $\interval$ agree on the light cone.\hwendpart
(d) Prove that the two definitions agree on $\interval$ in the spacelike case.
(e) What goes wrong if O doesn't move inertially?
