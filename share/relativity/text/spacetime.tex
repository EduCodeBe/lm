<% begin_sec("Spacetime",nil,'spacetime') %>
Let's compare how Aristotle, Galileo, and Einstein would describe space and time.

Aristotle: All observers agree on whether or not two things happen at the same \emph{time}, and also
on whether they happen at the same \emph{place}.

Galileo: Observers always agree on simultaneity, but not necessarily on whether things happen in the same
place.

The reason for the disagreement is shown in figure \figref{cow-and-car}. Aristotle says that the
only legitimate observers are those that are at rest relative to the ground,
while Galileo is willing to accept any inertial frame of reference, such as the driver's.
Galileo ended up winning this argument because of experiments verifying the principle of inertia.

<% marg(30) %>
<%
  fig(
    'cow-and-car',
    %q{As time passes, the driver says the car stays in the same place, but the cow says the car 
       is moving forward.}
  )
%>
<% end_marg %>

Einstein: Observers need not agree on whether two things happen at the same time \emph{or} the same place.

We accept Einstein's view because of evidence such as the atomic clock
experiment described on p.~\pageref{hafele-keating}.
Such experiments rule out both the instantaneous transmission of signals (p.~\pageref{subsec:time-delays})
and, as we will argue in more detail on p.~\pageref{hk-implies-nonsimultaneity}, Galileo's claim
about universal agreement on simultaneity.

One of the reasons that nineteenth-century Europeans found Marxism alarming was because it
was atheistic, and they felt that without the framework of religion, there could be no basis
for morality. For similar reasons, I was deeply disoriented when I first encountered relativity.
The idea had been firmly inculcated that the universe was described by mathematical functions,
and the natural habitat of those functions was graph paper. The graph paper provided what seemed
like a necessary framework. For a position-time graph, the vertical lines meant ``same time,''
and the horizontal ones ``same place.'' Somehow it didn't bother me much when Galileo
erased the same-place lines (or at least relegated them to subjectivity), but without the same-time
lines I felt lost, as if I were wandering in a landscape of Hieronymus Bosch's hell or
Dali's melting watches.

<% marg(300) %>
<%
  fig(
    'bosch',
    %q{Hell, according to Hieronymus Bosch (1450-1516).}
  )
%>
<% end_marg %>

One of the disorienting things about this vision of the universe is that it takes away the notion
that we can have a literal ``vision'' of the universe. We no longer have the idea of a
snapshot of the landscape at a certain moment frozen in time. The sense of vision is merely
a type of optical measurement, in which we receive signals that have traveled to our eyes
at some finite speed (the speed of light). What relativity substitutes for the Galilean
instantaneous snapshot is the concept of \emph{spacetime}, which is like the graph paper
when its lines have been erased. Every point on the paper is called an \emph{event}. How can we
even agree on the existence of an event, or define which one we are talking about, if we can't
necessarily agree on its time or position? The relativist's attitude is that if a firecracker
pops, that's an event, everyone agrees that it's an event, and $x$ and $t$ coordinates are just
an optional and arbitrary name or label for the event. Labeling an event with coordinates is like
God asking Adam to name all the birds and animals: the animals weren't consulted and didn't care.
%       Genesis 20:19, "Out of the ground Yahweh God formed every animal of the field,
%       and every bird of the sky, and brought them to the man to see what he would call
%       them. Whatever the man called every living creature became its name." -- http://ebible.org/web/GEN02.htm#V0

My grandparents' German shepherd lived for a certain amount of time, so he was
not just a pointlike event
in spacetime. Way back in ch.~\ref{ch:motion}, we saw how to represent the motion of such things
as curves on an $x$-$t$ graph. From the point of view of relativity, the curve \emph{is} the thing ---
we make no distinction between the dog and the dog's track through spacetime. Such a track is called
a \emph{world-line}.\index{world-line} A world-line is a set of events strung together continuously:
the dog as a puppy in Walnut Creek in 1964, the dog dozing next to the TV in 1970, and so on. The
strange terminology is translated from German, and is supposed to be a description of the idea that
the line is the thing's track \emph{through} the world, i.e., through spacetime.

Sometimes if we want to describe an event, we can describe it as the beginning or end of a world-line:
the dog's birth, or the firecracker's self-destruction. More commonly, we pick out an event of interest
as the intersection of two world-lines, as in figure \figref{cell-phone-signal}. In this figure, as is
common in relativity, we omit any indications of the axes, since the idea is that events and world-lines
are primary, and coordinates secondary. In this book, to be consistent with the familiar depiction of
$x$-$t$ graphs, we will use the convention that
later times on an object's world-line are to the right, but it is actually more common in relativity to
show time progressing from the bottom of the diagram to the top.

<% marg() %>
<%
  fig(
    'cell-phone-signal',
    %q{A phone transmits a 1 or 0 to a cell tower. The phone, the tower, and the signal
       all have world-lines. Two events, corresponding to the transmission and reception of
       the signal, can be defined by the intersections of the world-lines.}
  )
%>
<% end_marg %>

All observers agree on whether or not two world-lines intersect, and another aid in holding on to
our sanity is that they agree on whether or not world-lines are \emph{straight}. A straight world-line
is an object moving inertially, with no forces acting on it.\label{agreement-on-intersection}

<% marg(300) %>
<%
  fig(   
    'light-cone',
    %q{%
      Normally we only try to represent one spatial dimension in a depiction of spacetime, as in
      panel 1, where a light bulb is momentarily turned on, producing flashes of light that spread
      out in both directions.
      In panel 2 we attempt to show a second spatial dimension, so that the world-line becomes a
      surface: a ``world-sheet'' shaped like a cone.
    }
  )
%>
<% end_marg %>


If we wish to, we are able to draw a graph-paper grid on our picture of spacetime, and
assign $x$ and $t$ coordinates to events, but these are not built into the structure of spacetime,
and they are observer-dependent --- even more so than in Galilean spacetime.
They are best thought of as the sophisticated results of a laborious
process of collecting and analyzing data obtained by methods such as
consulting clocks or exchanging signals between different places.
Figure \figref{surveying} outlines such a process in a cartoonish way.
A fleet of rocket ships, carrying surveyors,
is sent out from Earth and dispersed throughout a vast region of space. The surveyors
look through their theodolites at images, which are formed by light rays (dashed lines)
that have arrived after traveling at a finite speed. Such light rays carry old, stale information
about various events. A nuclear war has broken out. Rock and roll music has arrived on Saturn.
The resulting data are then transmitted by various means (passenger pigeon, Morse-coded radio,
paper mail) and consolidated at the surveying office, where coordinates are charted.

<%
  fig(   
    'surveying',
    %q{%
      Coordinates like $x$ and $t$ are the after-the-fact result of a process analogous
      to surveying.
    },
    {
      'width'=>'fullpage'
    }
  )
%>


<% self_check('grandparents-house',<<-'SELF_CHECK'
Here is a spacetime graph for an empty object such as a house:
\anonymousinlinefig{../../../share/relativity/figs/grandparents-house-1}. Explain why
it looks like this. My grandparents had a dog door with a flap cut into their back door, so
that their dog could come in and out. Draw a spacetime diagram showing the dog going out into
the back yard. Can an observer using another frame of reference say  that the dog didn't go outside?
  SELF_CHECK
  ) %>

In the next section we turn to a more quantitative treatment of how time and distance
behave in relativity.


<% end_sec('spacetime') %> % 
