(a) In an inkjet printer, a series of droplets of ink, each with mass $m$, are squirted out in rapid succession
at speed $u$. They are then given charge $q$ by the beam of an electron gun, and
finally deflected by passing through a capacitor whose electric field $E$
is in the perpendicular direction. Let the width of the capacitor be $w$, and assume that
the electric field is uniform between its plates but zero outside. (We'll see in
section \ref{sec:something-is-missing}, p.~\pageref{sec:something-is-missing}, that this
cannot quite be true, and is at best an approximation.) Find the deflection angle $\theta$.\answercheck\hwendpart
%
% Got reasonable numbers from http://spiff.rit.edu/classes/phys213/lectures/inkjet/inkjet_long.html 
(b) Evaluate your result, in radians, for $m=2.0\times10^{-10}\ \kgunit$, $u=20\ \munit/\sunit$, $q=2.0\times10^{-10}\ \zu{C}$, 
$E=2.0\times10^5\ \zu{V}/\munit$, and $w=3\ \zu{mm}$.\answercheck\hwendpart

