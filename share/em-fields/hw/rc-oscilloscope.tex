Resistors have a standard color code on them, but capacitors often
have only cryptic part numbers printed on them that don't say the
value of the component. Suppose we have an unknown capacitor, and we
want to find its value. We take a known resistance $R=4.7\
\zu{k}\Omega$ and form a series RC circuit. Hooking it up to a
square-wave generator, and observing the results on a digital
oscilloscope, we see the results shown in the figures. These are two
versions of the same electrical signal on the scope.  Note the scales
shown at the bottom. The only difference between the two traces is
time scale. Find the unknown capacitance. Why would it be necessary to
look at both scales in order to get a good measurement?\answercheck
