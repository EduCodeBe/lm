(a) A line charge, with charge per unit length $\lambda$, moves at velocity
         $v$ along its own length. How much charge passes a given point in time $\der t$?
         What is the resulting current? <% hw_answer %>\hwendpart
        (b) Show that the units of your answer in part a work out correctly.
        \hwremark{This constitutes a physical model of an electric current, and it would
        be a physically realistic model of a beam of particles moving in a vacuum, such
        as the electron beam in a television tube. It is not a physically realistic
        model of the motion of the electrons in
        a current-carrying wire, or of the ions in your nervous system; the motion of
        the charge carriers in these systems is much more
        complicated and chaotic, and there are charges of both signs, so
        that the total charge is zero. But even when the model is physically unrealistic, it
        still gives the right answers when you use it to compute magnetic effects.
        This is a remarkable fact, which we will not prove. The interested reader is
        referred to E.M. Purcell, \emph{Electricity and Magnetism}, McGraw Hill, 1963.}
