A carbon dioxide molecule is structured like O-C-O, with all three
atoms along a line. The oxygen atoms grab a little bit of extra
negative charge, leaving the carbon positive.  The molecule's
symmetry, however, means that it has no overall dipole moment, unlike
a V-shaped water molecule, for instance.  Whereas the potential of a
dipole of magnitude $D$ is proportional to 
m4_ifelse(__lm_series,1,[:$D/r^2$,:],[:$D/r^2$, (see problem \ref{hw:dipolev}),:])
it turns out that the potential of a carbon dioxide
molecule at a distant point along the molecule's axis equals $b/r^3$,
where $r$ is the distance from the molecule and $b$ is a
m4_ifelse(__lm_series,1,[:constant.:],[:constant (cf.~problem \ref{hw:quadrupole}).:])
What would be the electric field of a
carbon dioxide molecule at a point on the molecule's axis, at a
distance $r$ from the molecule?\answercheck

