A hydrogen atom is electrically neutral, so at large distances, we expect that
        it will create essentially zero electric field. This is not true, however, near
        the atom or inside it. Very close to the proton, for example, the field
        is very strong. To see this, think of the electron as a spherically
        symmetric cloud that surrounds the
        proton, getting thinner and thinner as we get farther away from the proton. (Quantum
        mechanics tells us that this is a more correct picture than trying to imagine the
        electron orbiting the proton.)  Near the center of the atom, the electron cloud's
        field cancels out by symmetry, but the proton's field is strong, so the total field
        is very strong. The potential in and around the hydrogen atom can be approximated using
        an expression of the form $V=r^{-1}e^{-r}$. (The units come out wrong, because I've
        left out some constants.) Find the electric field corresponding to this potential, and
        comment on its behavior at very large and very small $r$.
        <% hw_solution %>
