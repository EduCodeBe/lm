The figure shows cross-sectional views of two cubical
capacitors, and a cross-sectional view of the same two
capacitors put together so that their interiors coincide. A
capacitor with the plates close together has a nearly
uniform electric field between the plates, and almost zero
field outside; these capacitors don't have their plates very
close together compared to the dimensions of the plates, but
for the purposes of this problem, assume that they still
have approximately the kind of idealized field pattern shown
in the figure. Each capacitor has an interior volume of 1.00
$\munit^3$, and is charged up to the point where its internal
field is 1.00 V/m. (a) Calculate the energy stored in the
electric field of each capacitor when they are separate. (b)
Calculate the magnitude of the interior field when the two
capacitors are put together in the manner shown. Ignore
effects arising from the redistribution of each capacitor's
charge under the influence of the other capacitor. (c)
Calculate the energy of the put-together configuration. Does
assembling them like this release energy, consume energy, or neither?\answercheck\hwendpart
