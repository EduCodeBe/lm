(a) An electric potential is given by $\phi=ar^2$, where $r$ is the distance
from the origin and $a$ is a constant. This can be written as
\begin{equation*}
  \phi = a(x^2+y^2+z^2).
\end{equation*}
Find the components of the corresponding electric field by computing the
gradient.\hwendpart
(b) Find the magnitude of the field.\answercheck\hwendpart
(c) By comparing with the result of example \ref{eg:gauss-law-uniform-sphere},
p.~\pageref{eg:gauss-law-uniform-sphere}, show that this is the potential of
a uniform sphere of charge, and determine $a$ in terms of the charge density.\answercheck\hwendpart
(d) Suppose that the potential had instead been
\begin{equation*}
  \phi = a(x^2+y^2+z^2)+b,
\end{equation*}
where $b$ is a constant. How would this have affected the results?
