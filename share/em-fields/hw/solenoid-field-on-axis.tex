(a) A solenoid can be imagined as a series of circular current loops that are spaced
        along their common axis.
        Integrate the result of example \ref{eg:biotloop} on page
        \pageref{eg:biotloop} to show that the field
        on the axis of a solenoid can be written as
        $B=(2\pi k\eta/c^2)(\cos\beta+\cos\gamma)$, where the angles $\beta$ and
        $\gamma$ are defined in the figure.\hwendpart
        (b) Show that in the limit where the solenoid is very long, this exact result
        agrees with the approximate one derived in example
        \ref{eg:amperesolenoid} on page \pageref{eg:amperesolenoid} using Amp\`{e}re's
        law.\hwendpart
        (c) Note that, unlike the calculation using Amp\`{e}re's law, this one is valid
        at points that are near the mouths of the solenoid, or even outside it entirely.
        If the solenoid is long, at what point on the axis is the field equal to one
        half of its value at the center of the solenoid?\hwendpart
        (d) What happens to your result when you apply it to points that are very far away
        from the solenoid? Does this make sense?
