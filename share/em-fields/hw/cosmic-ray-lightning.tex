In problem \ref{hw:lightning} on p.~\pageref{hw:lightning}, you estimated the energy released in a
bolt of lightning, based on the energy stored in the
electric field immediately before the lightning occurs. The
assumption was that the field would build up to a certain
value, which is what is necessary to ionize air. However,
real-life measurements always seemed to show electric fields
strengths roughly 10 times smaller than those required in
that model. For a long time, it wasn't clear whether the
field measurements were wrong, or the model was wrong.
Research carried out in 2003 seems to show that the model
was wrong. It is now believed that the final triggering of
the bolt of lightning comes from cosmic rays that enter the
atmosphere and ionize some of the air. If the field is 10
times smaller than the value assumed in problem \ref{hw:lightning}, what
effect does this have on the final result of problem \ref{hw:lightning}?\answercheck\hwendpart
