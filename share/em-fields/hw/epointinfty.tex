(a) Show that the energy in the electric field of a point charge
        is infinite! Does the integral diverge at small distances, at large
        distances, or both? <% hw_hint("epointinfty") %>\hwendpart[4]
        (b) Now calculate the energy in the electric field of a uniformly
        charged sphere with radius $b$. Based on the shell theorem,
        it can be shown that the field for $r>b$ is the same as for
        a point charge, while the field for $r<b$ is $kqr/b^3$. (Example
        \ref{eg:divsphere} shows this using a different technique.)\\
        \hwremark{The calculation in part a seems to show that infinite energy
        would be required in order to create a charged, pointlike particle.
        However, there are processes that, for example,
        create electron-positron pairs, and these processes don't require
        infinite energy. According to Einstein's famous equation
        $E=mc^2$, the energy required to create such a pair should only
        be $2mc^2$, which is finite. One way out of this difficulty is to
        assume that no particle is really pointlike, and this is in fact the
        main motivation behind a speculative physical theory called string
        theory, which posits that charged particles are actually tiny loops,
        not points.}\answercheck
