The top panel of the figure shows measurements (Settle \emph{et al.}, 1980) of the impedance of a human
body, from an ankle to the opposite wrist. The impedance is shown
in the complex plane, at a series of increasing frequencies. This
technique can be used as a cheap, noninvasive way of estimating a person's body
composition. The idea is that fat is effectively an insulator, while muscle
contains both fluids, which act like a resistance, and cell membranes, which
act like capacitors. The middle panel is a simplified version of the graph,
cooked up so as to provide round numbers that are easier to work with in a textbook
exercise. All the real and imaginary parts of the impedances are multiples of 100,
in units of ohms, for the three dots. The bottom panel shows a model used to explain the graph.
Use the model to determine the parameters
$R_1$, $R_2$, and $C$.\answercheck
% doi:10.1080/01635588009513660 
