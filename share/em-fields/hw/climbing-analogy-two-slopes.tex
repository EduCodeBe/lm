The figure shows a rock climbing route lying in a vertical plane.
It can be approximated by two line segments. Superimposed on the climb
is a grid of squares with sides of length $\ell$. The gravitational field
is $g$.\\
(a) Continuing the gravitational
analogy from problem \ref{hw:two-climbing-routes}, find
$\phi_B-\phi_A$ and $\phi_C-\phi_B$, where the ``electric potential''
$\phi$ is the gravitational potential energy per unit mass.\answercheck\hwendpart
(b) In the electrical version of this situation, the ``height'' is
not a physical distance in space at all, so we could say that
only the horizontal segments of the squares represent
distances, and the situation is effectively one-dimensional.
Find the ratio of the ``electric fields'' $E_{BC}/E_{AB}$.\answercheck\hwendpart
(c) In electromagnetism, we can always add an arbitrary constant to
the potential while still describing the same physical situation.
What would be the analogous statement for the climber in our graviational
analogy?
