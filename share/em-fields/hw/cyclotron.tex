A charged particle of mass $m$ and charge $q$
        moves in a circle due to a uniform magnetic field of magnitude $B$,
        which points perpendicular to the plane of the circle.\hwendpart
        (a) Assume the particle is positively charged.
        Make a sketch showing the direction of motion and the direction of the
        field, and show that the resulting force is in the right direction to
        produce circular motion.\hwendpart
        (b) Find the radius, $r$, of the circle, in terms of $m$, $q$,
         $v$, and $B$.\answercheck\hwendpart
        (c) Show that your result from part b has the right units.\hwendpart
        (d) Discuss all four variables occurring on the right-hand side of your
        answer from part b. Do they make sense? For instance, what should happen
        to the radius when the magnetic field is made stronger? Does your equation behave
        this way?\hwendpart
        (e) Restate your result so that it gives the particle's angular frequency,
        $\omega$, in terms of the other variables, and show that $v$ drops out.\answercheck\hwendpart
        \hwremark{A charged particle can be accelerated in a circular device called a cyclotron,
        in which a magnetic field is what keeps them from going off straight.
        This frequency is therefore known as the cyclotron 
        frequency.\index{cyclotron!cyclotron frequency}\index{cyclotron}
        The particles are accelerated by other forces (electric forces), which are AC.
        As long as the electric field
         is operated at the correct cyclotron frequency for the type of particles
        being manipulated, it will stay in sync with the particles, giving them a shove in
        the right direction each time they pass by. The particles are speeding up,
        so this only works because the cyclotron frequency is independent of velocity.}
