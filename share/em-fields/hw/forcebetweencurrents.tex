The purpose of this problem is to show that the magnetic interaction rules shown in figure
\figref{magtwobody} can be simplified by stating them in terms of current. 
Recall that, as discussed in discussion question \ref{dq:signsofcurrent} on page
\pageref{dq:signsofcurrent}, one type of charge moving in a particular direction produces
the same current as the other type of charge moving in the opposite direction. 
Let's say arbitrarily that the current made by the dark type of charged particle is
in the direction it's moving, while a light-colored particle produces a current in the
direction opposite to its motion. Redraw all four panels of figure \figref{magtwobody},
replacing each picture of a moving light or dark particle with an
arrow showing the direction of the current it makes. Show that the rules for attraction and
repulsion can now be made much simpler, and state the simplified rules explicitly.
