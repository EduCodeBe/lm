It sounds a bit odd when a scientist refers to a theory as
``beautiful,'' but to those in the know it makes perfect
sense. One mark of a beautiful theory is that it surprises
us by being simple. The mathematical theory of lenses and
curved mirrors gives us just such a surprise. We expect the
subject to be complex because there are so many cases: a
converging mirror forming a real image, a diverging lens
that makes a virtual image, and so on for a total of six
possibilities. If we want to predict the location of the
images in all these situations, we might expect to need six
different equations, and six more for predicting magnifications.
Instead, it turns out that we can use just one equation for
the location of the image and one equation for its
magnification, and these two equations work in all the
different cases with no changes except for plus and minus
signs. This is the kind of thing the physicist Eugene
\index{Wigner, Eugene}Wigner referred to as ``the unreasonable
effectiveness of mathematics.'' Sometimes we can find a
deeper reason for this kind of unexpected simplicity, but
sometimes it almost seems as if God went out of Her way to
make the secrets of universe susceptible to attack by the
human thought-tool called math.

<% begin_sec("A real image formed by a converging mirror",0,'locate-real-image') %>
\index{mirror!converging}

<% begin_sec("Location of the image") %>\index{images!location of}

We will now derive the equation for the location of a real
image formed by a converging mirror. We assume for
simplicity that the mirror is spherical, but actually this
isn't a restrictive assumption, because any shallow,
symmetric curve can be approximated by a sphere. The shape
of the mirror can be specified by giving the location of its
center, C. A deeply curved mirror is a sphere with a small
radius, so C is close to it, while a weakly curved mirror
has C farther away. Given the point O where the object
is, we wish to find the point I where the image will be formed.
<% marg(80) %>
<%
  fig(
    'thetai-and-thetao',
    %q{%
      The relationship between the object's position
      and the image's can be expressed in terms of the angles $\theta_o$ and $\theta_i$.
    }
  )
%>
<% end_marg %>

To locate an image, we need to track a minimum of two rays
coming from the same point. Since we have proved in the
previous chapter that this type of image is not distorted,
we can use an on-axis point, O, on the object, as in
figure \subfigref{thetai-and-thetao}{1}. The results we derive will also hold for
off-axis points, since otherwise the image would have to be
distorted, which we know is not true. We let one of the rays
be the one that is emitted along the axis; this ray is
especially easy to trace, because it bounces straight back
along the axis again. As our second ray, we choose one that
strikes the mirror at a distance of 1 from the axis. ``One
what?'' asks the astute reader. The answer is that it
doesn't really matter. When a mirror has shallow curvature,
all the reflected rays hit the same point, so 1 could be
expressed in any units you like. It could, for instance, be
1 cm, unless your mirror is smaller than 1 cm!

The only way to find out anything mathematical about the
rays is to use the sole mathematical fact we possess
concerning specular reflection: the incident and reflected
rays form equal angles with respect to the normal, which is
shown as a dashed line. Therefore the two angles shown in
figure \subfigref{thetai-and-thetao}{2} are the same, and skipping some straightforward
geometry, this leads to the visually reasonable result that
the two angles in figure \subfigref{thetai-and-thetao}{3} are related as follows:
\begin{equation*}
                \theta_i+\theta_o  =  \text{constant}
\end{equation*}
(Note that $\theta_i$ and $\theta_o$, which are measured
from the image and the object, not from the eye like the
angles we referred to in discussing angular magnification
on page \pageref{fig:angular-size}.)
For example, move O farther from the
mirror. The top angle in figure \subfigref{thetai-and-thetao}{2} is increased, so the
bottom angle must increase by the same amount, causing the
image point, I, to move closer to the mirror. In terms of
the angles shown in figure \subfigref{thetai-and-thetao}{3}, the more distant object has
resulted in a smaller angle $\theta_o$, while the closer
image corresponds to a larger $\theta_i;$ One angle
increases by the same amount that the other decreases, so
their sum remains constant. These changes are summarized in figure \subfigref{thetai-and-thetao}{4}.

<% marg(30) %>
<%
  fig(
    'focal-angle',
    %q{The geometrical interpretation of the focal angle.}
  )
%>
\spacebetweenfigs
<%
  fig(
    'eg-focal-angle-alternative',
    %q{%
      Example \ref{eg:focal-angle-alternative}, an alternative test for finding
      the focal angle. The mirror is the same as in figure \figref{focal-angle}.
    }
  )
%>

<% end_marg %>

The sum $\theta_i+\theta_o$ is a constant. What does this
constant represent? Geometrically, we interpret it as double
the angle made by the dashed radius line. Optically, it is a
measure of the strength of the mirror, i.e., how strongly
the mirror focuses light, and so we call it the focal
angle,\index{focal angle} $\theta_f$,
\begin{equation*}
                \theta_i+\theta_o  =  \theta_f\eqquad.
\end{equation*}
Suppose, for example, that we wish to use a quick and dirty
optical test to determine how strong a particular mirror is.
We can lay it on the floor as shown in figure \figref{eg-focal-angle-alternative}, and use
it to make an image of a lamp mounted on the ceiling
overhead, which we assume is very far away compared to the
radius of curvature of the mirror, so that the mirror
intercepts only a very narrow cone of rays from the lamp.
This cone is so narrow that its rays are nearly parallel,
and $\theta_o$ is nearly zero. The real image can be
observed on a piece of paper. By moving the paper nearer and
farther, we can bring the image into focus, at which point
we know the paper is located at the image point. Since
$\theta_o\approx 0$, we have $\theta_i\approx \theta_f$, and we can
then determine this mirror's focal angle either by measuring
$\theta_i$ directly with a protractor, or indirectly via
trigonometry. A strong mirror will bring the rays together to form
an image close to the mirror, and these rays will form a
blunt-angled cone with a large $\theta_i$ and $\theta_f$.

\begin{eg}{An alternative optical test}\label{eg:focal-angle-alternative}
\egquestion Figure \figref{eg-focal-angle-alternative} shows an alternative optical test. Rather
than placing the object at infinity as in figure \figref{focal-angle}, we
adjust it so that the image is right on top of the object.
Points O and I coincide, and the rays are reflected right
back on top of themselves. If we measure the angle $\theta $
shown in figure \figref{eg-focal-angle-alternative}, how can we find the focal angle?

\eganswer The object and image angles are the same; the
angle labeled $\theta $ in the figure equals both of them.
We therefore have $\theta_i+\theta_o=\theta =\theta_f$.
Comparing figures \figref{focal-angle} and \figref{eg-focal-angle-alternative}, it is indeed plausible that
the angles are related by a factor of two.
\end{eg}

At this point, we could consider our work to be done.
Typically, we know the strength of the mirror, and we want
to find the image location for a given object location.
Given the mirror's focal angle and the object location, we
can determine $\theta_o$ by trigonometry, subtract to find
$\theta_i=\theta_f-\theta_o$, and then do more trig to
find the image location.

There is, however, a shortcut that can save us from doing so
much work. Figure \subfigref{thetai-and-thetao}{3} shows two right triangles whose legs
of length 1 coincide and whose acute angles are $\theta_o$
and $\theta_i$. These can be related by trigonometry to the
object and image distances shown in figure \figref{di-and-do}:
\begin{equation*}
                \tan  \theta_o = 1/d_o \qquad \qquad        \tan  \theta_i  =  1/d_i
\end{equation*}
Ever since chapter 2, we've been assuming small angles. For
small angles, we can use the small-angle approximation 
$\tan x\approx x$ (for $x$ in radians), giving simply
\begin{equation*}
                \theta_o = 1/d_o \qquad\qquad \theta_i  =  1/d_i\eqquad.
\end{equation*}
We likewise define a distance called the focal length,\index{focal
length} $f$ according to $\theta_f=1/f$. In figure \figref{focal-angle}, $f$
is the distance from the mirror to the place where the rays cross. We can
now reexpress the equation relating the object and image positions as
\begin{equation*}
        \frac{1}{f} = \frac{1}{d_i}+\frac{1}{d_o}\eqquad.
\end{equation*}
Figure \figref{focal-length-concept} summarizes the interpretation of the focal
length and focal angle.\footnote{There is a standard piece of terminology
which is that the ``focal point'' is the point lying on the optical
axis at a distance from the mirror equal to the focal length.\index{focal point}
This term isn't particularly helpful, because it names a location where nothing
normally happens. In particular, it is \emph{not} normally the place where
the rays come to a focus! --- that would be the \emph{image} point.
In other words, we don't normally have $d_i=f$, unless perhaps
$d_o=\infty$. A recent online discussion among some physics teachers 
(https://carnot.physics.buffalo.edu/archives, Feb. 2006) showed that
many disliked the terminology, felt it was misleading, or didn't know it and would have
misinterpreted it if they had come across it. That is, it appears to be what grammarians
call a ``skunked term'' --- a word that bothers half the population when it's used
incorrectly, and the other half when it's used correctly.}

<% marg(79) %>
<%
  fig(
    'di-and-do',
    %q{The object and image distances}
  )
%>
\spacebetweenfigs
<%
  fig(
    'focal-length-concept',
    %q{%
      Mirror 1 is weaker than mirror 2. It has a shallower curvature, a longer focal length,
      and a smaller focal angle. It reflects rays at angles not much different than those that would be produced
      with a flat mirror.
    }
  )
%>

<% end_marg %>
Which form is better, $\theta_f=\theta_i+\theta_o$ or 
$1/f=1/d_i+1/d_o?$ The angular form has in its favor its
simplicity and its straightforward visual interpretation,
but there are two reasons why we might prefer the second
version. First, the numerical values of the angles depend on
what we mean by ``one unit'' for the distance shown as 1 in
figure \subfigref{thetai-and-thetao}{1}. Second, it is usually easier to measure
distances rather than angles, so the distance form is more
convenient for number crunching. Neither form is superior
overall, and we will often need to use both to solve any
given problem.\footnote{I would like to thank Fouad Ajami for
pointing out the pedagogical advantages of using both
equations side by side.}

\begin{eg}{A searchlight}
Suppose we need to create a parallel beam of light, as in a
searchlight. Where should we place the lightbulb? A parallel
beam has zero angle between its rays, so $\theta_i=0$. To
place the lightbulb correctly, however, we need to know a
distance, not an angle: the distance $d_o$ between the bulb
and the mirror. The problem involves a mixture of distances
and angles, so we need to get everything in terms of one or
the other in order to solve it. Since the goal is to find a
distance, let's figure out the image distance corresponding
to the given angle $\theta_i=0$. These are related by
$d_i=1/\theta_i$, so we have $d_i=\infty$. (Yes, dividing
by zero gives infinity. Don't be afraid of infinity.
Infinity is a useful problem-solving device.) Solving the
distance equation for $d_o$, we have
\begin{align*}
                d_o          &= (1/f - 1/d_i)^{-1} \\
                          &= (1/f - 0)^{-1} \\
                          &= f
\end{align*}
The bulb has to be placed at a distance from the mirror
equal to its focal point.
\end{eg}

\begin{eg}{Diopters}
An equation like $d_i=1/\theta_i$ really doesn't make sense
in terms of units. Angles are unitless, since radians aren't
really units, so the right-hand side is unitless. We can't
have a left-hand side with units of distance if the
right-hand side of the same equation is unitless. This is an
artifact of my cavalier statement that the conical bundles
of rays spread out to a distance of 1 from the axis where
they strike the mirror, without specifying the units used to
measure this 1. In real life, optometrists define the thing
we're calling $\theta_i=1/d_i$ as the ``dioptric strength''
of a lens or mirror, and measure it in units of inverse
meters ($\munit^{-1}$), also known as diopters (1 D=1 $\munit^{-1}$).\index{diopter}
\end{eg}

<% end_sec() %>
<% begin_sec("Magnification") %>

We have already discussed in the previous chapter how to
find the magnification of a virtual image made by a curved
mirror. The result is the same for a real image, and we omit
the proof, which is very similar. In our new notation, the
result is $M=d_i/d_o$. A numerical example is given in
__subsection_or_section(other-curved-mirrors).

\vspace{0mm plus 1mm}

<% end_sec() %>
<% end_sec() %>
<% begin_sec("Other cases with curved mirrors",0,'other-curved-mirrors') %>

The equation $d_i=(1/f-1/d_o)^{-1}$ can easily produce a negative result,
but we have been thinking of $d_i$ as a distance, and
distances can't be negative. A similar problem occurs with
$\theta_i=\theta_f-\theta_o$ for $\theta_o>\theta_f$.
What's going on here?

The interpretation of the angular equation is straightforward.
As we bring the object closer and closer to the image,
$\theta_o$ gets bigger and bigger, and eventually we reach a
point where $\theta_o=\theta_f$ and $\theta_i=0$. This large
object angle represents a bundle of rays forming a cone that
is very broad, so broad that the mirror can no longer bend
them back so that they reconverge on the axis. The image
angle $\theta_i=0$ represents an outgoing bundle of rays
that are parallel. The outgoing rays never cross, so this is
not a real image, unless we want to be charitable and say
that the rays cross at infinity. If we go on bringing the
object even closer, we get a virtual image.

m4_ifelse(__sn,1,[::],[:\pagebreak:])

<%
  fig(
    'graph-of-di-vs-do',
    %q{%
      A graph of the image distance $d_i$ as a function
      of the object distance $d_o$.
    },
    {
      'width'=>'wide',
      'sidecaption'=>true
    }
  )
%>

To analyze the distance equation, let's look at a graph of
$d_i$ as a function of $d_o$. The branch on the upper right
corresponds to the case of a real image. Strictly speaking,
this is the only part of the graph that we've proven
corresponds to reality, since we never did any geometry for
other cases, such as virtual images. As discussed in the
previous section, making $d_o$ bigger causes $d_i$ to become
smaller, and vice-versa.

Letting $d_o$ be less than $f$ is equivalent to $\theta_o>\theta_f:$
a virtual image is produced on the far side of the mirror.
This is the first example of Wigner's ``unreasonable
effectiveness of mathematics'' that we have encountered in
optics. Even though our proof depended on the assumption
that the image was real, the equation we derived turns out
to be applicable to virtual images, provided that we either
interpret the positive and negative signs in a certain way,
or else modify the equation to have different positive and negative signs.

<% self_check('asymptotes',<<-'SELF_CHECK'
Interpret the three places where, in physically realistic
parts of the graph, the graph approaches one of the dashed
lines. [This will come more naturally if you have learned
the concept of limits in a math class.]
  SELF_CHECK
  ) %>

\begin{eg}{A flat mirror}
We can even apply the equation to a flat mirror. As a sphere
gets bigger and bigger, its surface is more and more gently
curved. The planet Earth is so large, for example, that we
cannot even perceive the curvature of its surface. To
represent a flat mirror, we let the mirror's radius of
curvature, and its focal length, become infinite. Dividing
by infinity gives zero, so we have
\begin{align*}
                1/d_o  &=  -1/d_i\eqquad,\\
\intertext{or}
                d_o  &=  -d_i\eqquad.
\end{align*}
If we interpret the minus sign as indicating a virtual image
on the far side of the mirror from the object, this makes sense.
\end{eg}

It turns out that for any of the six possible combinations
of real or virtual images formed by converging or diverging
lenses or mirrors, we can apply equations of the form
\begin{gather*}
                \theta_f =  \theta_i+\theta_o \\
\intertext{and}
        \frac{1}{f}            =  \frac{1}{d_i}+\frac{1}{d_o}\eqquad,
\end{gather*}
with only a modification of plus or minus signs. There are
two possible approaches here. The approach we have been
using so far is the more popular approach in American
textbooks: leave the equation the same, but attach
interpretations to the resulting negative or positive values
of the variables. The trouble with this approach is that one
is then forced to memorize tables of sign conventions, e.g.,
that the value of $d_i$ should be negative when the image is
a virtual image formed by a converging mirror. Positive and
negative signs also have to be memorized for focal lengths.
Ugh! It's highly unlikely that any student has ever retained
these lengthy tables in his or her mind for more than five
minutes after handing in the final exam in a physics course.
Of course one can always look such things up when they are
needed, but the effect is to turn the whole thing into an
exercise in blindly plugging numbers into formulas.

As you have gathered by now, there is another method which I
think is better, and which I'll use throughout the rest of
this book. In this method, all distances and angles are
\emph{positive by definition}, and we put in positive and
negative signs in the \emph{equations} depending on the
situation. (I thought I was the first to invent this method,
but I've been told that this is known as the European sign
convention, and that it's fairly common in Europe.) Rather
than memorizing these signs, we start with the generic equations
\begin{align*}
                \theta_f &= \pm \theta_i \pm \theta_o \\
        \frac{1}{f}            &=  \pm\frac{1}{d_i}\pm\frac{1}{d_o}\eqquad,
\end{align*}
and then determine the signs by a two-step method that
depends on ray diagrams. There are really only two signs to
determine, not four; the signs in the two equations match up
in the way you'd expect. The method is as follows:

1. Use ray diagrams to decide whether $\theta_o$ and
$\theta_i$ vary in the same way or in opposite ways. (In
other words, decide whether making $\theta_o$ greater
results in a greater value of $\theta_i$ or a smaller one.)
Based on this, decide whether the two signs in the angle
equation are the same or opposite. If the signs are
opposite, go on to step 2 to determine which is positive
and which is negative.

2. If the signs are opposite, we need to decide which is the
positive one and which is the negative. Since the focal
angle is never negative, the smaller angle must be the
one with a minus sign.

In step 1, many students have trouble drawing the ray
diagram correctly. For simplicity, you should always do your
diagram for a point on the object that is on the axis of the
mirror, and let one of your rays be the one that is emitted
along the axis and reflected straight back on itself, as in
the figures in __subsection_or_section(locate-real-image). As shown in figure \subfigref{thetai-and-thetao}{4} in
__subsection_or_section(locate-real-image), there are four angles involved: two at the
mirror, one at the object $(\theta_o)$, and one at the image
$(\theta_i)$. Make sure to draw in the normal to the mirror
so that you can see the two angles at the mirror. These two
angles are equal, so as you change the object position, they
fan out or fan in, like opening or closing a book. Once
you've drawn this effect, you should easily be able to tell
whether $\theta_o$ and $\theta_i$ change in the same way
or in opposite ways.

Although focal lengths are always positive in the method
used in this book, you should be aware that diverging
mirrors and lenses are assigned negative focal lengths in
the other method, so if you see a lens labeled
$f=-30$ cm, you'll know what it means.

\begin{eg}{An anti-shoplifting mirror}\label{eg:shoplifting-mirror}
\egquestion Convenience stores often install a diverging
mirror so that the clerk has a view of the whole store and
can catch shoplifters. Use a ray diagram to show that the
image is reduced, bringing more into the clerk's field of
view. If the focal length of the mirror is 3.0 m, and the
mirror is 7.0 m from the farthest wall, how deep is
the image of the store?
<% marg(60) %>
<%
  fig(
    'eg-shoplifting-mirror',
    %q{Example \ref{eg:shoplifting-mirror}.}
  )
%>
<% end_marg %>

\eganswer 
As shown in ray diagram \figref{eg-shoplifting-mirror}/1, $d_i$ is less than
$d_o$. The magnification, $M= d_i/d_o$, will be less than
one, i.e., the image is actually reduced rather than magnified.

Apply the method outlined above for determining the
plus and minus signs. Step 1: 
The object is the point on the
opposite wall. As an experiment, \subfigref{eg-shoplifting-mirror}{2}, move the
object closer. I did these drawings using illustration
software, but if you were doing them by hand, you'd want to
make the scale much larger for greater accuracy. Also,
although I split figure \figref{eg-shoplifting-mirror} into two separate drawings in
order to make them easier to understand, you're less likely
to make a mistake if you do them on top of each other.

        The two angles at the mirror fan out from the normal.
Increasing $\theta_o$ has clearly made $\theta_i$ larger as
well. (All four angles got bigger.) There must be a
cancellation of the effects of changing the two terms on the
right in the same way, and the only way to get such a
cancellation is if the two terms in the angle equation
have opposite signs:
\begin{equation*}
                        \theta_f =  + \theta_i  -  \theta_o \qquad
\end{equation*}
or
\begin{equation*}
                        \theta_f =  - \theta_i  +  \theta_o\eqquad.
\end{equation*}
        Step 2: Now which is the positive term and which is
negative? Since the image angle is bigger than the object
angle, the angle equation must be
\begin{equation*}
                        \theta_f =     \theta_i  -  \theta_o\eqquad,
\end{equation*}
in order to give a positive result for the focal angle. The
signs of the distance equation behave the same way:
\begin{equation*}
                        \frac{1}{f}         = \frac{1}{d_i}-\frac{1}{d_o}\eqquad.
\end{equation*}
Solving for $d_i$, we find
\begin{align*}
                        d_i         &= \left(\frac{1}{f}+\frac{1}{d_o}\right)^{-1}\\
                                 &= 2.1\ \munit\eqquad.
\end{align*}
The image of the store is reduced by a factor of $2.1/7.0=0.3$,
i.e., it is smaller by 70\%.
\end{eg}

<%
  fig(
    'mirrorball',
    %q{%
      A diverging mirror in the shape of a sphere. The image is reduced ($M<1$).
      This is similar to example \ref{eg:shoplifting-mirror}, but here the image is
      distorted because the mirror's curve is not shallow.
    },
    {
      'width'=>'wide',
      'sidecaption'=>true
    }
  )
%>

\begin{eg}{A shortcut for real images}
In the case of a real image, there is a shortcut for step 1,
the determination of the signs. In a real image, the rays
cross at both the object and the image. We can therefore
time-reverse the ray diagram, so that all the rays are
coming from the image and reconverging at the object. Object
and image swap roles. Due to this time-reversal symmetry,
the object and image cannot be treated differently in any of
the equations, and they must therefore have the same signs.
They are both positive, since they must add up to a positive result.
\end{eg}

<% end_sec() %>
<% begin_sec("Aberrations",nil,'aberration',{'optional'=>true}) %>\index{aberration}

An imperfection or distortion in an image is called an
aberration. An aberration can be produced by a flaw in a
lens or mirror, but even with a perfect optical surface some
degree of aberration is unavoidable. To see why, consider
the mathematical approximation we've been making, which is
that the depth of the mirror's curve is small compared to
$d_o$ and $d_i$. Since only a flat mirror can satisfy this
shallow-mirror condition perfectly, any curved mirror will
deviate somewhat from the mathematical behavior we derived
by assuming that condition. There are two main types of
aberration in curved mirrors, and these also occur with lenses.

(1) An object on the axis of the lens or mirror may be
imaged correctly, but off-axis objects may be out of focus or distorted.
In a camera, this type of aberration would show up as a
fuzziness or warping near the sides of the picture when the center
was perfectly focused. An example of this is shown in figure \figref{short-lens-aberration},
and in that particular example, the aberration is not a sign that the equipment was of
low quality or wasn't right for the job but rather an inevitable result of trying to flatten
a panoramic view; in the limit of a 360-degree panorama, the problem would be similar to the
problem of representing the Earth's surface on a flat map, which can't be accomplished
without distortion.

<%
  fig(
    'short-lens-aberration',
    %q{%
      This photo was taken using a ``fish-eye lens,'' which
      gives an extremely large field of view. 
    },
    {
      'width'=>'wide',
      'sidecaption'=>true
    }
  )
%>

(2) The image may be sharp when the object is at certain
distances and blurry when it is at other distances. The
blurriness occurs because the rays do not all cross at
exactly the same point. If we know in advance the distance
of the objects with which the mirror or lens will be used,
then we can optimize the shape of the optical surface to
make in-focus images in that situation. For instance, a
spherical mirror will produce a perfect image of an object
that is at the center of the sphere, because each ray is
reflected directly onto the radius along which it was
emitted. For objects at greater distances, however, the
focus will be somewhat blurry. In astronomy the objects
being used are always at infinity, so a spherical mirror is
a poor choice for a telescope. A different shape (a
parabola) is better specialized for astronomy.

<%
  fig(
    'spherical-vs-parabolic',
    %q{%
      Spherical mirrors are the cheapest to
      make, but parabolic mirrors are better for making images of objects at
      infinity. A sphere has equal curvature everywhere, but a parabola has
      tighter curvature at its center and gentler curvature at the sides.
    },
    {
      'width'=>'wide',
      'sidecaption'=>true
    }
  )
%>

One way of decreasing aberration is to use a small-diameter
mirror or lens, or block most of the light with an opaque
screen with a hole in it, so that only light that comes in
close to the axis can get through. Either way, we are using
a smaller portion of the lens or mirror whose curvature will
be more shallow, thereby making the shallow-mirror (or
thin-lens) approximation more accurate. Your eye does this
by narrowing down the pupil to a smaller hole. In a camera,
there is either an automatic or manual adjustment, and
narrowing the opening is called ``stopping down.'' The
disadvantage of stopping down is that light is wasted, so
the image will be dimmer or a longer exposure must be used.

<%
  fig(
    'spherical-and-parabolic-stopped',
    %q{%
      Even though the spherical mirror
      (solid line) is not well adapted for viewing an object at infinity, we can
      improve its performance greatly by stopping it down. Now the only part of the mirror
      being used is the central portion, where its shape is virtually indistinguishable
      from a parabola (dashed line).
    },
    {
      'width'=>'wide',
      'sidecaption'=>true
    }
  )
%>

What I would suggest you take away from this discussion for
the sake of your general scientific education is simply an
understanding of what an aberration is, why it occurs, and
how it can be reduced, not detailed facts about specific
types of aberrations.

<%
  fig(
    'hubble-aberration',
    %q{%
      The Hubble Space Telescope was placed into orbit
      with faulty optics in 1990. Its main mirror was supposed to have been nearly parabolic, since
      it is an astronomical telescope, meant for producing images of objects at infinity.
      However, contractor Per\-kin Elmer had delivered a faulty mirror, which produced aberrations.
      The large photo shows astronauts putting correcting mirrors in place in 1993. The two small
      photos show images produced by the telescope before and after the fix.
    },
    {
      'width'=>'wide',
      'sidecaption'=>true
    }
  )
%>


<% end_sec() %>
