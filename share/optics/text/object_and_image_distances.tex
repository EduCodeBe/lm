\noindent Equipment:

\begin{indentedblock}
	optical benches

	converging mirrors

	illuminated objects
\end{indentedblock}

1. Set up the optical bench with the mirror at zero on the
centimeter scale. Set up the illuminated object on the bench as well.

2.  Each group will locate the image for their own value of
the object distance, by finding where a piece of paper has
to be placed in order to see the image on it. (The
instructor will do one point as well.) Note that you will
have to tilt the mirror a little so that the paper on which
you project the image doesn't block the light from
the illuminated object.

Is the image real or virtual? How do you know? Is it
inverted, or uninverted? 

Draw a ray diagram.
\vspace{70mm}


3. Measure the image distance and write your result in the
table on the board. Do the same for the magnification.

4. What do you notice about the trend of the data on the
board? Draw a second ray diagram with a different object
distance, and show why this makes sense. Some tips for doing
this correctly: (1) For simplicity, use the point on the
object that is on the mirror's axis. (2) You need to trace
two rays to locate the image. To save work, don't just do
two rays at random angles. You can either use the on-axis
ray as one ray, or do two rays that come off at the same
angle, one above and one below the axis. (3) Where each ray
hits the mirror, draw the normal line, and make sure the ray
is at equal angles on both sides of the normal.

5. We will find the mirror's focal length from the
instructor's data-point. Then, using this focal length,
calculate a theoretical prediction of the image distance,
and write it on the board next to the experimentally
determined image distance.
