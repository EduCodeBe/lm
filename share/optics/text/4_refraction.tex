Economists normally consider free markets to be the natural
way of judging the monetary value of something, but social
scientists also use questionnaires to gauge the relative
value of privileges, disadvantages, or possessions that
cannot be bought or sold. They ask people to \emph{imagine}
that they could trade one thing for another and ask which
they would choose. One interesting result is that the
average light-skinned person in the U.S. would rather lose
an arm than suffer the racist treatment routinely endured by
African-Americans. Even more impressive is the value of
sight. Many prospective parents can imagine without too much
fear having a deaf child, but would have a far more
difficult time coping with raising a blind one.

So great is the value attached to sight that some have
imbued it with mystical aspects. Joan of Arc saw visions,
and my college has a ``vision statement.'' Christian
fundamentalists who perceive a conflict between \index{evolution}evolution
and their religion have claimed that the eye is such a
perfect device that it could never have arisen through a
process as helter-skelter as evolution, or that it could not
have evolved because half of an \index{eye!evolution of}eye
would be useless. In fact, the structure of an eye is
fundamentally dictated by physics, and it has arisen
separately by evolution somewhere between eight and 40
times, depending on which biologist you ask. We humans have
a version of the eye that can be traced back to the
evolution of a light-sensitive ``eye spot'' on the head of
an ancient invertebrate. A sunken pit then developed so that
the eye would only receive light from one direction,
allowing the organism to tell where the light was coming
from. (Modern \index{flatworm}flatworms have this type of
eye.) The top of the pit then became partially covered,
leaving a hole, for even greater directionality (as in the
\index{nautilus}nautilus). At some point the cavity became
filled with jelly, and this jelly finally became a lens,
resulting in the general type of eye that we share with the
bony fishes and other vertebrates. Far from being a perfect
device, the vertebrate eye is marred by a serious design
flaw due to the lack of planning or intelligent design in
evolution: the nerve cells of the retina and the blood
vessels that serve them are all in front of the light-sensitive
cells, blocking part of the light. \index{Squid}Squids and
other \index{mollusc}molluscs, whose eyes evolved on a
separate branch of the evolutionary tree, have a more
sensible arrangement, with the light-sensitive cells out in front.

<% begin_sec("Refraction",0) %>\index{refraction!defined}

<% begin_sec("Refraction") %>


<% marg(m4_ifelse(__sn,1,[:80:],[:110:])) %>
<%
  fig(
    'eye-cross-section',
    %q{A human eye.}
  )
%>
\smspacebetweenfigs
<%
  fig(
    'eye-anatomy',
    %q{The anatomy of the eye.}
  )
%>
\smspacebetweenfigs
<%
  fig(
    'eye-simplified',
    %q{%
      A simplified optical diagram of the eye. Light rays are bent when
      they cross from the air into the eye. (A little of the incident rays' energy goes into
      the reflected rays rather than the ones transmitted into the eye.)
    }
  )
%>

<% end_marg %>
The fundamental physical phenomenon at work in the eye is
that when light crosses a boundary between two media (such
as air and the eye's jelly), part of its energy is
reflected, but part passes into the new medium. In the ray
model of light, we describe the original ray as splitting
into a reflected ray and a transmitted one (the one that
gets through the boundary). Of course the reflected ray goes
in a direction that is different from that of the original
one, according to the rules of reflection we have already
studied. More surprisingly --- and this is the crucial point
for making your eye focus light --- the transmitted ray is
bent somewhat as well. This bending phenomenon is called
\index{refraction!defined}\emph{refraction}. The origin of the
word is the same as that of the word ``fracture,'' i.e., the
ray is bent or ``broken.'' (Keep in mind, however, that
light rays are not physical objects that can really be
``broken.'') Refraction occurs with all waves, not just light waves.

The actual anatomy of the eye, \figref{eye-anatomy}, is quite complex, but in
essence it is very much like every other optical device
based on refraction. The rays are bent when they pass
through the front surface of the eye, \figref{eye-simplified}. Rays that enter
farther from the central axis are bent more, with the result
that an image is formed on the retina. There is only one
slightly novel aspect of the situation. In most human-built
optical devices, such as a movie projector, the light is
bent as it passes into a lens, bent again as it reemerges,
and then reaches a focus beyond the lens. In the eye,
however, the ``screen'' is inside the eye, so the rays are
only refracted once, on entering the jelly, and never emerge again.

A common misconception is that the ``lens'' of the
\index{eye!human}eye is what does the focusing. All the
transparent parts of the eye are made of fairly similar
stuff, so the dramatic change in medium is when a ray
crosses from the air into the eye (at the outside surface of
the cornea). This is where nearly all the refraction takes
place. The lens medium differs only slightly in its optical
properties from the rest of the eye, so very little
refraction occurs as light enters and exits the lens. The
lens, whose shape is adjusted by muscles attached to it, is
only meant for fine-tuning the focus to form images of
near or far objects.

<% end_sec() %>
<% begin_sec("Refractive properties of media") %>

What are the rules governing refraction? The first thing to
observe is that just as with reflection, the new, bent part
of the ray lies in the same plane as the normal (perpendicular)
and the incident ray, \figref{refr-3-rays}.

<% marg(m4_ifelse(__sn,1,[:50:],[:90:])) %>
<%
  fig(
    'refr-3-rays',
    %q{%
      The incident, reflected, and transmitted (refracted) rays
      all lie in a plane that includes the normal (dashed line).
    }
  )
%>
\smspacebetweenfigs
<%
  fig(
    'refr-angles',
    %q{%
      The angles $\theta_1$ and $\theta_2$ are related to
      each other, and also depend on the properties of the two media. Because
      refraction is time-reversal symmetric, there is no need to label the rays
      with arrowheads.
    }
  )
%>
\smspacebetweenfigs
<%
  fig(
    'refr-t-reversal',
    %q{%
      Refraction has time-reversal symmetry. Regardless of
      whether the light is going into or out of the water, the relationship
      between the two angles is the same, and the ray is closer to the normal
      while in the water.
    }
  )
%>

<% end_marg %>
If you try shooting a beam of light at the boundary between
two substances, say water and air, you'll find that
regardless of the angle at which you send in the beam, the
part of the beam in the water is always closer to the normal
line, \figref{refr-angles}. It doesn't matter if the ray is entering the
water or leaving, so refraction is symmetric with respect
to time-reversal, \figref{refr-t-reversal}.

If, instead of water and air, you try another combination of
substances, say plastic and gasoline, again you'll find that
the ray's angle with respect to the normal is consistently
smaller in one and larger in the other. Also, we find that
if substance A has rays closer to normal than in B, and
B has rays closer to normal than in C, then A has rays
closer to normal than C. This means that we can rank-order
all materials according to their refractive properties.
Isaac Newton did so, including in his list many amusing
substances, such as ``Danzig vitriol'' and ``a pseudo-topazius,
being a natural, pellucid, brittle, hairy stone, of a yellow
color.'' Several general rules can be inferred from such a list:

\begin{itemize}

\item Vacuum lies at one end of the list. In refraction across
the interface between vacuum and any other medium, the other
medium has rays closer to the normal.

\item Among gases, the ray gets closer to the normal if you
increase the density of the gas by pressurizing it more.

\item The refractive properties of liquid mixtures and solutions
vary in a smooth and systematic manner as the proportions of
the mixture are changed.

\item Denser substances usually, but not always, have rays
closer to the normal.

\end{itemize}

The second and third rules provide us with a method for
measuring the density of an unknown sample of gas, or the
concentration of a solution. The latter technique is very
commonly used, and the CRC Handbook of Physics and
Chemistry, for instance, contains extensive tables of the
refractive properties of sugar solutions, cat urine, and so on.

\index{Snell's law}
<% end_sec() %>
<% begin_sec("Snell's law") %>

The numerical rule governing refraction was discovered by
Snell, who must have collected experimental data something
like what is shown on this graph and then attempted by trial
and error to find the right equation. The equation he came up with was
\begin{equation*}
        \frac{\sin\theta_1}{\sin\theta_2}          =  \text{constant}\eqquad.
\end{equation*}
The value of the constant would depend on the combination of
media used. For instance, any one of the data points in the
graph would have sufficed to show that the constant was 1.3
for an air-water interface (taking air to be substance 1 and
water to be substance 2).
<% marg(50) %>
<%
  fig(
    'refr-graph',
    %q{The relationship between the angles in refraction.}
  )
%>
<% end_marg %>

Snell further found that if media A and B gave a constant
$K_{AB}$ and media B and C gave a constant $K_{BC}$,
then refraction at an interface between A and C would be
described by a constant equal to the product, $K_{AC}=K_{AB}K_{BC}$.
This is exactly what one would expect if the constant
depended on the ratio of some number characterizing one
medium to the number characteristic of the second medium.
This number is called the \index{index of refraction!defined}
\emph{index of refraction} of the medium, written as $n$ in
equations. Since measuring the angles would only allow him
to determine the \emph{ratio} of the indices of refraction
of two media, Snell had to pick some medium and define it as
having $n=1$. He chose to define vacuum as having $n=1$.
(The index of refraction of air at normal atmospheric
pressure is 1.0003, so for most purposes it is a good
approximation to assume that air has $n=1$.) He also had to
decide which way to define the ratio, and he chose to define
it so that media with their rays closer to the normal would
have larger indices of refraction. This had the advantage
that denser media would typically have higher indices of
refraction, and for this reason the index of refraction is
also referred to as the optical density. Written in terms of
indices of refraction, Snell's equation becomes
\begin{equation*}
  \frac{\sin\theta_1}{\sin\theta_2} = \frac{n_2}{n_1}\eqquad,
\end{equation*}
but rewriting it in the form

m4_ifelse(__sn,1,[::],[:\pagebreak:])

\begin{equation*}
n_1 \sin \theta_1=n_2 \sin \theta_2
\end{equation*}
\begin{longnoteafterequation}
    [relationship
    between angles of rays at the interface between media with
    indices of refraction $n_1$ and $n_2$; angles are defined
    with respect to the normal]
\end{longnoteafterequation}
\noindent makes us less likely to get the 1's and 2's mixed up, so
this the way most people remember Snell's law. A few indices
of refraction are given in the back of the book.

<% self_check('index-of-refraction',<<-'SELF_CHECK'
(1) What would the graph look like for two substances with
the same index of refraction?

(2) Based on the graph, when does refraction at an air-water
interface change the direction of a ray most strongly?
  SELF_CHECK
  ) %>

\begin{eg}{Finding an angle using Snell's law}\label{eg:yellow-submarine}
\egquestion A submarine shines its searchlight up toward the
surface of the water. What is the angle $\alpha $ shown in the figure?

\eganswer The tricky part is that Snell's law refers to the
angles with respect to the normal. Forgetting this is a very
common mistake. The beam is at an angle of $30\degunit$ with
respect to the normal in the water. Let's refer to the air
as medium 1 and the water as 2. Solving Snell's law
for $\theta_1$, we find
\begin{equation*}
                \theta_1         = \sin^{-1}\left(\frac{n_2}{n_1}\sin\theta_2\right)\eqquad.
\end{equation*}
As mentioned above, air has an index of refraction very
close to 1, and water's is about 1.3, so we find $\theta_1=40\degunit$.
The angle $\alpha $ is therefore $50\degunit$.
\end{eg}
<% marg(50) %>
<%
  fig(
    'eg-yellow-submarine',
    %q{Example \ref{eg:yellow-submarine}.}
  )
%>

<% end_marg %>

\index{index of refraction!related to speed of
light}
<% end_sec() %>
<% begin_sec("The index of refraction is related to the speed of light.") %>

What neither Snell nor Newton knew was that there is a very
simple interpretation of the index of refraction. This may
come as a relief to the reader who is taken aback by the
complex reasoning involving proportionalities that led to
its definition. Later experiments showed that the index of
refraction of a medium was inversely proportional to the
speed of light in that medium. Since $c$ is defined as the
speed of light in vacuum, and $n=1$ is defined as the index
of refraction of vacuum, we have
\begin{equation*}
        n=\frac{c}{v}\eqquad.
\end{equation*}
\begin{longnoteafterequation}
[$n=$ medium's index of refraction, $v=$ speed of
light in that medium, $c=$ speed of light in a vacuum]
\end{longnoteafterequation}

Many textbooks start with this as the definition of the
index of refraction, although that approach makes the
quantity's name somewhat of a mystery, and leaves students
wondering why $c/v$ was used rather than $v/c$. It should
also be noted that measuring angles of refraction is a far
more practical method for determining $n$ than direct
measurement of the speed of light in the substance of interest.

<% end_sec() %>
<% begin_sec("A mechanical model of Snell's law") %>
\index{Snell's law!mechanical model of}

Why should refraction be related to the speed of light? The
mechanical model shown in the figure may help to make this
more plausible. Suppose medium 2 is thick, sticky mud, which
slows down the car. The car's right wheel hits the mud
first, causing the right side of the car to slow down. This
will cause the car to turn to the right until it moves far
enough forward for the left wheel to cross into the mud.
After that, the two sides of the car will once again be
moving at the same speed, and the car will go straight.

<% marg(40) %>
<%
  fig(
    'mechanical-model',
    %q{A mechanical model of refraction.}
  )
%>
<% end_marg %>
Of course, light isn't a car. Why should a beam of light
have anything resembling a ``left wheel'' and ``right
wheel?'' After all, the mechanical model would predict that
a motorcycle would go straight, and a motorcycle seems like
a better approximation to a ray of light than a car. The
whole thing is just a model, not a description of physical reality.
m4_ifelse(__sn,1,[::],[:\vspace{0mm plus 5mm}:])
<%
  fig(
    'refr-derivation',
    %q{A derivation of Snell's law.},
    {
      'width'=>'fullpage'
    }
  )
%>

<% end_sec() %>
<% begin_sec("A derivation of Snell's law") %>\label{subsubsec:snell-derivation}\index{Snell's law!derivation of}

However intuitively appealing the mechanical model may be,
light is a wave, and we should be using wave models to
describe refraction. In fact Snell's law can be derived
quite simply from wave concepts. Figure \figref{refr-derivation} shows
the refraction of a water wave. The water in the upper left part of the tank
is shallower, so the speed of the waves is slower there, and their
wavelengths is shorter. The reflected part of the wave is also very faintly
visible.

m4_ifelse(__sn,1,[::],[:\pagebreak:])

In the close-up view on the right, the dashed lines are normals to the
interface. The two marked angles on the right side are both equal to
$\theta_1$, and the two on the left to $\theta_2$.

Trigonometry gives
\begin{align*}
\sin \theta_1 &= \lambda_1/h \qquad  \text{and} \\
\sin \theta_2 &= \lambda_2/h\eqquad.
\end{align*}
Eliminating $h$ by dividing the equations, we find
\begin{equation*}
 \frac{\sin\theta_1}{\sin\theta_2}=\frac{\lambda_1}{\lambda_2}\eqquad.
\end{equation*}
The frequencies of the two waves must be equal or else they
would get out of step, so by $v=f\lambda $ we know that
their wavelengths are proportional to their velocities.
Combining $\lambda\propto v$ with $v\propto 1/n$ gives $\lambda\propto 1/n$, so we find
\begin{equation*}
 \frac{\sin\theta_1}{\sin\theta_2}=\frac{n_2}{n_1}\eqquad,
\end{equation*}
which is one form of Snell's law.

\begin{eg}{Ocean waves near and far from shore}
Ocean waves are formed by winds, typically on the open sea,
and the wavefronts are perpendicular to the direction of the
wind that formed them. At the beach, however, you have
undoubtedly observed that waves tend come in with their
wavefronts very nearly (but not exactly) parallel to the
shoreline. This is because the speed of water waves in
shallow water depends on depth: the shallower the water, the
slower the wave. Although the change from the fast-wave
region to the slow-wave region is gradual rather than
abrupt, there is still refraction, and the wave motion is
nearly perpendicular to the normal in the slow region.
\end{eg}

\index{refraction!and color}\index{color}
<% end_sec() %>
<% begin_sec("Color and refraction") %>

In general, the speed of light in a medium depends both on
the medium and on the wavelength of the light. Another way
of saying it is that a medium's index of refraction varies
with wavelength. This is why a prism can be used to split up
a beam of white light into a rainbow. Each wavelength of
light is refracted through a different angle.

<% end_sec() %>
<% begin_sec("How much light is reflected, and how much is transmitted?") %>

In __section_or_chapter(bounded-waves) we developed an equation for the percentage of the
wave energy that is transmitted and the percentage reflected
at a boundary between media. This was only done in the case
of waves in one dimension, however, and rather than discuss
the full three dimensional generalization it will be more
useful to go into some qualitative observations about what
happens. First, reflection happens only at the interface
between two media, and two media with the same index of
refraction act as if they were a single medium. Thus, at the
interface between media with the same index of refraction,
there is no reflection, and the ray keeps going straight.
Continuing this line of thought, it is not surprising that
we observe very little reflection at an interface between
media with similar indices of refraction.

<% marg(m4_ifelse(__sn,1,[:120:],[:20:])) %>
<%
  fig(
    'total-internal-cable',
    %q{%
      Total internal reflection in a fiber-optic
      cable.
    }
  )
%>
\spacebetweenfigs
<%
  fig(
    'total-internal-endoscope',
    %q{%
      A simplified drawing of a surgical
      endoscope. The first lens forms a real image at one end of a bundle
      of optical fibers. The light is transmitted through the bundle, and
      is finally magnified by the eyepiece.
    }
  )
%>
\spacebetweenfigs
<%
  fig(
    'ulcer',
    %q{Endoscopic images of a duodenal ulcer.}
  )
%>

<% end_marg %>
The next thing to note is that it is possible to have
situations where no possible angle for the refracted ray can
satisfy Snell's law. Solving Snell's law for $\theta_2$, we find
\begin{equation*}
 \theta_2 = \sin^{-1}\left(\frac{n_1}{n_2}\sin\theta_1\right)\eqquad,
\end{equation*}
and if $n_1$ is greater than $n_2$, then there will be large
values of $\theta_1$ for which the quantity $(n_1/n_2)\sin\theta $
 is greater than one, meaning that your calculator
will flash an error message at you when you try to take the
inverse sine. What can happen physically in such a
situation? The answer is that all the light is reflected, so
there is no refracted ray.        This phenomenon is known as
\index{total internal reflection}\emph{total internal reflection},
and is used in the fiber-optic cables that nowadays carry
almost all long-distance telephone calls. The electrical
signals from your phone travel to a switching center, where
they are converted from electricity into light. From there,
the light is sent across the country in a thin transparent
fiber. The light is aimed straight into the end of the
fiber, and as long as the fiber never goes through any turns
that are too sharp, the light will always encounter the edge
of the fiber at an angle sufficiently oblique to give total
internal reflection. If the fiber-optic cable is thick
enough, one can see an image at one end of whatever the
other end is pointed at.

Alternatively, a bundle of cables can be used, since a
single thick cable is too hard to bend. This technique for
seeing around corners is useful for making surgery less
traumatic. Instead of cutting a person wide open, a surgeon
can make a small ``keyhole'' incision and insert a bundle of
fiber-optic cable (known as an \index{endoscope}endoscope) into the body.

Since rays at sufficiently large angles with respect to the
normal may be completely reflected, it is not surprising
that the relative amount of reflection changes depending on
the angle of incidence, and is greatest for large angles of incidence.

m4_ifelse(__sn,1,[::],[:\vspace{0mm plus 15mm}\pagebreak:])

\startdqs

\begin{dq}
What index of refraction should a fish have in order to
be invisible to other fish?
\end{dq}

\begin{dq}
Does a surgeon using an endoscope need a source of light
inside the body cavity? If so, how could this be done
without inserting a light bulb through the incision?
\end{dq}

\begin{dq}
A denser sample of a gas has a higher index of refraction
than a less dense sample (i.e., a sample under lower
pressure), but why would it not make sense for the index of
refraction of a gas to be proportional to density?
\end{dq}

\begin{dq}
The earth's atmosphere gets thinner and thinner as you go
higher in altitude. If a ray of light comes from a star
that is below the zenith, what will happen to it as it comes
into the earth's atmosphere?
\end{dq}

\begin{dq}
Does total internal reflection occur when light in a
denser medium encounters a less dense medium, or the other
way around? Or can it occur in either case?
\end{dq}

<% end_sec() %>
<% end_sec() %>
<% begin_sec("Lenses",3) %>\index{lens}

Figures \subfigref{sc-lenses-flame}{1} and \subfigref{sc-lenses-flame}{2} show examples of lenses forming images.
There is essentially nothing for you to learn about imaging
with lenses that is truly new. You already know how to
construct and use ray diagrams, and you know about real and
virtual images. The concept of the focal length of a lens is
the same as for a curved mirror. The equations for locating
images and determining magnifications are of the same form.
It's really just a question of flexing your mental muscles
on a few examples. The following self-checks and discussion
questions will get you started.  I've also made a video that demonstrates
some applications and how to explain them with ray diagrams:
\url{https://youtu.be/gL8awy6PWLQ}.

<%
  fig(
    'sc-lenses-flame',
    %q{%
      1. A converging lens forms an image
      of a candle flame. 2. A diverging lens.
    },
    {
      'width'=>'wide',
      'sidecaption'=>true
    }
  )
%>

<% self_check('lenses-flame',<<-'SELF_CHECK'
(1) In figures \subfigref{sc-lenses-flame}{1} and \subfigref{sc-lenses-flame}{2}, classify the images as real or virtual.

(2) Glass has an index of refraction that is greater than
that of air. Consider the topmost ray in figure \subfigref{sc-lenses-flame}{1}. Explain
why the ray makes a slight left turn upon entering the lens,
and another left turn when it exits.

(3) If the flame in figure \subfigref{sc-lenses-flame}{2} was moved closer to the lens,
what would happen to the location of the image?
  SELF_CHECK
  ) %>

\startdqs

\begin{dq}\label{dq:lens-no-bending-at-waist}
In figures \subfigref{sc-lenses-flame}{1} and \subfigref{sc-lenses-flame}{2}, the front and back surfaces are
parallel to each other at the center of the lens. What will
happen to a ray that enters near the center, but not
necessarily along the axis of the lens? Draw a BIG ray diagram,
and show a ray that comes from off axis.
\end{dq}

\emph{In discussion questions \ref{dq:real-image-symmetry}-\ref{dq:rose-image-focus}, don't draw ultra-detailed
ray diagrams as in \ref{dq:lens-no-bending-at-waist}.}

\begin{dq}\label{dq:real-image-symmetry}
Suppose you wanted to change the setup in figure \subfigref{sc-lenses-flame}{1} so
that the location of the actual flame in the figure would
instead be occupied by an image of a flame. Where would you
have to move the candle to achieve this? What about in \subfigref{sc-lenses-flame}{2}?
\end{dq}

\begin{dq}\label{dq:lens-image-types}
There are three qualitatively different types of image
formation that can occur with lenses, of which figures \subfigref{sc-lenses-flame}{1}
and \subfigref{sc-lenses-flame}{2} exhaust only two. Figure out what the third
possibility is. Which of the three possibilities can result
in a magnification greater than one? Cf.~problem \ref{hw:listmirrorimages}, p.~\pageref{hw:listmirrorimages}.
\end{dq}

m4_ifelse(__sn,1,[:\pagebreak:],[::])

\begin{dq}
Classify the examples shown in figure \figref{rose} according to
the types of images delineated in discussion question \ref{dq:lens-image-types}.
\end{dq}

\begin{dq}
In figures \subfigref{sc-lenses-flame}{1} and \subfigref{sc-lenses-flame}{2}, the only rays drawn were those
that happened to enter the lenses. Discuss this in
relation to figure \figref{rose}.
\end{dq}

\begin{dq}\label{dq:rose-image-focus}
In the right-hand side of figure \figref{rose}, the image viewed
through the lens is in focus, but the side of the rose that
sticks out from behind the lens is not. Why?
\end{dq}

<%
  fig(
    'rose',
    %q{Two images of a rose created by the same lens and recorded with the same camera.},
    {
      'width'=>'fullpage'
    }
  )
%>

<% end_sec() %>
<% begin_sec("The lensmaker's equation",nil,'',{'optional'=>true}) %>\index{lensmaker's equation}

<% marg(m4_ifelse(__sn,1,[:-50:],[:5:])) %>
<%
  fig(
    'radii-of-curvature',
    %q{The radii of curvature appearing in the lensmaker's equation.}
  )
%>
<% end_marg %>
The focal length of a spherical mirror is simply $r/2$, but
we cannot expect the focal length of a lens to be given by
pure geometry, since it also depends on the index of
refraction of the lens. Suppose we have a lens whose front
and back surfaces are both spherical. (This is no great loss
of generality, since any surface with a sufficiently shallow
curvature can be approximated with a sphere.) Then if the
lens is immersed in a medium with an index of refraction of
1, its focal length is given approximately by
\begin{equation*}
 f = \left[(n-1)\left|\frac{1}{r_1}\pm\frac{1}{r_2}\right|\right]^{-1}\eqquad,
\end{equation*}
where $n$ is the index of refraction and $r_1$ and $r_2$ are
the radii of curvature of the two surfaces of the lens. This
is known as the lensmaker's equation. In my opinion it is
not particularly worthy of memorization. The positive sign
is used when both surfaces are curved outward or both are
curved inward; otherwise a negative sign applies. The proof
of this equation is left as an exercise to those readers who
are sufficiently brave and motivated.

<% end_sec() %>
%%%%%%%%%%%%%%%%%%%%%%%%%%%%%%%%%%%%%%%%%%%%%%%%%%%%%%%%%%%%%%%%%%%%%%%%%%%%%%%%%%%%%%%%%%%%%%%%%%%%
m4_ifelse(__sn,1,[:\vfill:],[::])
<% begin_sec("Dispersion",nil,'optical-dispersion') %>\index{wave!dispersive}\index{dispersion}
For most materials, we observe that the index of refraction depends slightly on wavelength,
being highest at the blue end of the visible spectrum and lowest at the red. For example,
white light disperses into a rainbow when it passes through a prism,
\figref{dispersion-by-prism}. Even when the waves involved aren't light waves, and even when
refraction isn't of interest, the dependence of wave speed on wavelength is referred to
as dispersion.\index{dispersion} Dispersion inside spherical raindrops is responsible for
the creation of rainbows in the sky, and in an optical instrument such as the eye or a camera it
is responsible for a type of aberration called chromatic aberration 
(__subsection_or_section(aberration) and problem \ref{hw:refractorvsreflector}).\index{aberration!chromatic}
As we'll see in __subsection_or_section(dispersive-waves), dispersion causes a wave that is not a pure
sine wave to have its shape distorted as it travels, and also causes the speed at which energy and information are
transported by the wave to be different from what one might expect from a naive calculation.
m4_ifelse(__sn,0,[:%:],[:The microscopic reasons for dispersion of light in matter are discussed in optional __subsection_or_section(microscopic-refraction).:])
<% marg(m4_ifelse(__sn,1,80,0)) %>
<%
  fig(
    'dispersion-by-prism',
    %q{Dispersion of white light by a prism. White light is a mixture of all the wavelengths of the
       visible spectrum. Waves of different wavelengths undergo different amounts of refraction.}
  )
%>
<% end_marg %>

<% end_sec() %>
%%%%%%%%%%%%%%%%%%%%%%%%%%%%%%%%%%%%%%%%%%%%%%%%%%%%%%%%%%%%%%%%%%%%%%%%%%%%%%%%%%%%%%%%%%%%%%%%%%%%
m4_ifelse(__sn,1,[:\vfill:],[::])
<% begin_sec("The principle of least time for refraction",nil,'least-time-refraction',{'optional'=>true}) %>

We have seen previously how the rules governing straight-line
motion of light and reflection of light can be derived from
the principle of least time. What about refraction? In the
figure, it is indeed plausible that the bending of the ray
serves to minimize the time required to get from a point A
to point B. If the ray followed the unbent path shown with
a dashed line, it would have to travel a longer distance in
the medium in which its speed is slower. By bending the
correct amount, it can reduce the distance it has to cover
in the slower medium without going too far out of its way.
It is true that Snell's law gives exactly the set of angles
that minimizes the time required for light to get from one
point to another. The proof of this fact is left as an exercise
(problem \ref{hw:least-time-refraction}, p.~\pageref{hw:least-time-refraction}).\index{least time, principle of}
<% marg(m4_ifelse(__sn,1,[:50:],[:120:])) %>
<%
  fig(
    'refr-least-time',
    %q{The principle of least time applied to refraction.}
  )
%>
<% end_marg %>

<% end_sec() %>
%%%%%%%%%%%%%%%%%%%%%%%%%%%%%%%%%%%%%%%%%%%%%%%%%%%%%%%%%%%%%%%%%%%%%%%%%%%%%%%%%%%%%%%%%%%%%%%%%%%%
m4_ifelse(__sn,0,[::],[:
%--- For SN only.
<% begin_sec("Microscopic description of refraction",4,'microscopic-refraction',{'optional'=>true}) %>
Given that the speed of light is different in different media, we've seen two different explanations
(on p.~\pageref{subsubsec:snell-derivation} and in subsection \ref{subsec:least-time-refraction} above) of why
refraction must occur. What we haven't yet explained is why the speed of light does depend on the medium.

<%
  fig(
    'dispersion-of-glass',
    %q{%
      Index of refraction of silica glass, redrawn from Kitamura, Pilon, and Jonasz, Applied Optics 46 (2007) 8118,
      reprinted online at \url{http://www.seas.ucla.edu/~pilon/Publications/AO2007-1.pdf}.
    },
    {
      'width'=>'wide',
      'sidecaption'=>true
    }
  )
%>

A good clue as to what's going on comes from the figure \figref{dispersion-of-glass}. The relatively
minor variation of the index of refraction within the visible spectrum was misleading. At certain 
specific frequencies, $n$ exhibits wild swings in the positive and negative directions. After each
such swing, we reach a new, lower plateau on the graph. These frequencies are resonances. For example, the
visible part of the spectrum lies on the left-hand tail of a resonance at about $2\times10^{15}\ \zu{Hz}$, corresponding
to the ultraviolet part of the spectrum. This resonance arises from the vibration of the electrons,
which are bound to the nuclei as if by little springs. Because this resonance is narrow, the effect on visible-light
frequencies is relatively small, but it is stronger at the blue end of the spectrum than at the red end.
Near each resonance, not only does the index of refraction fluctuate wildly, but the glass becomes
nearly opaque; this is because the vibration becomes very strong, causing energy to be dissipated as heat.
The ``staircase'' effect is the same one visible in any resonance, e.g.,
figure \figref{fwhm-omega} on p.~\pageref{fig:fwhm-omega}: oscillators have a finite response for
$f \ll f_0$, but the response approaches zero for $f \gg f_0$.

So far, we have a qualitative explanation of the frequency-variation of the loosely defined ``strength''
of the glass's effect on a light wave, but we haven't explained why the effect is observed as a change
in speed, or why each resonance is an up-down swing rather than
a single positive peak. To understand these effects in more detail, we need to consider
the phase response of the oscillator.
As shown in the bottom panel of
figure \figref{resonance} on p.~\pageref{fig:resonance}, the phase response reverses itself as we pass
through a resonance. 

Suppose that a plane wave is normally incident on the left side of a thin sheet of glass,
\subfigref{lorentz-model}{1}, at $f \ll f_0$.
The light wave observed on the right side consists of a superposition of the incident wave consisting
of $\vc{E}_0$ and $\vc{B}_0$ with
a secondary wave $\vc{E}^*$ and $\vc{B}^*$
generated by the oscillating charges in the glass.
Since the frequency is far below resonance, the response $q\vc{x}$ of a vibrating charge $q$ is
in phase with the driving force $\vc{E}_0$. The current is the derivative of this quantity,
and therefore 90 degrees ahead of it in phase. The magnetic field generated by a sheet of current
has been analyzed in __subsection_or_section(superposwires), and the result, shown in figure
\figref{sheeteb} on p.~\pageref{fig:sheeteb}, is just what we would expect from the right-hand rule.
We find, \subfigref{lorentz-model}{1}, that the secondary wave is 90 degrees ahead of the incident one
in phase. The incident wave still exists on the right side of the sheet, but it is superposed with the
secondary one. Their addition is shown in \subfigref{lorentz-model}{2} using the complex number
representation introduced in __subsection_or_section(impedance).
The superposition of the two fields lags
behind the incident wave, which is the effect we would expect if
the wave had traveled more slowly through the glass.

<% marg(80) %>
<%
  fig(
    'lorentz-model',
    %q{1. A wave incident on a sheet of glass excites current in the glass, which produce a
        secondary wave. 2. The secondary wave superposes with the original wave,
        as represented in the complex-number representation introduced in __subsection_or_section(impedance).}
  )
%>
<% end_marg %>

In the case $f \gg f_0$, the same analysis applies except that the phase of the secondary wave is
reversed. The transmitted wave is advanced rather than retarded
in phase. This explains the dip observed in figure \figref{dispersion-of-glass} after each spike.



All of this is in accord with our understanding of relativity, ch.~\ref{ch:rel},
in which we saw that the universal speed $c$ was to be understood fundamentally as a conversion
factor between the units used to measure time and space --- not as the speed of light.
Since $c$ isn't defined as the speed of light, it's of no fundamental importance whether light has
a different speed in matter than it does in vacuum. In fact, the picture we've built up here
is one in which all of our electromagnetic waves travel at $c$; propagation at some other speed
is only what appears to happen because of the superposition of the $(\vc{E}_0,\vc{B}_0)$
and $(\vc{E}^*,\vc{B}^*)$ waves, both of which move at $c$.

But it is worrisome that
at the frequencies where $n<1$, the speed of the
wave is greater than $c$. According to special relativity, information is never supposed to be
transmitted at speeds greater than $c$, since this would produce situations in which a signal
could be received before it was transmitted! This difficulty is resolved in
__subsection_or_section(dispersive-waves), where we show that there are two different velocities that
can be defined for a wave in a dispersive medium, the phase velocity and the group velocity. The group
velocity is the velocity at which information is transmitted, and it is always less than $c$.

<% end_sec() %>
:])
%%%%%%%%%%%%%%%%%%%%%%%%%%%%%%%%%%%%%%%%%%%%%%%%%%%%%%%%%%%%%%%%%%%%%%%%%%%%%%%%%%%%%%%%%%%%%%%%%%%%
m4_ifelse(__sn,1,[::],[:
%--- It's a biology application, so don't include it in SN.
%\pagebreak
<% begin_sec("Case study: the eye of the jumping spider",nil,'jumping-spider',{'optional'=>true}) %>
Figure \figref{jumping-spider} shows an exceptionally cute jumping spider. The jumping spider does not build a web.
It stalks its prey like a cat, so it needs excellent eyesight. In some ways, its visual system is more sophisticated
and more functional than that of a human, illustrating how evolution does not progress systematically toward ``higher'' forms of life.
<%
  fig(
    'jumping-spider',
    %q{Top left: A female jumping spider, \emph{Phidippus mystaceus}. Top right: Cross-section in a horizontal plane, viewed from above, of the jumping spider
    \emph{Metaphidippus aeneolus}. The eight eyes are shown in white. Bottom: Close-up of one of the large principal eyes.},
    {'width'=>'wide','sidecaption'=>true}
  )
%>

One way in which the spider outdoes us is that it has eight eyes to our two. (Each eye is simple, not compound like that of a fly.)
The reason this works well has to do with the trade-off between magnification and field of view. The elongated principal eyes at the
front of the head have a large value of $d_i$, resulting in a large magnification $M=d_i/d_o$. This high magnification is used for
sophisticated visual tasks like distinguishing prey from a potential mate. (The pretty stripes on the legs in the photo are probably
evolved to aid in making this distinction, which is a crucial one on a Saturday night.) As always with a high magnification, this results
in a reduction in the field of view: making the image bigger means reducing the amount of the potential image that can
actually fit on the retina. The animal has tunnel vision in these forward eyes. To allow it to glimpse prey from other angles, it has
the additional eyes on the sides of its head. These are not elongated, and the smaller $d_i$ gives a smaller magnification but
a larger field of view. When the spider sees something moving in these eyes, it turns its body so that it can take a look with
the front eyes. The tiniest pair of eyes are too small to be useful. These vestigial organs, like the maladaptive human appendix,
are an example of the tendency of evolution to produce unfortunate accidents due to the lack of intelligent design. The use of multiple
eyes for these multiple purposes is far superior to the two-eye arrangement found in humans, octopuses, etc., especially because of its
compactness. If the spider had only two spherical eyes, they would have to have the same front-to-back dimension in order to produce
the same acuity, but then the eyes would take up nearly all of the front of the head.

Another beautiful feature of these eyes is that they will never need bifocals. A human eye uses muscles to adjust for
seeing near and far, varying $f$ in order to achive a fixed $d_i$ for differing values of $d_o$. On older models of
\emph{H. sap.}, this poorly engineered feature is usually one of the first things to break down. The spider's front eyes have
muscles, like a human's, that rotate the tube, but none that vary $f$, which is fixed. However, the retina consists of four
separate layers at slightly different values of $d_i$. The figure only shows the detailed cellular structure of the rearmost
layer, which is the most acute. Depending on $d_o$, the image may lie closest to any one of the four layers, and the spider can then
use that layer to get a well-focused view. The layering is also believed to help eliminate problems caused by the variation of the
index of refraction with wavelength (cf.~problem \ref{hw:refractorvsreflector}, p.~\pageref{hw:refractorvsreflector}).

Although the spider's eye is different in many ways from a human's or an octopus's, it shares the same fundamental
construction, being essentially a lens that forms a real image on a screen inside a darkened chamber. From this
perspective, the main difference is simply the scale, which is miniaturized by about a factor of $10^2$ in the linear dimensions.
How far down can this scaling go? Does an amoeba or a white blood cell lack an eye merely because it doesn't have a nervous
system that could make sense of the signals? In fact there is an optical limit on the miniaturization of any eye or camera.
The spider's eye is already so small that on the scale of the bottom panel in figure \figref{jumping-spider},
one wavelength of visible light would be easily distinguishable --- about the length of the comma in this sentence.
Chapter \ref{ch:wave-optics} is about optical effects that occur when the wave nature of light is important,
and problem \ref{hw:spider-diffraction-limited} on p.~\pageref{hw:spider-diffraction-limited} specifically
addresses the effect on this spider's vision.
<% end_sec() %>
:])
