The intensity of a beam of light is defined as the power per unit area
incident on a perpendicular surface. Suppose that a beam of light in a
medium with index of refraction $n$ reaches the surface of the medium,
with air on the outside. Its incident angle with respect to the normal is $\theta$.
(All angles are in radians.)
Only a fraction $f$ of the energy is
transmitted, the rest being reflected.  Because of this, we might
expect that the transmitted ray would always be less intense than the incident one. But because
the transmitted ray is refracted, it  becomes narrower, causing an additional
change in intensity by a factor $g>1$. The product of these factors $I=fg$
can be greater than one. The purpose of this problem is to estimate the maximum amount
of intensification.\\
We will use the small-angle approximation $\theta\ll 1$ freely, in order to make
the math tractable. In our previous studies of waves, we have only studied the
factor $f$ in the one-dimensional case where $\theta=0$. The generalization
to $\theta\ne0$ is rather complicated and depends on the polarization,
but for unpolarized light, we can use
Schlick's approximation, 
\begin{equation*}
  f(\theta) = f(0)(1-\cos\theta)^5,
\end{equation*}
where the value of $f$ at $\theta=0$ is found as in problem \ref{hw:maxtransmission}
on p.~\pageref{hw:maxtransmission}.\\
(a) Using small-angle approximations, obtain an expression for $g$ of the form
$g\approx 1+P\theta^2$, and find the constant $P$.<% hw_answer %>\hwendpart
(b) Find an expression for $I$ that includes the two leading-order terms in $\theta$. We will
call this expression $I_2$.
Obtain a simple expression for the angle at which $I_2$ is maximized.
As a check on your work, you should find that for $n=1.3$, $\theta=63\degunit$. (Trial-and-error
maximization of $I$ gives $60\degunit$.)\hwendpart
(c) Find an expression for the maximum value of $I_2$. You should find that for
$n=1.3$, the maximum intensification is 31\%.

