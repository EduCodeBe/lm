Suppose we have a polygonal room whose walls are mirrors, and there a pointlike
light source in the room. In most such examples, every point in the room ends up
being illuminated by the light source after some finite number of reflections.
A difficult mathematical question, first posed in the middle of the last century,
is whether it is ever possible to have an example in which the whole room is
\emph{not} illuminated. (Rays are assumed to be absorbed if they strike exactly at a vertex of the polygon,
or if they pass exactly through the plane of a mirror.)

The problem was finally solved in 1995 by G.W. Tokarsky,
who found an example of a room that was not illuminable from a certain point.
Figure \ref{hw:illuminable} shows a slightly simpler example found two years later
by D. Castro. If a light source is placed at either of the locations shown with
dots, the other dot remains unilluminated, although every other point is lit up.
It is not straightforward to prove rigorously that Castro's solution has this property.
However, the plausibility of the solution can be demonstrated as follows.

Suppose the light source is placed at the right-hand dot. Locate all the images
formed by single reflections. Note that they form a regular pattern. Convince yourself
that none of these images illuminates the left-hand dot. Because of the regular
pattern, it becomes plausible that even if we form images of images, images of images
of images, etc., none of them will ever illuminate the other dot.

There are various other versions of the problem, some of which remain unsolved.
The book by Klee and Wagon gives a good introduction to the topic, although it
predates Tokarsky and Castro's work.

References:\\
G.W. Tokarsky, ``Polygonal Rooms Not Illuminable from Every Point.'' Amer. Math. Monthly 102, 867-879, 1995. \\
D. Castro, ``Corrections.'' Quantum 7, 42, Jan. 1997.\\
V. Klee and S. Wagon, \emph{Old and New Unsolved Problems in Plane Geometry and Number Theory}. Mathematical
Association of America, 1991.
