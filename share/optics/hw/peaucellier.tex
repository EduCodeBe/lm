A mechanical linkage is a device that changes one type of motion into
another. The most familiar example occurs in a gasoline
car's engine, where a connecting rod changes the linear motion of the
piston into circular motion of the crankshaft.
The top panel of the figure shows a mechanical linkage invented by Peaucellier in 1864, and
independently by Lipkin around the same time. It consists of six rods joined
by hinges, the four short ones forming a rhombus. Point O is fixed in space,
but the apparatus is free to rotate about O. Motion at P is transformed
into a different motion at $\zu{P}'$ (or vice versa).\index{Peaucellier linkage}\index{Lipkin linkage}

Geometrically, the linkage is a mechanical implementation of
the ancient problem of inversion in a circle.
Considering the case in which the rhombus is folded flat, let
the $k$ be the distance from O to the point where P and  $\zu{P}'$
coincide. Form the circle of radius $k$ with its center at O.
As P and $\zu{P}'$ move in and out, points on the inside of the
circle are always mapped to points on its outside, such that
$rr'=k^2$. That is, the linkage is a type of analog computer
that exactly solves the problem of finding the inverse of a
number $r$. Inversion in a circle has many remarkable geometrical properties, discussed in
H.S.M. Coxeter, \emph{Introduction to Geometry}, Wiley, 1961.
If a pen is inserted through a hole at P, and $\zu{P}'$ is traced over
a geometrical figure, the Peaucellier linkage can be used to draw a kind
of image of the figure.

A related problem is the construction of pictures, like the one in the bottom
panel of the figure, called anamorphs.\index{anamorph}
The drawing of the column on the paper is highly distorted, but when the
reflecting cylinder is placed in the correct spot on top of the page,
an undistorted image is produced inside the cylinder. (Wide-format movie technologies
such as Cinemascope are based on similar principles.)

Show that the Peaucellier linkage does \emph{not} convert correctly between
an image and its anamorph, and design a modified version of the linkage that does.
Some knowledge of analytic geometry will be helpful.
