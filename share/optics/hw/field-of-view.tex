<% hw_solution %> Two standard focal lengths for camera lenses are 50
mm (standard) and 28 mm (wide-angle). To see how the focal
lengths relate to the angular size of the field of view, it
is helpful to visualize things as represented in the figure.
Instead of showing many rays coming from the same point on
the same object, as we normally do, the figure shows two
rays from two different objects. Although the lens will
intercept infinitely many rays from each of these points, we
have shown only the ones that pass through the center of the
lens, so that they suffer no angular deflection. (Any
angular deflection at the front surface of the lens is
canceled by an opposite deflection at the back, since the
front and back surfaces are parallel at the lens's center.)
What is special about these two rays is that they are aimed
at the edges of one 35-mm-wide frame of film; that is, they
show the limits of the field of view. Throughout this
problem, we assume that $d_o$ is much greater than $d_i$.
(a) Compute the angular width of the camera's field of view
when these two lenses are used. (b) Use small-angle
approximations to find a simplified equation for the angular
width of the field of view, $\theta $, in terms of the focal
length, $f$, and the width of the film, $w$. Your equation
should not have any trig functions in it. Compare the
results of this approximation with your answers from part
a. (c) Suppose that we are holding constant the aperture
(amount of surface area of the lens being used to collect
light). When switching from a 50-mm lens to a 28-mm lens,
how many times longer or shorter must the exposure be in
order to make a properly developed picture, i.e., one that is
not under- or overexposed? [Based on a problem by Arnold Arons.]
