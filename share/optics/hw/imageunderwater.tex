(a) Light is being reflected diffusely from an object
1.000 m underwater. The light that comes up to the
surface is refracted at the water-air interface. If the
refracted rays all appear to come from the same point, then
there will be a virtual image of the object in the water,
above the object's actual position, which will be visible to
an observer above the water. Consider three rays, A, B
and C, whose angles in the water with respect to the
normal are $\theta_i=0.000\degunit$, $1.000\degunit$ and
$20.000\degunit$ respectively. Find the depth of the point at
which the refracted parts of A and B appear to have
intersected, and do the same for A and C. Show that the
intersections are at nearly the same depth, but not quite.
[Check: The difference in depth should be about 4 cm.]

(b) Since all the refracted rays do not quite appear to have
come from the same point, this is technically not a virtual
image. In practical terms, what effect would this
have on what you see?

(c) In the case where the angles are all small, use algebra
and trig to show that the refracted rays do appear to come
from the same point, and find an equation for the depth of
the virtual image. Do not put in any numerical values for
the angles or for the indices of refraction --- just keep
them as symbols. You will need the approximation $\sin\theta\approx \tan\theta\approx \theta$, which is
valid for small angles measured in radians.
