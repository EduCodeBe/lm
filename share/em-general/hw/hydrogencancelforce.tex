If you put two hydrogen atoms near each other, they will
feel an attractive force, and they will pull together to
form a molecule. (Molecules consisting of two hydrogen
atoms are the normal form of hydrogen gas.) Why do they
feel a force if they are near each other, since each is
electrically neutral? Shouldn't the attractive and
repulsive forces all cancel out exactly? Use the raisin
cookie model. (Students who have taken chemistry often try
to use fancier models to explain this, but if you can't
explain it using a simple model, you probably don't
understand the fancy model as well as you thought you did!)
