<% marg(0) %>
<%
  fig(
    'chernobyl-map',
    %q{%
      A map showing levels of radiation near the site of the Chernobyl nuclear accident.
    }
  )
%>
<% end_marg %>

<% begin_sec("Biological effects of ionizing radiation",nil,'biological-effects-radiation') %>
<% begin_sec("Units used to measure exposure") %>
        As a science educator, I find it frustrating that nowhere in
        the massive amount of journalism devoted to nuclear safety
        does one ever find any numerical statements about
        the amount of radiation to which people have been exposed.
        Anyone capable of understanding sports statistics
        or weather reports ought to be able to understand such
        measurements, as long as something like the following
        explanatory text was inserted somewhere in the article:

        Radiation exposure is measured in units of \index{Sievert (unit)}Sieverts (Sv). The average person
        is exposed to about 2000 $\mu$Sv  (microSieverts) each year from natural background sources.

      With this context, people would be able to come to informed conclusions.
     For example, figure \figref{chernobyl-map} shows a scary-looking map of the levels of radiation
      in the area surrounding the 1986 nuclear accident at Chernobyl,\index{Chernobyl} Ukraine,
      the most serious that has ever occurred.
      At the boundary of the
      most highly contaminated (bright red) areas, people would be exposed to about 13,000 $\mu$Sv  per year, or
      about four times the natural background level. In the pink areas, which are still densely populated, the exposure
      is comparable to the natural level found in a high-altitude city such as Denver.


        What is a Sievert? It measures the amount of energy per
        kilogram deposited in the body by ionizing radiation,
        multiplied by a ``quality factor'' to account for the
        different health hazards posed by alphas, betas, gammas,
        neutrons, and other types of radiation. Only ionizing
        radiation is counted, since nonionizing radiation simply
        heats one's body rather than killing cells or altering
        \index{DNA}DNA. For instance, alpha particles are typically
        moving so fast that their kinetic energy is sufficient to
        ionize thousands of atoms, but it is possible for an alpha
        particle to be moving so slowly that it would not have
        enough kinetic energy to ionize even one atom.
<% marg(-100) %>
<%
  fig(
    'zombies',
    %q{%
      In this classic zombie flick, a newscaster speculates that the dead have been reanimated
      due to radiation brought back to earth by a space probe.
    }
  )
%>
\spacebetweenfigs
<%
  fig(
    'fifty-foot-woman',
    %q{%
      Radiation doesn't mutate entire multicellular organisms.
    }
  )
%>
<% end_marg %>

      Unfortunately, most people don't know much about radiation and tend to react to it based on 
      unscientific cultural notions. These may, as in figure \figref{zombies}, be based on
      fictional tropes silly enough to
      require the suspension of disbelief by the audience, but they can also be more subtle.
      People of my kids' generation are more familiar with the 2011 Fukushima nuclear accident
      than with the much more serious Chernobyl accident. The news coverage of Fukushima showed
      scary scenes of devastated landscapes and distraught evacuees, implying that people had been
      killed and displaced by the release of radiation from the reaction. In fact, there were no deaths
      at all due to the radiation released at Fukushima, and no excess cancer deaths are statistically
      predicted in the future. The devastation and the death toll of 16,000 were caused by the earthquake
      and tsunami, which were also what damaged the plant.

<% end_sec %>
<% begin_sec("Effects of exposure") %>

        Notwithstanding the pop culture images like figure \figref{fifty-foot-woman}, it is not possible for a multicellular
        animal to become ``mutated'' as a whole. In most cases, a
        particle of ionizing radiation will not even hit the DNA,
        and even if it does, it will only affect the DNA of a single
        cell, not every cell in the animal's body. Typically, that
        cell is simply killed, because the DNA becomes unable to
        function properly. Once in a while, however, the DNA may be
        altered so as to make that cell cancerous. For instance,
        skin cancer can be caused by UV light hitting a single skin
        cell in the body of a sunbather. If that cell becomes
        cancerous and begins reproducing uncontrollably, she will
        end up with a tumor twenty years later.

        Other than cancer, the only other dramatic effect that can
        result from altering a single cell's DNA is if that cell
        happens to be a sperm or ovum, which can result in nonviable
        or mutated offspring. Men are relatively immune to
        reproductive harm from radiation, because their sperm cells
        are replaced frequently. Women are more vulnerable because
        they keep the same set of ova as long as they live.
<% end_sec %>
<% begin_sec("Effects of high doses of radiation") %>

        A whole-body exposure of 5,000,000 $\mu$Sv 
        will kill a person
        within a week or so. Luckily, only a small number of humans
        have ever been exposed to such levels: one scientist working
        on the Manhattan  Project, some victims of the Nagasaki and
        Hiroshima explosions, and 31 workers at Chernobyl. Death
        occurs by massive killing of cells, especially in the
        blood-producing cells of the bone marrow.

<% end_sec %>
<% begin_sec("Effects of low doses radiation") %>
        Lower levels, on the order of 1,000,000 $\mu$Sv, were inflicted
        on some people at Nagasaki and \index{Hiroshima}Hiroshima.
        No acute symptoms result from this level of exposure, but
        certain types of cancer are significantly more common among
        these people. It was originally expected that the radiation
        would cause many mutations resulting in birth defects, but
        very few such inherited effects have been observed.

        A great deal of time has been spent debating the effects of
        very low levels of ionizing radiation. The following table gives some sample figures.

\begin{tabular}{p{70mm}r}
  maximum \emph{beneficial} dose per day     & $\sim$ 10,000 $\mu$Sv  \\
  CT scan                                    & $\sim$ 10,000 $\mu$Sv  \\
  natural background per year                & 2,000-7,000 $\mu$Sv  \\
  health guidelines for exposure to a fetus  & 1,000 $\mu$Sv  \\
  flying from New York to Tokyo              & 150 $\mu$Sv  \\
  chest x-ray                                & 50 $\mu$Sv 
\end{tabular}

\noindent Note that the largest number, on the first line of the table, is the maximum \emph{beneficial} dose.
The most useful evidence comes from experiments in animals, which can intentionally be exposed to
significant and well measured doses of radiation under controlled conditions.
Experiments show that low levels of radiation activate cellular damage control mechanisms, increasing
the health of the organism. For example, exposure to radiation up to a certain level makes mice grow
faster; makes guinea pigs' immune systems function better against diptheria; increases fertility in trout and mice;
improves fetal mice's resistance to disease;
increases the life-spans of flour beetles and mice; and reduces mortality from cancer in mice.
This type of effect is called radiation
hormesis.\index{hormesis}\index{radiation hormesis} 
<% marg(80) %>
<%
  fig(
    'hormesis',
    %q{%
      A typical example of radiation hormesis: the health of mice is improved by low levels of radiation.
      In this study, young mice were exposed to fairly high levels of x-rays, while a control group
      of mice was not exposed. The mice were weighed, and their rate of growth was taken as a measure of
      their health. At levels below about 50,000 $\mu$Sv, the radiation had a beneficial effect on the
      health of the mice, presumably by activating cellular damage control mechanisms. The two highest
      data points are statistically significant at the 99\% level. The curve is a fit to a theoretical model.
      Redrawn from T.D. Luckey, \emph{Hormesis with Ionizing Radiation}, CRC Press, 1980.
    }
  )
%>
<% end_marg %>

There is also some evidence that in humans, small doses of radiation increase fertility, reduce genetic abnormalities,
and reduce mortality from cancer. The human data, however, tend to be very poor compared to the animal data.
Due to ethical issues, one cannot do controlled experiments in humans. For example, one of the best sources of
information has been from the survivors of the Hiroshima and Nagasaki bomb blasts, but these people were also
exposed to high levels of carcinogenic chemicals in the smoke from their burning cities; for comparison, firefighters
have a heightened risk of cancer, and there are also significant concerns about cancer from the 9/11 attacks in New York.
The direct empirical evidence about radiation hormesis in humans is therefore not good enough to tell us
anything unambiguous,\footnote{For two opposing viewpoints, see Tubiana et al., ``The Linear No-Threshold Relationship 
Is Inconsistent with Radiation Biologic and Experimental Data,'' Radiology, 251 (2009) 13 and
Little et al., `` Risks Associated with Low Doses and Low Dose Rates of Ionizing Radiation: Why Linearity May Be (Almost) the Best We Can Do,''
Radiology, 251 (2009) 6.} and the most scientifically reasonable approach is to assume that the results in animals
also hold for humans: small doses of radiation in humans are beneficial, rather than harmful. However,
a variety of cultural and historical factors have led to a situation in which public health policy is based
on the assumption, known as ``linear no-threshold'' (LNT),\index{LNT}\index{linear no-threshold}
that even tiny doses of radiation are harmful, and that the risk they carry is proportional
to the dose. In other words, law and policy are made based on the assumption that the effects of radiation
on humans are dramatically different than its effects on mice and guinea pigs. Even with the unrealistic assumption of LNT,
one can still evaluate risks by comparing with natural background radiation. For example, we can see that the effect
of a chest x-ray is about a hundred times smaller than the effect of spending a year in Colorado, where the level of natural
background radiation from cosmic rays is higher than average, due to the high altitude.
Dropping the implausible LNT assumption, we can see that the impact on one's health of spending a year in Colorado is likely to be
\emph{positive}, because the excess radiation is below the maximum beneficial level.

<% end_sec %>
<% begin_sec("The green case for nuclear power") %>\index{global warming}\index{climate change}
        In the late twentieth century, antinuclear activists largely succeeded in bringing
        construction of new nuclear power plants to a halt in the U.S. Ironically, we now know
        that the burning of fossil fuels, which leads to global warming, is a far more grave
        threat to the environment than even the Chernobyl disaster.
        A team of biologists writes: ``During recent visits to Chernobyl, we experienced numerous
        sightings of moose (Alces alces), roe deer (Capreol capreolus), Russian wild boar (Sus scrofa), foxes
        (Vulpes vulpes), river otter (Lutra canadensis), and rabbits (Lepus europaeus) ...
        Diversity of flowers and other plants in the highly radioactive regions is 
        impressive and equals that observed in protected habitats outside the zone ...
        The observation that typical human activity (industrialization, farming, cattle raising,
        collection of firewood, hunting, etc.) is more devastating to biodiversity and abundance  of
        local flora and fauna than is the worst nuclear power plant disaster validates the negative impact
        the exponential growth of human populations has on wildlife.''\footnote{
        Baker and Chesser, 
        Env. Toxicology and Chem. 19 (1231) 2000.  %http://www.nsrl.ttu.edu/chornobyl/wildlifepreserve.htm
        Similar effects have been seen at the Bikini Atoll, the site of a 1954 hydrogen bomb test. Although some species
        have disappeared from the area, the coral reef is in many ways healthier than similar reefs elsewhere, because humans have tended to stay away for fear of
        radiation (Richards et al., Marine Pollution Bulletin 56 (2008) 503). % http://environment.newscientist.com/article/dn13668-nuked-coral-reef-bounces-back.html
        }
<% marg(150) %>
<%
  fig(
    'chernobyl-horses',
    %q{Wild Przewalski's horses prosper in the Chernobyl area.}
  )
%>
\spacebetweenfigs
<%
  fig(
    'polar-bear',
    %q{Fossil fuels have done incomparably more damage to the environment than nuclear power ever has. Polar bears' habitat
       is rapidly being destroyed by global warming.}
  )
%>
<% end_marg %>

        Nuclear power is the only
        source of energy that is sufficient to replace any significant percentage of energy from fossil
        fuels on the rapid schedule demanded by the speed at which global warming is progressing.
        People worried about the downside of nuclear energy might be better off putting their energy
        into issues related to nuclear weapons: the poor stewardship of the former Soviet Union's
        warheads; nuclear proliferation in unstable states such as Pakistan; and the poor safety and environmental
        history of the superpowers' nuclear weapons programs, including the loss of several warheads
        in plane crashes, and the environmental disaster at the Hanford, Washington, weapons plant.

<% end_sec %>
<% begin_sec("Protection from radiation") %>
People do sometimes work with strong enough radioactivity that there is a serious health risk.
Typically the scariest sources are those used in cancer treatment and in medical and biological
research. Also, a dental technician, for example, needs to take precautions to avoid accumulating
a large radiation dose from giving dental x-rays to many patients. There are three general ways to
reduce exposure: time, distance, and shielding. This is why a dental technician doing x-rays wears
a lead apron (shielding) and steps outside of the x-ray room while running an exposure (distance).
Reducing the time of exposure dictates, for example, that a person working with a hot cancer-therapy
source would minimize the amount of time spent near it.

Shielding against alpha and beta particles
is trivial to accomplish. (Alphas can't even penetrate the skin.) Gammas and x-rays interact most
strongly with materials that are dense and have high atomic numbers, which is why lead is so
commonly used. But other materials will also work. For example, the reason that bones show up
so clearly on x-ray images is that they are dense and contain plenty of calcium, which has a higher
atomic number than the elements found in most other body tissues, which are mostly made of water.

Neutrons are difficult to shield
against. Because they are electrically neutral, they don't interact intensely with matter in the same
way as alphas and betas. They only interact if they happen to collide head-on with a nucleus, and
that doesn't happen very often because nuclei are tiny targets. Kinematically, a collision can transfer
kinetic energy most efficiently when the target is as low in mass as possible compared to the projectile.
For this reason, substances that contain a lot of hydrogen make the best shielding against neutrons.
Blocks of paraffin wax from the supermarket are often used for this purpose.
<% end_sec %>

 %%----------------------------------------------
<% end_sec('biological-effects-radiation') %>
