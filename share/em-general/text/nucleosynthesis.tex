<% begin_sec("The creation of the elements",nil,'nucleosynthesis',{'optional'=>true}) %>
  <% begin_sec("Creation of hydrogen and helium in the Big Bang",nil,'bbn') %>

 Did all
        the chemical elements we're made of come into being in the
        big bang?\footnote{The evidence for
        the big bang theory of the origin of the universe was discussed 
        m4_ifelse(__sn,1,[:on p.~\pageref{subsubsec:hubble}:],[:in subsection \ref{subsec:hubble}:])%
        .} Temperatures
        in the first microseconds after the big bang were so high
        that atoms and nuclei could not hold together at all. 
        After things had cooled down enough for nuclei and atoms to
        exist, there was a period of about three minutes during which the
temperature and density were high enough for fusion to occur, but not so
high that atoms could hold together. We have a good, detailed understanding
of the laws of physics that apply under these conditions, so theorists are
able to say with confidence that the only element heavier than hydrogen that
was created in significant quantities was helium.


  <% end_sec('bbn') %>
<% marg(200) %>
<%
  fig(
    'crab-nebula',
    %q{%
      The Crab Nebula is a remnant of a supernova explosion. Almost all the elements
      our planet is made of originated in such explosions.
    }
  )
%>
\spacebetweenfigs
<%
  fig(
    'unilac',
    %q{%
      Construction of the UNILAC accelerator in Germany,
              one of whose uses is for experiments to create very heavy artificial
              elements. In such an experiment, fusion products recoil through
              a device called SHIP (not shown) that separates them based on their charge-to-mass ratios ---
              it is essentially just a scaled-up version of Thomson's apparatus.
              A typical experiment
              runs for several months, and out of the billions of fusion reactions
              induced during this time, only one or two may result in the production
              of superheavy atoms. In all the rest, the fused nucleus breaks up
              immediately. SHIP is used to identify the small number of
              ``good'' reactions and separate them from this intense
              background.
    }
  )
%>
<% end_marg %>

  <% begin_sec("We are stardust",nil,'stardust') %>

        In that case, where did all the other elements come from?
        Astronomers came up with the answer. By studying the
        combinations of wavelengths of light, called spectra,
        emitted by various stars, they had been able to determine
        what kinds of atoms they contained. (We will have more to
        say about spectra at the end of this book.) They found that the stars fell
        into two groups. One type was nearly 100\% hydrogen and
        helium, while the other contained 99\% hydrogen and helium
        and 1\% other elements. They interpreted these as two
        generations of stars. The first generation had formed out of
        clouds of gas that came fresh from the big bang, and their
        composition reflected that of the early universe. The
        nuclear fusion reactions by which they shine have mainly
        just increased the proportion of helium relative to
        hydrogen, without making any heavier elements.
        The members of the first generation that we see today,
        however, are only those that lived a long time. Small stars
        are more miserly with their fuel than large stars, which
        have short lives. The large stars of the first generation
        have already finished their lives. Near the end of its
        lifetime, a star runs out of hydrogen fuel and undergoes a
        series of violent and spectacular   reorganizations as it
        fuses heavier and heavier elements. Very large stars finish
        this sequence of events by undergoing supernova explosions,
        in which some of their material is flung off into the void
        while the rest collapses into an exotic object such as a
        black hole or neutron star.

        The second generation of stars, of which our own sun is an
        example, condensed out of clouds of gas that had been
        enriched in heavy elements due to supernova explosions. It
        is those heavy elements that make up our planet and our bodies.

  <% end_sec('stardust') %>

\startdqs
        
\begin{dq}
        Should the quality factor for neutrinos be very small,  
        because they mostly don't interact with your body?
\end{dq}
 %
\begin{dq}
        Would an alpha source be likely to cause different types
        of cancer depending on whether the source was external to
        the body or swallowed in contaminated food? What about a gamma source?
\end{dq}

<% end_sec('nucleosynthesis') %>
