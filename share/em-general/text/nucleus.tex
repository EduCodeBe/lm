<% begin_sec("The nucleus",nil,'nucleus') %>
  <% begin_sec("Radioactivity",nil,'radioactivity') %>
    <% begin_sec("Becquerel's discovery of radioactivity",nil,'becquerel') %>

        How did physicists figure out that the raisin cookie model
        was incorrect, and that the atom's positive charge was
        concentrated in a tiny, central nucleus? The story begins
        with the discovery of radioactivity by the French chemist
        Becquerel. Up until radioactivity was discovered, all the
        processes of nature were thought to be based on chemical
        reactions, which were rearrangements of combinations of
        atoms. Atoms exert forces on each other when they are close
        together, so sticking or unsticking them would either
        release or store electrical energy. That energy could be
        converted to and from other forms, as when a plant uses the
        energy in sunlight to make sugars and carbohydrates, or when
        a  child eats sugar, releasing the energy in the
        form of kinetic energy.

        Becquerel discovered a process that seemed to release energy
        from an unknown new source that was not chemical. Becquerel,
        whose father and grandfather had also been physicists, spent
        the first twenty years of his professional life as a
        successful civil engineer, teaching physics  on a
        part-time basis. He was awarded the chair of physics at the
        Mus\'{e}e d'Histoire Naturelle in Paris after the death of his
        father, who had previously occupied it. Having now a
        significant amount of time to devote to physics, he began
        studying the interaction of light and matter. He became
        interested in the phenomenon of phosphorescence, in which a
        substance absorbs energy from light, then releases the
        energy via a glow that only gradually goes away. One of the
        substances he investigated was a uranium compound, the salt
        $\zu{UKSO}_5$. One day in 1896, cloudy weather interfered with
        his plan to expose this substance to sunlight in order to
        observe its fluorescence. He stuck it in a drawer,
        coincidentally on top of a blank photographic plate --- the
        old-fashioned glass-backed counterpart of the modern plastic
        roll of film. The plate had been carefully wrapped, but
        several days later when Becquerel checked it in the darkroom
        before using it, he found that it was ruined, as if it had
        been completely exposed to light.

<% marg(112) %>
<%
  fig(
    'becquerel',
    %q{Henri Becquerel (1852-1908).}
  )
%>
\spacebetweenfigs
<%
  fig(
    'becquerel-plate',
    %q{%
      Becquerel's photographic plate. In the exposure at the
      bottom of the image, he has found that he could absorb the radiations, casting
      the shadow of a Maltese cross that was placed between the plate and the
      uranium salts.
    }
  )
%>

<% end_marg %>
        History provides many examples of scientific discoveries
        that happened this way: an alert and inquisitive mind
        decides to investigate a phenomenon that most people would
        not have worried about explaining. Becquerel first determined by
        further experiments that the effect was produced by the
        uranium salt, despite a thick wrapping of paper around the
        plate that blocked out all light. He tried a variety of
        compounds, and found that it was the uranium that did it:
        the effect was produced by any uranium compound, but not by
        any compound that didn't include uranium atoms. The effect
        could be at least partially blocked by a sufficient
        thickness of metal, and he was able to produce silhouettes
        of coins by interposing them between the uranium and the
        plate. This indicated that the effect traveled in a straight
        line., so that it must have been some kind of ray rather
        than, e.g., the seepage of chemicals through the paper. He
        used the word ``radiations,'' since the effect radiated out
        from the uranium salt.

        At this point Becquerel still believed that the uranium
        atoms were absorbing energy from light and then gradually
        releasing the energy in the form of the mysterious rays, and
        this was how he presented it in his first published lecture
        describing his experiments. Interesting, but not earth-shattering.
        But he then tried to determine how long it took for the
        uranium to use up all the energy that had supposedly
        been stored in it by light, and he found that it never
        seemed to become inactive, no matter how long he waited. Not
        only that, but a sample that had been exposed to intense
        sunlight for a whole afternoon was no more or less effective
        than a sample that had always been kept inside. Was this a
        violation of conservation of energy? If the energy didn't
        come from exposure to light, where did it come from?

    <% end_sec('becquerel') %>
    <% begin_sec("Three kinds of ``radiations''",nil,'alpha-beta-gamma') %>
        Unable to determine the source of the energy directly,
        turn-of-the-century physicists instead studied the behavior of the
        ``radiations'' once they had been emitted.
         Becquerel had already shown that the
        radioactivity could penetrate through cloth and paper, so
        the first obvious thing to do was to investigate in more
        detail what thickness of material the radioactivity could
        get through. They soon learned that a certain fraction of
        the radioactivity's intensity would be eliminated by even a
        few inches of air, but the remainder was not eliminated by
        passing through more air. Apparently, then, the radioactivity
        was a mixture of more than one type, of which one was
        blocked by air. They then found that of the part that could
        penetrate air, a further fraction could be eliminated by a
        piece of paper or a very thin metal foil. What was left
        after that, however, was a third, extremely penetrating
        type, some of whose intensity would still remain even after
        passing through a brick wall. They decided that this showed
        there were three types of radioactivity, and without having
        the faintest idea of what they really were, they made up
        names for them. The least penetrating type was arbitrarily
        labeled $\alpha$ (alpha), the first letter of the Greek alphabet,
        and so on through $\beta$ (beta) and finally $\mygamma$
        (gamma) for the most penetrating type.

    <% end_sec('alpha-beta-gamma') %>
    <% begin_sec("Radium: a more intense source of radioactivity",nil,'radium') %>

        The measuring devices used to detect radioactivity were
        crude: photographic plates or even human eyeballs (radioactivity
        makes flashes of light in the jelly-like fluid inside
        the eye, which can be seen by the eyeball's owner if it is
        otherwise very dark). Because the ways of detecting
        radioactivity were so crude and insensitive, further
        progress was hindered by the fact that the amount of
        radioactivity emitted by uranium was not really very great.
        The vital contribution of physicist/chemist Marie Curie and
        her husband Pierre was to discover the element radium, and
        to purify and isolate significant quantities of it. Radium
        emits about a million times more radioactivity per unit mass
        than uranium, making it possible to do the experiments that
        were needed to learn the true nature of radioactivity. The
        dangers of radioactivity to human health were then unknown,
        and Marie died of leukemia thirty years later. (Pierre was
        run over and killed by a horsecart.)

    <% end_sec('radium') %>
    <% begin_sec("Tracking down the nature of alphas, betas, and gammas",nil,'identifying-radiation') %>

        As radium was becoming available, an apprentice scientist
        named Ernest Rutherford arrived in England from his native
        New Zealand and began studying radioactivity at the
        Cavendish Laboratory.  The young colonial's first success
        was to measure the mass-to-charge ratio of beta rays.  The
        technique was essentially the same as the one Thomson had
        used to measure the mass-to-charge ratio of cathode rays by
        measuring their deflections in electric and magnetic fields.
         The only difference was that instead of the cathode of a
        vacuum tube, a nugget of radium was used to supply the beta
        rays.  Not only was the technique the same, but so was the
        result.  Beta rays had the same $m/q$ ratio as cathode rays,
        which suggested they were one and the same.  Nowadays, it
        would make sense simply to use the term ``electron,'' and
        avoid the archaic ``cathode ray'' and ``beta
        particle,'' but the old labels are still widely used, and it
        is unfortunately necessary for physics students to memorize
        all three names for the same thing.

        At first, it seemed that neither alphas or gammas could be
        deflected in electric or magnetic fields, making it appear
        that neither was electrically charged.  But soon Rutherford
        obtained a much more powerful magnet, and was able to use it
        to deflect the alphas but not the gammas.  The alphas had a
        much larger value of $m/q$ than the betas (about 4000 times
        greater), which was why they had been so hard to deflect. 
        Gammas are uncharged, and were later found to be a form of light.
        
        <% marg(50) %>
<%
  fig(
    'radon',
    %q{%
      A simplified version of Rutherford's 1908 experiment, showing
              that alpha particles were doubly ionized helium atoms.
    }
  )
%>
\spacebetweenfigs
<%
  fig(
    'nuclear-fuel-pellets',
    %q{%
      These pellets of uranium fuel will be inserted into
      the metal fuel rod and used in
      a nuclear reactor. The pellets emit alpha and beta radiation, which the gloves
      are thick enough to stop.
    }
  )
%>

<% end_marg %>
        The $m/q$ ratio of alpha particles turned out to be the same
        as those of two different types of ions, $\zu{He}^{++}$ (a helium atom
        with two missing electrons) and $\zu{H}_2^+$ (two hydrogen atoms
        bonded into a molecule, with one electron missing), so it
        seemed likely that they were one or the other of those. The
        diagram shows a simplified version of Rutherford's ingenious
        experiment proving that they were $\zu{He}^{++}$ ions. The gaseous
        element radon, an alpha emitter, was introduced into one
        half of a double glass chamber. The glass wall dividing the
        chamber was made extremely thin, so that some of the rapidly
        moving alpha particles were able to penetrate it. The other
        chamber, which was initially evacuated, gradually began to
        accumulate a population of alpha particles (which would
        quickly pick up electrons from their surroundings and become
        electrically neutral). Rutherford then determined that it
        was helium gas that had appeared in the second chamber. Thus
        alpha particles were proved to be $\zu{He}^{++}$ ions. The nucleus was
        yet to be discovered, but in modern terms, we would describe
        a $\zu{He}^{++}$ ion as the nucleus of a He atom.         

        To summarize, here are the three types of radiation emitted
        by radioactive elements, and their descriptions in modern 
        terms:\index{alpha particle|see {alpha decay}}\index{beta particle|see {beta decay}}\index{gamma ray|see {gamma decay}}%
        \index{alpha decay!nature of emitted particle}\index{beta decay!nature of emitted particle}\index{gamma decay!nature of emitted particle}

        \begin{tabular}{|l|l|l|}
                \hline
                $\alpha$ particle        &stopped by a few inches of air        &He nucleus\\
                \hline
                $\beta$ particle        &stopped by a piece of paper                &electron\\
                \hline
                $\mygamma$ ray        &penetrates thick shielding                &a type of light\\
                \hline
        \end{tabular}


\startdq
\begin{dq}
        Most sources of radioactivity emit alphas, betas, and
        gammas, not just one of the three. In the radon experiment,
        how did Rutherford know that he was studying the alphas?
\end{dq}
 %%----------------------------------------------
    <% end_sec('identifying-radiation') %>
  <% end_sec('radioactivity') %>
  <% begin_sec("The planetary model",3,'planetary-model') %>
        The stage was now set for the unexpected discovery that the
        positively charged part of the atom was a tiny, dense lump
        at the atom's center rather than the ``cookie dough'' of the
        raisin cookie model. By 1909, Rutherford was an established
        professor, and had students working under him. For a raw
        undergraduate named Marsden, he picked a research project he
        thought would be tedious but straightforward.

        It was already known that although alpha particles would be
        stopped completely by a sheet of paper, they could pass
        through a sufficiently thin metal foil. Marsden was to work
        with a gold foil only 1000 atoms thick. (The foil was
        probably made by evaporating a little gold in a vacuum
        chamber so that a thin layer would be deposited on a glass
        microscope slide. The foil would then be lifted off the
        slide by submerging the slide in water.) 

<% marg(123) %>
<%
  fig(
    'rutherford',
    %q{Ernest Rutherford (1871-1937).}
  )
%>
\spacebetweenfigs
<%
  fig(
    'rutherfordsetup',
    %q{Marsden and Rutherford's apparatus.}
  )
%>
<% end_marg %>
        Rutherford had already determined in his previous experiments
        the speed of the alpha particles emitted by radium, a
        fantastic $1.5\times10^7$  m/s. The experimenters in
        Rutherford's group visualized them as very small, very fast
        cannonballs penetrating the ``cookie dough'' part of the big
        gold atoms. A piece of paper has a thickness of a hundred
        thousand atoms or so, which would be sufficient to stop them
        completely, but crashing through a thousand would only slow
        them a little and turn them slightly off of their original paths. 

        Marsden's supposedly ho-hum assignment was to use the
        apparatus shown in figure \ref{fig:rutherfordsetup} to measure how often alpha
        particles were deflected at various angles. A tiny lump of
        radium in a box emitted alpha particles, and a thin beam was
        created by blocking all the alphas except those that
        happened to pass out through a tube. Typically deflected in
        the gold by only a small amount, they would reach a screen
        very much like the screen of a TV's picture tube, which
        would make a flash of light when it was hit. Here is the
        first example we have encountered of an experiment in which
        a beam of particles is detected one at a time. This was
        possible because each alpha particle carried so much kinetic
        energy; they were moving at about the same speed as the
        electrons in the Thomson experiment, but had ten thousand times more mass.

        Marsden sat in a dark room, watching the apparatus hour
        after hour and recording the number of flashes with the
        screen moved to various angles. The rate of the flashes was
        highest when he set the screen at an angle close to the line
        of the alphas' original path, but if he watched an area
        farther off to the side, he would also occasionally see an
        alpha that had been deflected through a larger angle. After
        seeing a few of these, he got the crazy idea of moving the
        screen to see if even larger angles ever occurred, perhaps
        even angles larger than 90 degrees.
        
        %
<%
  fig(
    'rutherfordscatt',
    %q{%
      %
              Alpha particles being scattered
              by a gold nucleus.  On this
              scale, the gold atom
              is the size of a car, so all
              the alpha particles shown
              here are ones that just
              happened to come 
              unusually close to the nucleus.
              For these exceptional alpha
              particles, the forces from the
              electrons are unimportant,
              because they are so much
              more distant than the nucleus.
    },
    {
      'width'=>'wide'
    }
  )
%>

        The crazy idea worked: a few alpha particles were deflected
        through angles of up to 180 degrees, and the routine
        experiment had become an epoch-making one. Rutherford said,
        ``We have been able to get some of the alpha particles
        coming backwards. It was almost as incredible as if you
        fired a 15-inch shell at a piece of tissue paper and it came
        back and hit you.'' Explanations were hard to come by in the
        raisin cookie model. What intense electrical forces could
        have caused some of the alpha particles, moving at such
        astronomical speeds, to change direction so drastically?
        Since each gold atom was electrically neutral, it would not
        exert much force on an alpha particle outside it. True, if
        the alpha particle was very near to or inside of a
        particular atom, then the forces would not necessarily
        cancel out perfectly; if the alpha particle happened to come
        very close to a particular electron, the $1/r^2$ form of the
        Coulomb force law would make for a very strong force. But
        Marsden and Rutherford knew that an alpha particle was 8000
        times more massive than an electron, and it is simply not
        possible for a more massive object to rebound backwards from
        a collision with a less massive object while conserving
        momentum and energy. It might be possible in principle for a
        particular alpha to follow a path that took it very close to
        one electron, and then very close to another electron, and
        so on, with the net result of a large deflection, but
        careful calculations showed that such multiple ``close
        encounters'' with electrons would be millions of times too
        rare to explain what was actually observed.
        
        <% marg(70) %>
<%
  fig(
    'planetarymodel',
    %q{The planetary model of the atom.}
  )
%>
<% end_marg %>
         At this point, Rutherford and Marsden dusted off an
        unpopular and neglected model of the atom, in which all the
        electrons orbited around a small, positively charged core or
        ``\index{nucleus!discovery}nucleus,'' just like the planets
        orbiting around the sun. All the positive charge and nearly
        all the mass of the atom would be concentrated in the
        nucleus, rather than spread throughout the atom as in the
        raisin cookie model. The positively charged alpha
        particles would be repelled by the gold atom's nucleus, but
        most of the alphas would not come close enough to any
        nucleus to have their paths drastically altered. The few
        that did come close to a nucleus, however, could rebound
        backwards from a single such encounter, since the nucleus of
        a heavy gold atom would be fifty times more massive than an
        alpha particle. It turned out that it was not even too
        difficult to derive a formula giving the relative frequency
        of deflections through various angles, and this calculation
        agreed with the data well enough (to within 15\%),
        considering the difficulty in getting good experimental
        statistics on the rare, very large angles.

        What had started out as a tedious exercise to get a student
        started in science had ended as a revolution in our
        understanding of nature. Indeed, the whole thing may sound a
        little too much like a moralistic fable of the scientific
        method with overtones of the Horatio Alger genre. The
        skeptical reader may wonder why the planetary model was
        ignored so thoroughly until Marsden and Rutherford's
        discovery. Is science really more of a sociological
        enterprise, in which certain ideas become accepted by the
        establishment, and other, equally plausible explanations are
        arbitrarily discarded? Some social scientists are currently
        ruffling a lot of scientists' feathers with critiques very
        much like this, but in this particular case, there were very
        sound reasons for rejecting the planetary model. As you'll
        learn in more detail later in this course, any charged
        particle that undergoes an acceleration dissipate energy in
        the form of light. In the planetary model, the electrons
        were orbiting the nucleus in circles or ellipses, which
        meant they were undergoing acceleration, just like the
        acceleration you feel in a car going around a curve. They
        should have dissipated energy as light, and eventually they
        should have lost all their energy. Atoms don't spontaneously
        collapse like that, which was why the raisin cookie model,
        with its stationary electrons, was originally preferred.
        There were other problems as well. In the planetary model,
        the one-electron atom would have to be flat, which would be
        inconsistent with the success of molecular modeling with
        spherical balls representing hydrogen and atoms. These
        molecular models also seemed to work best if specific sizes
        were used for different atoms, but there is no obvious
        reason in the planetary model why the radius of an
        electron's orbit should be a fixed number. In view of the
        conclusive Marsden-Rutherford results, however, these became
        fresh puzzles in atomic physics, not reasons for disbelieving
        the planetary model.

    <% begin_sec("Some phenomena explained with the planetary model",4,'successes-of-planetary-model') %>

        The planetary model may not be the ultimate, perfect model
        of the atom, but don't underestimate its power. It already
        allows us to visualize correctly a great many phenomena.

        As an example, let's consider the distinctions among
        nonmetals, metals that are magnetic, and metals that are
        nonmagnetic. As shown in figure \ref{fig:conductionelectrons}, a metal differs from a
        nonmetal because its outermost electrons are free to wander
        rather than owing their allegiance to a particular atom. A
        metal that can be magnetized is one that is willing to line
        up the rotations of some of its electrons so that their axes
        are parallel. Recall that magnetic forces are forces made by
        moving charges; we have not yet discussed the mathematics
        and geometry of magnetic forces, but it is easy to see how
        random orientations of the atoms in the nonmagnetic
        substance would lead to cancellation of the forces.

<% marg(100) %>
<%
  fig(
    'conductionelectrons',
    %q{%
      The planetary model applied to a nonmetal, 1, an unmagnetized
              metal, 2, and a magnetized metal, 3. 
              Note that these figures
               are all simplified in several ways. For
              one thing, the electrons of an individual atom do not all revolve
              around the nucleus in the same plane. It is also very unusual for a
              metal to become so strongly magnetized that 100\% of its atoms have
              their rotations aligned as shown in this figure.
    }
  )
%>
<% end_marg %>

        Even if the planetary model does not immediately answer such
        questions as why one element would be a metal and another a
        nonmetal, these ideas would be difficult or impossible to
        conceptualize in the raisin cookie model.


\startdq

\begin{dq}
        In reality, charges of the same type repel one another and
        charges of different types are attracted. Suppose the rules
        were the other way around, giving repulsion between opposite
        charges and attraction between similar ones. What would
        the universe be like?
\end{dq}


 %%----------------------------------------------
    <% end_sec('successes-of-planetary-model') %>
  <% end_sec('planetary-model') %>
  <% begin_sec("Atomic number",nil,'atomic-number') %>
        As alluded to in a discussion question in the previous
        section, scientists of this period had only a very
        approximate idea of how many units of charge resided in the
        nuclei of the various chemical elements. Although we now
        associate the number of units of nuclear charge with the
        element's position on the periodic table, and call it the
        \index{atomic number!defined}atomic number, they had no
        idea that such a relationship existed. Mendeleev's table
        just seemed like an organizational tool, not something with
        any necessary physical significance. And everything
        Mendeleev had done seemed equally valid if you turned the
        table upside-down or reversed its left and right sides, so
        even if you wanted to number the elements sequentially with
        integers, there was an ambiguity as to how to do it.
        Mendeleev's original table was in fact upside-down
        compared to the modern one.
<%
  fig(
    'periodictable',
    %q{%
        A modern periodic table, labeled with atomic numbers.
        Mendeleev's original table was upside-down compared to this one.
    },
    {
      'width'=>'wide',
      'sidecaption'=>m4_ifelse(__sn,1,[:false:],[:true:]),
      'suffix'=>'b'
    }
  )
%>
\index{periodic table}

        In the period immediately following the discovery of the
        nucleus, physicists only had rough estimates of the charges
        of the various nuclei. In the case of the very lightest
        nuclei, they simply found the maximum number of electrons
        they could strip off by various methods: chemical reactions,
        electric sparks, ultraviolet light, and so on. For example
        they could easily strip off one or two electrons from helium,
        making $\zu{He}^+$ or $\zu{He}^{++}$, but nobody
        could make $\zu{He}^{+++}$, presumably
        because the nuclear charge of helium was only $+2e$.
        Unfortunately only a few of the lightest elements could be
        stripped completely, because the more electrons were
        stripped off, the greater the positive net charge remaining,
        and the more strongly the rest of the negatively charged
        electrons would be held on. The heavy elements' atomic
        numbers could only be roughly extrapolated from the light
        elements, where the atomic number was about half the atom's
        mass expressed in units of the mass of a hydrogen atom.
        Gold, for example, had a mass about 197 times that of
        hydrogen, so its atomic number was estimated to be about
        half that, or somewhere around 100. We now know it to be 79.

        How did we finally find out? The riddle of the nuclear
        charges was at last successfully attacked using two
        different techniques, which gave consistent results. One set
        of experiments, involving x-rays, was performed by the young
        Henry Mosely, whose scientific brilliance was soon to be
        sacrificed in a battle between European imperialists over
        who would own the Dardanelles, during that pointless
        conflict then known as the War to End All Wars, and now
        referred to as World War I.
<%
  fig(
    'rutherfordz',
    %q{%
      An alpha particle has to come much closer
              to the low-charged copper nucleus in order to be deflected
              through the same angle.
    },
    {
      'width'=>'wide',
      'sidecaption'=>true
    }
  )
%>

        Since Mosely's analysis requires several concepts with which
        you are not yet familiar, we will instead describe the
        technique used by James Chadwick at around the same time. An
        added bonus of describing Chadwick's experiments is that
        they presaged the important modern technique of studying
        \emph{collisions} of subatomic particles. In grad school, I
        worked with a professor whose thesis adviser's thesis
        adviser was Chadwick, and he related some interesting
        stories about the man. Chadwick was apparently a little
        nutty and a complete fanatic about science, to the extent
        that when he was held in a German prison camp during World
        War II, he managed to cajole his captors into allowing him
        to scrounge up parts from broken radios so that he could
        attempt to do physics experiments.

        Chadwick's experiment worked like this. Suppose you perform
        two Rutherford-type alpha scattering measurements, first one
        with a gold foil as a target as in Rutherford's original
        experiment, and then one with a copper foil. It is possible
        to get large angles of deflection in both cases, but as
        shown in figure \ref{fig:crosssection}, the alpha particle must be heading
        almost straight for the copper nucleus to get the same angle
        of deflection that would have occurred with an alpha that
        was much farther off the mark; the gold nucleus' charge is
        so much greater than the copper's that it exerts a strong
        force on the alpha particle even from far off. The situation
        is very much like that of a blindfolded person playing
        darts. Just as it is impossible to aim an alpha particle at
        an individual nucleus in the target, the blindfolded person
        cannot really aim the darts. Achieving a very close
        encounter with the copper atom would be akin to hitting an
        inner circle on the dartboard. It's much more likely that
        one would have the luck to hit the outer circle, which
        covers a greater number of square inches. By analogy, if you
        measure the frequency with which alphas are scattered by
        copper at some particular angle, say between 19 and 20
        degrees, and then perform the same measurement at the same
        angle with gold, you get a much higher percentage for
        gold than for copper.
\enlargethispage{-2\baselineskip}
<%
  fig(
    'crosssection',
    %q{%
      An alpha particle must be headed for the ring on the front of
              the imaginary cylindrical pipe in order to produce scattering at an
              angle between 19 and 20 degrees.  The area of this ring is called the
              ``cross-section'' for scattering at 19-20\degunit because it is
              the cross-sectional area of a cut through the pipe. 
    },
    {
      'width'=>'wide',
      'sidecaption'=>true
    }
  )
%>

        In fact, the numerical ratio of the two nuclei's charges can
        be derived from this same experimentally determined ratio.
        Using the standard notation $Z$ for the atomic number
        (charge of the nucleus divided by $e$), the following
        equation can be proved (example \ref{eg:zscatt}):
        \begin{equation*}
                \frac{Z_{gold}^2}{Z_{copper}^2}
                        = \frac{\text{number of alphas scattered by gold at 19-20$\degunit$}}
                                {\text{number of alphas scattered by copper at 19-20$\degunit$}}
        \end{equation*}
        By making such measurements for targets constructed from all
        the elements, one can infer the ratios of all the atomic
        numbers, and since the atomic numbers of the light elements
        were already known, atomic numbers could be assigned to the
        entire periodic table. According to Mosely, the atomic
        numbers of copper, silver and platinum were 29, 47, and 78,
        which corresponded well with their positions on the periodic
        table. Chadwick's figures for the same elements were 29.3,
        46.3, and 77.4, with error bars of about 1.5 times the
        fundamental charge, so the two experiments were in good agreement.

\enlargethispage{-2\baselineskip}
        The point here is absolutely not that you should be ready to
        plug numbers into the above equation for a homework or exam
        question! My overall goal in this chapter is to explain how
        we know what we know about atoms. An added bonus of
        describing Chadwick's experiment is that the approach is
        very similar to that used in modern particle physics
        experiments, and the ideas used in the analysis are closely
        related to the now-ubiquitous concept of a ``cross-section.''
        In the dartboard analogy, the cross-section would be the
        area of the circular ring you have to hit. The reasoning
        behind the invention of the term ``cross-section'' can be
        visualized as shown in figure \ref{fig:crosssection}. In this language,
        Rutherford's invention of the planetary model came from his
        unexpected discovery that there was a nonzero cross-section
        for alpha scattering from gold at large angles, and Chadwick
        confirmed Mosely's determinations of the atomic numbers by
        measuring cross-sections for alpha scattering. 

        \begin{eg}{Proof of the relationship between Z and scattering}\label{eg:zscatt}
        The equation above can be derived by the following not very rigorous
        proof. To deflect the alpha particle by a certain angle requires that
        it acquire a certain momentum component in the direction perpendicular
        to its original momentum. Although the nucleus's force on the alpha
        particle is not constant, we can pretend that it is approximately
        constant during the time when the alpha is within a distance equal to,
        say, 150\% of its distance of closest approach, and that the force is
        zero before and after that part of the motion. (If we chose 120\% or
        200\%, it shouldn't make any difference in the final result, because
        the final result is a ratio, and the effects on the numerator and
        denominator should cancel each other.) In the approximation of
        constant force, the change in the alpha's perpendicular momentum
        component is then equal to $F\Delta  t$. The Coulomb force law says the force
        is proportional to $Z/ r^2$. Although $r$ does change somewhat during the time
        interval of interest, it's good enough to treat it as a constant
        number, since we're only computing the ratio between the two
        experiments' results. Since we are approximating the force as acting
        over the time during which the distance is not too much greater than
        the distance of closest approach, the time interval $\Delta  t$ must be
        proportional to $r$, and the sideways momentum imparted to the alpha,
        $F\Delta  t$, is proportional to
         $( Z/ r^2) r$, or $Z/ r$. If we're comparing alphas
        scattered at the same angle from gold and from copper, then $\Delta  p$ is the
        same in both cases, and the proportionality 
        $\Delta  p\propto  Z/ r$  tells us that the ones
        scattered from copper at that angle had to be headed in along a line
        closer to the central axis by a factor equaling 
        $Z_\zu{gold}/ Z_\zu{copper}$. If you imagine a
        ``dartboard ring'' that the alphas have to hit, then the ring for the
        gold experiment has the same proportions as the one for copper, but it
        is enlarged by a factor equal to $Z_\zu{gold}/ Z_\zu{copper}$. 
        That is, not only is the radius of
        the ring greater by that factor, but unlike the rings on a normal
        dartboard, the thickness of the outer ring is also greater in
        proportion to its radius. When you take a geometric shape and scale it
        up in size like a photographic enlargement, its area is increased in
        proportion to the square of the enlargement factor, so the area of the
        dartboard ring in the gold experiment is greater by a factor equal to         
        $( Z_\zu{gold}/ Z_\zu{copper})^2$.
         Since the alphas are aimed entirely randomly, the chances of an
        alpha hitting the ring are in proportion to the area of the ring,
        which proves the equation given above.
        \end{eg}


        As an example of the modern use of scattering experiments
        and cross-section measurements, you may have heard of the
        recent experimental evidence for the existence of a particle
        called the top quark. Of the twelve subatomic particles
        currently believed to be the smallest constituents of
        matter, six form a family called the quarks, distinguished
        from the other six by the intense attractive forces that
        make the quarks stick to each other. (The other six consist
        of the electron plus five other, more exotic particles.) The
        only two types of quarks found in naturally occurring matter
        are the ``up quark'' and ``down quark,'' which are what
        protons and neutrons are made of, but four other types were
        theoretically predicted to exist, for a total of six. (The
        whimsical term ``quark'' comes from a line by James Joyce
        reading ``Three quarks for master Mark.'') Until recently,
        only five types of quarks had been proven to exist via
        experiments, and the sixth, the top quark, was only
        theorized. There was no hope of ever detecting a top quark
        directly, since it is radioactive, and only exists for a
        zillionth of a second before evaporating. Instead, the
        researchers searching for it at the Fermi National
        Accelerator Laboratory near Chicago measured cross-sections
        for scattering of nuclei off of other nuclei. The experiment
        was much like those of Rutherford and Chadwick, except that
        the incoming nuclei had to be boosted to much higher speeds
        in a particle accelerator. The resulting encounter with a
        target nucleus was so violent that both nuclei were
        completely demolished, but, as Einstein proved, energy can
        be converted into matter, and the energy of the collision
        creates a spray of exotic, radioactive particles, like the
        deadly shower of wood fragments produced by a cannon ball in
        an old naval battle. Among those particles were some top
        quarks. The cross-sections being measured were the
        cross-sections for the production of certain combinations of
        these secondary particles. However different the details,
        the principle was the same as that employed at the turn of
        the century: you smash things together and look at the
        fragments that fly off to see what was inside them. The
        approach has been compared to shooting a clock with a rifle
        and then studying the pieces that fly off to figure out
        how the clock worked.


\startdqs

\begin{dq}
        The diagram, showing alpha particles being deflected by a
        gold nucleus, was drawn with the assumption that alpha
        particles came in on lines at many different distances from
        the nucleus. Why wouldn't they all come in along the same
        line, since they all came out through the same tube?
\end{dq}

\begin{dq}        
        Why does it make sense that, as shown in the figure, the
        trajectories that result in $19\degunit$ and $20\degunit$
        scattering cross each other?
\end{dq}
 %
\begin{dq}
        Rutherford knew the velocity of the alpha particles
        emitted by radium, and guessed that the positively charged
        part of a gold atom had a charge of about $+100e$ (we now
        know it is $+79e)$. Considering the fact that some alpha
        particles were deflected by $180\degunit$, how could he then
        use conservation of energy to derive an upper limit on the
        size of a gold nucleus? (For simplicity, assume the size of
        the alpha particle is negligible compared to that of the
        gold nucleus, and ignore the fact that the gold nucleus
        recoils a little from the collision, picking up a little kinetic energy.)
\end{dq}

 %%----------------------------------------------
  <% end_sec('atomic-number') %>
  <% begin_sec("The structure of nuclei",nil,'nuclear-structure') %>
    <% begin_sec("The proton",nil,'proton') %>

        The fact that the nuclear charges were all integer multiples
        of $e$ suggested to many physicists that rather than being a
        pointlike object, the nucleus might contain smaller
        particles having individual charges of $+e$. Evidence in
        favor of this idea was not long in arriving. Rutherford
        reasoned that if he bombarded the atoms of a very light
        element with alpha particles, the small charge of the target
        nuclei would give a very weak repulsion. Perhaps those few
        alpha particles that happened to arrive on head-on collision
        courses would get so close that they would physically crash
        into some of the target nuclei. An alpha particle is itself
        a nucleus, so this would be a collision between two nuclei,
        and a violent one due to the high speeds involved.
        Rutherford hit pay dirt in an experiment with alpha
        particles striking a target containing nitrogen atoms.
        Charged particles were detected flying out of the target
        like parts flying off of cars in a high-speed crash.
        Measurements of the deflection of these particles in
        electric and magnetic fields showed that they had the same
        charge-to-mass ratio as singly-ionized hydrogen atoms.
        Rutherford concluded that these were the conjectured
        singly-charged particles that held the charge of the
        nucleus, and they were later named protons. The hydrogen
        nucleus consists of a single proton, and in general, an
        element's atomic number gives the number of protons
        contained in each of its nuclei. The mass of the proton is
        about 1800 times greater than the mass of the electron.

    <% end_sec('proton') %>
    <% begin_sec("The neutron",nil,'neutron') %>

        It would have been nice and simple if all the nuclei could
        have been built only from protons, but that couldn't be the
        case. If you spend a little time looking at a periodic
        table, you will soon notice that although some of the atomic
        masses are very nearly integer multiples of hydrogen's mass,
        many others are not. Even where the masses are close whole
        numbers, the masses of an element other than hydrogen is
        always greater than its atomic number, not equal to it.
        Helium, for instance, has two protons, but its mass is four
        times greater than that of hydrogen.

        Chadwick cleared up the confusion by proving the existence
        of a new subatomic particle. Unlike the electron and proton,
        which are electrically charged, this particle is electrically
        neutral, and he named it the neutron. m4_ifelse(__mod,0,[:Chadwick's experiment
        has been described in detail %
        m4_ifelse(__lm_series,1,[:in section \ref{sec:collisions-1d},:],
        [:on p.~\pageref{chadwick-as-example-of-collision},:]) %
        but briefly the method:],[:The method Chadwick used:])  was to expose a sample of the
        light element beryllium to a stream of alpha particles from
        a lump of radium. Beryllium has only four protons, so an
        alpha that happens to be aimed directly at a beryllium
        nucleus can actually hit it rather than being stopped short
        of a collision by electrical repulsion. Neutrons were
        observed as a new form of radiation emerging from the
        collisions, and Chadwick correctly inferred that they were
        previously unsuspected components of the nucleus that had
        been knocked out. As described earlier, Chadwick also
        determined the mass of the neutron; it is very nearly the
        same as that of the proton.

        <% marg(20) %>
<%
  fig(
    'hhe',
    %q{%
      Examples of the construction of atoms: hydrogen (top) and
              helium (bottom). On this scale, the electrons' orbits would be the
              size of a college campus.
    }
  )
%>
<% end_marg %>
        To summarize, atoms are made of three types of particles:

        \begin{tabular}{|l|l|p{30mm}|p{30mm}|}
        \hline
                        & \emph{charge}        & \emph{mass in units of the proton's mass} & \emph{location in atom} \\
        \hline
        proton        & $+e$        & 1                & in nucleus \\
        \hline
        neutron        & 0                & 1.001        & in nucleus \\
        \hline
        electron        & $-e$        & 1/1836        & orbiting nucleus \\
        \hline
        \end{tabular}

        The existence of neutrons explained the mysterious masses of
        the elements. Helium, for instance, has a mass very close to
        four times greater than that of hydrogen. This is because it
        contains two neutrons in addition to its two protons. The
        mass of an atom is essentially determined by the total
        number of neutrons and protons. The total number of neutrons
        plus protons is therefore referred to as the atom's \emph{mass number}.

    <% end_sec('neutron') %>
    <% begin_sec("Isotopes",nil,'isotopes') %>
        We now have a clear interpretation of the fact that helium
        is close to four times more massive than hydrogen, and
        similarly for all the atomic masses that are close to an
        integer multiple of the mass of hydrogen. But what about
        copper, for instance, which had an atomic mass 63.5 times
        that of hydrogen? It didn't seem reasonable to think that it
        possessed an extra half of a neutron! The solution was found
        by measuring the mass-to-charge ratios of singly-ionized
        atoms (atoms with one electron removed). The technique is
        essentially that same as the one used by Thomson for cathode
        rays, except that whole atoms do not spontaneously leap out
        of the surface of an object as electrons sometimes do. 
        Figure \ref{fig:massspectrometer} shows an example of how the ions can be created and
        injected between the charged plates for acceleration.

        Injecting a stream of copper ions into the device, we find a
        surprise --- the beam splits into two parts! Chemists had
        elevated to dogma the assumption that all the atoms of a
        given element were identical, but we find that 69\% of
        copper atoms have one mass, and 31\% have another.  Not only
        that, but both masses are very nearly integer multiples of
        the mass of hydrogen (63 and 65, respectively).  Copper gets
        its chemical identity from the number of protons in its
        nucleus, 29, since chemical reactions work by electric
        forces. But apparently some copper atoms have $63-29=34$
        neutrons while others have $65-29=36$. The atomic mass of
        copper, 63.5, reflects the proportions of the mixture of the
        mass-63 and mass-65 varieties.  The different mass varieties
        of a given element are called \index{isotopes}\emph{isotopes} of that element.

m4_ifelse(__sn,1,[::],[:\pagebreak:])

        Isotopes can be named by giving the mass number as a
        subscript to the left of the chemical symbol, e.g.,
        $^{65}\zu{Cu}$. Examples:

        \begin{tabular}{|l|l|l|l|}
                \hline
                &\emph{protons}                        & \emph{neutrons}        & \emph{mass number} \\
                \hline
                $^1\zu{H}$                &1                        &0                &0+1 = 1  \\
                \hline
                $^4\zu{He}$                &2                        &2                &2+2 = 4  \\
                \hline
                $^{12}\zu{C}$                &6                        &6                &6+6 = 12  \\
                \hline
                $^{14}\zu{C}$                &6                        &8                & 6+8 = 14  \\
                \hline
                $^{262}\zu{Ha}$        &105                &157        &105+157 = 262   \\
                \hline
        \end{tabular}

        <% self_check('reversedthomson',<<-'SELF_CHECK'
        Why are the positive and negative charges of the accelerating
        plates reversed in the isotope-separating apparatus compared
        to the Thomson apparatus?
          SELF_CHECK
  ) %>
<% marg(100) %>
<%
  fig(
    'massspectrometer',
    %q{%
      A version of the Thomson apparatus modified for measuring the
              mass-to-charge ratios of ions rather than electrons. A small sample of
              the element in question, copper in our example, is boiled in the oven
              to create a thin vapor. (A vacuum pump is continuously sucking on the
              main chamber to keep it from accumulating enough gas to stop the beam
              of ions.) Some of the atoms of the vapor are ionized by a spark or by
              ultraviolet light. Ions that wander out of the nozzle and into the
              region between the charged plates are then accelerated toward the top
              of the figure. As in the Thomson experiment, mass-to-charge ratios are
              inferred from the deflection of the beam.
    }
  )
%>
<% end_marg %>

        Chemical reactions are all about the exchange and sharing of
        electrons: the nuclei have to sit out this dance because the
        forces of electrical repulsion prevent them from ever
        getting close enough to make contact with each other.
        Although the protons do have a vitally important effect on
        chemical processes because of their electrical forces, the
        neutrons can have no effect on the atom's chemical
        reactions. It is not possible, for instance, to separate
        $^{63}\zu{Cu}$ from $^{65}\zu{Cu}$ by chemical reactions. This is why
        chemists had never realized that different isotopes existed.
        (To be perfectly accurate, different isotopes do behave
        slightly differently because the more massive atoms move
        more sluggishly and therefore react with a tiny bit less
        intensity. This tiny difference is used, for instance, to
        separate out the isotopes of uranium needed to build a
        nuclear bomb. The smallness of this effect makes the
        separation process a slow and difficult one, which is what
        we have to thank for the fact that nuclear weapons have not
        been built by every terrorist cabal on the planet.m4_ifelse(__calc,1,[: See also example \ref{eg:deuterium-chemistry}, 
        p.~\pageref{eg:deuterium-chemistry}.:]))

    <% end_sec('isotopes') %>
    <% begin_sec("Sizes and shapes of nuclei",nil,'nuclear-sizes') %>

        Matter is nearly all nuclei if you count by weight, but in
        terms of volume nuclei don't amount to much. The radius of
        an individual neutron or proton is very close to 1 fm (1
        fm=$10^{-15}$  m), so even a big lead nucleus with a mass
        number of 208 still has a diameter of only about 13 fm,
        which is ten thousand times smaller than the diameter of a
        typical atom. Contrary to the usual imagery of the nucleus
        as a small sphere, it turns out that many nuclei are
        somewhat elongated, like an American football, and a few
        have exotic asymmetric shapes like pears or kiwi fruits.


\startdqs

\begin{dq}
        Suppose the entire universe was in a (very large) cereal
        box, and the nutritional labeling was supposed to tell a
        godlike consumer what percentage of the contents was nuclei.
        Roughly what would the percentage be like if the labeling
        was according to mass? What if it was by volume? 
\end{dq}

<%
  fig(
    'nuclear-power-plant',
    %q{%
      A nuclear power plant at Cattenom, France.
      Unlike the coal and oil plants that supply most of the U.S.'s electrical power, a nuclear power plant like this
      one releases no pollution or greenhouse gases into the Earth's atmosphere, and therefore doesn't contribute to
      global warming. The white stuff puffing out of this plant is non-radioactive water vapor. Although nuclear
      power plants generate long-lived nuclear waste, this waste arguably poses much less of a threat to the biosphere than
      greenhouse gases would.
    },
    {
      'width'=>'wide',
      'sidecaption'=>true,
    }
  )
%>

   <% end_sec('nuclear-sizes') %>
  <% end_sec('nuclear-structure') %>
  <% begin_sec("The strong nuclear force, alpha decay and fission",0,'strong-force') %>\index{strong nuclear force}

        Once physicists realized that nuclei consisted of positively
        charged protons and uncharged neutrons, they had a problem
        on their hands. The electrical forces among the protons are
        all repulsive, so the nucleus should simply fly apart! The
        reason all the nuclei in your body are not spontaneously
        exploding at this moment is that there is another force
        acting. This force, called the \index{nuclear forces}\index{strong
        nuclear force}\emph{strong nuclear force}, is always attractive,
        and acts between neutrons and neutrons, neutrons and
        protons, and protons and protons with roughly equal
        strength. The strong nuclear force does not have any effect
        on electrons, which is why it does not influence chemical reactions.

        <% marg(50) %>
<%
  fig(
    'strongforcegraph',
    %q{%
      The strong nuclear force cuts off very
              sharply at a range of about 1 fm.
    }
  )
%>
<% end_marg %>
        Unlike electric forces, whose strengths are given by the
        simple Coulomb force law, there is no simple formula for how
        the strong nuclear force depends on distance. Roughly
        speaking, it is effective over ranges of $\sim1$ fm, but
        falls off extremely quickly at larger distances (much faster
        than $1/r^2)$. Since the radius of a neutron or proton is
        about 1 fm, that means that when a bunch of neutrons and
        protons are packed together to form a nucleus, the strong
        nuclear force is effective only between neighbors.
<%
  fig(
    'stronginteractions',
    %q{%
      1. The forces cancel. 2. The forces
              don't cancel. 3. In a heavy nucleus, the large number of electrical
              repulsions can add up to a force that is comparable to the
              strong nuclear attraction. 4. Alpha emission. 5. Fission.
    },
    {
      'width'=>'wide'
    }
  )
%>



        Figure \ref{fig:stronginteractions} illustrates how the strong nuclear force acts to
        keep ordinary nuclei together, but is not able to keep very
        heavy nuclei from breaking apart. In \ref{fig:stronginteractions}/1,
        a proton in the
        middle of a carbon nucleus feels an attractive strong
        nuclear force (arrows) from each of its nearest neighbors.
        The forces are all in different directions, and tend to
        cancel out. The same is true for the repulsive electrical
        forces (not shown). In figure
        \ref{fig:stronginteractions}/2, a proton at the edge of the nucleus
        has neighbors only on one side, and therefore all the strong
        nuclear forces acting on it are tending to pull it back in.
        Although all the electrical forces from the other five
        protons (dark arrows) are all pushing it out of the nucleus,
        they are not sufficient to overcome the strong nuclear forces.

        In a very heavy nucleus, \ref{fig:stronginteractions}/3, a proton that finds itself
        near the edge has only a few neighbors close enough to
        attract it significantly via the strong nuclear force, but
        every other proton in the nucleus exerts a repulsive
        electrical force on it. If the nucleus is large enough, the
        total electrical repulsion may be sufficient to overcome the
        attraction of the strong force, and the nucleus may spit out
        a proton. Proton emission is fairly rare, however; a more
        common type of radioactive decay\footnote{Alpha decay is more common
        because an alpha particle happens to be a very stable
        arrangement of protons and neutrons.}
        in heavy nuclei is alpha
        decay, shown in \ref{fig:stronginteractions}/4.\index{alpha decay}
        The imbalance of the forces is similar,
        but the chunk that is ejected is an alpha particle (two
        protons and two neutrons) rather than a single proton.

        It is also possible for the nucleus to split into two pieces
        of roughly equal size, \ref{fig:stronginteractions}/5, a process known as fission.  
        Note that in addition to the two large fragments, there is a spray of
        individual neutrons. In a nuclear fission bomb or a nuclear fission reactor,
        some of these neutrons fly off and hit other nuclei, causing them to undergo
        fission as well. The result is a chain reaction.\index{chain reaction}

        When a nucleus is able to undergo one of these processes, it
        is said to be radioactive, and to undergo radioactive decay.
        Some of the naturally occurring nuclei on earth are
        radioactive. The term ``radioactive'' comes from Becquerel's
        image of rays radiating out from something, not from radio
        waves, which are a whole different phenomenon. The term
        ``decay'' can also be a little misleading, since it implies
        that the nucleus turns to dust or simply disappears --
        actually it is splitting into two new nuclei with the
        same total number of neutrons and protons, so the term
        ``radioactive transformation'' would have been more
        appropriate. Although the original atom's electrons are mere
        spectators in the process of weak radioactive decay, we
        often speak loosely of ``radioactive atoms'' rather than
        ``radioactive nuclei.''

    <% begin_sec("Randomness in physics",nil,'randomness') %>

        How does an atom decide when to decay? We might imagine that
        it is like a termite-infested house that gets weaker and
        weaker, until finally it reaches the day on which it is
        destined to fall apart. Experiments, however, have not
        succeeded in detecting such ``ticking clock'' hidden below
        the surface; the evidence is that all atoms of a given
        isotope are absolutely identical. Why, then, would one
        uranium atom decay today while another lives for another
        million years? The answer appears to be that it is entirely
        random. We can make general statements about the average
        time required for a certain isotope to decay, or how long it
        will take for half the atoms in a sample to decay (its
        half-life), but we can never predict the behavior of a particular atom.

        This is the first example we have encountered of an
        inescapable randomness in the laws of physics. If this kind
        of randomness makes you uneasy, you're in good company. 
        Einstein's famous quote is ``...I am convinced that He [God]
        does not play dice.``  Einstein's distaste for randomness,
        and his association of determinism with divinity, goes back
        to the Enlightenment conception of the universe as a
        gigantic piece of clockwork that only had to be set in
        motion initially by the Builder. Physics had to be entirely
        rebuilt in the 20th century to incorporate the fundamental
        randomness of physics, and this modern revolution is the
        topic of %
        m4_ifelse(__lm_series,1,[:chapters \ref{ch:randomness}-\ref{ch:atom}.:],
        [:chapter \ref{ch:quantum}.:]) In particular, we will delay
        the mathematical development of the half-life concept until then.

    <% end_sec('randomness') %>
  <% end_sec('strong-force') %>
  <% begin_sec("The weak nuclear force; beta decay",m4_ifelse(__sn,1,[:nil:],[:4:]),'weak-force') %>\index{weak nuclear force}

        All the nuclear processes we've discussed so far have
        involved rearrangements of neutrons and protons, with no
        change in the total number of neutrons or the total number
        of protons. Now consider the proportions of neutrons and
        protons in your body and in the planet earth: neutrons and
        protons are roughly equally numerous in your body's carbon
        and oxygen nuclei, and also in the nickel and iron that make
        up most of the earth. The proportions are about 50-50. But, as discussed in more
        detail on p.~\pageref{subsec:nucleosynthesis},
        the only chemical elements produced in any significant
        quantities by the big m4_ifelse(__mod,0,[:bang\footnote{The evidence for
        the big bang theory of the origin of the universe was discussed
        on p.~\pageref{bigbang}.}:],[:bang:]) were hydrogen (about 90\%) and
        helium (about 10\%). If the early universe was almost
        nothing but hydrogen atoms, whose nuclei are protons, where
        did all those neutrons come from?

        The answer is that there is another nuclear force, the weak
        nuclear force, that is capable of transforming neutrons into
        protons and vice-versa. Two possible reactions are
        \begin{equation*}
                \text{n} \rightarrow \text{p} + \zu{e}^- +  \bar{\nu}\qquad        \hfill        \text{[electron decay]}
        \end{equation*}
        and
        \begin{equation*}
                \text{p} \rightarrow \text{n} + \zu{e}^+ + \nu\eqquad.
                                                \qquad \hfill        \text{[positron decay]}        
        \end{equation*}
        (There is also a third type called electron capture, in
        which a proton grabs one of the atom's electrons and they
        produce a neutron and a neutrino.)\label{electroncapture}
        \index{electron capture}\index{beta decay}\index{electron decay}\index{positron decay}

        Whereas alpha decay and fission are just a redivision of the
        previously existing particles, these reactions involve the
        destruction of one particle and the creation of three new
        particles that did not exist before.

        There are three new particles here that you have never
        previously encountered. The symbol $\zu{e}^+$ stands for an
        antielectron, which is a particle just like the electron in
        every way, except that its electric charge is positive
        rather than negative. Antielectrons are also known as
        positrons. Nobody knows why electrons are so common in the
        universe and antielectrons are scarce. When an antielectron
        encounters an electron, they annihilate each other, producing gamma rays, and this
        is the fate of all the antielectrons that are produced by
        natural radioactivity on earth.\index{antielectron}\index{antimatter}\index{positron}
        Antielectrons are an example of antimatter. A complete atom of antimatter
        would consist of antiprotons, antielectrons, and antineutrons. Although
        individual particles of antimatter occur commonly in nature due to natural
        radioactivity and cosmic rays, only a few complete atoms of antihydrogen have
        ever been produced artificially.

        The notation $\nu $ stands for a particle called a neutrino,
        and $\bar{\nu}$ means an antineutrino.  Neutrinos and antineutrinos
        have no electric charge (hence the name). 

        We can now list all four of the known fundamental forces of physics:

        \begin{itemize}
        \item gravity

        \item electromagnetism

        \item strong nuclear force

        \item weak nuclear force
        \end{itemize}

        The other forces we have learned about, such as friction and
        the normal force, all arise from electromagnetic interactions
        between atoms, and therefore are not considered to be
        fundamental forces of physics.

\begin{eg}{Decay of $^{212}\zu{Pb}$}
        As an example, consider the radioactive isotope of lead
        $^{212}\zu{Pb}$. It contains 82 protons and 130 neutrons. It
        decays by the process $n\rightarrow p + e^- + \bar{\nu}$ . The newly
        created proton is held inside the nucleus by the strong
        nuclear force, so the new nucleus contains 83 protons and
        129 neutrons. Having 83 protons makes it the element
        bismuth, so it will be an atom of $^{212}\zu{Bi}$.
\end{eg}

        In a reaction like this one, the electron flies off at high
        speed (typically close to the speed of light), and the
        escaping electrons are the things that make large amounts of
        this type of radioactivity dangerous. The outgoing electron
        was the first thing that tipped off scientists in the early
        1900s to the existence of this type of radioactivity.
        Since they didn't know that the outgoing particles were
        electrons, they called them beta particles, and this type of
        radioactive decay was therefore known as beta decay. A
        clearer but less common terminology is to call the two
        processes electron decay and positron decay.

        The neutrino or antineutrino emitted 
        in such a reaction pretty much ignores all matter, because its
        lack of charge makes it immune to electrical forces, and it
        also remains aloof from strong nuclear interactions. Even if
        it happens to fly off going straight down, it is almost
        certain to make it through the entire earth without
        interacting with any atoms in any way. It ends up flying
        through outer space forever. The neutrino's behavior makes
        it exceedingly difficult to detect, and when beta decay was
        first discovered nobody realized that neutrinos even
        existed. We now know that the neutrino carries off some of
        the energy produced in the reaction, but at the time it
        seemed that the total energy afterwards (not counting the
        unsuspected neutrino's energy) was greater than the total
        energy before the reaction, violating conservation of
        energy. Physicists were getting ready to throw conservation
        of energy out the window as a basic law of physics when
        indirect evidence led them to the conclusion that neutrinos existed.

\startdqs

\begin{dq}
        In the reactions 
                $\text{n} \rightarrow \text{p} + \zu{e}^- +  \bar{\nu}$
        and
                $\text{p} \rightarrow \text{n} + \zu{e}^+ + \nu$,
        verify that charge is conserved. In beta
        decay, when one of these reactions happens to a neutron or
        proton within a nucleus, one or more gamma rays may also be
        emitted. Does this affect conservation of charge? Would it
        be possible for some extra electrons to be released without
        violating charge conservation?
\end{dq}
 %
\begin{dq}
        When an antielectron and an electron annihilate each
        other, they produce two gamma rays. Is charge conserved in this reaction?
\end{dq}

 %%----------------------------------------------
<%
  fig(
    'fusion-collage',
    %q{%
      %
      1. Our sun's source of energy is nuclear fusion, so nuclear fusion is also the
      source of power for all life on earth, including, 2, this rain forest in Fatu-Hiva.
      3. The first release of energy by nuclear fusion through human technology was
      the 1952 Ivy Mike test at the Enewetak Atoll. 
      4. This array of gamma-ray detectors is called GAMMASPHERE.
      During        operation, the array is closed up, and a beam of ions produced by a
              particle accelerator strikes a target at its center, producing nuclear
              fusion reactions. The gamma rays can be studied for information about
              the structure of the fused nuclei, which are typically varieties not
              found in nature.
      5. Nuclear fusion promises to be a clean, inexhaustible source of energy.
      However, the goal of commercially viable nuclear fusion power has remained
      elusive, due to the engineering difficulties involved in magnetically
      containing a plasma (ionized gas) at a sufficiently high temperature and density.
      This photo shows the experimental JET reactor, with the device opened up on the
      left, and in action on the right.
      
    },
    {
      'width'=>'fullpage'
    }
  )
%>

  <% end_sec('weak-force') %>
<% begin_sec("Fusion",nil,'fusion') %>

        As we have seen, heavy nuclei tend to fly apart because each
        proton is being repelled by every other proton in the
        nucleus, but is only attracted by its nearest neighbors. The
        nucleus splits up into two parts, and as soon as those two
        parts are more than about 1 fm apart, the strong nuclear
        force no longer causes the two fragments to attract each
        other. The electrical repulsion then accelerates them,
        causing them to gain a large amount of kinetic energy. This
        release of kinetic energy is what powers nuclear reactors and fission bombs.

        It might seem, then, that the lightest nuclei would be the
        most stable, but that is not the case. Let's compare an
        extremely light nucleus like $^4\zu{He}$ with a somewhat heavier
        one, $^{16}\zu{O}$. A neutron or proton in $^4\zu{He}$ can be
        attracted by the three others, but in $^{16}\zu{O}$, it might
        have five or six neighbors attracting it. The $^{16}\zu{O}$
        nucleus is therefore more stable. 

        It turns out that the most stable nuclei of all are those
        around nickel and iron, having about 30 protons and 30
        neutrons. Just as a nucleus that is too heavy to be stable
        can release energy by splitting apart into pieces that are
        closer to the most stable size, light nuclei can release
        energy if you stick them together to make bigger nuclei that
        are closer to the most stable size. Fusing one nucleus with
        another is called nuclear fusion. Nuclear fusion is what
        powers our sun and other stars.

 %%----------------------------------------------
<% end_sec('fusion') %>
<% begin_sec("Nuclear energy and binding energies",4,'nuclearenergy') %>
        In the same way that chemical reactions can be classified as
        exothermic (releasing energy) or endothermic (requiring
        energy to react), so nuclear reactions may either release or
        use up energy. The energies involved in nuclear reactions
        are greater by a huge factor. Thousands of tons of coal
        would have to be burned to produce as much energy as would
        be produced in a nuclear power plant by one kg of fuel.

        Although nuclear reactions that use up energy (endothermic
        reactions) can be initiated in accelerators, where one
        nucleus is rammed into another at high speed, they do not
        occur in nature, not even in the sun. The amount of kinetic
        energy required is simply not available.

        To find the amount of energy consumed or released in a
        nuclear reaction, you need to know how much nuclear interaction energy, $U_{nuc}$,
        was stored or released. Experimentalists have determined the
        amount of nuclear energy stored in the nucleus of every
        stable element, as well as many unstable elements. This is
        the amount of mechanical work that would be required to pull
        the nucleus apart into its individual neutrons and protons,
        and is known as the nuclear\index{binding energy!nuclear} binding energy.

\begin{eg}{A reaction occurring in the sun}
        The sun produces its energy through a series of nuclear
        fusion reactions. One of the reactions is
        \begin{equation*}
                ^1\zu{H} + ^2\zu{H}\rightarrow ^3\zu{He} + \mygamma
        \end{equation*}
        The excess energy is almost all carried off by the gamma ray
        (not by the kinetic energy of the helium-3 atom). The
        binding energies in units of pJ (picojoules) are:

        \noindent\begin{tabular}{ll}
                $^1\zu{H}$                 & 0 J \\
                $^2\zu{H}$                & 0.35593 pJ \\
                $^3\zu{He}$                 & 1.23489 pJ 
        \end{tabular}

        The total initial nuclear energy is 0 pJ+0.35593 pJ, and
        the final nuclear energy is 1.23489 pJ, so by conservation
        of energy, the gamma ray must carry off 0.87896 pJ of
        energy. The gamma ray is then absorbed by the sun and converted to heat.
\end{eg}

<% self_check('hbindingenergy',<<-'SELF_CHECK'
          Why is the binding energy of $^1\zu{H}$ exactly equal to zero?
          SELF_CHECK
  ) %>

<%
  fig(
    'chartofnuclei',
    %q{%
      The known nuclei, represented on a chart
              of proton number versus neutron number. Note the two nuclei in the bottom row with zero protons.
    },
    {
      'width'=>'fullpage'
    }
  )
%>
        
        Figure \ref{fig:chartofnuclei} is a compact way of showing the vast
        variety of the nuclei. Each box represents a particular
        number of neutrons and protons. The black boxes are nuclei
        that are stable, i.e., that would require an input of energy
        in order to change into another. The gray boxes show all the
        unstable nuclei that have been studied experimentally. Some
        of these last for billions of years on the average before
        decaying and are found in nature, but most have much shorter
        average lifetimes, and can only be created and studied in the laboratory.

        The curve along which the stable nuclei lie is called the
        line of stability. Nuclei along this line have the most
        stable proportion of neutrons to protons. For light nuclei
        the most stable mixture is about 50-50, but we can see that
        stable heavy nuclei have two or three times more neutrons
        than protons. This is because the electrical repulsions of
        all the protons in a heavy nucleus add up to a powerful
        force that would tend to tear it apart. The presence of a
        large number of neutrons increases the distances among the
        protons, and also increases the number of attractions due to
        the strong nuclear force.

<% end_sec('nuclearenergy') %>
