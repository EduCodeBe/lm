(a) We have 
\begin{align*}
  & \der P = \rho g \der y \\
  & \Delta P = \int \rho g \der y,
\end{align*}
and since we're taking water to be incompressible, and $g$ doesn't change very much over 11 km of
height, we can treat $\rho$ and $g$ as constants and take them outside the integral.
\begin{align*}
  \Delta P &= \rho g \Delta y \\
           &= (1.0\ \zu{g}/\zu{cm}^3)(9.8\ \munit/\sunit^2)(11.0\ \zu{km}) \\
           &= (1.0\times10^3\ \kgunit/\munit^3)(9.8\ \munit/\sunit^2)(1.10\times10^4\ \munit) \\
           &= 1.0\times10^8\ \zu{Pa} \\
           &= 1.0\times10^3\ \text{atm}.
\end{align*}
The precision of the result is limited to a few percent, due to the compressibility of the water,
so we have at most two significant figures. If the change in pressure were exactly a thousand
atmospheres, then the pressure at the bottom would be 1001 atmospheres; however, this distinction is
not relevant at the level of approximation we're attempting here.\\
(b) Since the air in the bubble is in thermal contact with the water, it's reasonable to assume
that it keeps the same temperature the whole time. The ideal gas law is $PV=nkT$, and rewriting this
as a proportionality gives
\begin{equation*}
  V\propto P^{-1},
\end{equation*}
or
\begin{equation*}
  \frac{V_f}{V_i} = \left(\frac{P_f}{P_i}\right)^{-1} \approx 10^3.
\end{equation*}
Since the volume is proportional to the cube of the linear dimensions, the growth in
radius is about a factor of 10.
