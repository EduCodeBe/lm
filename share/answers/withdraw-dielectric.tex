An (idealized) battery is a circuit
element that always maintains the same voltage difference across
itself, so by the loop rule, the voltage difference across the
capacitor must remain unchanged, even while the dielectric is being
withdrawn. The bound charges on the surfaces of the dielectric have
been attracting the free charges in the plates, causing them to charge
up more than they ordinarily would have. As the dielectric is
withdrawn, the capacitor will be partially discharged, and we will
observe a current in the ammeter. Since the dielectric is attracted to
the plates, positive work is done in extracting it, indicating that
there must be an increase in the electrical energy stored in the
capacitor. This may seem paradoxical, since the energy stored in a
capacitor is $(1/2)CV^2$, and we are decreasing the capacitance.
However, the energy $(1/2)CV^2$ is calculated in terms of the work
required to deposit the free charge on the plates. In addition to this
energy, there is also energy stored in the dielectric itself.  By
moving its bound charges farther away from the free charges in the
plates, to which they are attracted, we have increased their
electrical energy. This energy of the bound charges is inaccessible to
the electric circuit.
