(a) By Newton's third law, the forces are $F$ and $-F$. Pick a coordinate system in which skater 1
moves in the negative $x$ direction due to a force $-F$.
Since the forces are constant, the accelerations are also constant, and
the distances moved by their centers of mass are $\Delta x_1=(1/2)a_1T^2$ and
$\Delta x_2=(1/2)a_2T^2$. The accelerations are $a_1=-F/m_1$ and
$a_2=F/m_2$. We then have
\begin{align*}
  \ell_f-\ell_\zu{0} &= \Delta x_2-\Delta x_1 \\
                     &= \frac{1}{2} F \left(\frac{1}{m_1}+\frac{1}{m_2}\right)T^2,
\end{align*}
resulting in
\begin{equation*}
  T = \sqrt{\frac{2(\ell_f-\ell_\zu{0})}{F\left(\frac{1}{m_1}+\frac{1}{m_2}\right)}}
\end{equation*}
(b) 
\begin{equation*}
  \sqrt{\frac{\munit}{\nunit/\kgunit}} = \sqrt{\frac{\munit}{\kgunit\unitdot\munit\sunit^{-2}\kgunit^{-1}}}
              = \sunit
\end{equation*}
(c) If the force is bigger, we expect physically that they will reach arm's length more quickly.
Mathematically, a bigger $F$ on the bottom results in a smaller $T$.\\
(d) If one of the masses is very small, then $1/m_1+1/m_2$ gets very big, and $T$ gets very small.
This makes sense physically. If you flick a flea off of yourself, contact is broken very quickly.
