(a) If there was no friction, the angle of repose would
be zero, so the coefficient of static friction, $\mu_s$, will
definitely matter. We also make up symbols $\theta$, $m$ and $g$ for
the angle of the slope, the mass of the object, and the
acceleration of gravity. The forces form a triangle just
like the one in example \ref{eg:rampforces}
 on page \pageref{eg:rampforces}, but instead of a force applied
by an external object, we have static friction, which is
less than $\mu_sF_n$. As in that example, $F_s=mg \sin \theta$, and
$F_s<\mu_s F_n$, so
\begin{equation*}
	mg \sin \theta<\mu_sF_n\eqquad.
\end{equation*}
From the same triangle, we have $F_n=mg cos \theta$, so
\begin{equation*}
	mg \sin \theta < \mu_s mg \cos \theta\eqquad.
\end{equation*}
Rearranging,
\begin{equation*}
	\theta < \tan^{-1} \mu_s\eqquad.
\end{equation*}
(b) Both $m$ and $g$ canceled out, so the angle of repose would
be the same on an asteroid.
