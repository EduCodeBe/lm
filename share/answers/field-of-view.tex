Since $d_o$ is much greater than $d_i$, the lens-film
distance $d_i$ is essentially the same as $f$. (a) Splitting
the triangle inside the camera into two right triangles,
straightforward trigonometry gives
\begin{equation*}
		 \theta =  2 \tan^{-1}\frac{w}{2f}
\end{equation*}
 for the field of view. This comes out to be $39\degunit$ and
$64\degunit$ for the two lenses. (b) For small angles, the
tangent is approximately the same as the angle itself,
provided we measure everything in radians. The equation
above then simplifies to
\begin{equation*}
		 \theta  =  \frac{w}{f}
\end{equation*}
The results for the two lenses are $.70\ \text{rad}=40\degunit$, and
$1.25\ \text{rad}=72\degunit$. This is a decent approximation.

(c) With the 28-mm lens, which is closer to the film, the
entire field of view we had with the 50-mm lens is now
confined to a small part of the film. Using our small-angle
approximation $\theta =w/f$, the amount of light contained
within the same angular width $\theta $ is now striking a
piece of the film whose linear dimensions are smaller by the
ratio 28/50. Area depends on the square of the linear
dimensions, so all other things being equal, the film would
now be overexposed by a factor of $(50/28)^2=3.2$. To
compensate, we need to shorten the exposure by a factor of 3.2.



