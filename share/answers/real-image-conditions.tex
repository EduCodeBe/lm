(a) The situation being described requires a real image, since the rays need
to converge at a point on Becky's neck.
See the ray diagram drawn with thick lines, showing object location $\zu{o}$
and image location $\zu{i}$.

\anonymousinlinefig{real-image-conditions}

If we move the object farther away, to $\zu{o}'$ the cone of rays
intercepted by the lens (thin lines) is less strongly diverging, and the
lens is able to bring it to a closer focus, at $\zu{i}'$. In the diagrams,
we see that a smaller $\theta_o$ leads to a larger
$\theta_i$, so the signs in the equation $\pm\theta_o\pm\theta_i=\theta_f$ must be the same, and
therefore both positive, since $\theta_f$ is positive by definition.
The equation relating the image and object locations must be
 $1/f=1/d_o+1/d_i$.\\

(b) The case with $d_i=f$ is not possible, because then we need $1/d_o=0$,
i.e., $d_o=\infty$. Although it is possible in principle to have an object
so far away that it is practically at infinity, that is not possible in
this situation, since Zahra can't take her lens very far away from the fire.
By the way, this means that the \emph{focal length} $f$ is not where the
\emph{focus} happens --- the focus happens at $d_i$.

For similar reasons, we can't have $d_o=f$.

Since all the variables are positive, we must have $1/d_o$ and
$1/d_i$ both less than $1/f$. This implies that $d_o>f$ and
$d_i>f$. Of the nine logical possibilities in the table, only
this one is actually possible for this real image.
