This is a case where it's probably easiest to draw the
acceleration graph first. While the ball is in the air (bc,
de, etc.), the only force acting on it is gravity, so it
must have the same, constant acceleration during each hop.
Choosing a coordinate system where the positive $x$ axis
points up, this becomes a negative acceleration (force in
the opposite direction compared to the axis). During the
short times between hops when the ball is in contact with
the ground (cd, ef, etc.), it experiences a large acceleration,
which turns around its velocity very rapidly. These short
positive accelerations probably aren't constant, but it's
hard to know how they'd really look. We just idealize them
as constant accelerations. Similarly, the hand's force on
the ball during the time ab is probably not constant, but we
can draw it that way, since we don't know how to draw it
more realistically. Since our acceleration graph consists of
constant-acceleration segments, the velocity graph must
consist of line segments, and the position graph must
consist of parabolas. On the $x$ graph, I chose zero to be
the height of the center of the ball above the floor when
the ball is just lying on the floor. When the ball is
touching the floor and compressed, as in interval cd, its
center is below this level, so its $x$ is negative.

\anonymousinlinefig{soln-graph-bouncing}



