(a) Let's do 10.0 g and 1000 g. The arithmetic mean
is 505 grams. It comes out to be 0.505 kg, which is
consistent. (b) The geometric mean comes out to be 100 g
or 0.1 kg, which is consistent. (c) If we multiply meters by
meters, we get square meters. Multiplying grams by grams
should give square grams! This sounds strange, but it makes
sense. Taking the square root of square grams ($\zu{g}^2$) gives
grams again. (d) No. The superduper mean of two quantities
with units of grams wouldn't even be something with units of
grams! Related to this shortcoming is the fact that the
superduper mean would fail the kind of consistency test
carried out in the first two parts of the problem.




