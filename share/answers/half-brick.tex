If the full-sized brick A undergoes some process, such as heating it with a blowtorch,
then we want to be able to apply the equation $\Delta S=Q/T$
to either the whole brick or half of it, which would be identical to B.
When we redefine the boundary of the system to contain only half of the brick,
the quantities $\Delta S$ and $Q$ are each half as big, because entropy and energy are
additive quantities. $T$, meanwhile, stays the
same, because temperature isn't additive --- two cups of coffee aren't twice as hot as one.
These changes to the variables leave the equation consistent, since each side has been divided by 2.
