By symmetry, the field is always directly toward or away from the center.
We can therefore calculate it along the $x$ axis, where $r=x$, and the
result will be valid for any location at that distance from the center.
The electric field is minus the derivative of the potential,
\begin{align*}
        E       &= -\frac{\der m4_ifelse(__fac,1,[:\phi:],[:V:])}{\der x} \\
                &= -\frac{\der}{\der x}\left(x^{-1}e^{-x}\right) \\
                &= x^{-2}e^{-x}+x^{-1}e^{-x} \\
\end{align*}
At small $x$, near the proton, the first term dominates, and the exponential
is essentially 1, so we have $E\propto x^{-2}$, as we expect from the Coulomb
force law. At large $x$, the second term dominates, and
the field approaches zero faster than an exponential.
