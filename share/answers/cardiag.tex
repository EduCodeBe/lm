There are two reasonable possibilities we could imagine --- neither
of which ends up making much sense --- if we insist on the
straight-line trajectory. (1) If the car has constant speed along the
line, then in the * frame we see it going straight down at constant
speed. It makes sense that it goes straight down in the * frame of
reference, since in that frame it was never moving horizontally, and
there's no reason for it to start. However, it doesn't make sense
that it goes down with constant speed, since falling objects are
supposed to speed up the whole time they fall. This violates both
Galilean relativity and conservation of energy. (2) If it's speeding
up and moving along a diagonal line in the original frame, then it
might be conserving energy in one frame or the other. But if it's
speeding up along a line, then as seen in the original frame, both
its vertical motion and its horizontal motion must be speeding up. If
its horizontal velocity is increasing in the original frame, then it
can't be zero and remain zero in the * frame. This violates Galilean
relativity, since in the * frame the car apparently starts moving
sideways for no reason.
