 (a) Based on our knowledge of the field pattern of a
current-carrying loop, we know that the magnetic field must
be either into or out of the page. This makes sense, since
that would mean the field is always perpendicular to the
plane of the electrons' motion; if it was in their plane of
motion, then the angle between the $v$ and $B$ vectors would
be changing all the time, but we see no evidence of such
behavior. With the field turned on, the force vector is
apparently toward the center of the circle. Let's analyze
the force at the moment when the electrons have started
moving, which is at the right side of the circle. The force
is to the left. Since the electrons are negatively charged
particles, we know that if we sight along the force vector,
the $B$ vector must be counterclockwise from the $v$ vector.
The magnetic field must be out of the page. (b) Looking at
figure \figref{magnetic-field-equations} on page \pageref{fig:magnetic-field-equations}, we can tell that the
current in the coils must be counterclockwise as viewed from
the perspective of the camera. (c) Electrons are negatively
charged, so to produce a counterclockwise current, the
electrons in the coils must be going clockwise, i.e.,
they are counterrotating compared to the beam. (d) The
current in the coils is keep the electrons in the beam from
going straight, i.e. the force is a repulsion. This makes
sense by comparison with figure w in section 23.2:
like charges moving in opposite directions repel one another.
