The boat's velocity relative to the land equals
the vector sum of its velocity with respect to the water and
the water's velocity with respect to the land,
\begin{equation*}
            \vc{v}_{BL} =  \vc{v}_{BW}+  \vc{v}_{WL}\eqquad.
\end{equation*}
If the boat is to travel straight across the river, i.e.,
along the $y$ axis, then we need to have $\vc{v}_{BL,x}=0$. This $x$
component equals the sum of the $x$ components of the other two vectors,
\begin{equation*}
            \vc{v}_{BL,x}  =  \vc{v}_{BW,x} +  \vc{v}_{WL,x}\eqquad,
\end{equation*}
or
\begin{equation*}
            0  =  -|\vc{v}_{BW}| \sin  \theta  + |\vc{v}_{WL}|\eqquad.
\end{equation*}
Solving for $\theta $, we find
\begin{align*}
            \sin  \theta  &=  |\vc{v}_{WL}|/|\vc{v}_{BW}|\eqquad,\\
\intertext{so}
            \theta &= \sin^{-1}\frac{|\vc{v}_{WL}|}{|\vc{v}_{BW}|}\eqquad.
\end{align*}
