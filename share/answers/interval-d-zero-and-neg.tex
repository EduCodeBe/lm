For $d=0$, the diagram becomes flat. The segments defining $t$ and $\interval$ coincide,
and we get $\interval=t$. This makes sense, because that's what we expect to happen in
the equation $\interval^2=t^2-(\text{const.})d^2$, in the case where $d=0$.

When $d<0$, the triangle just flips over. By symmetry, we expect that the effect on
time should be the same as for a positive $d$. This matches up in a sensible way
with the behavior of the equation $\interval^2=t^2-(\text{const.})d^2$. A negative
$d$ doesn't matter, because $d$ appears as a square in the equation.
