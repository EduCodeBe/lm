The reasoning is reminiscent of section \ref{sec:newton-gravity}. From
Newton's second law we have
\begin{equation*}
 F=ma=mv^2/r=m(2\pi r/T)^2/r=4\pi^2mr/T^2\eqquad,
\end{equation*}
and Newton's law of gravity gives
$F=GMm/r^2$, where $M$ is the mass of the earth. Setting
these expressions equal to each other, we have
\begin{equation*}
        4\pi ^2 mr/T^2  =  GMm/r^2\eqquad,
\end{equation*}
which gives
\begin{align*}
        r     &=  \left(\frac{GMT^2}{4\pi^2}\right)^{1/3}    \\
             &=    4.22\times10^4\  \zu{km}\eqquad.
\end{align*}
This is the distance from the center of the earth, so to
find the altitude, we need to subtract the radius of the
earth. The altitude is $3.58\times10^4$  km.



