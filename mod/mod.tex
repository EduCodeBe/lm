% (c) 2019 Ben Crowell, licensed under the Creative Commons
% Attribution-ShareAlike license,
% http://creativecommons.org/licenses/by-sa/1.0/
%
\documentclass{lmseries}
% \let\ifpdf\relax % http://tex.stackexchange.com/questions/11414/package-ifpdf-error
% --------> this fails with TeX Live 2013, conflicts with xparse
%\selectlanguage{english}
\usepackage{lmlanguage}
%\includeonly{n1/ctemp}
\inputprotcode
\makeindex
\pdfmapfile{=fullembed.map} % created by the script create_fullembed_file
\begin{document}
\myeqnspacing % Do this early and often, since it gets reset by \normalsize
% 
% The following is all related to margin kerning.
% This is activated in dp.cls using the boolean wantmarginkerning.
% The constant 1 is to allow margin kerning, but to keep it from affecting
% line breaks.
\ifthenelse{\boolean{wantmarginkerning}}{
 \setprotcode\font
 {\it \setprotcode \font}
 {\bf \setprotcode \font}
 {\bf \it \setprotcode \font}
 \pdfprotrudechars=1
}{}
% Override lmcommon.cls, show subsection numbers:
\titleformat{\subsection}
  {\normalfont\normalsize\bfseries\sffamily\raggedright\protect\sansmath}{\showsecnum{\thesubsection}}{0.6em}{}   
% Override lmcommon.cls, no special label for calculus-based problems:
\renewcommand{\hwtopboilerplate}{%
  \noindent\formatlikesubsection{\langkey}\\*
  \begin{tabular}{lll}
    \hwcheckmark & A computerized answer check is available online. \\
    \displayhwdifficulty{2} & A difficult problem.
  \end{tabular}
}
%========================= frontmatter =========================
\formatchtoc{\Large}{}{4mm}
\frontmatter
\yesiwantarabic
\renewcommand{\chapdir}{front}
\thispagestyle{empty}
\raisebox{0mm}[0mm][0mm]{%
\parbox{8.5in}{
\vspace*{236mm}\hspace{-38.5mm}\includegraphics{\chapdir/figs/cover}\\
}
}%
\\

\pagebreak[4]

  \zerosizebox{-10mm}{140mm}{
    \noindent{}copyright 2006 Benjamin Crowell\vspace{10mm}
  }

  \zerosizebox{-10mm}{159mm}{
    rev. \today\vspace{10mm}
  }



  \zerosizebox{-10mm}{213mm}{
    \noindent{}\begin{minipage}{100mm}
    \noindent{}\begin{tabular}{p{9mm}p{95mm}}
    \zerosizebox{0mm}{14mm}{\anonymousinlinefig{cc-by-sa}} &
    This book is licensed under the 
Creative Commons
Attribution-ShareAlike license, version 3.0,
http://creativecommons.org/licenses/by-sa/3.0/,
    except for those photographs and
    drawings of which I am not the author, as listed in the photo credits.
    If you agree to the license, it grants you certain privileges that
    you would not otherwise have, such as the right to copy the book,
    or download the digital version free of charge from
    www.lightandmatter.com. At your option, you may also copy this book
    under the GNU Free Documentation License version 1.2, http://www.gnu.org/licenses/fdl.txt,
    with no invariant sections, no front-cover texts, and no back-cover texts.
    \end{tabular}
    \end{minipage}
}

\yesiwantarabic
\nomarginlayout
\vspace{10mm}\begin{center}\bfseries\sffamily{}\noindent{}{\huge Brief Contents}\end{center}\vspace{10mm}

\newcommand{\brieftocpartstyle}{\large\sffamily{}}
\newcommand{\brieftocchstyle}{\normalsize\sffamily{}}
\newcommand{\brieftocvert}{7mm}
\newcommand{\brieftochoriz}{\hspace{20mm}}
\newcommand{\brieftoctabularwidth}{90mm}

\vspace{\brieftocvert}\noindent\brieftochoriz%
\brieftocchstyle\begin{tabular}{rp{\brieftoctabularwidth}}
& \textit{\brieftocpartstyle Waves and relativity}\\
\brieftocentry[\hfill]{ch:time}{Time} \\
\brieftocentry[\hfill]{ch:waves}{Waves} \\
\brieftocentry[\hfill]{ch:em-waves}{Electromagnetic waves} \\
\brieftocentry[\hfill]{ch:lorentz}{The Lorentz transformation} \\
\brieftocentry[\hfill]{ch:media}{Waves done medium well} \\
\brieftocentry[\hfill]{ch:rel-dynamics}{Relativistic energy and momentum} \\
& \textit{\brieftocpartstyle Thermodynamics}\\
\brieftocentry[\hfill]{ch:stat}{Statistics and the ideal gas} \\
\brieftocentry[\hfill]{ch:macro}{The macroscopic picture} \\
\brieftocentry[\hfill]{ch:entropy}{Entropy} \\
& \textit{\brieftocpartstyle Optics}\\
\brieftocentry[\hfill]{ch:images-1}{Images, qualitatively} \\
\brieftocentry[\hfill]{ch:images-2}{Images, quantiitatively} \\
\brieftocentry[\hfill]{ch:wave-optics}{Wave optics} \\
& \textit{\brieftocpartstyle The microscopic description of matter and quantum physics}\\
\brieftocentry[\hfill]{ch:atom}{The atom and the nucleus} \\
\brieftocentry[\hfill]{ch:decay}{Probability distributions and a first glimpse of quantum physics} \\
\brieftocentry[\hfill]{ch:light-as-a-particle}{Light as a particle} \\
\brieftocentry[\hfill]{ch:matter-as-a-wave}{Matter as a wave} \\
\brieftocentry[\hfill]{ch:schrodinger}{The Schr\"odinger equation} \\
\brieftocentry[\hfill]{ch:quantum-ang-mom}{Quantization of angular momentum} \\
\end{tabular}




\onecolumn\vfill
\mynormaltype

\pagebreak[4]

\vspace{0mm}
\begin{center}
\noindent\huge\bfseries\sffamily{}Contents\mynormaltype
\end{center}
\vspace{0mm}
\begin{multicols}{2}
  \tableofcontents
  \setcounter{unbalance}{0}
\end{multicols}
\normallayout\onecolumn

%========================= main matter =========================
\mainmatter
%-- I want the whole book numbered sequentially, arabic:
  \pagenumbering{arabic} 
  \addtocounter{page}{10}
\parafmt
\myeqnspacing % Do this early and often, since it gets reset by \normalsize
%========================= chapters =========================
\wugga%used to be needed after preface
    \renewcommand{\chapdir}{waves}\include{waves/atemp}
    \renewcommand{\chapdir}{waves}\include{waves/btemp}
    \renewcommand{\chapdir}{waves}\include{waves/ctemp}
    \renewcommand{\chapdir}{waves}\include{waves/dtemp}
    \renewcommand{\chapdir}{waves}\include{waves/etemp}
    \renewcommand{\chapdir}{waves}\include{waves/ftemp}
    \renewcommand{\chapdir}{thermo}\include{thermo/atemp}
    \renewcommand{\chapdir}{thermo}\include{thermo/btemp}
    \renewcommand{\chapdir}{thermo}\include{thermo/ctemp}
    \renewcommand{\chapdir}{optics}\include{optics/atemp}
    \renewcommand{\chapdir}{optics}\include{optics/btemp}
    \renewcommand{\chapdir}{optics}\include{optics/ctemp}
    \renewcommand{\chapdir}{quantum}\include{quantum/atemp}
    \renewcommand{\chapdir}{quantum}\include{quantum/btemp}
    \renewcommand{\chapdir}{quantum}\include{quantum/ctemp}
    \renewcommand{\chapdir}{quantum}\include{quantum/dtemp}
    \renewcommand{\chapdir}{quantum}\include{quantum/etemp}
	\formatchtoc{\large}{\quad\contentspage}{4mm} % This has to go before the last chapter.
    \renewcommand{\chapdir}{quantum}\include{quantum/ftemp}
        \addtocontents{toc}{\vspace{5mm}}% For reasons I don't understand, this actually comes *after* the final chapter.
%
    \widelayout
    \renewcommand{\chapdir}{end}
    \include{end/skillstemp}
    \onecolumn\include{end/hwanstemp}
    \include{end/mathtemp}
    \onecolumn\include{end/photocreditstemp}
%========================= index =========================
\label{splits:index}\printindex
%========================= data tables, etc. =========================
        \blankchaptermarks\label{splits:data}
        \begin{datatablepage}\label{datatable}

%------------------------------------------------------------------------------

\begin{datatable}{Metric Prefixes}
\noindent\begin{tabular}{lll}
M-	& mega-			& $10^6$ \\
k-	& kilo-			& $10^3$ \\
m-	& milli-		& $10^{-3}$ \\
$\mu$- (Greek mu) & micro-	& $10^{-6}$ \\
n-	& nano-			& $10^{-9}$ \\
p-	& pico-			& $10^{-12}$ \\
f-	& femto-		& $10^{-15}$
\end{tabular}

\noindent\datatablenote{(Centi-, $10^{-2}$, is used only in the centimeter.)}
\end{datatable}

\vspace{2.5mm}

%------------------------------------------------------------------------------

\begin{datatable}{The Greek Alphabet}
\noindent\begin{tabular}{lll|lll}
$\alpha$	& A			& alpha	  & $\nu$	& N		& nu \\
$\beta$		& B			& beta    & $\xi$		& $\Xi$		& xi \\
$\gamma$	& $\Gamma$	        & gamma	  & o	& O		& omicron \\
$\delta$	& $\Delta$		& delta	  & $\pi$		& $\Pi$		& pi \\
$\epsilon$	& E			& epsilon &  $\rho$	& P		& rho \\
$\zeta$		& Z			& zeta	  & $\sigma$	& $\Sigma$	& sigma \\
$\eta$		& H			& eta	  & $\tau$ & T & tau\\
$\theta$	& $\Theta$		& theta	  & $\upsilon$ & Y & upsilon \\
$\iota$		& I		        & iota    & $\phi$ & $\Phi$ & phi\\
$\kappa$	& K	                & kappa   & $\chi$ & X & chi\\
$\lambda$       & $\Lambda$             & lambda  & $\psi$ & $\Psi$ & psi\\
$\mu$	        & M                     & mu      & $\omega$ & $\Omega$ & omega
\end{tabular}
\end{datatable}

\vspace{2.5mm}

%------------------------------------------------------------------------------

\begin{datatable}{Subatomic Particles}
\noindent\begin{tabular}{lll}
\datatablecolhdr{particle}	& \datatablecolhdr{mass (kg)}	& \datatablecolhdr{radius (fm)} \\
electron	& $9.109\times10^{-31}$		& $\lesssim0.01$\\
proton	& $1.673\times10^{-27}$		& $\sim{}1.1$\\
neutron	& $1.675\times10^{-27}$			& $\sim{}1.1$
\end{tabular}

\noindent\datatablenote{The radii of protons and neutrons can only be given
approximately, since they have fuzzy surfaces. For
comparison, a typical atom is about a million fm in radius.}
\end{datatable}

\vfill\pagebreak[4]

%------------------------------------------------------------------------------

\begin{datatable}{Notation and Units}
\noindent\begin{tabular}{lll}
\datatablecolhdr{quantity}	& \datatablecolhdr{unit}	& \datatablecolhdr{symbol} \\
distance	& meter, m	& $x, \Delta{}x$ \\
time	& second, s	& $t, \Delta{}t$ \\
mass	& kilogram, kg	& $m$ \\
density	& $\kgunit/\munit^3$	& $\rho$  \\
velocity	& m/s	& $\vc{v}$ \\
acceleration	& $\munit/\sunit^2$	& $\vc{a}$ \\
force	& $\nunit=\kgunit\unitdot\munit/\sunit^2$	& $\vc{F}$ \\
pressure & Pa=$1\ \nunit/\munit^2$	& $P$ \\
energy	& $\junit=\kgunit\unitdot\munit^2/\sunit^2$	& $E$ \\
power	& $\zu{W} = 1\ \junit/\sunit$	& $P$ \\
momentum	& $\kgunit\unitdot\munit/\sunit$	& $\vc{p}$ \\
angular momentum	& $\kgunit\unitdot\munit^2/\sunit$ or $\junit\unitdot\sunit$	& $\vc{L}$ \\
period	& s	& $T$\\
wavelength & m & $\lambda$ \\
frequency & $\sunit^{-1}$ or Hz & $f$\\
gamma factor & unitless & $\gamma$\\
probability & unitless & $P$\\
prob. distribution & various & $D$\\
electron wavefunction & $\munit^{-3/2}$ & $\Psi$\\
\end{tabular}

%\begin{tabular}{ll}
%\datatablecolhdr{symbol} & \datatablecolhdr{meaning}\\
%$\propto$ & is proportional to \\
%$\approx$ & is approximately equal to \\
%$\sim$ & is on the order of
%\end{tabular}

\end{datatable}

%------------------------------------------------------------------------------

\begin{datatable}{Earth, Moon, and Sun}
\noindent\begin{tabular}{llll}
\datatablecolhdr{body}		&	\datatablecolhdr{mass (kg)}		&	\datatablecolhdr{radius (km)}	&	\datatablecolhdr{radius of orbit (km)} \\
earth	&	$5.97\times10^{24}$	&	$6.4\times10^{3}$	&	$1.49\times10^{8}$\\
moon	&	$7.35\times10^{22}$	&	$1.7\times10^{3}$	&	$3.84\times10^{5}$\\
sun		&	$1.99\times10^{30}$	&	$7.0\times10^{5}$	&	---\\
\end{tabular}
\end{datatable}

%------------------------------------------------------------------------------
\begin{datatable}{Fundamental Constants}
\noindent\begin{tabular}{ll}
gravitational constant	& $G=6.67\times10^{-11}\ \nunit\unitdot\munit^2/\kgunit^2$ \\
Coulomb constant        & $k=8.99\times10^9\ \nunit\unitdot\munit^2/\zu{C}^2$ \\
quantum of charge       & $e=1.60\times10^{-19}\ \zu{C}$\\
speed of light	        & $c=3.00\times10^8\ \zu{m/s}$ \\
Planck's constant       & $h=6.63\times10^{-34}\ \junit\unitdot\sunit$
\end{tabular}
\end{datatable}


\end{datatablepage}


\end{document}
